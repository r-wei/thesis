\section{Dg cocategories: $B(n)$}\label{sec:B(n)}
%
\subsection{Background on dg cocategories}
\begin{defn} A \textbf{dg cocategory} is a 
cocategory enriched over chain complexes. 
More explicitly, a dg cocategory $B$ consists 
of the following data:
\begin{itemize}
\item A collection of objects 
  denoted $Obj(B)$;
\item For each pair of objects, $x,z \in 
  Obj(B)$, a complex $B^\bullet(x,z)$ and 
  a morphism of complexes
  $$
  \Delta_B(x,z): B^\bullet(x,z) \to 
  \prod \limits_{y \in Obj(B)}
  B^\bullet(x,y) \otimes
  B^\bullet(y,z)
  $$
  such that the following diagrams 
  commute (coassociativity):
  $$
  \xymatrixcolsep{5pc}
  \xymatrixrowsep{5pc}
  \xymatrix{
  B^\bullet(x,z)
  \ar[r]^{\Delta_B(x,z)}
  \ar[d]_{\Delta_B(x,z)}
  & \prod \limits_{y \in Obj(B)}
    B^\bullet(x,y) \otimes
    B^\bullet(y,z)
  \ar[d]^{\prod\limits_y 
    id_{B(x,y)} \otimes \Delta_B(y,z)}\\
  \prod \limits_{y \in Obj(B)}
    B^\bullet(x,y) \otimes
    B^\bullet(y,z)
  \ar[r]^{\prod\limits_y 
    \Delta_B(x,y) \otimes id_{B(y,z)}}
  & \prod \limits_{y, y^\prime \in Obj(B)}
    B^\bullet(x,y) \otimes
    B^\bullet(y,y^\prime) \otimes
    B^\bullet(y^\prime,z)  
  }
  $$
\item For each pair of objects, $x,z \in
  Obj(B)$, a morphism of complexes
  $$
  \epsilon_B(x,z): B^\bullet(x,z) \to k
  $$
  where $k$ is the ground field considered 
  as a chain complex concentrated in degree 
  0 and $\epsilon_B(x,z) = 0$ if $x\neq z$, 
  such that the following diagrams commute 
  (counitality):
  $$
  \xymatrixcolsep{5pc}
  \xymatrixrowsep{5pc}
  \xymatrix{
  B^\bullet(x,z)
  \ar[r]^{\Delta_B(x,z)}
  \ar[d]_{\Delta_B(x,z)}
  \ar[rd]_{id}
  & \prod \limits_{y \in Obj(B)} 
    B^\bullet(x,y) \otimes
    B^\bullet(y,z)
  \ar[d]^{\prod \limits_y
    \epsilon_B(x,y)
    \otimes id_{B(y,z)}} \\
  %
  \prod \limits_{y \in Obj(B)} 
    B^\bullet(x,y) \otimes
    B^\bullet(y,z)
  \ar[r]_{\prod \limits_y
    id_{B(x,y)} \otimes 
    \epsilon_B(y,z)}
  & B^\bullet(x,z).  
  }
  $$
\end{itemize}
\end{defn}
We will denote a dg cocategory with its 
cocomposition and counit as $(B, \Delta_B, 
\epsilon_B)$. To make the notation more 
readable, when the meaning is clear, 
we will omit references to the 
objects and write $\Delta_B$ instead of 
$\Delta_B(x,z)$, $\epsilon_B$ instead of 
$\epsilon_B(x,z)$, and for the differentials 
on morphisms, $d_B$ instead of $d_B(x,z)$.
%
\begin{defn} A \textbf{functor} $F: A \to B$ 
between two dg cocategories is a functor 
between the cocategories satisfying 
$d_B\circ F(f) = F\circ d_A(f)$ for all 
morphisms $f$ in $A$.
\end{defn}
%
\begin{defn} A \textbf{conilpotent} dg 
cocategory is a dg cocategory $(B, 
\Delta_B, \epsilon_B)$ satisfying: for each 
morphism $f:x\to y$ in $B$, there exists 
$n_f \in \mathbb{N}$ such that 
$\bar{\Delta}_B^{n_f}(f) = 0$
where
\begin{align*}
\bar{\Delta}_B(x,z): B^\bullet(x,z) 
&\to 
  \prod \limits_{y \in Obj(B)} 
  B^\bullet(x,y) \otimes
  B^\bullet(y,z)\\
f
&\mapsto
\Delta_B(f)
- \sum \limits_{e_x \in 
  \epsilon_B(x,x)^{-1}(1)}
  e_x \otimes f
- \sum \limits_{e_z \in 
  \epsilon_B(z,z)^{-1}(1)}
  f \otimes e_z. 
\end{align*}
\end{defn}
%
\textbf{Fact (needs reference?):} 
If $B$ is a conilpotent dg cocategory, 
then for all $x \in Obj(B)$, 
$\epsilon_B(x,x)^{-1}(1)$ has exactly 
one element, which we will denote $e_x$.
%
\subsection{Structure of $B(n)$}
For each sequence of algebras, 
$A_0, A_1, \cdots, A_n$, 
we will define a conilpotent dg cocategory, 
$B(A_0 \to A_1 \to \cdots \to A_n \to A_0)$. 
In this chapter, we fix the sequence of 
algebras, and abbreviate 
$$
B(n):= B(A_0 \to A_1 \to \cdots \to A_n \to A_0).
$$
%
\subsubsection{Objects}
$B(n)$ has objects tuples $(f_0, f_1, \cdots, f_n)$ where $f_i : A_i \to A_{i+1 \textrm{ (mod n+1)}}$, $0 \leq i \leq n$, are maps of algebras. We can picture an object in $B(n)$ as follows:

\begin{equation*}
\xymatrix{
A_0 \ar[r]^{f_0}
& A_1 \ar[r]^{f_1}
& A_2 \ar[r]^{f_2}
& \cdots \ar[r]^{f_{n-1}}
& A_n \ar[r]^{f_n}
& A_0
}
\end{equation*}

\subsubsection{Morphisms}
The graded vector space of morphisms in 
$B(n)$ between two objects, 
$(f_0, \smdots, f_n)$ and 
$(g_0, \smdots, g_n)$, is 
%
\begin{equation*}
Bar(C^\bullet(A_0, _{f_0}{A_1}_{g_0})) \otimes Bar(C^\bullet(A_1, _{f_1}{A_2}_{g_1})) \otimes \cdots
\otimes Bar(C^\bullet(A_n, _{f_n}{A_0}_{g_n}))
\end{equation*}
%
where $Bar(C^\bullet(A, _fB_g))$ is 
the following complex: 
%
\begin{align*}
Bar(C^\bullet(A, _fB_g)) 
&:= Bar_0(C^\bullet(A, _fB_g)) \oplus 
  \bigoplus_{\substack{m \geq 1}} 
    Bar_m(C^\bullet(A, _fB_g)) \\
Bar_0(C^\bullet(A, _fB_g))    
&:= k\\ 
Bar_m(C^\bullet(A, _fB_g))
&:= 
  \bigoplus \limits_{\substack{
  h_0 = f, \\
  h_m = g, \\
  h_1, \dots, h_{m-1}\\ 
	\textrm{algebra maps}}} 
  C^\bullet(A, \, _{h_0}B_{h_1})[1] \otimes 
  C^\bullet(A, \, _{h_1}B_{h_2})[1] 
  \otimes \dots \otimes 
  C^\bullet(A, \, _{h_{m-1}}B_{h_m})[1]\\
(C^\bullet(A, _{h_i}B_{h_j}), \;
  _{h_i}\delta_{h_j})
&=
\textrm{Hochschild cochain complex, 
see Appendix \ref{chap:hochschild}}
\end{align*}
\begin{align*}
d_{Bar(C^\bullet(A, _fB_g))} 
&= 
\tilde{\delta} + b^\prime\\
\tilde{\delta}(\phi_1 \otimes \smdots \otimes \phi_m)
&=
\sum \limits_{1 \leq i \leq m}
  (-1)^{1+ \sum \limits_{j<i} |\phi_i|+1}
  \phi_1 \otimes \smdots \otimes 
  [_{h_{i-1}}\delta_{h_i}](\phi_i)
  \otimes \smdots \otimes \phi_m\\
b^\prime(\phi_1 \otimes \smdots \otimes \phi_m)
&=
\sum \limits_{1 \leq i \leq m-1}
  (-1)^{\sum \limits_{j\leq i} |\phi_i|+1}
  \phi_1 \otimes \smdots \otimes 
  \phi_i \cup \phi_{i+1}
  \otimes \smdots \otimes \phi_m\\  
\cup
&= 
\textrm{cup product on Hochschild cochains, 
see Appendix \ref{chap:hochschild}.}  
\end{align*}
(This sign convention is consistent with 
Reference \cite{T}, Section 4.6.)
%
\begin{figure}
\centerline{\xymatrix{
A_0 \ar@/^5pc/[r]^{f_{0,0}} 
\ar@/^2pc/[r]^{\big\Downarrow \phi_{0,1}}_{f_{0,1}} 
\ar@/_2pc/[r]^{\substack{\vdots\\ f_{0,k_0-1}\\ \\}}
\ar@/_5pc/[r]^{\substack{\phi_{0,k_0} \\ \big\Downarrow}}_{f_{0,k_0}}
& A_1 \ar@/^5pc/[r]^{f_{1,0}} 
\ar@/^2pc/[r]^{\big\Downarrow \phi_{1,1}}_{f_{1,1}} 
\ar@/_2pc/[r]^{\substack{\vdots\\ f_{1,k_1-1}\\ \\}}
\ar@/_5pc/[r]^{\substack{\phi_{1,k_1} \\ \big\Downarrow}}_{f_{1,k_1}}
& A_2 \ar@{.>}@/^5pc/[r] 
\ar@{.>}@/^2pc/[r]_{\substack{\\ \\ \vdots}} 
\ar@{.>}@/_2pc/[r]
\ar@{.>}@/_5pc/[r]
& \cdots \ar@{.>}@/^5pc/[r] 
\ar@{.>}@/^2pc/[r]_{\substack{\\ \\ \vdots}} 
\ar@{.>}@/_2pc/[r]
\ar@{.>}@/_5pc/[r]
& A_n \ar@/^5pc/[r]^{f_{n,0}} 
\ar@/^2pc/[r]^{\big\Downarrow \phi_{n,1}}_{f_{n,1}} 
\ar@/_2pc/[r]^{\substack{\vdots\\ f_{n,k_n-1}\\ \\}}
\ar@/_5pc/[r]^{\substack{\phi_{n,k_n} \\ \big\Downarrow}}_{f_{n,k_n}}
& A_0 
}}
\caption{A morphism in $B(n)$
from $(f_{0,0}, f_{1,0}, \cdots, f_{n,0})$ to  
$(f_{0,k_0}, f_{1,k_1}, \cdots, f_{n,k_n})$ where
$\phi_{i,j} \in C^\bullet(A_i, _{f_{j-1}}{A_{i+1 \textrm{ (mod n+1)}}}_{f_j})$}
 \label{fig:phi}
\end{figure}
%
\subsubsection{Aside on notation} \label{sec:phi_notation}
When referring to an arbitrary morphism in $B(n)$,
we will assume it is a morphism from object $(f_{0,0}, f_{1,0}, \cdots, f_{n,0})$ 
to object $(f_{0,k_0}, f_{1,k_1}, \cdots, f_{n,k_n})$.
We will denote the morphism 
$$\phi_{0,1}\smdots\phi_{0,k_0} | 
\phi_{1,1}\smdots\phi_{1,k_1} | \dots |
\phi_{n,1}\smdots\phi_{n,k_n}$$ 
where $\phi_{i,j} \in C^\bullet(A_i, _{f_{j-1}}
{A_{i+1 \textrm{ (mod n+1)}}}_{f_j})$. 
See Figure \ref{fig:phi} for a picture of 
this morphism.
%
\subsubsection{Differential on $B(n)$}
Putting everything together, 
the differential on \newline $B(n)(
(f_{0,0}, \smdots, f_{n,0}), 
(f_{0,k_0}, \smdots, f_{n,k_n}))$ is
\begin{align*}
&\phantom{{}={}}
d_{B(n)} \big(
  \phi_{0,1}\smdots\phi_{0,k_0} | \smdots |
  \phi_{n,1}\smdots\phi_{n,k_n} \big)\\
&= 
\sum \limits_{0 \leq i \leq n}
  (-1)^{\sum \limits_{p<i;q}
  |\phi_{p,q}|+1}
  \phi_{0,1}\smdots | \smdots |
  d_{Bar(C^\bullet(A_i, A_{i+1}))} 
  (\phi_{i,1}\smdots\phi_{i,k_i}) | \smdots |
  \smdots\phi_{n,k_n} 
\end{align*}
%
\subsubsection{Counit}
Define
\begin{align*}
\epsilon_{B(n)}((f_{0,0},&\smdots,f_{n,0}), 
  (f_{0,k_0},\smdots,f_{n,k_n})):
B(n)((f_{0,0},\smdots,f_{n,0}), 
  (f_{0,k_0},\smdots,f_{n,k_n})) =\\
&=
Bar(C^\bullet(A_0, _{f_{0,0}}{A_1}_{f_{0,k_0}})) 
  \otimes \cdots \otimes 
  Bar(C^\bullet(A_0, _{f_{n,0}}{A_1}_{f_{n,k_n}}))
  \to \\
&\xrightarrow{project}
  Bar_0(C^\bullet(A_0, _{f_{0,0}}{A_1}_{f_{0,k_0}})) 
  \otimes \cdots \otimes 
  Bar_0(C^\bullet(A_0, _{f_{n,0}}{A_1}_{f_{n,k_n}}))
\cong k.
\end{align*}
%
\subsubsection{Cocomposition}
We have a coassociative map of complexes
%
\begin{align*}
\Delta_{A, _fB_g}: Bar(C^\bullet(A, _fB_g)) 
&\to \bigoplus\limits_{h: A \to B} Bar(C^\bullet(A, _fB_h)) \otimes Bar(C^\bullet(A, _hB_g)) \\
\phi_1 \cdots \phi_k 
&\mapsto \sum\limits_{1\leq i \leq k-1} 
\phi_1 \cdots \phi_i \otimes 
  \phi_{i+1} \cdots \phi_k\\
&\phantom{{}\mapsto \sum{}}
  + e_f\otimes \phi_1 \dots \phi_k + 
  \phi_1 \dots \phi_k \otimes e_g  
\end{align*}
%
where 
$e_f = 1$ in $Bar_0(C^\bullet(A, _fB_f)) \cong k$.
Extend $\Delta_{A, _fB_g}$ to a 
cocomposition on $B(n)$ by taking 
(up to signs)
$$
\Delta_{B(n)}((f_{0,0},\smdots,f_{n,0}), 
  (f_{0,k_0},\smdots,f_{n,k_n})) :=
\Delta_{A_0, _{f_{0,0}}{A_1}_{f_{0,k_0}}} 
\otimes \dots \otimes
\Delta_{A_0, _{f_{n,0}}{A_1}_{f_{n,k_n}}}.
$$
The sign on the term
$
(\phi_{0,1}\smdots \phi_{0,i_0} | \smdots |
\phi_{n,1}\smdots \phi_{n,i_n}) \otimes
(\phi_{0,i_0+1}\smdots \phi_{0,k_0} | \smdots |
\phi_{n,i_n+1}\smdots \phi_{n,k_n}) 
$
in the cocomposition is:
$$
(-1)^{\sum \limits_{\substack{
  1 \leq p \leq n\\ 1 \leq r<p}}\;
(\sum \limits_{1\leq q \leq i_p} |\phi_{p,q}|+1)
(\sum \limits_{i_r+1 \leq s \leq k_r} |\phi_{r,s}|+1)}
$$
%
It's clear from the definitions that
$(B(n), \Delta_{B(n)}, \epsilon_{B(n)})$ 
satisfy the diagrams needed to form a dg cocategory. 
We also see that $B(n)$ is conilpotent: 
$$
\bar{\Delta}_{B(n)}^{min(k_0,\smdots,k_n)}
(\phi_{0,1}\smdots\phi_{0,k_0}|\smdots|
\phi_{n,1}\smdots\phi_{n,k_n}) = 0.
$$