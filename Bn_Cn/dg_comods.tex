\section{Dg comodules: $C(n)$}
%
\subsection{Background on dg comodules}
\begin{defn} A \textbf{dg comodule} $C$ 
over a dg cocategory $B$ consists
of the following data:
\begin{itemize}
\item for each object $f \in B$, a complex $C^\bullet(f)$, and
\item maps of complexes
\begin{equation*}
\Delta_C(f): C^\bullet(f) \to 
\prod\limits_{g \in Obj(B)} B^\bullet(f,g) \otimes C^\bullet(g).
\end{equation*}
\end{itemize}
such that the following diagrams 
for coassociativity and 
counitality commute:
$$
\xymatrixcolsep{5pc}
\xymatrixrowsep{5pc}
\xymatrix{
C^\bullet(f)
\ar[r]^{\Delta_C(f)}
\ar[d]_{\Delta_C(f)}
& \prod \limits_{g \in Obj(B)}
B^\bullet(f,g) \otimes
C^\bullet(g)
\ar[d]^{\prod\limits_g 
id_{B(f,g)} \otimes \Delta_C(g)}\\
\prod \limits_{g \in Obj(B)}
B^\bullet(f,g) \otimes
C^\bullet(g)
\ar[r]^{\prod\limits_g 
\Delta_B(f,g) \otimes id_{C(g)}}
& \prod \limits_{g, g^\prime \in Obj(B)}
B^\bullet(f,g) \otimes
B^\bullet(g,g^\prime) \otimes
C^\bullet(g^\prime)  
}
$$
$$
\xymatrixcolsep{5pc}
\xymatrixrowsep{5pc}
\xymatrix{
C^\bullet(f)
\ar[r]^{\Delta_C(f)}
\ar[rd]_{id}
& \prod \limits_{g \in Obj(B)}
B^\bullet(f,g) \otimes
C^\bullet(g)
\ar[d]^{\prod \limits_g
\epsilon_B(f,g)
\otimes id_{C(g)}} \\
%
& C^\bullet(f).  
}
$$
\end{defn}
To simplify notation, we will 
write $\Delta_C$ instead of 
$\Delta_C(f)$ when the meaning 
is clear.
%
\begin{eg}
A dg comodule over a dg cocategory $B$ with one object, $*$, is 
a dg comodule over the counital dg coalgebra $B^\bullet(*,*)$.
\end{eg}
%
\begin{defn} A \textbf{morphism of dg comodules}
$H: C \to D$
over a dg category $B$ consists of 
maps of complexes
$\big( H_f: C^\bullet(f) \to D^\bullet(f)
\big)_{f \in Obj(B)}$ such that
for each $f\in Obj(B)$, the
following diagram commutes:
\begin{equation*}
\begin{CD}
C^\bullet(f)  @>H_f>>  D^\bullet(f) \\
@VV\Delta_{C}V  @VV\Delta_{D}V \\
\prod\limits_{g \in Obj(B)} B^\bullet(f,g) \otimes C^\bullet(g) 
@>\prod\limits_g id_B \otimes H_g>> 
\prod\limits_{g \in Obj(B)} B^\bullet(f,g)) \otimes D^\bullet(g).
\end{CD}
\end{equation*}
\end{defn}
Again, when the meaning is clear, 
we may write $H$ instead of $H_f$.
%
\begin{defn} A \textbf{conilpotent} dg 
comodule over a dg cocategory $B$ 
is a dg comodule $(C, \Delta_C)$ 
over $B$ satisfying: for each 
$f \in Obj(B)$ and each element 
$\alpha \in C^\bullet(f)$, there exists 
$n_\alpha \in \mathbb{N}$ such that 
$\bar{\Delta}_f^{n_\alpha}(\alpha) = 0$
where
\begin{align*}
\bar{\Delta}_C(f): C^\bullet(f) 
&\to 
  \prod \limits_{g \in Obj(B)} 
  B^\bullet(f,g) \otimes
  C^\bullet(g)\\
\alpha
&\mapsto
\Delta_B(\alpha) 
  - \sum \limits_{e_f \in 
  \epsilon_B(f,f)^{-1}(1)}
  e_f \otimes f.
\end{align*}
\end{defn}
%

\subsection{Structure of $C(n)$}
\textbf{Reminder:} In this chapter, we fix algebras $A_0, A_1, \cdots, A_n$.
$C(n)$ and $B(n)$ are short for $C(A_0 \to A_1 \to \cdots \to A_n \to A_0)$
and $B(A_0 \to A_1 \to \cdots \to A_n \to A_0)$, respectively.
\newline
\newline
We now give dg comodules $C(n)$ over $B(n)$.
First, we will describe the graded comodule structure;
then, we will describe the differentials.
For an object $f = (f_0, f_1, \cdots, f_n) \in B(n)$, we have
%
\begin{equation}\label{eq:C(f)}
C(n)^\bullet(f) 
= \bigoplus\limits_{g \in Obj(B(n))} B(n)^\bullet(f,g) \otimes 
C_{-\bullet}(A_0, _{comp(g)}{A_0}_{id})
\end{equation}
%
where, for $g = (g_0, g_1, \cdots, g_n)$, 
we write $comp(g) = g_n \circ g_{n-1} \circ \cdots \circ g_0$,
and $(C_\bullet(A,B),\; _gb)$ is the 
Hochschild chain complex 
(see Appendix \ref{chap:hochschild}).
We will denote a typical element of $C(n)^\bullet(f)$ 
as $$\phi_{0,1}\smdots\phi_{0,k_0} | 
\phi_{1,1}\smdots\phi_{1,k_1} | \smdots |
\phi_{n,1}\smdots\phi_{n,k_n} | \alpha$$
where $\phi_{0,1}\cdots \phi_{n,k_n}$ is a morphism
in $B(n)$ (see Section \ref{sec:phi_notation}) and
$\alpha \in C_{-\bullet}(A_0, _{f_{k_n}\cdots f_{k_0}}{A_0}_{id})$.
See Figure \ref{fig:phi|alpha} for a picture of 
a typical element of $C(n)^\bullet(f)$.
% long form of equation
% \begin{align*}
% C^\bullet(f) 
% = C_{-\bullet}(A_0, _{f_n\cdots f_1 f_0}{A_0}_{id}) \oplus \\
%   \bigoplus_{\substack{f_{i,j} \textrm{ maps of algebras}, \\
% 					   \textrm{and } f_{r,0} = f_r, \\
% 					   0\leq r\leq n}} 
% & C^\bullet(A_0, \, _{f_{0,0}}{A_1}_{f_{0,1}}) \otimes 
%   C^\bullet(A_0, \, _{f_{0,1}}{A_1}_{f_{0,2}}) \otimes \cdots \otimes 
%   C^\bullet(A_0, \, _{f_{0,k_0-1}}{A_1}_{f_{0,k_0}}) \otimes \\ \otimes
% & C^\bullet(A_1, \, _{f_{1,0}}{A_2}_{f_{1,1}}) \otimes 
%   C^\bullet(A_1, \, _{f_{1,1}}{A_2}_{f_{1,2}}) \otimes \cdots \otimes 
%   C^\bullet(A_1, \, _{f_{1,k_1-1}}{A_2}_{f_{1,k_1}}) \otimes \\ \otimes
% & \cdots \otimes \\ \otimes
% & C^\bullet(A_n, \, _{f_{n,0}}{A_0}_{f_{n,1}}) \otimes 
%   C^\bullet(A_n, \, _{f_{n,1}}{A_0}_{f_{n,2}}) \otimes \cdots \otimes 
%   C^\bullet(A_n, \, _{f_{n,k_n-1}}{A_0}_{f_{n,k_n}}) \otimes \\ \otimes
% & C_{-\bullet}(A_0, _{f_{n,k_n}\cdots f_{1, k_1} f_{0, k_0}}{A_0}_{id})
% \end{align*}
\begin{figure}
\centerline{\xymatrix{
A_0 \ar@/^5pc/[r]^{f_{0,0}} 
\ar@/^2pc/[r]^{\bm{\big\Downarrow \phi_{0,1}}}_{f_{0,1}} 
\ar@/_2pc/[r]^{\substack{\vdots\\ f_{0,k_0}}}
\ar@/_9pc/[rrrrr]_{id}^{\substack{\alpha \\ \\ \\ \\ }}
& A_1 \ar@/^5pc/[r]^{f_{1,0}} 
\ar@/^2pc/[r]^{\bm{\big\Downarrow \phi_{1,1}}}_{f_{1,1}} 
\ar@/_2pc/[r]^{\substack{\vdots\\ f_{1,k_1}}}
& A_2 \ar@{.>}@/^5pc/[r] 
\ar@{.>}@/^2pc/[r]_{\substack{\\ \\ \vdots}} 
\ar@{.>}@/_2pc/[r]
& \cdots \ar@{.>}@/^5pc/[r] 
\ar@{.>}@/^2pc/[r]_{\substack{\\ \\ \vdots}} 
\ar@{.>}@/_2pc/[r]
& A_n \ar@/^5pc/[r]^{f_{n,0}} 
\ar@/^2pc/[r]^{\bm{\big\Downarrow \phi_{n,1}}}_{f_{n,1}} 
\ar@/_2pc/[r]^{\substack{\vdots\\ f_{n,k_n}}}
& A_0 
}}
\caption{Picture of an element of $C(n)^\bullet(f)$ 
where $f = (f_{0,0}, f_{1,0}, \cdots, f_{n,0})$, 
$\phi_{i,j} \in C^\bullet(A_i, _{f_{j-1}}
{A_{i+1 \textrm{ (mod n+1)}}}_{f_j})$, and 
$\alpha \in C_{-\bullet}(A_0, _{f_n\cdots f_1 f_0}
{A_0}_{id})$}
 \label{fig:phi|alpha}
\end{figure}
%
\subsubsection{Comodule structure}\label{sec:comod_strre}
The comodule maps on $C(n)^\bullet(f)$ are given by the
cocomposition maps in $B(n)$: 
%
\begin{equation*}
\begin{CD}
C(n)^\bullet(f) @>\Delta_C>>
\bigoplus\limits_{h \in Obj(B(n))} B(n)^\bullet(f,h) \otimes C(n)^\bullet(h) \\
@|  @| \\
\bigoplus\limits_{g \in Obj(B(n))} B(n)^\bullet(f,g) \otimes C_{-\bullet}(A_0, _g{A_0}_{id})
@>\Delta_{B(n)}\otimes id_{C_{-\bullet}}>>
\bigoplus\limits_{g,h\in Obj(B(n))} \substack{B(n)^\bullet(f,h)\otimes B(n)^\bullet(h,g) \\ \otimes 
C_{-\bullet}(A_0, _g{A_0}_{id})}
\end{CD} 
\end{equation*}
%
Because $\Delta_{C(n)}$ is induced by 
$\Delta_{B(n)}$, we have that $\Delta_{C(n)}$ 
satisfies coassocitivity and counitality 
and is conilpotent. 

$C(n)$ cofree as a comodule in the sense that 
a morphism to $C(n)$ is determined by projections
to its Hochschild-chains component.
More precisely, there is a one-to-one correspondence
\begin{equation}\label{eq:quasicofree}
\begin{split}
\bigg\{ \let\scriptstyle\textstyle
\substack{\textrm{maps of comodules}\\
  D \to C(n)\textrm{ over } B(n)}
\bigg\}
&\overset{1:1}\longleftrightarrow
\bigg\{ \bigg( \let\scriptstyle\textstyle
\substack{\textrm{maps of graded vector spaces}\\
  D^\bullet(f) \to C_{-\bullet}(A_0, _f{A_0}_{id})}
\bigg)_{f \in Obj(B(n))} \bigg\}\\
\bigg( F:D\to C(n) \bigg)
&\mapsto
\bigg( \substack{\begin{split}
  D(f) 
  &\xrightarrow{F_f} C(n)(f) \\
  &\xrightarrow{project}
  C_{-\bullet}(A_0, _f{A_0}_{id})
  \end{split}} 
  \bigg)_f\\
\begin{pmatrix}
  D(f)
  &\xrightarrow{\Delta_D}
  \bigoplus \limits_{g \in Obj(B(n))}
  B(n)(f,g) \otimes D(g)\\
  &\xrightarrow{\oplus_g id_B \otimes F_g}
  \oplus_g
  B(n)(f,g) \otimes C_{-\bullet}(A_0, _g{A_0}_{id})\\
  &\cong C(n)(f)
  \end{pmatrix} _f
&\leftmapsto  
\bigg( D(f) \to
  C_{-\bullet}(A_0, _f{A_0}_{id})
  \bigg)_f\\
\end{split}
\end{equation}
%
\begin{defn}\label{def:cogenerators}
We will call elements of 
$T(A_0 \to \smdots \to A_n \to A_0)(f) :=
C_{-\bullet}(A_0, _f{A_0}_{id})$ 
the \textbf{cogenerators} of 
$C(A_0 \to \smdots A_n \to A_0)(f)$. 
More generally, we will refer to 
the set $T(A_0 \to \smdots \to A_n \to A_0) 
= \{T(A_0 \to \smdots \to A_n \to A_0)(f) | 
f \in Obj(B(A_0 \to \smdots \to A_n \to A_0))\}$ 
as the \textbf{cogenerators} of 
$C(A_0 \to \smdots \to A_n \to A_0)$. When we 
have fixed a sequence of algebras, 
$A_0, \smdots A_n$, we will 
use $T(n)$ to denote 
$T(A_0 \to \smdots \to A_n \to A_0)$.
\end{defn}
%

%
\subsubsection{Differential}
The differential $d_{C(n)}$ on $C(n)$ is: 
%
\begin{equation} 
\label{eq:C(n)_differential}
\begin{split}
d_{C(n)} &= 
\tilde{d}_{B(n)} + \tilde{b} + \mathcal{I}\\
%
\tilde{d}_{B(n)} \big(
\phi_{0,1}\smdots\phi_{0,k_0} | \smdots |
\phi_{n,1}\smdots\phi_{n,k_n} | \alpha \big) 
&=
d_{B(n)}( \phi_{0,1}\smdots | 
\smdots | \smdots\phi_{n,k_n} ) | \alpha\\
%
\tilde{b} \big(
\phi_{0,1}\smdots\phi_{0,k_0} | \smdots |
\phi_{n,1}\smdots\phi_{n,k_n} | \alpha \big) 
&=
(-1)^{\sum \limits_{i,j}|\phi_{i,j}|+1}
\phi_{0,1}\smdots | \smdots | \smdots 
\phi_{n,k_n} | [_{f_{n,k_n}\smdots f_{0,k_0}}
b](\alpha)
\end{split}
\end{equation}
%
where $d_{B(n)}$ is the differential on $B(n)$, 
and $\mathcal{I}$ is a term that captures an action of cochains on chains described by the equations below:
%
\begin{equation}\label{eq:action_term}
\begin{split}
\mathcal{I} 
&= 
(id_{B(n)} \otimes \eta_{C(n)})
  \circ \Delta_{C(n)}\\
\eta_{C(n)}(
\phi_{0,1}\smdots\phi_{0,k_0} | \dots |
\phi_{n,1}\smdots\phi_{n,k_n} | \alpha)
&= 
\iota_{C(0)}( \pi_{B(0)}( 
(\phi_{0,1}\smdots \phi_{0,k_0}) \bullet \dots \bullet
(\phi_{n,1}\smdots \phi_{n,k_n})), \alpha)\\
\bullet
&= 
\textrm{brace operation on cochains, 
  see Section \ref{sec:def_braces}}\\
\pi_{B(0)}: 
B(0)^\bullet
(f_{n,1}\smdots f_{1,1} f_{0,1}, 
f_{n,k_n}\smdots f_{1,k_1} f_{0,k_0})
&\xrightarrow[component]{project\;onto}
C^\bullet(A_0, _{f_{n,1}\smdots f_{0,1}}{A_0}_{f_{n,k_n}\smdots f_{0,k_0}})\\
\iota
&=
\textrm{contraction operation, 
see Section \ref{chap:hochschild}}
\end{split}
\end{equation}
%
Given Equation \ref{eq:C(n)_differential}, 
it's easy to check that we can promote 
Equation \ref{eq:quasicofree} to a dg statement:
%
\begin{equation*}
\bigg\{ \let\scriptstyle\textstyle
\substack{\textrm{maps of dg comodules}\\
  D \to C(n)\textrm{ over }B(n)} 
\bigg\}
\overset{1:1}{\longleftrightarrow}
\bigg\{ \bigg( \let\scriptstyle\textstyle
\substack{\textrm{maps of complexes}\\
  D^\bullet(f) \to C_{-\bullet}(A_0, _f{A_0}_{id})}
\bigg)_{f \in Obj(B(n))} \bigg\}.
\end{equation*}