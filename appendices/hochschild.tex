\chapter{Background on Hochschild chains and cochains}
\label{chap:hochschild}
In this section, we give some known 
constructions on Hochschild chains and 
cochains for the reader's convenience. 
%
\section{Standard constructions and notation}
Let $k$ be a field of characteristic zero, 
$A$ a flat unital $k$-algebra, and $M$ be an 
$A$-$A$-bimodule. Then, we can take 
$(C_\bullet(A,M), b)$, the 
(reduced or standard) Hochschild chain 
complex of $A$ 
with coefficients in $M$ (see Reference 
\cite{T}, Equation 2.1). When $M = B$ is also 
an algebra over $k$ with left and right 
module structure given by two maps of algebras 
$f:A \to B$ and $g:A \to B$, respectively, 
we may write $_fB_g$ to clarify 
the module structure.

Let $k, A, M$ be as above. We can also 
take $(C^\bullet(A,M), \delta)$, the 
(reduced) Hochschild cochain complex of $A$ 
with coefficients in $M$ (see Reference 
\cite{T}, Equations 2.12-13, 2.19-21). When $M=B$ is 
an algebra, $(C^\bullet(A,B), \delta, \cup)$ 
is a dga where the cup product $\cup$ is 
given in Reference \cite{T}, Equation 2.14.

Let $f,g,h:A \to A$ be maps of algebras. 
We have a contraction operation of Hochschild 
cochains and chains, which is a map of complexes:
\begin{align*}
\iota: C^p(A, _fA_g) \bigotimes C_{-q}(A, _gA_h) 
&\longrightarrow 
C_{-(q-p)}(A, _fA_h) \\
\phi \bigotimes a_0\otimes \dots \otimes a_q 
&\mapsto 
\iota(\phi, a_0\otimes \dots \otimes a_q)
:= \phi \cdot (a_0\otimes \dots \otimes a_q) :=\\
&:=
(-1)^{p(q+1)}
\phi(a_{q-p+1},\dots,a_q)\cdot a_0 
\otimes a_1 \otimes \dots \otimes a_{q-p}.
\end{align*}

% \tilde{\delta} 
% &:= 
% \substack{\textrm{extension of the Hochschild 
% cochain differential}\\\textrm{to a differential on 
% a bar complex}}\\
% b^\prime 
% &:= 
% \sum_{i=0}^{r-1} (-1)^{i}b_i, 
%   \textrm{ for appropriate r}\\
% b_i 
% &:= \textrm{cup product on Hochschild cochains 
% between the $i^{th}$ and $i+1^{th}$ terms}\\

%\lambda()
% , $\iota$ is 
% defined in Equation \ref{eq:action_term}, and 
% the braces $\cdot \{ \cdot \}$ are defined in ....



\section{Brace operation on Hochschild cochains}
\label{sec:def_braces}
Fix algebras $A_0, A_1$ and maps of algebras 
$f_0,f_n: A_0 \rightrightarrows A_1$, 
$A_0 \leftleftarrows A_1: g_0, g_m$. We will 
define a map of complexes 
\begin{align*}
&\phantom{{}={}}
C(A_0 \to A_1 \to A_0)((f_0,g_0), (f_n,g_m))\\
&:=
\Big( \bigoplus \limits_{\substack{
  i \in \mathbb{N}\\
  f_1,\smdots,f_i \textrm{ maps of algebras}\\
  f_{i+1} = f_n}}
  C^\bullet(A_0, _{f_0}{A_1}_{f_1}) \otimes \smdots \otimes 
  C^\bullet(A_0, _{f_i}{A_1}_{f_{i+1}}) \Big) \otimes \\
&\phantom{{}={}}  
\Big( \bigoplus \limits_{\substack{
	j \in \mathbb{N}\\
	g_1,\smdots,g_j \textrm{ maps of algebras}\\
	g_{j+1} = g_m}}
  C^\bullet(A_1, _{g_0}{A_0}_{g_1}) \otimes \smdots \otimes 
  C^\bullet(A_1, _{g_j}{A_0}_{g_{j+1}}) \Big)\\
&\phantom{{}moveovermoveovermoveover{}}
  \Big\downarrow -\bullet -\\
&\phantom{{}={}}
C(A_0 \to A_0)(g_0f_0, g_mf_n)\\
&:=
\bigoplus \limits_{\substack{
  i \in \mathbb{N}\\
  h_1,\smdots,h_i \textrm{ maps of algebras}\\
  h_{i+1} = g_mf_n}}
  C^\bullet(A_0, _{g_0f_0}{A_0}_{h_1}) \otimes \smdots \otimes 
  C^\bullet(A_0, _{h_i}{A_0}_{h_{i+1}}).
\end{align*}
First, for 
$$
(\phi_1\smdots\phi_n | 1) = 
\xymatrix{
A_0 
\ar@/^5pc/[r]^{f_0} 
\ar@/^2pc/[r]^{\big\Downarrow \phi_1}_{f_1} 
\ar@/_2pc/[r]^{\substack{\vdots\\ f_n}}
& A_0 
\ar@/^2pc/[r]^{}^{id}_{\big\Downarrow 1}
\ar@/_2pc/[r]_{id}
& A_0
}
\quad \textrm{and} \quad
(1|\phi_1\smdots\phi_n) = 
\xymatrix{
A_0
\ar@/^2pc/[r]^{}^{id}_{\big\Downarrow 1}
\ar@/_2pc/[r]_{id}
& A_0 
\ar@/^5pc/[r]^{f_0} 
\ar@/^2pc/[r]^{\big\Downarrow \phi_1}_{f_1} 
\ar@/_2pc/[r]^{\substack{\vdots\\ f_n}}
& A_0,
}$$
define $(\phi_1\smdots\phi_n | 1) 
\overset{\bullet}{\mapsto} 
(\phi_1\smdots\phi_n) \bullet 1 = 
(\phi_1\smdots\phi_n)$ and 
$(1| \phi_1\smdots\phi_n) 
\overset{\bullet}{\mapsto} 
1 \bullet (\phi_1\smdots\phi_n) = 
(\phi_1\smdots\phi_n)$.
Then, for $n,m\geq 1$, let 
$$
(\phi_1\smdots\phi_n | \psi_1\smdots\psi_m)
=
\xymatrix{
A_0 \ar@/^5pc/[r]^{f_0} 
\ar@/^2pc/[r]^{\big\Downarrow \phi_1}_{f_1} 
\ar@/_2pc/[r]^{\substack{\vdots\\ f_n}}
& A_1 \ar@/^5pc/[r]^{g_0} 
\ar@/^2pc/[r]^{\big\Downarrow \psi_1}_{g_1} 
\ar@/_2pc/[r]^{\substack{\vdots\\ g_m}}
& A_0
}
$$
and define $(\phi_1\smdots\phi_n) \bullet
(\psi_1\smdots\psi_m) \in C(A_0 \to A_0)
(g_0f_0, g_mf_n)$ as follows:
\begin{align*}
&\phantom{{}={}}
(\phi_1\smdots\phi_n) \bullet
(\psi_1\smdots\psi_m)\\
&= 
\sum \limits_{
  0 \leq i_1 \leq \smdots
  \leq i_{2m} \leq n
}
  g_0\phi_1 \otimes \smdots \otimes g_0\phi_{i_1} \otimes 
  \psi_1\{\phi_{i_1+1}\smdots \phi_{i_2}\} \otimes\\
&\phantom{{}move{}}  
  \otimes g_1\phi_{i_2+1} \otimes \smdots \otimes g_1\phi_{i_3} 
  \otimes \psi_2\{\phi_{i_3+1}\smdots \phi_{i_4}\} \otimes \\
&\phantom{{}move{}}    
  \otimes \dots \otimes
  \psi_m\{\phi_{i_{2m-1}+1}\smdots \phi_{i_{2m}}\} \otimes \\
&\phantom{{}move{}}    
  \otimes g_m\phi_{i_{2m}+1} \otimes \smdots \otimes g_m\phi_n \\
&= \sum  
\xymatrix{
A_0 
\ar@/^14pc/[rrrrrrrrrr]^{g_0f_0}_{\big\Downarrow g_0\phi_1}
\ar@{.>}@/^12pc/[rrrrrrrrrr]_{\substack{
  g_0f_1\\ \vdots\\\\g_0f_{i_1-1}}} 
\ar@{.>}@/^8pc/[rrrrrrrrrr]
\ar@{.>}@/^6pc/[rrrrrrrrrr]^{\big\Downarrow g_0\phi_{i_1}}_{g_0f_{i_1}}
\ar@{.>}@/^3pc/[rrrrrrrrrr]^{\big\Downarrow 
  \psi_1\{\phi_{i_1+1}\smdots \phi_{i_2}\}}_{\substack{
  g_1f_{i_2}\\ \vdots \\\\ g_{m-1}f_{i_{2m-1}}}}
\ar@{.>}@/_1pc/[rrrrrrrrrr]
\ar@{.>}@/_3pc/[rrrrrrrrrr]^{\big\Downarrow 
 \psi_m\{\phi_{i_{2m-1}+1}\smdots \phi_{i_{2m}}\}}_{g_m
 f_{i_{2m}}}
\ar@{.>}@/_6pc/[rrrrrrrrrr]^{\big\Downarrow g_m
  \phi_{i_{2m}+1}}_{\substack{
  g_mf_{i_{2m}+1}\\ \vdots\\\\g_mf_{n-1} }} 
\ar@{.>}@/_10pc/[rrrrrrrrrr]_{\big\Downarrow g_m\phi_n}
\ar@/_12pc/[rrrrrrrrrr]_{g_mf_n}
&&&&&&&&&& A_0  
}
\end{align*}
where $\psi_j\{\phi_{i_r+1}\smdots \phi_{i_r+s}\}
\in C^\bullet(A_0, _{g_{j-1}f_{i_r}} {A_0}_{g_jf_{i_r+s}})$ 
is the following cochain: 
\begin{align*}
&\phantom{{}={}}
\psi_j\{\phi_{i_r+1}\smdots \phi_{i_r+s}\}
  (b_0 \otimes a_1 \otimes \smdots \otimes a_p)\\
&= 
\sum \limits_{
  0 \leq k_1 \leq \smdots \leq k_{2s} \leq p}
\psi_j \Bigg( b_0, f_{i_r}a_1, \smdots, f_{i_r}a_{k_1},
  \phi_{i_r+1}(a_{k_1+1}, \smdots, a_{k_2}),\\
&\phantom{{}moveovermoveover{}}  
  f_{i_r+1}a_{k_2+1}, \smdots, f_{i_r+1}a_{k_3},
  \phi_{i_r+2}(a_{k_3+1}, \smdots, a_{k_4}), \\
&\phantom{{}moveovermoveover{}}   
  ,\smdots, 
  \phi_{i_r+s}(a_{k_{2s-1}}, \smdots, a_{k_{2s}}), \\
&\phantom{{}moveovermoveover{}} 
  f_{i_r+s}a_{k_{2s}+1}, \smdots, f_{i_r+s}a_p \Bigg)
\end{align*}
for $b_0 \in A_1$, $a_1,\smdots,a_p \in A_0$, 
$\psi_j \in C^\bullet(A_1, _{g_{j-1}}{A_0}_{g_j})$, 
$\phi_i \in C^\bullet(A_), _{f_{i-1}}{A_1}_{f_i})$.
% -reduced hochschild chains
% -reference for braces and signs
% -lambda(\vec\phi) notation


% -computational props:
% - dot notation = iota notation
% - lambda(\phi) notation