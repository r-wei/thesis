% \begin{figure} \label{fig:upsilon}
% \centerline{\xymatrix{
% A_0 \ar@/^5pc/[r]^{f_0} 
% \ar@/^2pc/[r]^{\big\Downarrow \phi_1}_{f_1} 
% \ar@/_2pc/[r]^{\substack{\vdots\\ f_n}}
% \ar@/_4pc/[rr]_{id}^{\substack{\alpha \\ \\ \\ \\ }}
% & A_1 \ar@/^5pc/[r]^{g_0} 
% \ar@/^2pc/[r]^{\big\Downarrow \psi_1}_{g_1} 
% \ar@/_2pc/[r]^{\substack{\vdots\\ g_m}}
% & A_0
% }
% $\overset{\Upsilon}\longrightarrow$
% \xymatrix{
% A_1 \ar@/^5pc/[r]^{g_0} 
% \ar@/^2pc/[r]^{\big\Downarrow \psi_1}_{g_1} 
% \ar@/_2pc/[r]^{\substack{\vdots\\ g_m}}
% \ar@/_4pc/[rr]_{id}^{\substack{\alpha \\ \\ \\ \\ }}
% & A_0 \ar@/^5pc/[r]^{f_0} 
% \ar@/^2pc/[r]^{\big\Downarrow \phi_1}_{f_1} 
% \ar@/_2pc/[r]^{\substack{\vdots\\ f_n}}
% & A_1
% }}
% \caption{A picture of the domain and target of $\Upsilon$}
% \end{figure}

\begin{prop}
\label{prop:c1}
Let $\hat{\tau}_1: 
B(A_0 \to A_1 \to A_0) 
\longrightarrow B(A_1 \to A_0 \to A_1)$ 
be as defined in Section 
\ref{sec:cyclic_B(n)}.
Recall from Example \ref{eg:pb5} that 
$\hat{\tau}_1^*C(A_1 \to A_0 \to A_0)
\cong C(A_1 \to A_0 \to A_1)$ 
as complexes. Define a map 
$$
\Upsilon_{A_0,A_1}: C(A_0 \to A_1 \to A_0)
\to \hat{\tau}_1^*C(A_1 \to A_0 \to A_1)
$$
of comodules over 
$B(A_0 \to A_1 \to A_0)$ as follows:
\begin{align*}
\upsilon^{f_0, g_0}: C(A_0 \to A_1 \to A_0)(g_0f_0) 
&\to
\hat{\tau}_1^*C(A_1 \to A_0 \to A_1)(f_0g_0)
\cong 
C(A_1 \to A_0 \to A_1)(f_0g_0)\\
&\xrightarrow[cogenerators]{\textrm{project onto}}
C_{-\bullet}(A_1, _{f_0g_0}{A_1}_{id})\\
\upsilon_{n,m}^{f_0,g_0} 
(\vec{\phi} | \vec{\psi} | \alpha) = 
& \sum_{\substack{I_1I_2 = \{1,2,\cdots,n\} \\
                          \textrm{as ordered sets}}}
  \phi_1(\lambda(\vec{\psi})\lambda(\vec{\phi_{I_2}})\cdot \mathfrak{a}_3, a_0, \mathfrak{a}_1) \otimes \lambda(\vec{\phi_{I_1}}) \cdot \mathfrak{a}_2 \\
&\phantom{{}move{}}
\bigg( + f_0a_0 \otimes \lambda(\vec{\phi}) \mathfrak{a}_1 
  \; \; \; \; if \; \; m = 0 \bigg).
\end{align*}
Then, $\Upsilon_{A_0,A_1}: C(A_0 \to A_1 \to A_0)
\to \hat{\tau}^*C(A_1 \to A_0 \to A_1)$ 
is a map of dg comodules over 
$B(A_0 \to A_1 \to A_0)$.
\end{prop}
%
\begin{proof}
We must show: (1) $\Upsilon$ is a map of comodules, and 
(2) $\Upsilon$ commutes with the differentials. (In this 
proof, we drop the subscripts and write 
$\Upsilon := \Upsilon_{A_0, A_1}$.)

(1) This proof is standard for cofree comodules. 
Let ($\vec{\phi} | \vec{\psi} | \alpha$) be as 
in the statement of the proposition. We want to 
show that $\Upsilon$ commutes with the coproducts. 
On one hand,
\begin{align*}
&\phantom{{}={}}
[(id_B \otimes \Upsilon) \circ 
  \Delta_{C(A_0 \to A_1 \to A_0)}] 
  ( \vec{\phi} | \vec{\psi} | \alpha ) \\
&= [id_B \otimes \Upsilon]
	\big( \sum_{\substack{I_1I_2 = \{1,2,\cdots,n\} \, \textrm{and} \\ 
						  J_1J_2 = \{1,2,\cdots,m\} \\
				          \textrm{as ordered sets}}} 
    (\vec{\phi}_{I_1} | \vec{\psi}_{J_1}) \otimes (\vec{\phi}_{I_2} | \vec{\psi}_{J_2} | \alpha) \, \big) \\
&= \sum_{\substack{I_1I_2I_3 = \{1,2,\cdots,n\} \, \textrm{and} \\ 
				   J_1J_2J_3 = \{1,2,\cdots,m\} \\
				   \textrm{as ordered sets}}} 
    (\vec{\phi}_{I_1} | \vec{\psi}_{J_1}) \otimes 
    (\vec{\phi}_{I_2} | \vec{\psi}_{J_2}) \otimes 
    \upsilon_{|I_3|,|J_3|}(\vec{\phi}_{I_3} | \vec{\psi}_{J_3} | \alpha) \\
\end{align*}

On the other hand,
\begin{align*}
&\phantom{{}={}}
[\Delta_{\hat{\tau}^*C(A_1 \to A_0 \to A_1)} 
  \circ \Upsilon ]
  ( \vec{\phi} | \vec{\psi} | \alpha ) \\
&= \Delta_{\hat{\tau}^*C(A_1 \to A_0 \to A_1)}
	\big( \sum_{\substack{I_1I_2 = \{1,2,\cdots,n\} \, \textrm{and} \\ 
						  J_1J_2 = \{1,2,\cdots,m\} \\
				          \textrm{as ordered sets}}}
	(\vec{\phi}_{I_1} | \vec{\psi}_{J_1}) \otimes 
    \upsilon_{|I_2|,|J_2|}(\vec{\phi}_{I_2} | \vec{\psi}_{J_2} | \alpha) \, \big)\\
&= \sum_{\substack{I_1I_2I_3 = \{1,2,\cdots,n\} \, \textrm{and} \\ 
				   J_1J_2J_3 = \{1,2,\cdots,m\} \\
				   \textrm{as ordered sets}}} 
    (\vec{\phi}_{I_1} | \vec{\psi}_{J_1}) \otimes 
    (\vec{\phi}_{I_2} | \vec{\psi}_{J_2}) \otimes 
    \upsilon_{|I_3|,|J_3|}(\vec{\phi}_{I_3} | \vec{\psi}_{J_3} | \alpha).   				          
\end{align*}
Clearly $(id_B \otimes \Upsilon) \circ 
\Delta_{C(A_0 \to A_1 \to A_0)} = 
\Delta_{\hat{\tau}^*C(A_1 \to A_0 \to A_1)} 
\circ \Upsilon$.

(2) We will show that $\Upsilon$ commutes with 
the differentials by direct computation. Since 
$\Upsilon$ is a map of cofree comodules, we only 
need to check that $\pi_1 \circ D(\Upsilon) = 0$ 
where $D(\Upsilon)$ is the differential applied 
to $\Upsilon$ as a linear map between complexes 
and $\pi_1$ denotes projection of a comodule 
onto its cogenerators. More explicitly, we want 
to check that
\begin{equation} \label{eq:upsilon}
\begin{aligned}
&\upsilon_{n, m} ( \tilde{\delta}(\vec{\phi}) | \vec{\psi} | \alpha ) \; + 
\upsilon_{n, m} ( \vec{\phi} | \tilde{\delta}(\vec{\psi}) | \alpha ) \; + 
\upsilon_{n-1, m} ( b^\prime(\vec{\phi}) | \vec{\psi} | \alpha ) \; + 
\upsilon_{n, m-1} ( \vec{\phi} | b^\prime(\vec{\psi}) | \alpha ) \; + \\
&\upsilon_{n, m} ( \vec{\phi} | \vec{\psi} | b(\alpha) ) \; + 
b \circ \upsilon_{n, m} ( \vec{\phi} | \vec{\psi} | \alpha ) \; + \\
&\upsilon_{|I_1|, m-1}(\vec{\phi}_{I_1} | \vec{\psi}_{\{1,\cdots, m-1\}} | \psi_{m} \{\vec{\phi}_{I_2}\} \cdot \alpha ) \; + 
\upsilon_{n-1, m}(\vec{\phi}_{\{1,\cdots, n-1\}} |\vec{\psi}_{m} | \phi_{n}\cdot \alpha) \; + \\
&\phi_1 \{ \psi_{J_1}\}\cdot \upsilon_{\vec{\phi}|-1, |J_2|}(\phi_{\{2,\cdots , n\}} | \psi_{J_2} | \alpha ) \; + 
\psi_1\cdot \upsilon_{n,m-1} ( \vec{\phi} | \vec{\psi}_{\{2,\cdots, |\vec{\psi}\}} | \alpha) \\ 
&= 0.
\end{aligned}
\end{equation}
In Equation \ref{eq:upsilon}, we will call the 
terms in the first two rows the ``standard terms'', 
and the terms in the second two rows the 
``extra terms''.

We compute the sum of the standard terms. In the chart below, the leftmost column lists the expressions that don't cancel in the sum of the standard terms, the middle column gives the standard term from which the expression comes, and the rightmost column gives the term (extra or standard) that cancels the expression. 
\newpage
\begin{landscape}
\begin{center}
  \begin{tabular}{ p{3.25in} | p{2in} | p{2.5in} }
    \hline
    Expression & Comes from Standard Term & Cancelling Term \\ \hline
    
    \breakcell{$f_0\psi_1(\lambda(\vec{\phi}_{I_2}) \mathfrak{a}_3 ) \phi_1(\lambda(\vec{\psi}_{\{2,\cdots, m\}} \lambda(\vec{\phi}_{I_3}) \mathfrak{a}_4, a_0, \mathfrak{a}_1)$ \\ $\otimes \lambda(\vec{\phi}_{I_1}) \mathfrak{a}_2$} &
    $\upsilon_{n, m} (\delta(\phi_1)\phi_2 \cdots \phi_n | \vec{\psi} | \alpha)$ & 
    $f_0 \psi_1 \cdot \upsilon_{n,m-1} ( \vec{\phi} | \vec{\psi}_{\{2,\cdots, m\}} | \alpha)$ \\ \hline

    \breakcell{$\phi_1( \lambda(\vec{\psi}_{\{1,\cdots, m-1\}}) \lambda(\vec{\phi}_{I_2}) \mathfrak{a}_3, \psi_{m} ( \lambda(\vec{\phi}_{I_3}) \mathfrak{a}_4) \cdot a_0, \mathfrak{a}_1 )$ \\ $\otimes \lambda(\vec{\phi}_{I_1}) \mathfrak{a}_2$} &
    $\upsilon_{n, m} (\delta(\phi_1)\phi_2 \cdots \phi_n | \vec{\psi} | \alpha)$ &
    $\upsilon_{|I_1|, m-1}(\vec{\phi}_{I_1} | \vec{\psi}_{\{1,\cdots, m-1\}} | \psi_{m} \{\vec{\phi}_{I_2}\}\cdot \alpha )$ \\ \hline

    \breakcell{$\phi_1( \lambda(\vec{\psi}) \lambda(\vec{\phi}_{I_2}) \mathfrak{a}_3, g_m \phi_n(\mathfrak{a}_4) \cdot a_0, \mathfrak{a}_1) \otimes \lambda(\vec{\phi}_{I_1}) \mathfrak{a}_2$} &
    $\upsilon_{n, m} (\delta(\phi_1)\phi_2 \cdots \phi_n | \vec{\psi} | \alpha)$ &
    $\upsilon_{n-1, m}(\vec{\phi}_{\{1, \cdots, n-1\}} | \vec{\psi} | g_m \phi_{n} \cdot \alpha )$ \\ \hline

    \breakcell{$\phi_1( \lambda(\vec{\psi}) \lambda(\vec{\phi}_{I_2}) \mathfrak{a}_2) \cdot f_1(a_0) \otimes \lambda(\vec{\phi}_{I_1}) \mathfrak{a}_1$} &
    $\upsilon_{n, m} (\delta(\phi_1)\phi_2 \cdots \phi_n | \vec{\psi} | \alpha)$ &
    $\phi_1 \cdot \upsilon_{n-1, 0}(\vec{\phi}_{\{2, \cdots, n\}} | \vec{\psi} |\alpha )$ \\ \hline

    \breakcell{$f_0a_0 \cdot \phi_1(\mathfrak{a}_1) \otimes \lambda(\vec{\phi}_{\{1,\cdots,n-1\}}) \mathfrak{a}_2$} &
    \breakcell{$\upsilon_{n, m} (\delta(\phi_1)\phi_2 \cdots \phi_n | \vec{\psi} | \alpha)$ \\ if $\vec{\psi} = \emptyset$} & 
    \breakcell{$b \circ \upsilon_{n, m} (\vec{\phi} | \vec{\psi} | \alpha)$ \\ if $\vec{\psi} = \emptyset$} \\ \hline

    \breakcell{$f_0 g_m \phi_n(\mathfrak{a}_2) f_0a_0 \otimes \lambda(\vec{\phi}_{\{1,\cdots,n-1\}}) \mathfrak{a}_1$} &
    \breakcell{$b \circ \upsilon_{n, m} (\vec{\phi} | \vec{\psi} | \alpha)$ \\ if $\vec{\psi} = \emptyset$} &
    \breakcell{$\upsilon_{n-1, m}(\vec{\phi}_{\{1, \cdots, n-1\}} | \vec{\psi} | g_m \phi_{n} \cdot \alpha )$ \\ if $\vec{\psi} = \emptyset$} \\ \hline

    \breakcell{$\phi_1(\lambda(\vec{\psi}) \lambda(\vec{\phi}_{I_2}) \mathfrak{a}_4, a_0, \mathfrak{a}_1) \cdot \phi_2(\mathfrak{a}_2) \otimes \lambda(\vec{\phi}_{I_1}) \mathfrak{a}_3$} &
    $b \circ \upsilon_{n, m} (\vec{\phi} | \vec{\psi} | \alpha)$ &
    $\upsilon_{n-1, m}(\phi_1 \cup \phi_2 \phi_3 \cdots \phi_n | \vec{\psi} | \alpha)$ \\ \hline
    
    \breakcell{$\phi_1(\lambda(\vec{\psi}_{J_1}) \lambda(\vec{\phi}_{I_2}) \mathfrak{a}_3) \phi_2(\lambda(\vec{\psi}_{J_2} \lambda(\vec{\phi}_{I_3}) \mathfrak{a}_3, a_0, \mathfrak{a}_1)$ \\ $\otimes \lambda(\vec{\phi}_{I_1}) \mathfrak{a}_2$} &
    $\upsilon_{n-1, m}(\phi_1 \cup \phi_2 \phi_3 \cdots \phi_n | \vec{\psi} | \alpha)$ &
     $\phi_1 \{ \vec{\psi}_{J_1} \} \cdot \upsilon_{n-1, |J_2|}(\vec{\phi}_{\{2, \cdots, n\}} | \vec{\psi}_{J_2} |\alpha )$\\ \hline

    \breakcell{$f_0 \psi_1(\lambda(\vec{\phi}_{I_2}) \mathfrak{a}_2) \cdot f_0a_0 \otimes \lambda(\vec{\phi}_{I_1}) \mathfrak{a}_1$} &  
    \breakcell{$f_0 \psi_1 \cdot \upsilon_{n, 0}(\vec{\phi} | 1 | \alpha)$ \\ if $\vec{\psi} = \psi_1$} &
    \breakcell{$ \upsilon_{|I_1|, 0} (\vec{\phi}_{I_1} | 1 | \psi_1 \{ \vec{\phi}_{I_2} \} \cdot \alpha )$ \\ if $\vec{\psi} = \psi_1$} \\ \hline
    \hline
  \end{tabular}
\end{center}
(Technically, the last term in the middle column is not a standard term, but we include it in the table for convenience.)
\end{landscape}
All of the terms in the table describing the expansion of equation \ref{eq:upsilon} cancel, so $\Upsilon$ is a map of complexes.
\end{proof}