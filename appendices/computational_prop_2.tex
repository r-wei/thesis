% \begin{figure} \label{fig:upsilon}
% \centerline{\xymatrix{
% A_0 \ar@/^5pc/[r]^{f_0} 
% \ar@/^2pc/[r]^{\big\Downarrow \phi_1}_{f_1} 
% \ar@/_2pc/[r]^{\substack{\vdots\\ f_n}}
% \ar@/_4pc/[rr]_{id}^{\substack{\alpha \\ \\ \\ \\ }}
% & A_1 \ar@/^5pc/[r]^{g_0} 
% \ar@/^2pc/[r]^{\big\Downarrow \psi_1}_{g_1} 
% \ar@/_2pc/[r]^{\substack{\vdots\\ g_m}}
% & A_0
% }
% $\overset{B}\longrightarrow$
% \xymatrix{
% A_0 \ar@/^5pc/[r]^{f_0} 
% \ar@/^2pc/[r]^{\big\Downarrow \phi_1}_{f_1} 
% \ar@/_2pc/[r]^{\substack{\vdots\\ f_n}}
% \ar@/_4pc/[rr]_{id}^{\substack{\alpha \\ \\ \\ \\ }}
% & A_1 \ar@/^5pc/[r]^{g_0} 
% \ar@/^2pc/[r]^{\big\Downarrow \psi_1}_{g_1} 
% \ar@/_2pc/[r]^{\substack{\vdots\\ g_m}}
% & A_0
% }}
% \caption{A picture of the domain and target of $B$}
% \end{figure}
%
\begin{prop}
\label{prop:c2}
Let 
$B_{A_0,A_1} = B:C(A_0 \to A_1 \to A_0)
\longrightarrow C(A_0 \to A_1 \to A_0)$ 
be the map of cofree comodules defined by 
the following maps to cogenerators:
\begin{equation}
\label{eq:def_sigma}
B_{n, m} (\vec{\phi} | \vec{\psi} | \alpha) 
= (-1)^{|\mathfrak{a}_1|(|\mathfrak{a}_1|+|\mathfrak{a}_2|)}
1 \otimes \lambda(\psi)\lambda(\phi) \mathfrak{a}_2 \otimes a_0 \otimes \mathfrak{a}_1.
\end{equation}
Then, $D(B_{A_0,A_1}) = \Upsilon_{A_1,A_0}\Upsilon_{A_0,A_1} - id$ where
$\Upsilon$ is defined in Proposition 
\ref{prop:c1}.
\end{prop}
%
\begin{proof}
We prove the statement by direct computation. 
Since all of the maps are maps of cofree comodules, 
we only need to check that $\pi_1(D(B_{A_0,A_1}) - 
\Upsilon_{A_1,A_0}\Upsilon_{A_0,A_1} - id) = 0$ 
where $\pi_1$ denotes projection of the comodule 
onto cogenerators. More explicitly, for an element 
$(\vec{\phi}|\vec{\psi}|\alpha)$, we want to check that
\begin{equation} \label{eq:upsilon_homotopy}
\begin{aligned}
&B_{n, m} ( \tilde{\delta}(\vec{\phi}) | \vec{\psi} | \alpha ) \; + 
B_{n, m} ( \vec{\phi} | \tilde{\delta}(\vec{\psi}) | \alpha ) \; + 
B_{n-1, m} ( b^\prime(\vec{\phi}) | \vec{\psi} | \alpha ) \; + 
B_{n, m-1} ( \vec{\phi} | b^\prime(\vec{\psi}) | \alpha ) \; + \\
&B_{n, m} ( \vec{\phi} | \vec{\psi} | b(\alpha) ) \; + 
b \circ B_{n, m} ( \vec{\phi} | \vec{\psi} | \alpha ) \; + \\
%
& (-1)^{(\sum \limits_{1 \leq q \leq m}|\psi_q|+1)(|\phi_n|+1)}
B_{n-1, m}(\vec{\phi}_{\{1,\cdots, n-1\}} |\vec{\psi}_{m} | \phi_{n} \cdot \alpha) \; + \\
%
& (-1)^{(|\psi_1|+1)(\sum \limits_{1 \leq p \leq n}|\phi_p|+1)}
\psi_1 \cdot B_{n,m-1} ( \vec{\phi} | \vec{\psi}_{\{2,\cdots, m\}} | \alpha)  + \\ 
%
& \sum \limits_{\substack{
    I_1I_2 = \{1,\smdots,n\}\\
    \textrm{as ordered setts}}}
(-1)^{(\sum \limits_{1 \leq q \leq m-1}|\psi_q|+1)(\sum \limits_{p \in I_2}|\phi_p|+1)}
B_{|I_1|, m-1}(\vec{\phi}_{I_1} | \vec{\psi}_{\{1,\cdots, m-1\}} | \psi_{m} \{\vec{\phi}_{I_2}\}\cdot \alpha ) \; + \\
%
& \sum \limits_{\substack{
    J_1J_2 = \{1,\smdots,m\}\\
    \textrm{as ordered setts}}}
(-1)^{(\sum \limits_{q \in J_1}|\psi_q|+1)(\sum \limits_{2 \leq p \leq n}|\phi_p|+1)}
\phi_1 \{ \psi_{J_1}\} \cdot B_{n-1, |J_2|}(\phi_{\{2,\cdots , n\}} | \psi_{J_2} | \alpha ) \; - \\
%
& \sum \limits_{\substack{
    I_1I_2 = \{1,\smdots,n\}\\
    J_1J_2 = \{1,\smdots,m\}\\
    \textrm{as ordered setts}}}
(-1)^{(\sum \limits_{q \in J_1}|\psi_q|+1)(\sum \limits_{p \in I_2}|\phi_p|+1)}
\upsilon_{|J_1|, |I_1|} (\vec{\psi}_{J_1} | \vec{\phi}_{I_1} | \upsilon_{|I_2|, |J_2|} (\vec{\phi}_{I_2} | \vec{\psi}_{J_2} | \alpha ))  - \pi_1(\vec{\phi} | \vec{\psi} | \alpha) \\
&= 0.
\end{aligned}
\end{equation}

We will call the terms in rows 1-2 the ``standard terms'' in the computation of $D(B_{A_0,A_1})$, and the terms in rows 3-6 the ``extra terms'' in the computation of $D(B_{A_0,A_1})$. The seventh row is $\pi_1(\Upsilon_{A_1,A_0}\Upsilon_{A_0,A_1} - id)$. 

We compute the sum of the standard terms. 
In Table \ref{table:t21}, the leftmost column 
lists the expressions that don't cancel in the 
sum of the standard terms, the middle column 
gives the standard term from which the expression 
comes, and the rightmost column gives the extra 
term that cancels the expression. 
Table \ref{table:t22} lists the remaining terms 
from the seventh row that are not already listed in 
Table \ref{table:t21}. In Table \ref{table:t22}, 
the left column lists the remaining expressions 
that don't cancel in the seventh row, and the 
right column gives the extra term that cancels 
the expression.

All of the terms in the tables describing the 
expansion of equation \ref{eq:upsilon_homotopy} 
cancel, so $D(B_{A_0,A_1}) = \Upsilon_{A_1,A_0}
\Upsilon_{A_0,A_1} - id$.
\end{proof}
%
\begin{landscape}
\begin{center}
\begin{table}
  \begin{tabular}{ p{3.25in} | p{2in} | p{2.5in} }
    \hline
    Expression & Comes from Standard Term & Cancels with Extra Term \\ \hline

    $\psi_1(\lambda(\vec{\phi}_{I_1}) \mathfrak{a}_2) \otimes \lambda(\vec{\psi}_{\{2,\cdots,m\}}) \lambda(\vec{\phi}_{I_2}) \mathfrak{a}_3 \otimes a_0 \otimes \mathfrak{a}_1$ &
    $b \circ B_{n,m} (\vec{\phi} | \vec{\psi} | \alpha)$ & 
    $\psi_1 \{ \vec{\phi}_{I_1} \} \cdot B_{|I_2|, m-1} (\vec{\phi}_{I_2} | \vec{\psi}_{\{2, \cdots, m \}} | \alpha)$ \\ \hline

    $g_0\phi_1( \mathfrak{a}_2 ) \otimes \lambda(\vec{\psi}) \lambda(\vec{\phi}_{\{2, \cdots, n\}}) \mathfrak{a}_3 \otimes a_0 \otimes \mathfrak{a}_1$ &
    $b \circ B_{n,m} (\vec{\phi} | \vec{\psi} | \alpha)$ & 
    $\phi_1 \cdot B_{n-1, m} (\vec{\phi}_{\{2, \cdots, n\}} | \vec{\psi} | \alpha)$ \\ \hline

    $1 \otimes \lambda(\vec{\psi}) \lambda(\vec{\phi}_{\{1, \cdots, n-1\}}) \mathfrak{a}_2 \otimes g_m \phi_n(\mathfrak{a}_3 \cdot a_0 \otimes \mathfrak{a}_1$ &
    $b \circ B_{n,m} (\vec{\phi} | \vec{\psi} | \alpha)$ & 
    $B_{n-1, m} (\vec{\phi}_{\{1, \cdots, n-1 \}} | \vec{\psi} | \phi_n \cdot \alpha)$ \\ \hline

    $1 \otimes \lambda(\vec{\psi}_{\{1, \cdots, m-1 \}}) \lambda(\vec{\phi}_{I_1}) \mathfrak{a}_2 \otimes g_m \psi_m( \lambda(\vec{\phi}_{I_2} \mathfrak{a}_3) \cdot a_0 \otimes \mathfrak{a}_1$ &
    $b \circ B_{n,m} (\vec{\phi} | \vec{\psi} | \alpha)$ & 
    $B_{|I_1|, m-1} (\vec{\phi}_{I_2} | \vec{\psi}_{\{1, \cdots, m-1\}} | \psi_m \{ \vec{\phi}_{I_2} \} \cdot \alpha)$ \\ \hline

    $g_0f_0a_0 \otimes \lambda(\vec{\psi}) \lambda(\vec{\phi}) \mathfrak{a}_1$ &
    $b \circ B_{n,m} (\vec{\phi} | \vec{\psi} | \alpha)$ & 
    $\upsilon_{|J_1|, |I_1|} (\vec{\psi}_{J_1} | \vec{\phi}_{I_1} | \upsilon_{|I_2|, |J_2|} (\vec{\phi}_{I_2} | \vec{\psi}_{J_2} | \alpha ))$ \\ \hline

    \hline
  \end{tabular}
\caption{Expansion of $``$standard terms'' in 
Equation \ref{eq:upsilon_homotopy} and the 
$``$extra terms'' that cancel them}
\label{table:t21}  
(Technically, the last term in the right column is not an extra term, but we include it in the table for convenience.)
\end{table} 
\end{center}
%
$\newline
\newline$
\begin{table}
\begin{center}
  \begin{tabular}{ p{6.25in} | p{2.5in} }
    \hline
    Expression from seventh Row & Cancels with Extra Term \\ \hline

    $\psi_1(\lambda(\vec{\phi}_{I_1}) \lambda(\vec{\psi}_{J_2}) \lambda(\vec{\phi}_{I_4}) \mathfrak{a}_4, \phi_{|I_1|+1} (\lambda(\vec{\psi}_{J_3}) \lambda(\vec{\phi}_{I_5}) \mathfrak{a}_5, a_0, \mathfrak{a}_1), \lambda(\vec{\phi}_{I_2 \backslash |I_1| + 1}) \mathfrak{a}_2) \otimes \lambda(\vec{\psi}_{J_1}) \lambda(\vec{\phi}_{I_3}) \mathfrak{a}_3$ &
     $\psi_1 \{ \vec{\phi}_{I_1} \} \cdot B_{|I_2|, m-1} (\vec{\phi}_{I_2} | \vec{\psi}_{\{2,\cdots,m\}} | \alpha)$ \\ \hline

    $\psi_1(\lambda(\vec{\phi}_{I_1}) \lambda(\vec{\psi}_{J_2}) \lambda(\vec{\phi}_{I_4}) \mathfrak{a}_4, f_{|I_1|+1}a_0, \lambda(\vec{\phi}_{I_2 \backslash |I_1| + 1}) \mathfrak{a}_1) \otimes \lambda(\vec{\psi}_{J_1}) \lambda(\vec{\phi}_{I_3}) \mathfrak{a}_2$ &
    $\phi_1 \cdot B_{n-1, m} (\vec{\phi}_{\{2,\cdots,n\}} | \vec{\psi} | \alpha)$ \\ \hline

    $g_0\phi_1(\lambda(\vec{\psi}_{J_2}) \lambda(\vec{\phi}_{I_2}) \mathfrak{a}_3, a_0, \mathfrak{a}_1) \otimes \lambda(\vec{\psi}_{J_1}) \lambda(\vec{\phi}_{I_1}) \mathfrak{a}_2$ &
    $\psi_1 \{ \vec{\phi}_{I_1} \} \cdot B_{|I_2|, m-1} (\vec{\phi}_{I_2} | \vec{\psi}_{\{2,\cdots,m\}} | \alpha)$ \\ \hline

    \hline
  \end{tabular}
\end{center}
\caption{Expansion of remaining $``$seventh-row terms'' in 
Equation \ref{eq:upsilon_homotopy} and the 
$``$extra terms'' that cancel them}
\label{table:t22}
\end{table}
\end{landscape}