\begin{prop}
\label{prop:c3}
Let $\Upsilon_{A_0,A_1}: 
C(A_0 \to A_1 \to A_0) \longrightarrow
C(A_1 \to A_0 \to A_1)$ and 
$B_{A_0,A_1}:C(A_0 \to A_1 \to A_0) 
\longrightarrow C(A_0 \to A_1 \to A_0)$ 
be the maps defined in Propositions 
\ref{prop:c1} and \ref{prop:c2} above. 
Then, $[\Upsilon, B] := 
\Upsilon_{A_0,A_1} B_{A_0,A_1} - 
B_{A_1,A_0}\Upsilon_{A_0,A_1} = 0$.
\end{prop}
%
\begin{proof}
We show that $[\Upsilon, B] = 0$ by direct 
computation. Since all of the maps are maps 
of cofree comodules, we only need to check 
that $\pi_1([\Upsilon, B]) = 0$ where 
$\pi_1$ denotes projection of the comodule 
onto cogenerators. We check this directly.
%
\begin{equation*}
\begin{aligned}
\pi_1 \circ \Upsilon_{A_0,A_1} \circ B_{A_0,A_1} 
  (\vec{\phi} | \vec{\psi} | \alpha ) 
&= \upsilon_{|I_1|, |J_1|} (\vec{\phi}_{I_1} | \vec{\psi}_{J_1} | 
  B_{|I_2|, |J_2|} (\vec{\phi}_{I_1} | \vec{\psi}_{J_1} | \alpha)) \\
&= \upsilon_{|I_1|, |J_1|} (\vec{\phi}_{I_1} | \vec{\psi}_{J_1} | 
  1 \otimes \lambda(\vec{\psi}_{J_2}) \lambda(\vec{\phi}_{I_2}) 
  \mathfrak{a}_2, a_0, \mathfrak{a}_1) \\
&= 1 \otimes \lambda(\vec{\phi}_{I_1}) \big( 
  \lambda(\vec{\psi}) \lambda(\vec{\phi}_{I_2}) 
  \mathfrak{a}_2, a_0, \mathfrak{a}_1 \big)
\end{aligned}
\end{equation*}
%
\begin{align*}
\pi_1 \circ B_{A_1,A_0} \circ \Upsilon_{A_0,A_1} 
  (\vec{\phi} | \vec{\psi} | \alpha ) 
&= B_{|J_1|, |I_1|} (\vec{\psi}_{J_1} | \vec{\phi}_{I_1} | 
  \upsilon_{|I_2|, |J_2|} (\vec{\phi}_{I_1} | \vec{\psi}_{J_1} | \alpha)) \\
&= B_{|J_1|, |I_1|} \big(\vec{\psi}_{J_1} | \vec{\phi}_{I_1} | \phi_{|I_1|+1} (
  \lambda(\vec{\psi}_{J_2}) \lambda(\vec{\phi}_{I_3}) 
  \mathfrak{a}_3, a_0, \mathfrak{a}_1) \otimes 
  \lambda(\vec{\phi}_{I_2 \backslash |I_1| + 1}) 
  \mathfrak{a}_2 \; + \\
&\hphantom{{}moveovermoveover{}} 
  + a_0 \otimes \lambda(\vec{\phi}_{I_2 \backslash |I_1| + 1}) 
  \mathfrak{a}_1 \; \; \; 
  \text{if }J_2 = \emptyset \big) \\
&= 1 \otimes \lambda(\vec{\phi}_{I_1}) \lambda(\vec{\psi}_{J_1}) 
  \lambda(\vec{\phi}_{I_3}) \mathfrak{a}_3 \otimes 
  \phi_{|I_1|+1} (\lambda(\vec{\psi}_{J_2}) \lambda(\vec{\phi}_{I_4}) 
  \mathfrak{a}_4, a_0, \mathfrak{a}_1) \otimes 
  \lambda(\vec{\phi}_{I_2 \backslash |I_1| + 1}) 
  \mathfrak{a}_2 \; + \\
&\hphantom{{}==} 
  + 1 \otimes \lambda(\vec{\phi}_{I_1}) \lambda(\vec{\psi}) 
  \lambda(\vec{\phi}_{I_3}) \mathfrak{a}_2 \otimes 
  a_0 \otimes \lambda(\vec{\phi}_{I_2}) \mathfrak{a}_1
\end{align*}
%
It's clear that $\pi_1 \circ \Upsilon_{A_0,A_1} 
\circ B_{A_0,A_1} =  \pi_1 \circ B_{A_1,A_0} 
\circ \Upsilon_{A_0,A_1}$: The final expansion of 
$\pi_1 \circ \Upsilon_{A_0,A_1} \circ B_{A_0,A_1}$ 
is the sum of the two terms in the final expansion 
of $\pi_1 \circ B_{A_1,A_0} \circ \Upsilon_{A_0,A_1}$, 
which is the sum of terms in which one of 
the $\phi$'s contains $a_0$ and the terms in which 
none of the $\phi$'s contains $a_0$).
\end{proof}