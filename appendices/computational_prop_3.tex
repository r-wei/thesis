\begin{prop}
$[\Upsilon, B] = 0$.
\end{prop}

\begin{proof}
We show that $[\Upsilon, B] = 0$ by direct computation. Since all of the maps are maps of cofree comodules, we only need to check that $\pi_1([\Upsilon, B]) = 0$ where $\pi_1$ denotes projection of the comodule onto its degree-1 component, (i.e., $\pi_1: Bar(C^\bullet(A_0, A_1)) \otimes Bar(C^\bullet(A_1, A_0)) \otimes C_{-\bullet}(A_0, A_0) \rightarrow C_{-\bullet}(A_0, A_0)$). We check this directly.

\begin{equation*}
\begin{aligned}
\pi_1 \circ \Upsilon \circ B (\vec{\phi} | \vec{\psi} | \alpha ) 
&= \upsilon_{|I_1|, |J_1|} (\vec{\phi}_{I_1} | \vec{\psi}_{J_1} | B_{|I_2|, |J_2|} (\vec{\phi}_{I_1} | \vec{\psi}_{J_1} | \alpha)) \\
&= \upsilon_{|I_1|, |J_1|} (\vec{\phi}_{I_1} | \vec{\psi}_{J_1} | 1 \otimes \lambda(\vec{\psi}_{J_2}) \lambda(\vec{\phi}_{I_2}) \mathfrak{a}_2, a_0, \mathfrak{a}_1) \\
&= 1 \otimes \lambda(\vec{\phi}_{I_1}) \big( \lambda(\vec{\psi}) \lambda(\vec{\phi}_{I_2}) \mathfrak{a}_2, a_0, \mathfrak{a}_1 \big)
\end{aligned}
\end{equation*}

\begin{align*}
\pi_1 \circ B \circ \Upsilon (\vec{\phi} | \vec{\psi} | \alpha ) 
&= B_{|J_1|, |I_1|} (\vec{\psi}_{J_1} | \vec{\phi}_{I_1} | \upsilon_{|I_2|, |J_2|} (\vec{\phi}_{I_1} | \vec{\psi}_{J_1} | \alpha)) \\
&= B_{|J_1|, |I_1|} \big(\vec{\psi}_{J_1} | \vec{\phi}_{I_1} | \phi_{|I_1|+1} (\lambda(\vec{\psi}_{J_2}) \lambda(\vec{\phi}_{I_3}) \mathfrak{a}_3, a_0, \mathfrak{a}_1) \otimes \lambda(\vec{\phi}_{I_2 \backslash |I_1| + 1}) \mathfrak{a}_2 \; + \\
&\hphantom{{}=B_{|J_1|, |I_1|} \big(\vec{\psi}_{J_1} | \vec{\phi}_{I_1} |} + a_0 \otimes \lambda(\vec{\phi}_{I_2 \backslash |I_1| + 1}) \mathfrak{a}_1 \; \; \; \text{if }\vec{\psi}_{J_2} = \emptyset \big) \\
&= 1 \otimes \lambda(\vec{\phi}_{I_1}) \lambda(\vec{\psi}_{J_1}) \lambda(\vec{\phi}_{I_3}) \mathfrak{a}_3 \otimes \phi_{|I_1|+1} (\lambda(\vec{\psi}_{J_2}) \lambda(\vec{\phi}_{I_4}) \mathfrak{a}_4, a_0, \mathfrak{a}_1) \otimes \lambda(\vec{\phi}_{I_2 \backslash |I_1| + 1}) \mathfrak{a}_2 \; + \\
&\hphantom{{}==} + 1 \otimes \lambda(\vec{\phi}_{I_1}) \lambda(\vec{\psi}) \lambda(\vec{\phi}_{I_3}) \mathfrak{a}_2 \otimes a_0 \otimes \lambda(\vec{\phi}_{I_2}) \mathfrak{a}_1
\end{align*}

It's clear that $\pi_1 \circ \Upsilon \circ B =  \pi_1 \circ B \circ \Upsilon$. The final expansion of $\pi_1 \circ \Upsilon \circ B$ is the sum of the two terms in the final expansion of $\pi_1 \circ B \circ \Upsilon$, (i.e., the sum of terms in which one of the $\phi$'s contains $a_0$ and the terms in which none of the $\phi$'s contains $a_0$).
\end{proof}