\chapter{Computations}
In this appendix, we give the 
computational propositions 
needed to establish the 
homotopically sheafy-cyclic structure 
on dg comodules. All the comodules we work 
with will be cofree, 
and we will define maps into them by 
giving maps into cogenerators 
(see Equation \ref{eq:quasicofree}).

\section{Computational notation}
For this section's propositions, we 
establish the following notation:
\begin{align*}
A_0, A_1 
&\textrm{ fixed algebras}\\
(\vec{\phi} | \vec{\psi} | \alpha) 
&:= 
(\phi_1\smdots\phi_n | \psi_1\smdots\psi_m| \alpha)\\
&= 
\xymatrix{
A_0 \ar@/^5pc/[r]^{f_0} 
\ar@/^2pc/[r]^{\big\Downarrow \phi_1}_{f_1} 
\ar@/_2pc/[r]^{\substack{\vdots\\ f_n}}
\ar@/_4pc/[rr]_{id}^{\substack{\alpha \\ \\ \\ \\ }}
& A_1 \ar@/^5pc/[r]^{g_0} 
\ar@/^2pc/[r]^{\big\Downarrow \psi_1}_{g_1} 
\ar@/_2pc/[r]^{\substack{\vdots\\ g_m}}
& A_0
}
\in C(A_0 \to A_1 \to A_0)(g_0f_0)\\
\vec{\phi}_{\{i_1, i_2,\smdots, i_k\}}
&:= 
\phi_{i_1}\phi_{i_2}\smdots\phi_{i_k}
 \quad \textrm{ where $\{i_1, i_2,\smdots, i_k\}$ 
 is an ordered subset of $\{1,\smdots,n\}$}\\
\vec{\phi}_{\{\}}
&:= 
1 \in k \cong Bar_0(C^\bullet(A_0, A_1))\\
\vec{\psi}_{\{\}}
&:= 
1 \in k \cong Bar_0(C^\bullet(A_1, A_0))\\
|I| 
&:=
\textrm{number of elements in a set $I$}\\
I_1I_2 
&:= 
\textrm{concatenation as ordered sets 
of possibly-empty sets $I_1$ and $I_2$}\\
\lambda({\vec{\psi}}),\, \tilde{\delta},\, 
b^\prime,\, b,\, \psi\{\vec{\phi}\}\cdot \alpha
&=
\textrm{see Appendix \ref{chap:hochschild} 
for standard operations on Hochschild (co)chains}\\
\end{align*}
%
\subsection{Notation for elements of Hochschild chains}
Let $a_0 \otimes a_1 \otimes \cdots \otimes a_n$ 
denote a typical element of $C_{-\bullet}(A,A)$ where 
$A$ is some algebra. At times, we wish to feed a portion 
of $a_0 \otimes a_1 \otimes \smdots \otimes a_n$ to a 
Hochschild cochain (or other map on chains) without 
specifying the degree of the cochain. To do this, 
we will rewrite $a_0 \otimes a_1 \otimes \smdots \otimes a_n 
= a_0 \otimes \mathfrak{a}_1 \otimes \smdots \otimes \mathfrak{a}_r$ 
where each $\mathfrak{a}_i = a_{j_i} \otimes a_{j_i+1} \otimes 
\smdots \otimes a_{j_{i+1}-1}$ and $\mathfrak{a}_i$ 
is an empty chain if $j_i = j_{i+1}$.

For example, if $\phi \in C^2(A,A)$, then we 
rewrite 
$$
\sum \limits_{1\leq i \leq n-1} 
a_0 \otimes a_1 \otimes \smdots a_{i-1} \otimes 
\phi(a_i, a_{i+1}) \otimes a_{i+2} \otimes \smdots \otimes a_n
=
\sum
a_0 \otimes \mathfrak{a}_1 \otimes \phi(\mathfrak{a}_2)
\otimes \mathfrak{a}_3.
$$

\section{Computational Propositions}
% \begin{figure} \label{fig:upsilon}
% \centerline{\xymatrix{
% A_0 \ar@/^5pc/[r]^{f_0} 
% \ar@/^2pc/[r]^{\big\Downarrow \phi_1}_{f_1} 
% \ar@/_2pc/[r]^{\substack{\vdots\\ f_n}}
% \ar@/_4pc/[rr]_{id}^{\substack{\alpha \\ \\ \\ \\ }}
% & A_1 \ar@/^5pc/[r]^{g_0} 
% \ar@/^2pc/[r]^{\big\Downarrow \psi_1}_{g_1} 
% \ar@/_2pc/[r]^{\substack{\vdots\\ g_m}}
% & A_0
% }
% $\overset{\Upsilon}\longrightarrow$
% \xymatrix{
% A_1 \ar@/^5pc/[r]^{g_0} 
% \ar@/^2pc/[r]^{\big\Downarrow \psi_1}_{g_1} 
% \ar@/_2pc/[r]^{\substack{\vdots\\ g_m}}
% \ar@/_4pc/[rr]_{id}^{\substack{\alpha \\ \\ \\ \\ }}
% & A_0 \ar@/^5pc/[r]^{f_0} 
% \ar@/^2pc/[r]^{\big\Downarrow \phi_1}_{f_1} 
% \ar@/_2pc/[r]^{\substack{\vdots\\ f_n}}
% & A_1
% }}
% \caption{A picture of the domain and target of $\Upsilon$}
% \end{figure}

\begin{prop}
\label{prop:c1}
Let $\hat{\tau}_1: 
B(A_0 \to A_1 \to A_0) 
\longrightarrow B(A_1 \to A_0 \to A_1)$ 
be as defined in Section 
\ref{sec:cyclic_B(n)}.
Recall from Example \ref{eg:pb5} that 
$\hat{\tau}_1^*C(A_1 \to A_0 \to A_0)
\cong C(A_1 \to A_0 \to A_1)$ 
as complexes. Define a map 
$$
\Upsilon_{A_0,A_1}: C(A_0 \to A_1 \to A_0)
\to \hat{\tau}_1^*C(A_1 \to A_0 \to A_1)
$$
of comodules over 
$B(A_0 \to A_1 \to A_0)$ by mapping into 
cogenerators as follows:
\begin{align}\label{eq:define_upsilon}
\upsilon^{f_0, g_0}: C(A_0 \to A_1 \to A_0)(f_0,g_0) 
&\to
\hat{\tau}_1^*C(A_1 \to A_0 \to A_1)(g_0,f_0)\\
&\cong 
C(A_1 \to A_0 \to A_1)(g_0,f_0)\\
&\xrightarrow[cogenerators]{\textrm{project onto}}
C_{-\bullet}(A_1, _{f_0g_0}{A_1}_{id})\\
\upsilon_{n,m}^{f_0,g_0} 
(\vec{\phi} | \vec{\psi} | \alpha) = 
& \sum_{\substack{I_1I_2 = \{2,\cdots,n\} \\
                          \textrm{as ordered sets}}}
  \phi_1(\lambda(\vec{\psi})\lambda(\vec{\phi_{I_2}})\cdot \mathfrak{a}_3, a_0, \mathfrak{a}_1) \otimes \lambda(\vec{\phi_{I_1}}) \cdot \mathfrak{a}_2 \\
&\phantom{{}move{}}
\bigg( + f_0a_0 \otimes \lambda(\vec{\phi}) \mathfrak{a}_1 
  \; \; \; \; if \; \; m = 0 \bigg).
\end{align}
Then, $\Upsilon_{A_0,A_1}: C(A_0 \to A_1 \to A_0)
\to \hat{\tau}^*C(A_1 \to A_0 \to A_1)$ 
is a map of dg comodules over 
$B(A_0 \to A_1 \to A_0)$.
\end{prop}
%
\begin{proof}
We must show: (1) $\Upsilon$ is a map of comodules, and 
(2) $\Upsilon$ commutes with the differentials. (In this 
proof, we drop the subscripts and write 
$\Upsilon := \Upsilon_{A_0, A_1}$.)

(1) This proof is standard for cofree comodules. 
Let ($\vec{\phi} | \vec{\psi} | \alpha$) be as 
in the statement of the proposition. We want to 
show that $\Upsilon$ commutes with the coproducts. 
On one hand,
\begin{align*}
&\phantom{{}={}}
[(id_B \otimes \Upsilon) \circ 
  \Delta_{C(A_0 \to A_1 \to A_0)}] 
  ( \vec{\phi} | \vec{\psi} | \alpha ) \\
&= [id_B \otimes \Upsilon]
	\big( \sum_{\substack{I_1I_2 = \{1,2,\cdots,n\} \, \textrm{and} \\ 
						  J_1J_2 = \{1,2,\cdots,m\} \\
				          \textrm{as ordered sets}}} 
    \epsilon_{I_2,J_1}\cdot 
    (\vec{\phi}_{I_1} | \vec{\psi}_{J_1}) \otimes (\vec{\phi}_{I_2} | \vec{\psi}_{J_2} | \alpha) \, \big) \\
&= \sum_{\substack{I_1I_2I_3 = \{1,2,\cdots,n\} \, \textrm{and} \\ 
				   J_1J_2J_3 = \{1,2,\cdots,m\} \\
				   \textrm{as ordered sets}}} 
    \epsilon_{I_2I_3,J_1}\cdot \epsilon_{I_3,J_2}\cdot
    (\vec{\phi}_{I_1} | \vec{\psi}_{J_1}) \otimes 
    (\vec{\phi}_{I_2} | \vec{\psi}_{J_2}) \otimes 
    \upsilon_{|I_3|,|J_3|}(\vec{\phi}_{I_3} | \vec{\psi}_{J_3} | \alpha). \\
\end{align*}
On the other hand,
\begin{align*}
&\phantom{{}={}}
[\Delta_{\hat{\tau}^*C(A_1 \to A_0 \to A_1)} 
  \circ \Upsilon ]
  ( \vec{\phi} | \vec{\psi} | \alpha ) \\
&= \Delta_{\hat{\tau}^*C(A_1 \to A_0 \to A_1)}
	\big( \sum_{\substack{I_1I_2 = \{1,2,\cdots,n\} \, \textrm{and} \\ 
						  J_1J_2 = \{1,2,\cdots,m\} \\
				          \textrm{as ordered sets}}}
	\epsilon_{I_2,J_1} \cdot 
  (\vec{\phi}_{I_1} | \vec{\psi}_{J_1}) \otimes 
    \upsilon_{|I_2|,|J_2|}(\vec{\phi}_{I_2} | \vec{\psi}_{J_2} | \alpha) \, \big)\\
&= \sum_{\substack{I_1I_2I_3 = \{1,2,\cdots,n\} \, \textrm{and} \\ 
				   J_1J_2J_3 = \{1,2,\cdots,m\} \\
				   \textrm{as ordered sets}}} 
    \epsilon_{I_2I_3,J_1}\cdot \epsilon_{I_3,J_2}\cdot
    (\vec{\phi}_{I_1} | \vec{\psi}_{J_1}) \otimes 
    (\vec{\phi}_{I_2} | \vec{\psi}_{J_2}) \otimes 
    \upsilon_{|I_3|,|J_3|}(\vec{\phi}_{I_3} | \vec{\psi}_{J_3} | \alpha).   				          
\end{align*}
Clearly 
$(id_B \otimes \Upsilon) \circ 
\Delta_{C(A_0 \to A_1 \to A_0)} = 
\Delta_{\hat{\tau}^*C(A_1 \to A_0 \to A_1)} 
\circ \Upsilon$.

(2) We will show that $\Upsilon$ commutes with 
the differentials by direct computation. Since 
$\Upsilon$ is a map of cofree comodules, we only 
need to check that $\pi_1 \circ D(\Upsilon) = 0$ 
where $D(\Upsilon)$ is the differential applied 
to $\Upsilon$ as a linear map between complexes 
and $\pi_1$ denotes projection of a comodule 
onto its cogenerators. More explicitly, we want 
to check that
\begin{equation} \label{eq:upsilon}
\begin{aligned}
&\upsilon_{n, m} ( \tilde{\delta}(\vec{\phi}) | \vec{\psi} | \alpha ) \; + 
\upsilon_{n, m} ( \vec{\phi} | \tilde{\delta}(\vec{\psi}) | \alpha ) \; + 
\upsilon_{n-1, m} ( b^\prime(\vec{\phi}) | \vec{\psi} | \alpha ) \; + 
\upsilon_{n, m-1} ( \vec{\phi} | b^\prime(\vec{\psi}) | \alpha ) \; + \\
&\upsilon_{n, m} ( \vec{\phi} | \vec{\psi} | b(\alpha) ) \; + 
b \circ \upsilon_{n, m} ( \vec{\phi} | \vec{\psi} | \alpha ) \; + \\
& \sum \limits_{\substack{
  I_1I_2 = \{1,\smdots,n\}\\ 
  \textrm{as ordered sets}}}
  \epsilon_{I_2,\{1,\smdots,m-1\}}\cdot
  \upsilon_{|I_1|, m-1}(\vec{\phi}_{I_1} | \vec{\psi}_{\{1,\cdots, m-1\}} | \psi_{m} \{\vec{\phi}_{I_2}\} \cdot \alpha ) \; + \\
& \sum \limits_{\substack{
  J_1J_2 = \{1,\smdots,m\}\\ 
  \textrm{as ordered sets}}}
  \epsilon_{\{2,\smdots,n\},J_1}\cdot
\phi_1 \{ \psi_{J_1}\}\cdot \upsilon_{n-1, |J_2|}(\phi_{\{2,\cdots , n\}} | \psi_{J_2} | \alpha ) \; + \\  
& \epsilon_{\{n\},\{1,\smdots,m\}}\cdot
  \upsilon_{n-1, m}(\vec{\phi}_{\{1,\cdots, n-1\}} |\vec{\psi} | \phi_{n}\cdot \alpha) \; + \\
& \epsilon_{\{1,\smdots,n\},\{1\}}\cdot
\psi_1\cdot \upsilon_{n,m-1} ( \vec{\phi} | \vec{\psi}_{\{2,\cdots, m\}} | \alpha) \\ 
&= 0.
\end{aligned}
\end{equation}
In Equation \ref{eq:upsilon}, we will call the 
terms in rows 1-2 the ``standard terms'', 
and the terms in rows 3-6 the 
``extra terms''.

We compute the sum of the standard terms. 
In Table \ref{table:t1}, the leftmost column 
lists the expressions that don't cancel in 
the sum of the standard terms, the middle 
column gives the standard term from which 
the expression comes, and the rightmost 
column gives the term (extra or standard) 
that cancels the expression. 

All of the terms in Table \ref{table:t1} 
cancel, so $\Upsilon$ is a map of complexes.
\end{proof}
%
\begin{landscape}
\begin{center}
\begin{table}
  \begin{tabular}{ p{3in} | p{2in} | p{2.5in} }
    \hline
    Expression (Expansion) & 
      \breakcell{Comes from Standard Term\\in Equation \ref{eq:upsilon}} & 
      \breakcell{Cancelling Term\\in Equation \ref{eq:upsilon}} \\ \hline
    
    \breakcell{$f_0\psi_1(\lambda(\vec{\phi}_{I_2}) \mathfrak{a}_3 ) \cdot$\\
    $\phi_1(\lambda(\vec{\psi}_{\{2,\cdots, m\}} \lambda(\vec{\phi}_{I_3}) \mathfrak{a}_4, a_0, \mathfrak{a}_1) \otimes \lambda(\vec{\phi}_{I_1}) \mathfrak{a}_2$} &
    $\upsilon_{n, m} (\delta(\phi_1)\phi_2 \cdots \phi_n | \vec{\psi} | \alpha)$ & 
    $f_0 \psi_1 \cdot \upsilon_{n,m-1} ( \vec{\phi} | \vec{\psi}_{\{2,\cdots, m\}} | \alpha)$ \\ \hline

    \breakcell{$\phi_1( \lambda(\vec{\psi}_{\{1,\cdots, m-1\}}) \lambda(\vec{\phi}_{I_2}) \mathfrak{a}_3,$\\ 
    $\phantom{mo} \psi_{m} ( \lambda(\vec{\phi}_{I_3}) \mathfrak{a}_4) \cdot a_0, \mathfrak{a}_1 ) \otimes \lambda(\vec{\phi}_{I_1}) \mathfrak{a}_2$} &
    $\upsilon_{n, m} (\delta(\phi_1)\phi_2 \cdots \phi_n | \vec{\psi} | \alpha)$ &
    $\upsilon_{|I_1|, m-1}(\vec{\phi}_{I_1} | \vec{\psi}_{\{1,\cdots, m-1\}} | \psi_{m} \{\vec{\phi}_{I_2}\}\cdot \alpha )$ \\ \hline

    \breakcell{$\phi_1( \lambda(\vec{\psi}) \lambda(\vec{\phi}_{I_2}) \mathfrak{a}_3, g_m \phi_n(\mathfrak{a}_4) \cdot a_0, \mathfrak{a}_1)\otimes$\\ 
    $\otimes\, \lambda(\vec{\phi}_{I_1}) \mathfrak{a}_2$} &
    $\upsilon_{n, m} (\delta(\phi_1)\phi_2 \cdots \phi_n | \vec{\psi} | \alpha)$ &
    $\upsilon_{n-1, m}(\vec{\phi}_{\{1, \cdots, n-1\}} | \vec{\psi} | g_m \phi_{n} \cdot \alpha )$ \\ \hline

    \breakcell{$\phi_1( \lambda(\vec{\psi}) \lambda(\vec{\phi}_{I_2}) \mathfrak{a}_2) \cdot f_1(a_0) \otimes \lambda(\vec{\phi}_{I_1}) \mathfrak{a}_1$} &
    $\upsilon_{n, m} (\delta(\phi_1)\phi_2 \cdots \phi_n | \vec{\psi} | \alpha)$ &
    $\phi_1 \cdot \upsilon_{n-1, 0}(\vec{\phi}_{\{2, \cdots, n\}} | \vec{\psi} |\alpha )$ \\ \hline

    \breakcell{$f_0a_0 \cdot \phi_1(\mathfrak{a}_1) \otimes \lambda(\vec{\phi}_{\{1,\cdots,n-1\}}) \mathfrak{a}_2$} &
    \breakcell{$\upsilon_{n, m} (\delta(\phi_1)\phi_2 \cdots \phi_n | \vec{\psi} | \alpha)$ \\ if $\vec{\psi} = 1$} & 
    \breakcell{$b \circ \upsilon_{n, m} (\vec{\phi} | \vec{\psi} | \alpha)$ \\ if $\vec{\psi} =1$} \\ \hline

    \breakcell{$f_0 g_m \phi_n(\mathfrak{a}_2) f_0a_0 \otimes \lambda(\vec{\phi}_{\{1,\cdots,n-1\}}) \mathfrak{a}_1$} &
    \breakcell{$b \circ \upsilon_{n, m} (\vec{\phi} | \vec{\psi} | \alpha)$ \\ if $\vec{\psi} = 1$} &
    \breakcell{$\upsilon_{n-1, m}(\vec{\phi}_{\{1, \cdots, n-1\}} | \vec{\psi} | g_m \phi_{n} \cdot \alpha )$ \\ if $\vec{\psi} = 1$} \\ \hline

    \breakcell{$\phi_1(\lambda(\vec{\psi}) \lambda(\vec{\phi}_{I_2}) \mathfrak{a}_4, a_0, \mathfrak{a}_1) \cdot \phi_2(\mathfrak{a}_2) \otimes \lambda(\vec{\phi}_{I_1}) \mathfrak{a}_3$} &
    $b \circ \upsilon_{n, m} (\vec{\phi} | \vec{\psi} | \alpha)$ &
    $\upsilon_{n-1, m}(\phi_1 \cup \phi_2 \phi_3 \cdots \phi_n | \vec{\psi} | \alpha)$ \\ \hline
    
    \breakcell{$\phi_1(\lambda(\vec{\psi}_{J_1}) \lambda(\vec{\phi}_{I_2}) \mathfrak{a}_3) \phi_2(\lambda(\vec{\psi}_{J_2} \lambda(\vec{\phi}_{I_3}) \mathfrak{a}_3,$\\
    $\phantom{mo} a_0, \mathfrak{a}_1) \otimes \lambda(\vec{\phi}_{I_1}) \mathfrak{a}_2$} &
    $\upsilon_{n-1, m}(\phi_1 \cup \phi_2 \phi_3 \cdots \phi_n | \vec{\psi} | \alpha)$ &
     $\phi_1 \{ \vec{\psi}_{J_1} \} \cdot \upsilon_{n-1, |J_2|}(\vec{\phi}_{\{2, \cdots, n\}} | \vec{\psi}_{J_2} |\alpha )$\\ \hline

    \breakcell{$f_0 \psi_1(\lambda(\vec{\phi}_{I_2}) \mathfrak{a}_2) \cdot f_0a_0 \otimes \lambda(\vec{\phi}_{I_1}) \mathfrak{a}_1$} &  
    \breakcell{$f_0 \psi_1 \cdot \upsilon_{n, 0}(\vec{\phi} | 1 | \alpha)$ \\ if $\vec{\psi} = \psi_1$} &
    \breakcell{$ \upsilon_{|I_1|, 0} (\vec{\phi}_{I_1} | 1 | \psi_1 \{ \vec{\phi}_{I_2} \} \cdot \alpha )$ \\ if $\vec{\psi} = \psi_1$} \\ \hline
    \hline
  \end{tabular}
\caption{Expansion of terms in Equation \ref{eq:upsilon}}
\label{table:t1}
(Technically, the last term in the middle column is not a standard term, but we include it in the table for convenience.)
\end{table}
\end{center}
\end{landscape}
\begin{figure} \label{fig:upsilon}
\centerline{\xymatrix{
A_0 \ar@/^5pc/[r]^{f_0} 
\ar@/^2pc/[r]^{\big\Downarrow \phi_1}_{f_1} 
\ar@/_2pc/[r]^{\substack{\vdots\\ f_n}}
\ar@/_4pc/[rr]_{id}^{\substack{\alpha \\ \\ \\ \\ }}
& A_1 \ar@/^5pc/[r]^{g_0} 
\ar@/^2pc/[r]^{\big\Downarrow \psi_1}_{g_1} 
\ar@/_2pc/[r]^{\substack{\vdots\\ g_m}}
& A_0
}
$\overset{B}\longrightarrow$
\xymatrix{
A_0 \ar@/^5pc/[r]^{f_0} 
\ar@/^2pc/[r]^{\big\Downarrow \phi_1}_{f_1} 
\ar@/_2pc/[r]^{\substack{\vdots\\ f_n}}
\ar@/_4pc/[rr]_{id}^{\substack{\alpha \\ \\ \\ \\ }}
& A_1 \ar@/^5pc/[r]^{g_0} 
\ar@/^2pc/[r]^{\big\Downarrow \psi_1}_{g_1} 
\ar@/_2pc/[r]^{\substack{\vdots\\ g_m}}
& A_0
}}
\caption{A picture of the domain and target of $B$}
\end{figure}

\begin{prop}
$D(B) = \Upsilon^2 - Id$ where
\begin{align*}
B_{n, m} (\vec{\phi} | \vec{\psi} | \alpha) 
= 1 \otimes \lambda(\psi)\lambda(\phi) \mathfrak{a}_2 \otimes a_0 \otimes \mathfrak{a}_1                          
\end{align*}
\end{prop}

\begin{proof}
We must show: (1) $B$ is a map of comodules, and (2) $D(B) = \Upsilon^2 - Id$.

(1) This proof is standard for cofree comodules. 

(2) We show that $D(B) = \Upsilon^2 - Id$ by direct computation. Since all of the maps are maps of cofree comodules, we only need to check that $\pi_1(D(B) - \Upsilon^2 - Id) = 0$ where $\pi_1$ denotes projection of the comodule onto its degree-1 component, (i.e., $\pi_1: Bar(C^\bullet(A_0, A_1)) \otimes Bar(C^\bullet(A_1, A_0)) \otimes C_{-\bullet}(A_0, A_0) \rightarrow C_{-\bullet}(A_0, A_0)$). More explicitly, we want to check that
\begin{equation} \label{eq:upsilon_homotopy}
\begin{aligned}
&B_{n, m} ( \tilde{\delta}(\vec{\phi}) | \vec{\psi} | \alpha ) \; + 
B_{n, m} ( \vec{\phi} | \tilde{\delta}(\vec{\psi}) | \alpha ) \; + 
B_{n-1, m} ( b^\prime(\vec{\phi}) | \vec{\psi} | \alpha ) \; + 
B_{n, m-1} ( \vec{\phi} | b^\prime(\vec{\psi}) | \alpha ) \; + \\
&B_{n, m} ( \vec{\phi} | \vec{\psi} | b(\alpha) ) \; + 
b \circ B_{n, m} ( \vec{\phi} | \vec{\psi} | \alpha ) \; + \\
&B_{|I_1|, m-1}(\vec{\phi}_{I_1} | \vec{\psi}_{\{1,\cdots, m-1\}} | \psi_{m} \{\vec{\phi}_{I_2}\}\cdot \alpha ) \; + 
B_{n-1, m}(\vec{\phi}_{\{1,\cdots, n-1\}} |\vec{\psi}_{m} | \phi_{n} \cdot \alpha) \; + \\
&\phi_1 \{ \psi_{J_1}\} \cdot B_{\vec{\phi}|-1, |J_2|}(\phi_{\{2,\cdots , n\}} | \psi_{J_2} | \alpha ) \; + 
\psi_1 \cdot B_{n,m-1} ( \vec{\phi} | \vec{\psi}_{\{2,\cdots, |\vec{\psi}\}} | \alpha)  - \\ 
&\upsilon_{|J_1|, |I_1|} (\vec{\psi}_{J_1} | \vec{\phi}_{I_1} | \upsilon_{|I_2|, |J_2|} (\vec{\phi}_{I_2} | \vec{\psi}_{J_2} | \alpha ))  - \pi_1(\vec{\phi} | \vec{\psi} | \alpha) \\
&= 0.
\end{aligned}
\end{equation}

We will call the terms in the first two rows the ``standard terms'' in the computaion of $D(B)$, and the terms in the second two rows the ``extra terms'' in the computation of $D(B)$. The fifth row is $\pi_1(\Upsilon^2 - Id)$. 

Following the logic of Lemma [number], we compute the sum of the standard terms. In the chart below, the leftmost column lists the expressions that don't cancel in the sum of the standard terms, the middle column gives the standard term from which the expression comes, and the rightmost column gives the extra term that cancels the expression. 

\newpage

\begin{landscape}
\begin{center}
  \begin{tabular}{ p{3.25in} | p{2in} | p{2.5in} }
    \hline
    Expression & Comes from Standard Term & Cancels with Extra Term \\ \hline

    $\psi_1(\lambda(\vec{\phi}_{I_1}) \mathfrak{a}_2) \otimes \lambda(\vec{\psi}_{\{2,\cdots,m\}}) \lambda(\vec{\phi}_{I_2}) \mathfrak{a}_3 \otimes a_0 \otimes \mathfrak{a}_1$ &
    $b \circ B_{n,m} (\vec{\phi} | \vec{\psi} | \alpha)$ & 
    $\psi_1 \{ \vec{\phi}_{I_1} \} \cdot B_{|I_2|, m-1} (\vec{\phi}_{I_2} | \vec{\psi}_{\{2, \cdots, m \}} | \alpha)$ \\ \hline

    $g_0\phi_1( \mathfrak{a}_2 ) \otimes \lambda(\vec{\psi}) \lambda(\vec{\phi}_{\{2, \cdots, n\}}) \mathfrak{a}_3 \otimes a_0 \otimes \mathfrak{a}_1$ &
    $b \circ B_{n,m} (\vec{\phi} | \vec{\psi} | \alpha)$ & 
    $\phi_1 \cdot B_{n-1, m} (\vec{\phi}_{\{2, \cdots, n\}} | \vec{\psi} | \alpha)$ \\ \hline

    $1 \otimes \lambda(\vec{\psi}) \lambda(\vec{\phi}_{\{1, \cdots, n-1\}}) \mathfrak{a}_2 \otimes g_m \phi_n(\mathfrak{a}_3 \cdot a_0 \otimes \mathfrak{a}_1$ &
    $b \circ B_{n,m} (\vec{\phi} | \vec{\psi} | \alpha)$ & 
    $B_{n-1, m} (\vec{\phi}_{\{1, \cdots, n-1 \}} | \vec{\psi} | \phi_n \cdot \alpha)$ \\ \hline

    $1 \otimes \lambda(\vec{\psi}_{\{1, \cdots, m-1 \}}) \lambda(\vec{\phi}_{I_1}) \mathfrak{a}_2 \otimes g_m \psi_m( \lambda(\vec{\phi}_{I_2} \mathfrak{a}_3) \cdot a_0 \otimes \mathfrak{a}_1$ &
    $b \circ B_{n,m} (\vec{\phi} | \vec{\psi} | \alpha)$ & 
    $B_{|I_1|, m-1} (\vec{\phi}_{I_2} | \vec{\psi}_{\{1, \cdots, m-1\}} | \psi_m \{ \vec{\phi}_{I_2} \} \cdot \alpha)$ \\ \hline

    $g_0f_0a_0 \otimes \lambda(\vec{\psi}) \lambda(\vec{\phi}) \mathfrak{a}_1$ &
    $b \circ B_{n,m} (\vec{\phi} | \vec{\psi} | \alpha)$ & 
    $\upsilon_{|J_1|, |I_1|} (\vec{\psi}_{J_1} | \vec{\phi}_{I_1} | \upsilon_{|I_2|, |J_2|} (\vec{\phi}_{I_2} | \vec{\psi}_{J_2} | \alpha ))$ \\ \hline

    \hline
  \end{tabular}
\end{center}
(Technically, the last term in the right column is not an extra term, but we include it in the table for convenience.)
\end{landscape}

\newpage
\begin{landscape}

Now, we compute the remaining terms from the fifth row. In the chart below, the left column lists the remaining expressions that don't cancel in the fifth row, and the right column gives the extra term that cancels the expression. 

\begin{center}
  \begin{tabular}{ p{6.25in} | p{2.5in} }
    \hline
    Expression from Fifth Row & Cancels with Extra Term \\ \hline

    $\psi_1(\lambda(\vec{\phi}_{I_1}) \lambda(\vec{\psi}_{J_2}) \lambda(\vec{\phi}_{I_4}) \mathfrak{a}_4, \phi_{|I_1|+1} (\lambda(\vec{\psi}_{J_3}) \lambda(\vec{\phi}_{I_5}) \mathfrak{a}_5, a_0, \mathfrak{a}_1), \lambda(\vec{\phi}_{I_2 \backslash |I_1| + 1}) \mathfrak{a}_2) \otimes \lambda(\vec{\psi}_{J_1}) \lambda(\vec{\phi}_{I_3}) \mathfrak{a}_3$ &
     $\psi_1 \{ \vec{\phi}_{I_1} \} \cdot B_{|I_2|, m-1} (\vec{\phi}_{I_2} | \vec{\psi}_{\{2,\cdots,m\}} | \alpha)$ \\ \hline

    $\psi_1(\lambda(\vec{\phi}_{I_1}) \lambda(\vec{\psi}_{J_2}) \lambda(\vec{\phi}_{I_4}) \mathfrak{a}_4, f_{|I_1|+1}a_0, \lambda(\vec{\phi}_{I_2 \backslash |I_1| + 1}) \mathfrak{a}_1) \otimes \lambda(\vec{\psi}_{J_1}) \lambda(\vec{\phi}_{I_3}) \mathfrak{a}_2$ &
    $\phi_1 \cdot B_{n-1, m} (\vec{\phi}_{\{2,\cdots,n\}} | \vec{\psi} | \alpha)$ \\ \hline

    $g_0\phi_1(\lambda(\vec{\psi}_{J_2}) \lambda(\vec{\phi}_{I_2}) \mathfrak{a}_3, a_0, \mathfrak{a}_1) \otimes \lambda(\vec{\psi}_{J_1}) \lambda(\vec{\phi}_{I_1}) \mathfrak{a}_2$ &
    $\psi_1 \{ \vec{\phi}_{I_1} \} \cdot B_{|I_2|, m-1} (\vec{\phi}_{I_2} | \vec{\psi}_{\{2,\cdots,m\}} | \alpha)$ \\ \hline

    \hline
  \end{tabular}
\end{center}
\end{landscape}

All of the terms in the table describing the expansion of equation \ref{eq:upsilon_homotopy} cancel, so $D(B) = \Upsilon^2 - Id$.
\end{proof}
\begin{prop}
\label{prop:c3}
Let $\tau_{1!}(A_0,A_1): 
T(A_0 \to A_1 \to A_0) \longrightarrow
T(A_1 \to A_0 \to A_1)$ and 
$B(A_0,A_1): T(A_0 \to A_1 \to A_0) 
\longrightarrow T(A_0 \to A_1 \to A_0)$ 
be the maps defined in Propositions 
\ref{prop:c1} and \ref{prop:c2} above. 
Then, $$[\tau_{1!}, B] := 
\tau_{1!}(A_0,A_1) \circ B(A_0,A_1) - 
B(A_1,A_0) \circ \tau_{1!}(A_0,A_1) = 0.$$
\end{prop}
%
\begin{proof}
We show that $[\tau_{1!}, B] = 0$ by direct 
computation. Since all of the maps are maps 
of cofree comodules, we only need to check 
that $\pi_1([\tau_{1!}, B]) = 0$ where 
$\pi_1$ denotes projection of the comodule 
onto cogenerators. We check this directly.
%
\begin{align*}
&\phantom{=}
[\pi_1 \circ \tau_{1!}(A_0,A_1) \circ B(A_0,A_1)] 
  (\vec{\phi} | \vec{\psi} | \alpha ) \\
&= \sum \limits_{\substack{
  I_1I_2 = \{1,\smdots,n\}\\
  J_1J_2 = \{1,\smdots,m\}\\
  \textrm{as ordered sets}}}
\epsilon_{I_1,J_2} \cdot
  \tau_{1!}^{|I_1|, |J_1|} (\vec{\phi}_{I_1} | \vec{\psi}_{J_1} | 
    B^{|I_2|, |J_2|} (\vec{\phi}_{I_2} | \vec{\psi}_{J_2} | \alpha)) \\
&= 
\sum \limits_{\substack{
  I_1I_2 = \{1,\smdots,n\}\\
  J_1J_2 = \{1,\smdots,m\}\\
  \textrm{as ordered sets}}}
\begin{array}{l}  
\epsilon_{I_1,J_2} \cdot 
\eta_{\mathfrak{a_1},\mathfrak{a_2}} \cdot\\
\tau_{1!}^{|I_1|, |J_1|} (\vec{\phi}_{I_1} | \vec{\psi}_{J_1} | 
  1 \otimes \lambda(\vec{\psi}_{J_2}) \lambda(\vec{\phi}_{I_2}) 
  \mathfrak{a}_2, a_0, \mathfrak{a}_1)
\end{array} \\
&= 
\sum \limits_{\substack{
  I_1I_2 = \{1,\smdots,n\}\\
  J_1J_2 = \{1,\smdots,m\}\\
  \textrm{as ordered sets}}}
\epsilon_{I_1,J_2} \cdot 
\eta_{\mathfrak{a_1},\mathfrak{a_2}} \cdot
1 \otimes \lambda(\vec{\phi}_{I_1}) \big( 
  \lambda(\vec{\psi}) \lambda(\vec{\phi}_{I_2}) 
  \mathfrak{a}_2, a_0, \mathfrak{a}_1 \big)
\end{align*}
%
\begin{align*}
& \phantom{{}={}}
[\pi_1 \circ B(A_1,A_0) \circ \tau_{1!}(A_0,A_1)] 
  (\vec{\phi} | \vec{\psi} | \alpha ) \\
&=
\sum \limits_{\substack{
  I_1I_2 = \{1,\smdots,n\}\\
  J_1J_2 = \{1,\smdots,m\}\\
  \textrm{as ordered sets}}}
\epsilon_{I_1,J_2} \cdot
B^{|J_1|, |I_1|} (\vec{\psi}_{J_1} | \vec{\phi}_{I_1} | 
  \tau_{1!}^{|I_2|, |J_2|} (\vec{\phi}_{I_2} | \vec{\psi}_{J_2} | \alpha)) \\
&= 
\sum \limits_{\substack{
  I_1I_2 = \{1,\smdots,n\}\\
  J_1J_2 = \{1,\smdots,m\}\\
  \textrm{as ordered sets}}}
\begin{array}{l}{}
\epsilon_{I_1,J_2} \cdot
B^{|J_1|, |I_1|} \big(\vec{\psi}_{J_1} | \vec{\phi}_{I_1} | \phi_{|I_1|+1} (
  \lambda(\vec{\psi}_{J_2}) \lambda(\vec{\phi}_{I_3}) 
  \mathfrak{a}_3, a_0, \mathfrak{a}_1) \otimes 
  \lambda(\vec{\phi}_{I_2 \backslash |I_1| + 1}) 
  \mathfrak{a}_2 \; + \\
\hphantom{{}moveovermoveovermoveover{}} 
  + a_0 \otimes \lambda(\vec{\phi}_{I_2 \backslash |I_1| + 1}) 
  \mathfrak{a}_1 \; \; \; 
  \text{if }J_2 = \emptyset \big)
\end{array} \\
&= 
\sum \limits_{\substack{
  I_1I_2 = \{1,\smdots,n\}\\
  J_1J_2 = \{1,\smdots,m\}\\
  \textrm{as ordered sets}}}
\begin{array}{l}  
\epsilon_{I_1,J_2} \cdot  
\eta_{\mathfrak{a}_2,\mathfrak{a}_3} \cdot  
1 \otimes \lambda(\vec{\phi}_{I_1}) \lambda(\vec{\psi}_{J_1}) 
  \lambda(\vec{\phi}_{I_3}) \mathfrak{a}_3 \otimes 
  \phi_{|I_1|+1} (\lambda(\vec{\psi}_{J_2}) \lambda(\vec{\phi}_{I_4}) 
  \mathfrak{a}_4, a_0, \mathfrak{a}_1) \otimes \\
  \phantom{{}moveovermov{}}\otimes 
  \lambda(\vec{\phi}_{I_2 \backslash |I_1| + 1}) 
  \mathfrak{a}_2 \; + 
\end{array}\\
&\hphantom{{}moveovermove{}} 
  +\epsilon_{I_1,J_2} \cdot  
  \eta_{\mathfrak{a}_1,\mathfrak{a}_2} \cdot  
  1 \otimes \lambda(\vec{\phi}_{I_1}) \lambda(\vec{\psi}) 
  \lambda(\vec{\phi}_{I_3}) \mathfrak{a}_2 \otimes 
  a_0 \otimes \lambda(\vec{\phi}_{I_2}) \mathfrak{a}_1
\end{align*}
%
It's clear that $\pi_1 \circ \tau_{1!}(A_0,A_1) 
\circ B(A_0,A_1) =  \pi_1 \circ B(A_1,A_0) 
\circ \tau_{1!}(A_0,A_1)$: The final expansion of 
$\pi_1 \circ \tau_{1!}(A_0,A_1) \circ B(A_0,A_1)$ 
is the sum of the two terms in the final expansion 
of $\pi_1 \circ B(A_1,A_0) \circ \tau_{1!}(A_0,A_1)$, 
which is the sum of terms in which one of 
the $\phi$'s contains $a_0$ and the terms in which 
none of the $\phi$'s contains $a_0$).
\end{proof}

\section{More notation}
For the next two propositions, we will need 
some more notation. Set
\begin{align*}
A_0, A_1, A_2
&\textrm{ fixed algebras}\\
(\vec{\phi} | \vec{\psi} | \vec{\theta}| \alpha) 
&:= 
(\phi_1\smdots\phi_n | \psi_1\smdots\psi_m| 
  \theta_1 \smdots \theta_r|\alpha)\\
&= 
\xymatrix{
A_0 \ar@/^5pc/[r]^{f_0} 
\ar@/^2pc/[r]^{\big\Downarrow \phi_1}_{f_1} 
\ar@/_2pc/[r]^{\substack{\vdots\\ f_n}}
\ar@/_5pc/[rrr]_{id}^{\substack{\alpha \\ \\ \\ \\ }}
& A_1 \ar@/^5pc/[r]^{g_0} 
\ar@/^2pc/[r]^{\big\Downarrow \psi_1}_{g_1} 
\ar@/_2pc/[r]^{\substack{\vdots\\ g_m}}
& A_2 \ar@/^5pc/[r]^{h_0} 
\ar@/^2pc/[r]^{\big\Downarrow \theta_1}_{h_1} 
\ar@/_2pc/[r]^{\substack{\vdots\\ h_p}}
& A_0
}\\
&\in 
C(A_0 \to A_1 \to A_2 \to A_0)(h_0g_0f_0)\\
\Upsilon_{A_0\bullet A_1, A_2}:
  C(A_0 \to A_1 \to A_2 \to A_0) 
&\to
\hat{\tau}_2^*C(A_2 \to A_0 \to A_1 \to A_2)
  \quad \textrm{ map of dg comodules}\\
(\vec{\phi} | \vec{\psi} | \vec{\theta}| \alpha) 
&\mapsto 
\Upsilon_{A_0,A_2}
  (\vec{\phi} \bullet \vec{\psi} | \vec{\theta}| \alpha)\\
\Upsilon_{A_0,A_1\bullet A_2}:
  C(A_0 \to A_1 \to A_2 \to A_0) 
&\to
\hat{\tau}_2^{*2}C(A_1 \to A_2 \to A_0 \to A_1)
  \quad \textrm{ map of dg comodules}\\
(\vec{\phi} | \vec{\psi} | \vec{\theta}| \alpha) 
&\mapsto 
\Upsilon_{A_0,A_1}
  (\vec{\phi}|\vec{\psi} \bullet \vec{\theta}| \alpha)\\
\end{align*}

\section{More Propositions}
% \begin{figure} \label{fig:upsilon}
% \centerline{\xymatrix{
% A_0 \ar@/^5pc/[r]^{f_0} 
% \ar@/^2pc/[r]^{\big\Downarrow \phi_1}_{f_1} 
% \ar@/_2pc/[r]^{\substack{\vdots\\ f_n}}
% \ar@/_5pc/[rrr]_{id}^{\substack{\alpha \\ \\ \\ \\ }}
% & A_1 \ar@/^5pc/[r]^{g_0} 
% \ar@/^2pc/[r]^{\big\Downarrow \psi_1}_{g_1} 
% \ar@/_2pc/[r]^{\substack{\vdots\\ g_m}}
% & A_2 \ar@/^5pc/[r]^{h_0} 
% \ar@/^2pc/[r]^{\big\Downarrow \theta_1}_{h_1} 
% \ar@/_2pc/[r]^{\substack{\vdots\\ h_p}}
% & A_0
% }
% $\overset{\mathcal{B}}\longrightarrow$
% \xymatrix{
% A_2 \ar@/^5pc/[r]^{h_0} 
% \ar@/^2pc/[r]^{\big\Downarrow \theta_1}_{h_1} 
% \ar@/_2pc/[r]^{\substack{\vdots\\ h_p}}
% \ar@/_5pc/[rrr]_{id}^{\substack{\alpha \\ \\ \\ \\ }}
% & A_0 \ar@/^5pc/[r]^{f_0} 
% \ar@/^2pc/[r]^{\big\Downarrow \phi_1}_{f_1} 
% \ar@/_2pc/[r]^{\substack{\vdots\\ f_n}}
% & A_1 \ar@/^5pc/[r]^{g_0} 
% \ar@/^2pc/[r]^{\big\Downarrow \psi_1}_{g_1} 
% \ar@/_2pc/[r]^{\substack{\vdots\\ g_m}}
% & A_2
% }}
% \caption{A picture of the domain and target of $\mathcal{B}$}
% \end{figure}
%
\begin{prop} \label{prop:c4}
Let 
$$
\mathcal{B}_{A_0,A_1,A_2} = \mathcal{B}: 
C(A_0 \to A_1 \to A_2 \to A_0)
\to \hat{\tau}_2^{*2}C(A_1 \to A_2 \to A_0 \to A_1)
$$ 
be a map of comodules over 
$B(A_0 \to A_1 \to A_2 \to A_0)$ 
determined by the following maps to 
cogenerators:
\begin{equation}
\label{eq:def_sigma2}
\begin{split}
\mathcal{B}^{f_0, g_0,h_0}_{A_0,A_1,A_2}: 
  C(A_0 \to A_1 \to A_0)(h_0g_0f_0) 
&\to
\hat{\tau}_2^{*2}C(A_1 \to A_2 \to A_0 \to A_1)
  (f_0h_0g_0)\\
&\xrightarrow[cogenerators]{\textrm{project onto}}
C_{-\bullet}(A_1, _{f_0h_0g_0}{A_1}_{id})\\
\mathcal{B}_{n, m, p} (\vec{\phi} | \vec{\psi} | \vec{\theta} | \alpha) 
= & \sum_{\substack{I_1I_2 = \{1,2,\cdots,n\} \\
                          \textrm{as ordered sets}}}
  (-1)^{|\mathfrak{a}_1|(|\mathfrak{a}_1 + \mathfrak{a}_2|)}
  1 \otimes \lambda(\vec{\phi}_{I_1})\big( \lambda(\vec{\theta}) \lambda(\vec{\psi}) \lambda(\vec{\phi}_{I_2})
  \mathfrak{a}_2 \otimes a_0 \otimes \mathfrak{a}_1 \big)
\end{split}      
\end{equation}
Then, 
\begin{equation} \label{eq:prop4}
D(\mathcal{B}_{A_0,A_1,A_2}) = 
  \Upsilon_{A_2\bullet A_0, A_1} \circ
  \Upsilon_{A_0\bullet A_1, A_2} 
   - \Upsilon_{A_0, A_1\bullet A_2}.
\end{equation}
\end{prop}

\begin{proof}
We will show that Equation \ref{eq:prop4} 
holds by direct computation. Since all of 
the maps are maps of cofree comodules, we 
only need to check that $\pi_1($
Equation \ref{eq:prop4}$)$ holds where 
$\pi_1$ denotes projection of the comodule 
onto cogenerators. More explicitly, we 
want to check that
\begin{equation} \label{eq:prop4_expand}
\begin{aligned}
%hochschild cochain delta
\mathcal{B}_{n, m, p} ( \tilde{\delta}(\vec{\phi}) | \vec{\psi} | \vec{\theta} | \alpha ) \; + 
\mathcal{B}_{n, m, p} ( \vec{\phi} | \tilde{\delta}(\vec{\psi}) | \vec{\theta} | \alpha ) \; + 
\mathcal{B}_{n, m, p} ( \vec{\phi} | \vec{\psi} | \tilde{\delta}(\vec{\theta}) | \alpha ) \; + \\
%cochain b prime
\mathcal{B}_{n-1, m, p} ( b^\prime(\vec{\phi}) | \vec{\psi} | \vec{\theta} | \alpha ) \; + 
\mathcal{B}_{n, m-1, p} ( \vec{\phi} | b^\prime(\vec{\psi}) | \vec{\theta} | \alpha ) \; + 
\mathcal{B}_{n, m, p-1} ( \vec{\phi} | \vec{\psi} | b^\prime(\vec{\theta}) | \alpha ) \; + \\
%chain b
\mathcal{B}_{n, m, p} ( \vec{\phi} | \vec{\psi} | \vec{\theta} | b(\alpha) ) \; + 
b \circ \mathcal{B}_{n, m, p} ( \vec{\phi} | \vec{\psi} | \vec{\theta} | \alpha ) \; + \\
%twist before
\sum \limits_{\substack{
  I_1I_2 = \{1,\smdots,n\}\\
  J_1J_2 = \{1,\smdots,m\}\\
  \textrm{as ordered sets}}}
  \epsilon_{I_1, J_1,\{1,\smdots,p-1\}} \cdot
 \mathcal{B}_{|I_1|, |J_1|, p-1}(\vec{\phi}_{I_1} | \vec{\psi}_{J_1} | \vec{\theta}_{\{1,\cdots, p-1\}} |
     \theta_{p} \{\vec{\psi}_{J_2}\} \{\vec{\phi}_{I_2}\} \cdot \alpha ) \; + \\
%
\sum \limits_{\substack{
  I_1I_2 = \{1,\smdots,n\}\\
  \textrm{as ordered sets}}}
  \epsilon_{I_1,\{1,\smdots,m-1\},\{1,\smdots,p\}} \cdot   
 \mathcal{B}_{|I_1|, m-1, p}(\vec{\phi}_{I_1} | \vec{\psi}_{\{1,\cdots, m-1\}} | \vec{\theta} |
     \psi_{m} \{\vec{\phi}_{I_2}\}\cdot \alpha ) \; + \\
%     
\epsilon_{\{1,\smdots,n-1\},\{1,\smdots,m\},\{1,\smdots,p\}}\cdot
\mathcal{B}_{n-1, m, p}(\vec{\phi}_{\{1,\cdots, n-1\}} |\vec{\psi}_{m} | \vec{\theta} | 
     \phi_{n} \cdot \alpha) \; + \\
%twist after
\sum \limits_{\substack{
  J_1J_2 = \{1,\smdots,m\}\\
  K_1K_2 = \{1,\smdots,p\}\\
  \textrm{as ordered sets}}}
\epsilon_{\{1\},J_1,K_1} \cdot
\phi_1 \{\vec{\theta}_{K_1}\} \{\vec{\psi}_{J_1}\} \cdot
     \mathcal{B}_{n-1, |J_2|, |K_2|}
     (\vec{\phi}_{\{2,\cdots,n\}} | \vec{\psi}_{J_2} | \vec{\theta}_{K_2} | \alpha) \; + \\
\sum \limits_{\substack{
  J_1J_2 = \{1,\smdots,m\}\\
  \textrm{as ordered sets}}}
\epsilon_{\{\},J_1\{1\}} \cdot     
\theta_1 \{\vec{\psi}_{J_1}\} \cdot
     \mathcal{B}_{n, |J_2|, p-1}
     (\vec{\phi} | \vec{\psi}_{J_2} | \vec{\theta}_{\{2,\cdots,p\}} | \alpha) \; +\\
\epsilon_{\{\},\{1\},\{\}} \cdot        
\psi_1 \cdot
     \mathcal{B}_{n, m-1, p}
     (\vec{\phi} | \vec{\psi}_{\{2,\cdots,m\}} | \vec{\theta} | \alpha) \; + \\
%prop4
\upsilon_{n, p \leq * \leq m+p}(\vec{\phi} | \vec{\psi} \bullet \vec{\theta} | \alpha ) \; + \\
\sum \limits_{\substack{
  I_1I_2 = \{1,\smdots,n\}\\
  J_1J_2 = \{1,\smdots,m\}\\
  K_1K_2 = \{1,\smdots,p\}\\
  \textrm{as ordered sets}}}
\epsilon_{I_1,J_1,K_1} \cdot 
\upsilon_{|I_1| \leq * \leq |I_1| + |K_1|,|J_1|}(\vec{\theta}_{K_1} \bullet \vec{\phi}_{I_1}, \vec{\psi}_{J_1}, 
    \upsilon_{|J_2| \leq * \leq |I_2| + |J_2|,|K_2|}(\vec{\phi}_{I_2} \bullet \vec{\psi}_{J_2} | \vec{\theta}_{K_2} | \alpha )) \\
%
=0.
\end{aligned}
\end{equation}
where $\epsilon_{I_1,J_1,K_1} = 
(-1)^{(\sum \limits_{r \in I_2}|\phi_r|+1)
  ((\sum \limits_{s \in J_1}|\psi_s|+1) + 
  (\sum \limits_{t \in K_1}|\theta_t|+1)) + 
  (\sum \limits_{s \in J_2}|\psi_s|+1)
  (\sum \limits_{t \in K_1}|\theta_t|+1)}$.
In Equation \ref{eq:prop4_expand} above, 
we call the terms in rows 1-3 the 
``standard terms'' in the computation of 
$D(\mathcal{B}_{A_0,A_1,A_2})$, and the terms in rows 
4-9 the ``extra terms'' in the computation 
of $D(\mathcal{B}_{A_0,A_1,A_2})$. The terms in rows 10-11 
are $\pi_1$ of the righthand side of Equation 
\ref{eq:prop4}; we will call these the 
``10$^{th}$- and 11$^{th}$-row terms''.

We compute the sum of the standard terms. 
In Table \ref{table:t41}, the leftmost column lists 
the expressions that don't cancel in the sum 
of the standard terms, the middle column gives 
the standard term from which the expression comes, 
and the rightmost column gives the term that 
cancels the expression. Table \ref{table:t42} 
lists the remaining ninth row terms that aren't 
already listed in Table \ref{table:t41}. In 
Table \ref{table:t42}, the left column lists 
the remaining expressions that don't cancel in 
the ninth row, and the right column gives 
the extra term that cancels the expression. 

All of the terms in the tables describing 
the expansion of Equation \ref{eq:prop4_expand} 
cancel, so we're done.
\end{proof}
%
\begin{landscape}
\begin{center}
\begin{table}
  \begin{tabular}{ p{3.25in} | p{1.75in} | p{2.75in} }
    \hline
    Expression & Comes from Standard Term & Cancelling Term \\ \hline
    %
    % twist before
    $1 \otimes \lambda(\vec{\phi}_{I_1}) [
    \lambda(\vec{\theta}_{\{1,\cdots,p-1\}}
    \lambda(\vec{\psi}_{J_1})
    \lambda(\vec{\phi}_{I_2})
    \mathfrak{a}_2 \otimes 
    \theta_p(\lambda(\vec{\psi}_{J_2}) \lambda(\vec{\phi}_{I_3}) \mathfrak{a}_3) \cdot a_0 \otimes
    \mathfrak{a}_1 ]$ & 
    $b \circ \mathcal{B}_{n,m,p} (\vec{\phi} | \vec{\psi} | \vec{\theta} | \alpha)$ & 
    $\mathcal{B}_{|I_1|, |J_1|, p-1}(\vec{\phi}_{I_1} | \vec{\psi}_{J_1} | \vec{\theta}_{\{1,\cdots, p-1\}} |
     \theta_{p} \{\vec{\psi}_{J_2}\} \{\vec{\phi}_{I_2}\} \cdot \alpha )$ \\ \hline

    $1 \otimes \lambda(\vec{\phi}_{I_1}) [
    \lambda(\vec{\theta}
    \lambda(\vec{\psi}_{\{1,\cdots,m-1\}})
    \lambda(\vec{\phi}_{I_2})
    \mathfrak{a}_2 \otimes 
    \psi_m(\lambda(\vec{\phi}_{I_3}) \mathfrak{a}_3) \cdot a_0 \otimes
    \mathfrak{a}_1 ]$ & 
    $b \circ \mathcal{B}_{n,m,p} (\vec{\phi} | \vec{\psi} | \vec{\theta} | \alpha)$ & 
    $\mathcal{B}_{|I_1|, m-1, p}(\vec{\phi}_{I_1} | \vec{\psi}_{\{1,\cdots, m-1\}} | \vec{\theta} |
     \psi_m \{\vec{\phi}_{I_2}\} \cdot \alpha )$ \\ \hline

    $1 \otimes \lambda(\vec{\phi}_{I_1}) [
    \lambda(\vec{\theta}
    \lambda(\vec{\psi}
    \lambda(\vec{\phi}_{\{1,\cdots,n-1\}})
    \mathfrak{a}_2 \otimes 
    \psi_n(\mathfrak{a}_3) \cdot a_0 \otimes
    \mathfrak{a}_1 ]$ & 
    $b \circ \mathcal{B}_{n,m,p} (\vec{\phi} | \vec{\psi} | \vec{\theta} | \alpha)$ & 
    $\mathcal{B}_{n-1, m, p}(\vec{\phi}_{\{1,\cdots, n-1\}} | \vec{\psi} | \vec{\theta} |
     \phi_n \cdot \alpha )$ \\ \hline
    %
    %twist after
    $\phi_1(\lambda(\vec{\theta}_{K_1}) \lambda(\vec{\psi}_{J_1}) \lambda(\vec{\phi}_{I_2}) \mathfrak{a}_2)
    \otimes \lambda(\vec{\phi}_{I_1\backslash 1})[
    \lambda(\vec{\theta}_{K_2}) \lambda(\vec{\psi}_{J_3}) \lambda(\vec{\phi}_{I_3}) \mathfrak{a}_3
    \otimes a_0 \otimes \mathfrak{a}_1]$ &
    $b \circ \mathcal{B}_{n,m,p} (\vec{\phi} | \vec{\psi} | \vec{\theta} | \alpha)$ & 
    $\phi_1 \{\vec{\theta}_{K_1}\} \{\vec{\psi}_{J_1}\} \cdot
     \mathcal{B}_{n-1, |J_2|, |K_2|}
     (\vec{\phi}_{\{2,\cdots,n\}} | \vec{\psi}_{J_2} | \vec{\theta}_{K_2} | \alpha)$ \\ \hline

    $f_0\theta_1( \lambda(\vec{\psi}_{J_1}) \lambda(\vec{\phi}_{I_2}) \mathfrak{a}_2)
    \otimes \lambda(\vec{\phi}_{I_1})[
    \lambda(\vec{\theta}_{\{2,\cdots,p\}}) \lambda(\vec{\psi}_{J_2}) \lambda(\vec{\phi}_{I_3}) \mathfrak{a}_3
    \otimes a_0 \otimes \mathfrak{a}_1]$ &
    $b \circ \mathcal{B}_{n,m,p} (\vec{\phi} | \vec{\psi} | \vec{\theta} | \alpha)$ & 
    $\theta_1 \{\vec{\psi}_{J_1}\} \cdot
     \mathcal{B}_{n, |J_2|, p-1}
     (\vec{\phi} | \vec{\psi}_{J_2} | \vec{\theta}_{\{2,\cdots,p\}} | \alpha)$ \\ \hline

    $f_0h_0\psi_1( \lambda(\vec{\phi}_{I_2}) \mathfrak{a}_2)
    \otimes \lambda(\vec{\phi}_{I_1})[
    \lambda(\vec{\theta}) \lambda(\vec{\psi}_{\{2,\cdots,m\}}) \lambda(\vec{\phi}_{I_3}) \mathfrak{a}_3
    \otimes a_0 \otimes \mathfrak{a}_1]$ &
    $b \circ \mathcal{B}_{n,m,p} (\vec{\phi} | \vec{\psi} | \vec{\theta} | \alpha)$ & 
    $\psi_1 \cdot
     \mathcal{B}_{n, m-1, p}
     (\vec{\phi} | \vec{\psi}_{\{2,\cdots,m\}} | \vec{\theta} | \alpha)$ \\ \hline

    %wrap around
    $f_0h_0g_0 \phi_{i_1} ( \lambda(\vec{\theta}_{K_2}) \lambda(\vec{\psi}_{J_2}) \lambda(\vec{\phi}_{I_3})
       \mathfrak{a}_3 \otimes a_0 \otimes \mathfrak{a}_1) \otimes
       \lambda(\vec{\phi}_{I_1}) \lambda(\vec{\theta}_{K_1}) \lambda(\vec{\psi}_{J_1}) 
       \lambda(\vec{\phi}_{I_2 \backslash i_1}) \mathfrak{a}_2$ &
    $b \circ \mathcal{B}_{n,m,p} (\vec{\phi} | \vec{\psi} | \vec{\theta} | \alpha)$ & 
    11$^{th}$ row \\ \hline

    $f_0h_0g_0f_{i_1}a_0 \otimes \lambda(\vec{\phi}_{I_1}) \lambda(\vec{\theta}) 
       \lambda(\vec{\psi}_{J_1}) \lambda(\vec{\phi}_{I_2}) \mathfrak{a}_1$ &
    $b \circ \mathcal{B}_{n,m,p} (\vec{\phi} | \vec{\psi} | \vec{\theta} | \alpha)$ & 
    11$^{th}$ row \\ \hline

    $\phi_1( \lambda(\vec{\phi}_{I_1}) \lambda(\vec{\theta}) 
       \lambda(\vec{\psi}_{J_1}) \lambda(\vec{\phi}_{I_2}) \mathfrak{a}_3, a_0, \mathfrak{a}_1 ) \otimes
       \lambda(\vec{\phi}_{I_1 \backslash 1}) \mathfrak{a}_2$ &
    $b \circ \mathcal{B}_{n,m,p} (\vec{\phi} | \vec{\psi} | \vec{\theta} | \alpha)$ & 
    10$^{th}$ row \\ \hline

  \end{tabular}
\caption{Expansion of $``$standard terms'' in 
Equation \ref{eq:prop4_expand} and the 
terms that cancel them}
\label{table:t41}
\end{table}  
\end{center}
%
%
\begin{center}
\begin{table}
  \begin{tabular}{ p{6.25in} | p{2.5in} }
    \hline
    Expression from 11$^{th}$ Row & Cancels with Extra Term \\ \hline
    %twist after
    $\phi_1(\lambda(\vec{\theta}_{K_1}) \lambda(\vec{\psi}_{J_1}) \lambda(\vec{\phi}_{I_2}) [
      \lambda(\vec{\theta}_{K_3}) \lambda(\vec{\psi}_{J_4}) \lambda(\vec{\phi}_{I_5})
      \mathfrak{a}_3, a_0, \mathfrak{a}_1])
      \otimes \lambda(\vec{\phi}_{I_1\backslash 1}) \lambda(\vec{\theta}_{K_2}) 
      \lambda(\vec{\psi}_{J_3}) \lambda(\vec{\phi}_{I_4}) \mathfrak{a}_2$ &
    $\phi_1 \{\vec{\theta}_{K_1}\} \{\vec{\psi}_{J_1}\} \cdot
     \mathcal{B}_{n-1, |J_2|, |K_2|}
     (\vec{\phi}_{\{2,\cdots,n\}} | \vec{\psi}_{J_2} | \vec{\theta}_{K_2} | \alpha)$ \\ \hline

    $f_0\theta_1( \lambda(\vec{\psi}_{J_1}) \lambda(\vec{\phi}_{I_2}) [
      \lambda(\vec{\theta}_{K_2}) \lambda(\vec{\psi}_{J_3}) \lambda(\vec{\phi}_{I_4})
      \mathfrak{a}_3, a_0, \mathfrak{a}_1])
      \otimes \lambda(\vec{\phi}_{I_1}) \lambda(\vec{\theta}_{K_1 \backslash 1}) 
      \lambda(\vec{\psi}_{J_2}) \lambda(\vec{\phi}_{I_3}) \mathfrak{a}_2$ &
    $\theta_1 \{\vec{\psi}_{J_1}\} \cdot
     \mathcal{B}_{n, |J_2|, p-1}
     (\vec{\phi} | \vec{\psi}_{J_2} | \vec{\theta}_{\{2,\cdots,p\}} | \alpha)$ \\ \hline

    $f_0h_0\psi_1( \lambda(\vec{\phi}_{I_2}) [
      \lambda(\vec{\theta}_{K_2}) \lambda(\vec{\psi}_{J_2}) \lambda(\vec{\phi}_{I_4})
      \mathfrak{a}_3, a_0, \mathfrak{a}_1])
      \otimes \lambda(\vec{\phi}_{I_1}) \lambda(\vec{\theta}_{K_1}) 
      \lambda(\vec{\psi}_{J_1 \backslash 1}) \lambda(\vec{\phi}_{I_3}) \mathfrak{a}_2$ & 
    $\psi_1 \cdot
     \mathcal{B}_{n, m-1, p}
     (\vec{\phi} | \vec{\psi}_{\{2,\cdots,m\}} | \vec{\theta} | \alpha)$ \\ \hline

    \hline
  \end{tabular}
\caption{Expansion of remaining $``$11$^{th}$ row terms'' in 
Equation \ref{eq:prop4_expand} and the $``$extra terms''
that cancel them}
\label{table:t42}
\end{table}  
\end{center}
\end{landscape}
% \begin{figure} \label{fig:upsilon}
% \centerline{\xymatrix{
% A_0 \ar@/^5pc/[r]^{f_0} 
% \ar@/^2pc/[r]^{\big\Downarrow \phi_1}_{f_1} 
% \ar@/_2pc/[r]^{\substack{\vdots\\ f_n}}
% \ar@/_5pc/[rrr]_{id}^{\substack{\alpha \\ \\ \\ \\ }}
% & A_1 \ar@/^5pc/[r]^{g_0} 
% \ar@/^2pc/[r]^{\big\Downarrow \psi_1}_{g_1} 
% \ar@/_2pc/[r]^{\substack{\vdots\\ g_m}}
% & A_2 \ar@/^5pc/[r]^{h_0} 
% \ar@/^2pc/[r]^{\big\Downarrow \theta_1}_{h_1} 
% \ar@/_2pc/[r]^{\substack{\vdots\\ h_p}}
% & A_0
% }
% $\overset{[d_0^* \Upsilon, \mathcal{B}]}\longrightarrow$
% \xymatrix{
% A_0 \ar@/^5pc/[r]^{f_0} 
% \ar@/^2pc/[r]^{\big\Downarrow \phi_1}_{f_1} 
% \ar@/_2pc/[r]^{\substack{\vdots\\ f_n}}
% \ar@/_5pc/[rrr]_{id}^{\substack{\alpha \\ \\ \\ \\ }}
% & A_1 \ar@/^5pc/[r]^{g_0} 
% \ar@/^2pc/[r]^{\big\Downarrow \psi_1}_{g_1} 
% \ar@/_2pc/[r]^{\substack{\vdots\\ g_m}}
% & A_2 \ar@/^5pc/[r]^{h_0} 
% \ar@/^2pc/[r]^{\big\Downarrow \theta_1}_{h_1} 
% \ar@/_2pc/[r]^{\substack{\vdots\\ h_p}}
% & A_0
% }}
% \caption{A picture of the domain and target of $[d_0^* \Upsilon, \mathcal{B}]$}
% \end{figure}

\begin{prop}
\label{prop:c5}
Let $\Upsilon$ and $\mathcal{B}$ be as 
defined in the previous propositions. 
Then, $[\Upsilon, \mathcal{B}] := 
\Upsilon_{A_1\bullet A_2, A_0} 
\mathcal{B}_{A_0,A_1,A_2} - 
\mathcal{B}_{A_2, A_0, A_1} 
\Upsilon_{A_0 \bullet A_1, A_2} = 0$. 
(Note that $[\Upsilon, \mathcal{B}]$ is 
a map from $C(A_0 \to A_1 \to A_2 \to A_0)$ 
to itself.)
\end{prop}
%
\begin{proof}
We show the proposition by direct computation. 
Since all of the maps are maps of cofree 
comodules, we only need to check that 
$\pi_1([\Upsilon, \mathcal{B}]) = 0$ where 
$\pi_1$ denotes projection of the comodule 
onto cogenerators. We check this directly.
%
\begin{equation*}
\begin{aligned}
&\phantom{{}={}}
\pi_1 \circ \Upsilon_{A_1\bullet A_2, A_0} 
  \mathcal{B}_{A_0,A_1,A_2} 
  (\vec{\phi} | \vec{\psi} | \vec{\theta} | \alpha ) \\
&= 
\sum \limits_{\substack{
  I_1I_2 = \{1,\smdots,n\}\\
  J_1J_2 = \{1,\smdots,m\}\\
  K_1K_2 = \{1,\smdots,p\}\\
  \textrm{as ordered sets}}}
\epsilon_{I_2,J_1,J_2,K_1}\cdot
\upsilon_{|K_1| \leq * \leq |K_1|+|J_1|, |I_1|} (
   \vec{\psi}_{J_1} \bullet \vec{\theta}_{K_1} | 
   \vec{\phi}_{I_1} | 
   \mathcal{B}_{|I_2|, |J_2|, |K_2|} (
   \vec{\phi}_{I_2} | \vec{\psi}_{J_2} | \vec{\theta}_{K_2} | \alpha)) \\
&= 
\sum \limits_{\substack{
  I_1I_2 = \{1,\smdots,n\}\\
  J_1J_2 = \{1,\smdots,m\}\\
  K_1K_2 = \{1,\smdots,p\}\\
  \textrm{as ordered sets}}}
\begin{array}{l}  
\epsilon_{I_2,J_1,J_2,K_1}\cdot
\eta_{\mathfrak{a}_1, \mathfrak{a}_2} \cdot\\
\upsilon_{|K_1| \leq * \leq |K_1|+|J_1|, |I_1|} (
   \vec{\psi}_{J_1} \bullet \vec{\theta}_{K_1} | 
   \vec{\phi}_{I_1} |
   1 \otimes \lambda(\vec{\phi}_{I_2})[
     \lambda(\vec{\theta}_{K_2}) \lambda(\vec{\psi}_{J_2}) 
     \lambda(\vec{\phi}_{I_3}) 
     \mathfrak{a}_2, a_0, \mathfrak{a}_1] 
\end{array} \\
&= 
\sum \limits_{\substack{
  I_1I_2 = \{1,\smdots,n\}\\
  J_1J_2 = \{1,\smdots,m\}\\
  K_1K_2 = \{1,\smdots,p\}\\
  \textrm{as ordered sets}}}
\epsilon_{I_2,J_1,J_2,K_1}\cdot
\eta_{\mathfrak{a}_1, \mathfrak{a}_2} \cdot
1 \otimes \lambda(\vec{\theta}_{K_1}) \lambda(\vec{\psi}_{J_1}) 
  \lambda(\vec{\phi}_{I_1})[
     \lambda(\vec{\theta}_{K_2}) \lambda(\vec{\psi}_{J_2}) 
     \lambda(\vec{\phi}_{I_2})
     \mathfrak{a}_2, a_0, \mathfrak{a}_1]    
\end{aligned}
\end{equation*}
%
\begin{align*}
& \phantom{{}={}}
\pi_1 \circ \mathcal{B}_{A_2, A_0, A_1} 
  \Upsilon_{A_0 \bullet A_1, A_2} 
  (\vec{\phi} | \vec{\psi} | \vec{\theta} | \alpha ) \\
&= 
\sum \limits_{\substack{
  I_1I_2 = \{1,\smdots,n\}\\
  J_1J_2 = \{1,\smdots,m\}\\
  K_1K_2 = \{1,\smdots,p\}\\
  \textrm{as ordered sets}}}
\epsilon_{I_2,J_1,J_2,K_1}\cdot
B_{|K_1|, |I_1|, |J_1|} 
   (\vec{\theta}_{K_1} | \vec{\phi}_{I_1} | \vec{\psi}_{J_1} | 
   \upsilon_{|J_2| \leq * \leq |I_2|+|J_2|, |K_2|} 
   (\vec{\phi}_{I_2} \bullet \vec{\psi}_{J_2} | \vec{\theta}_{K_2} | \alpha)) \\
&= 
\sum \limits_{\substack{
  I_1I_2 = \{1,\smdots,n\}\\
  J_1J_2 = \{1,\smdots,m\}\\
  K_1K_2 = \{1,\smdots,p\}\\
  \textrm{as ordered sets}}}
\epsilon_{I_2,J_1,J_2,K_1}\cdot
\eta_{\mathfrak{a}_1, \mathfrak{a}_2} \cdot
1 \otimes \lambda(\vec{\theta}_{K_1}) \lambda(\vec{\psi}_{J_1}) 
  \lambda(\vec{\phi}_{I_1})[
     \lambda(\vec{\theta}_{K_2}) \lambda(\vec{\psi}_{J_2}) 
     \lambda(\vec{\phi}_{I_2})
     \mathfrak{a}_2, a_0, \mathfrak{a}_1]  
\end{align*}
It's clear that $\pi_1([\Upsilon, \mathcal{B}]) = 0$.
\end{proof}