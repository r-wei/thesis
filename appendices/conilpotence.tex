\section{Conilpotence} \label{sec:conilpotence}
%
In this section, we show that the 
dg categories and dg comodules we have been 
working with are conilpotent. For completeness, 
we start with the definition of a dg 
cocategory.
%
\begin{defn} A dg cocategory is a 
cocategory enriched over chain complexes. 
More explicitly, a dg cocategory $B$ consists 
of the following data:
\begin{itemize}
\item A collection of objects 
  denoted $Obj(B)$;
\item For each pair of objects, $x,z \in 
  Obj(B)$, a complex $B^\bullet(x,z)$ and 
  a morphism of complexes
  $$
  \Delta_B(x,z): B^\bullet(x,z) \to 
  \prod \limits_{y \in Obj(B)}
  B^\bullet(x,y) \otimes
  B^\bullet(y,z)
  $$
  such that the following diagrams 
  commute (coassociativity):
  $$
  \xymatrixcolsep{5pc}
  \xymatrixrowsep{5pc}
  \xymatrix{
  B^\bullet(x,z)
  \ar[r]^{\Delta_B(x,z)}
  \ar[d]_{\Delta_B(x,z)}
  & \prod \limits_{y \in Obj(B)}
    B^\bullet(x,y) \otimes
    B^\bullet(y,z)
  \ar[d]^{\prod\limits_y 
    id_{B(x,y)} \otimes \Delta_B(y,z)}\\
  \prod \limits_{y \in Obj(B)}
    B^\bullet(x,y) \otimes
    B^\bullet(y,z)
  \ar[r]^{\prod\limits_y 
    \Delta_B(x,y) \otimes id_{B(y,z)}}
  & \prod \limits_{y, y^\prime \in Obj(B)}
    B^\bullet(x,y) \otimes
    B^\bullet(y,y^\prime) \otimes
    B^\bullet(y^\prime,z)  
  }
  $$
\item For each pair of objects, $x,z \in
  Obj(B)$, a morphism of complexes
  $$
  \epsilon_B(x,z): B^\bullet(x,z) \to k
  $$
  where $k$ is the ground field considered 
  as a chain complex concentrated in degree 
  0 and $\epsilon_B(x,z) = 0$ if $x\neq z$, 
  such that the following diagrams commute 
  (counitality):
  $$
  \xymatrixcolsep{5pc}
  \xymatrixrowsep{5pc}
  \xymatrix{
  B^\bullet(x,z)
  \ar[r]^{\Delta_B(x,z)}
  \ar[d]_{\Delta_B(x,z)}
  \ar[rd]_{id}
  & \prod \limits_{y \in Obj(B)} 
    B^\bullet(x,y) \otimes
    B^\bullet(y,z)
  \ar[d]^{\prod \limits_y
    \epsilon_B(x,y)
    \otimes id_{B(y,z)}} \\
  %
  \prod \limits_{y \in Obj(B)} 
    B^\bullet(x,y) \otimes
    B^\bullet(y,z)
  \ar[r]_{\prod \limits_y
    id_{B(x,y)} \otimes 
    \epsilon_B(y,z)}
  & B^\bullet(x,z).  
  }
  $$
\end{itemize}
\end{defn}
We will denote a dg cocategory with its 
cocomposition and counit as $(B, \Delta_B, 
\epsilon_B)$. To make the notation more 
readable, when the meaning is clear, 
we will omit references to the 
objects and write $\Delta_B$ instead of 
$\Delta_B(x,z)$, $\epsilon_B$ instead of 
$\epsilon_B(x,z)$, and for the differentials 
on morphisms, $d_B$ instead of $d_B(x,z)$.
%
\begin{defn} A (dg) functor $F: A \to B$ 
between two dg cocategories is a functor 
between the cocategories satisfying 
$d_B\circ F(f) = F\circ d_A(f)$ for all 
morphisms $f$ in $A$.
\end{defn}
%
\begin{defn} A conilpotent dg 
cocategory is a dg cocategory $(B, 
\Delta_B, \epsilon_B)$ satisfying: for each 
morphism $f:x\to y$ in $B$, there exists 
$n_f \in \mathbb{N}$ such that 
$\bar{\Delta}_B^{n_f}(f) = 0$
where
\begin{align*}
\bar{\Delta}_B(x,z): B^\bullet(x,z) 
&\to 
  \prod \limits_{y \in Obj(B)} 
  B^\bullet(x,y) \otimes
  B^\bullet(y,z)\\
f
&\mapsto
\Delta_B(f)
- \sum \limits_{e_x \in 
  \epsilon_B(x,x)^{-1}(1)}
  e_x \otimes f
- \sum \limits_{e_z \in 
  \epsilon_B(z,z)^{-1}(1)}
  f \otimes e_z. 
\end{align*}
\end{defn}
%
The following fact follows from the 
definitions: 
If $B$ is a conilpotent dg cocategory, 
then for all $x \in Obj(B)$, 
$\epsilon_B(x,x)^{-1}(1)$ has exactly 
one element, which we will denote $e_x$.
%
\begin{eg} Let $\mathcal{C}$ 
be the category in dg cocategories defined in 
Equation \ref{eq:cat_in_dgcocat} and 
$A_0, \dots, A_n$ be algebras. Then, 
$\mathcal{C}(A_0, A_1) \otimes \dots \otimes 
\mathcal{C}(A_n,A_0)$ is conilpotent:
$$
\bar{\Delta}^{min(k_0,\smdots,k_n)}
(\phi_{0,1}\smdots\phi_{0,k_0}|\smdots|
\phi_{n,1}\smdots\phi_{n,k_n}) = 0.
$$
\end{eg}
%

Now, we will discuss conilpotence of the 
dg comodules. Recall the definition of a 
dg comodule in Definition \ref{def:dg_comod}.
%
\begin{defn} A conilpotent dg 
comodule over a dg cocategory $B$ 
is a dg comodule $(C, \Delta_C)$ 
over $B$ satisfying: for each 
$f \in Obj(B)$ and each element 
$\alpha \in C^\bullet(f)$, there exists 
$n_\alpha \in \mathbb{N}$ such that 
$\bar{\Delta}_f^{n_\alpha}(\alpha) = 0$
where
\begin{align*}
\bar{\Delta}_C(f): C^\bullet(f) 
&\to 
  \prod \limits_{g \in Obj(B)} 
  B^\bullet(f,g) \otimes
  C^\bullet(g)\\
\alpha
&\mapsto
\Delta_B(\alpha) 
  - \sum \limits_{e_f \in 
  \epsilon_B(f,f)^{-1}(1)}
  e_f \otimes f.
\end{align*}
\end{defn}
%
\begin{eg} 
Since all of the dg comodules we use 
are cofree, their comodule structure maps 
are induced by the cocompositions of the 
dg cocategories. Any cofree dg comodule 
over a conilpotent dg cocategory is 
conilpotent.
\end{eg}