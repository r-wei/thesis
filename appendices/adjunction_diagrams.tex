\begin{landscape}
\begin{figure}
\centerline{\xymatrixrowsep{5pc}
\xymatrix{
\bigoplus \limits_{f^\prime \in \lambda^{-1}f}
  C^\bullet(f^\prime)
\ar@[red][d]^{\bigoplus \limits_{f^\prime} F_{f^\prime}}\\
%
\bigoplus \limits_{\substack{
  f^\prime \in \lambda^{-1}f\\
  h^\prime \in Obj(B_1)}}
  B_1^\bullet(f^\prime, h^\prime)
  \otimes D^\bullet(\lambda h^\prime)
\ar@[red][r]^{\bigoplus \limits_{f^\prime, h^\prime} 
  \lambda \otimes id_{D}}  
\ar@/_/[d]_{\bigoplus \limits_{f^\prime, h^\prime, r^\prime}
  \substack{
  (\Delta_{B_1} \otimes id_D) \circ \\
  (id_{B_1} \otimes \lambda \otimes id_D)}}
\ar@/^/[d]^{\bigoplus \limits_{f^\prime, h^\prime} 
  id_{B_1} \otimes \Delta_{D}}
& \bigoplus \limits_{h \in Obj(B_0)}
  B_0^\bullet(f,h) \otimes D^\bullet(h)
\ar@[red][r]^{\bigoplus \limits_h 
  \epsilon_{B_0} \otimes id_D} 
& D^\bullet(f)
\ar@[red][d]^{\Delta_D} \\
%
\bigoplus \limits{\substack{
  f^\prime \in \lambda^{-1}f,\\
  h^\prime \in Obj(B_1),\\
  r \in Obj(B_0)}}
  B_0^\bullet(f^\prime, h^\prime) \otimes 
  B_1^\bullet(\lambda h^\prime, r) 
  \otimes D^\bullet(r)
\ar[rr]_{\bigoplus \limits_{f^\prime, h^\prime, r}
  \epsilon_{B_0} \lambda \otimes id_{B_1} 
  \otimes id_D}
& & \bigoplus \limits_{h \in Obj(B_0)}
  B_0^\bullet(f,h) \otimes D^\bullet(h) 
}}
\caption{Commuting diagram 
involving $\Delta_D \circ 
\Phi^{-1}F$} \label{fig:delta_phi-1}
$\Delta_D \circ 
\Phi^{-1}F$ = composition 
of red arrows. 
The fact that $F: C \to \lambda^*D$ and 
the universal property of $\lambda^*D$ 
imply that the diagram commutes.  
\end{figure}
%
\begin{figure}
\centerline{
\resizebox{3.5\vsize}{!}{\xymatrixrowsep{5pc} \xymatrixcolsep{1pc}
\xymatrix{
\bigoplus \limits_{f^\prime \in \lambda^{-1}f}
  C^\bullet(f^\prime)
\ar[d]_{\bigoplus \limits_{f^\prime} F_{f^\prime}}
\ar@[red][r]^{\bigoplus \limits_{f^\prime} \Delta_C}
& \bigoplus \limits_{\substack{
  f^\prime \in \lambda^{-1}f\\
  h^\prime \in Obj(B_1)}}
  B_1^\bullet(f^\prime, h^\prime)
  \otimes C^\bullet(h^\prime)
\ar@[red][r]^{\bigoplus \limits_{f^\prime, h^\prime} 
  \lambda \otimes id_{C}}  
\ar[ddl]^{\bigoplus \limits_{f^\prime, h^\prime}
  id_{B_1} \otimes 
  F_{\lambda h^\prime}|_{h^\prime}}
& \bigoplus \limits_{h^\prime \in Obj(B_1)}
  B_0^\bullet(f,\lambda h^\prime) 
  \otimes C^\bullet(h^\prime)
\ar@[red][d]^{\bigoplus \limits_{h^\prime}
  id_{B_0} \otimes 
  F_{\lambda h^\prime}|_{h^\prime}}\\
%  
\bigoplus \limits{\substack{
  f^\prime \in \lambda^{-1}f,\\
  r^\prime \in Obj(B_1)}}
  B_1^\bullet(f^\prime, r^\prime) 
  \otimes D^\bullet(\lambda r^\prime)
\ar[d]_{\substack{\Delta_{\lambda^*D} =\\
  \bigoplus \limits_{f^\prime, r^\prime}
  \Delta_{B_1} \otimes id_D}}
& 
& \bigoplus \limits_{h^\prime, r^\prime 
  \in Obj(B_1)}
  B_0^\bullet(f,\lambda h^\prime) \otimes
  B_1^\bullet(h^\prime, r^\prime)
  \otimes D^\bullet(\lambda r^\prime)  
\ar@[red][d]^{\bigoplus \limits_{h^\prime, r^\prime}
  id_{B_0} \otimes 
  \epsilon_{B_0} \lambda \otimes id_{D}}\\
%
 \bigoplus \limits{\substack{
  f^\prime \in \lambda^{-1}f,\\
  h^\prime, r^\prime \in Obj(B_1)}}
  B_1^\bullet(f^\prime, h^\prime) \otimes
  B_1^\bullet(h^\prime, r^\prime) 
  \otimes D^\bullet(\lambda r^\prime)
\ar[rru]_{\bigoplus 
  \limits_{f^\prime, h^\prime, r^\prime}
  \lambda \otimes id_{B_0} \otimes id_D}
& & \bigoplus \limits_{h\in Obj(B_0)}
  B_0^\bullet(f,h) \otimes 
  D^\bullet(h)
}}}
\caption{Commuting diagram
involving $(id_{B_1}\otimes 
\Phi^{-1}F) \circ \Delta_{\lambda_\# C}$} \label{fig:phi-1_delta}
$(id_{B_1}\otimes 
\Phi^{-1}F) \circ \Delta_{\lambda_\# C}$
= composition of red arrows. 
The fact that $F$ respects coproducts  
implies that the left square commutes. 
\end{figure}
%
\begin{figure}
\centerline{\xymatrixrowsep{5pc}
\xymatrix{
C^\bullet(f^\prime)
\ar@[red][r]^{\Delta_C}
\ar[d]^{F_{f^\prime}}
& \bigoplus \limits_{g^\prime \in Obj(B_1)}
  B_1^\bullet(f^\prime, g^\prime)
  \otimes C^\bullet(g^\prime)
\ar@[red][d]^{\bigoplus \limits_{g^\prime}
  id_{B_1}\otimes F_{g^\prime}}\\
%
\bigoplus \limits_{h^\prime \in Obj(B_1)}  
  B_1^\bullet(f^\prime, h^\prime)
  \otimes D^\bullet(\lambda h^\prime)
\ar[r]^{\Delta_{\lambda^* D} = \bigoplus
  \limits_{h^\prime} \Delta_{B_1} \otimes id_D}
\ar@/_3pc/[rr]^{id}  
& \bigoplus \limits_{g^\prime, h^\prime \in Obj(B_1)}
  B_1^\bullet(f^\prime, g^\prime)
  \otimes B_1^\bullet(g^\prime, h^\prime)
  \otimes D^\bullet(\lambda h^\prime)
\ar@[red][r]^{\bigoplus \limits_{g^\prime, h^\prime}
  id_{B_1} \otimes 
  (\epsilon_{B_0}\lambda = \epsilon_{B_1})
  \otimes id_D}
& \bigoplus \limits_{g^\prime \in Obj(B_1)}
  B_1^\bullet(f^\prime, g^\prime)
  \otimes D^\bullet(\lambda g^\prime)
}}
\caption{Commuting diagram 
involving $\Phi\Phi^{-1}F_{f^\prime}$}  \label{fig:phi_phi-1}
$\Phi\Phi^{-1}F_{f^\prime}$
= composition of red arrows. 
The square commutes because $F$ 
respects coproducts; 
the composition of the bottom row 
of horizontal arrows is equal to 
the identity because $\lambda_\#D$ 
satisfies counitality.
\end{figure}
%
\begin{figure}
\centerline{
\xymatrixrowsep{5pc}
\xymatrixcolsep{5pc}
\xymatrix{
\bigoplus \limits_{f^\prime \in \lambda^{-1}f}
  C^\bullet(f^\prime)
\ar@[red][r]^{\bigoplus \limits_{f^\prime}
  \Delta_C}
\ar[dd]_{F_f}
& \bigoplus \limits_{\substack{
  f^\prime \in \lambda^{-1}f,\\
  g^\prime \in Obj(B_1)}}
  B_1^\bullet(f^\prime, g^\prime)
  \otimes C^\bullet(g^\prime)
\ar@[red][r]^{\bigoplus \limits_{
  f^\prime, g^\prime} 
  id_{B_1} \otimes 
  F_{\lambda g^\prime}|_{g^\prime}}
\ar[d]^{\bigoplus \limits_{
  f^\prime, g^\prime} 
  \lambda \otimes id_C}
& \bigoplus \limits_{\substack{
  f^\prime \in \lambda^{-1}f,\\
  g^\prime \in Obj(B_1)}} 
  B_1^\bullet(f^\prime, g^\prime)
  \otimes D\bullet(\lambda g^\prime)
\ar@[red][d]^{\bigoplus \limits_{
  f^\prime, g^\prime} 
  \lambda \otimes id_D}\\
%
& \bigoplus \limits_{g^\prime \in Obj(B_1)}
  B_0^\bullet(f, \lambda g^\prime)
  \otimes C^\bullet(g^\prime)
\ar[r]^{\bigoplus \limits_{g^\prime}
  id_{B_0} \otimes 
  F_{\lambda g^\prime}|_{g^\prime}}
& \bigoplus \limits_{g \in Obj(B_0)}
  B_0^\bullet(f, g)
  \otimes D^\bullet(g)
\ar@[red][r]^{\bigoplus \limits_g
  \epsilon_{B_0} \otimes id_D}
& D^\bullet(f)\\
%
D^\bullet(f)
\ar[rru]_{\Delta_D}  
\ar@/_3pc/[rrru]_{id}
}}
\caption{Commuting diagram 
involving $\Phi^{-1}\Phi F_f$}  \label{fig:phi-1_phi}
$\Phi^{-1}\Phi F_f$ = composition of red arrows. 
The concave pentagon on the left 
side commutes because $F$ 
respects coproducts; the triangle in the 
bottom right corner commutes 
because $D$ satisfies counitality.
\end{figure}
\end{landscape}