\documentclass[12pt]{nuthesis} %use [draft] to check on margins

\usepackage{bm} 	%bold in any mode
\usepackage{amscd}        % For writing commutative diagrams.
\usepackage{eucal}        % Euler fonts.
\usepackage{verbatim}     % Quotations
\usepackage{amsthm}	  % newtheorems
\usepackage{lscape} % landscape
\usepackage{amssymb} % arrows
\usepackage{mathtools} %extensible arrows; underbrace
\usepackage[all]{xy} %diagrams
\usepackage{color} %colored arrows
\usepackage{float} %figures appear here [H]


%%%%%%%%%%%%%%%%%%%%%%% Some Math support %%%%%%%%%%%%%%%%%%%%

%% \theoremstyle{plain} %% This is the default
\newtheorem{thm}{Theorem}[chapter]
\newtheorem{cor}[thm]{Corollary}
\newtheorem{lem}[thm]{Lemma}
\newtheorem{prop}[thm]{Proposition}
\newtheorem{ax}{Axiom}
\theoremstyle{definition}
\newtheorem{defn}{Definition}[section]
\theoremstyle{remark}
\newtheorem{rem}{Remark}[section]
\newtheorem*{notation}{Notation}
\theoremstyle{example}
\newtheorem{eg}{Example}[section]
\newcommand{\breakcell}[2][l]{% break cell in tabular
  \begin{tabular}[#1]{@{}l@{}}#2\end{tabular}}
\newcommand \smdots{\makebox[1em][c]{.\hfil.\hfil.}} %small \dots
\newcommand{\leftmapsto}{\leftarrow\!\shortmid} %\left mapsto

\newenvironment{paren_enclose}
  {\left( \left.\begin{aligned}\end{aligned}\right\r)}
%%%%%%%%%%%%%%%%%%%%%%%%%%%%%%%%%%%
% DATA OF AUTHOR AND DISSERTATION %
%%%%%%%%%%%%%%%%%%%%%%%%%%%%%%%%%%%

\author{Ann Rebecca Wei}

\title{What Do Algebras Form?}

%\degree{DOCTOR OF PHILOSOPHY}  % Default: DOCTOR OF PHILOSOPHY

%\field{Mathematics}            % Default: Mathematics

\graduationmonth{March}         % The default is June or December
                                % depending on current date.

%\graduationyear{2003}          % Default: current year.


% Use \includeonly to select the 
\includeonly{intro/abstract, intro/intro, 
cat_in_dgcocat/motivation, cat_in_dgcocat/define_bar_hoch,
cat_in_dgcocat/composition, 
2cat_trace/motivation, 2cat_trace/2cat_trace_example, 
2cat_trace/generalize_trace, 
trace_to_dg_functor/motivation, 
bibliography, 
appendices/lambda, appendices/computations, 
appendices/hochschild}	

\begin{document}
%%%%%%%%%%%%%%%%%%%%%%
% Some initial stuff %
%%%%%%%%%%%%%%%%%%%%%%

\frontmatter		% Preliminary pages start here.
\maketitle		% Produces the title page.
%\copyrightpage		% Creates the copyright page.

\abstract		% Abstract.
In 2008, Dolgushev, Tamarkin and Tsygan showed that, for $M$ a smooth real manifold, the pair $(\textrm{Hochschild cochains }C^\bullet(C^\infty(M)), \textrm{ Hochschild chains }C_\bullet(C^\infty(M)))$ is quasi-isomorphic as an $\infty$-calculus algebra to $(\Lambda^\bullet(T_M), \Omega^\bullet(M))$. In this thesis, we use their $\infty$-calculus structure on Hochschild cochains and chains to construct a $``$homotopically sheafy-cyclic object in dg categories with a dg module'', i.e., an $A_\infty$-functor $\chi_\infty \to \mathcal{E}_\infty$. To give a sheafy-cyclic object is to give a dg functor out of $\chi_\infty$, a category fibred over Connes' cyclic category with fibre category $\chi_{\infty [n]} =$ the discrete category on the set of objects $\{\textrm{diagrams }A_0 \to \dots \to A_n \to A_0: A_i \textrm{ is an algebra} \}$. $\mathcal{E}_\infty$ is the dg category of objects small dg categories with a dg module, and morphisms given by a dg functor between dg categories with a (not necessarily graded) linear map from the source module to a pullback along the functor of the target module. Interestingly, in the construction of our $A_\infty$-functor, the formulas we give for cyclically rotating a dg module resemble the Lie derivative of Hochschild cochains on chains, and the homotopy between the image of $\tau_n^{n+1}$ and the image of identity resembles the $B$ operator, the analogue of the deRham differential on Hochschild chains.

\acknowledgements	% Acknowledgements (optional).
Text for acknowledgments.

%\preface		% Preface (optional).

%\listofabbreviations 
% This is the list of abbreviations (optional).

%\glossary
%This is the glossary (optional).

\nomenclature %optional
\begin{align*}
k 
&-
\textrm{a fixed ground field of char 0}\\
%
1
&-
\textrm{the unit in (a vector space 
isomorphic to) $k$}\\
%
[1]
&-
\textrm{shift operator on complexes, }
C^\bullet[1] = C^{\bullet+1}\\
\Lambda
&-
\textrm{Connes cyclic category, 
see Appendix \ref{chap:lambda}}\\
%
\Delta(b) = \sum_{(b)} b_{(1)} \otimes b_{(2)}
&-
\textrm{Sweedler notation for coproducts}
\end{align*}

%
%% Note that the dedication text must be passed as an argument
%% of the \dedication command
%\dedication{This is the dedication (optional).}
%
\clearpage\phantomsection % needed for the hyperlinks to work correctly
%\tableofcontents	% Table of Contents will be automatically
			% generated and placed here.

%list of tables and figures are not optional...
% \clearpage\phantomsection % needed for the hyperlinks to work correctly
% \listoftables		% List of Tables and List of Figures will be placed

% \clearpage\phantomsection % needed for the hyperlinks to work correctly
% \listoffigures		% here, if applicable (optional).



\mainmatter             % Actual text starts here.

% If there is an introduction it must be the first chapter
\chapter{Introduction}
	\documentclass[12pt]{article} %use [draft] to check on margins
\begin{document}

This dissertation builds on work in noncommutative differential geometry, and we start with a brief history of the subject. In 1962, Hochschild, Kostant and Rosenberg gave the first hint that, to build a notion of differential geometry for noncommutative algebras, one should start by studying Hochschild homology and cohomology. Their famous HKR theorem states that, for a regular, commutative algebra over a field of characteristic zero, (1) Hochschild homology is quasi-isomorphic to deRham forms on the algebra (with zero differential), and (2) Hochschild cohomology is quasi-isomorphic to poly-vector fields on the algebra (also with zero differential) (Reference \cite{HKR}).

However, when the algebra is an algebra of smooth functions on a real manifold, we have much more differential-geometric structure. For example, (shifted) poly-vector fields form a dgla with the Schouten-Nijenhuis bracket, an extension of the Lie bracket on vector fields to poly-vector fields; and in 1997, Kontsevich proved that Hochschild cohomology with its known Gerstenhaber bracket is quasi-isomorphic to poly-vector fields, not as dglas, but as $L_\infty$ algebras (Reference \cite{K}). Moreover, the associative wedge product satisfies a Leibniz relation with the Schouten-Nijenhuis bracket, making poly-vector fields a Gerstenhaber algebra. In 1998, Tamarkin gave, for each Drinfeld associator, a quasi-isomorphism between Hochschild cochains and poly-vector fields as Gerstenhaber$_\infty$ algebras (Reference \cite{Tam}, Theorem 2.1).

But what about differential forms? Poly-vector fields act on forms in two ways: (1) via contraction as a dga, and (2) via the Lie derivative as a dgla. Together, these two actions satisfy Cartan's formula. We can axiomatize all of this structure (a Gerstenhaber algebra and the two actions on a module satsifying relations) into the notion of an algebra over the Calc operad. In 2008, Dolgushev, Tamarkin and Tsygan proved that the pair (Hochschild cochains, Hochschild chains) are quasi-isomorphic to (poly-vector fields, deRham forms) as Calc$_\infty$ algebras (Reference \cite{DTT}, Corollary 4).

Loosely given, the motivating question of this thesis is: Can we use the Calc$_\infty$ structure on Hochschild cochains and chains to create a cyclic object? First, we repackage and generalize the $L_\infty$ structure on Hochschild cochains to construct dg cocategories, $B(n)$ (Section ??). Then, we show that these dg cocategories form, not a cyclic, but a sheafy-cyclic object (Section ??). We introduce dg comodules $C(n)$ over $B(n)$, which are categorified versions of the bar construction of Hochschild chains as a module over Hochschild cochains via contraction (Section ??). Given this background, we ask the motivating question in a more technical way: Can we extend the sheafy-cyclic structure on dg cocategories to include the dg comodules?

In Sections ??, we establish the theoretical background needed to work with the dg comodules. Then, we give a reasonable candidate for a sheafy-cyclic structure on dg comodules (Section ??). Here, we find, surprisingly, that the formulas for cyclically rotating a dg comodule resemble the formulas for the Lie derivative of Hochschild cochains on chains. Unfortunately, our candidate only respects composition of morphisms up to homotopy. Yet, we are able to give these homotopies explicitly, and interestingly, they are also familiar formulas--they look like generalizations of the B operator. We show that no higher homotopies are needed, and repackage our $``$functor up to homotopy'' into an $A_\infty$ functor. Finally, we apply (categorified) Cobar to get a $``$homotopically sheafy-cyclic object in dg categories with dg modules'' (Section ??).

\end{document}

\chapter{A category in dg cocategories}
	\section{Motivation of this chapter}
In this chapter, we show that algebras 
form a category in dg cocategories. 
As stated in the introduction, we will 
construct such a category with 
\begin{align*}
\begin{split}
	&\textrm{Objects: $k$-algebras } A,B,\dots\\
	&\textrm{Morphisms: dg cocategory } Bar(Hoch(A,B)\\
	&\textrm{Composition: } \bullet: Bar(Hoch(A,B)) \otimes 
	Bar(Hoch(B,C)) \to Bar(Hoch(A,C))\\
	&\phantom{Composition: }
	\textrm{ associative map of dg cocategories.}
\end{split}
\end{align*}
First, we define the dg cocategories 
$Bar(Hoch(A,B)$ using Hochschild cochains 
as morphisms, then we define the composition 
$\bullet$ using the brace operator on 
Hochschild cochains.
	\section{Dg cocategories $Bar(Hoch(A,B))$}
Let $A$, $B$ be $k$-algebras. We define 
a dg category, $Hoch(A,B)$, as follows:
\begin{align*}
	&\textrm{Objects: algebra maps } f:A \to B\\
	&\textrm{Morphisms: } 
	  Hoch(A)(f,g) = (C^\bullet(A,\, _fB_g),\, _f\delta_g)\\
	&\textrm{Composition: cup product on cochains.}
\end{align*}
(See Appendix \ref{chap:hochschild} for notation and 
standard operations on Hochschild complexes.) 
The cup product is an associative map of complexes, 
so $Hoch(A,B)$ is a dg category.

Now, we will take $Bar(-)$ of $Hoch(A,B)$, which 
is a categorified bar construction:
$$Bar: DGCat \to DGCocat.$$
	\section{Associative Composition $\bullet$} \label{sec:assoc_comp}
Now, we define an associative composition of 
dg cocategories 
$$Bar(Hoch(A,B)) \otimes Bar(Hoch(B,C))
\to Bar(Hoch(A,C))$$
where $A,B,C$ are $k$-algebras. To define 
the composition, we use the fact that 
$Bar(Hoch(A,C))$ is the cofree dg 
cocategory over $Hoch(A,C)$. In other 
words, $Bar(Hoch(A,C))$ satisfies 
the following universal property:
%
\begin{figure}[H]
\centerline{\xymatrix{
  \mathcal{B}
  %Bar(Hoch(A,B)) \otimes Bar(Hoch(B,C)) 
  \ar[r]
  \ar@{.>}[rd]
  & Hoch(A,C)\\
  & Bar(Hoch(A,C))
  \ar@{->>}[u]}}
\caption{Universal Property of $Bar$}	
\end{figure}  
where $\mathcal{B}$ is any dg cocategory, 
the horizontal map is a map of underlying 
structure (i.e., an association on objects 
and maps of complexes of morphisms), and the 
diagonal lift arrow is a map of dg cocategories. 
For us, $\mathcal{B} = Bar(Hoch(A,B)) 
\otimes Bar(Hoch(B,C))$. We will define a map of 
underlying structure $Bar(Hoch(A,B)) \otimes 
Bar(Hoch(B,C)) \to Hoch(A,C)$, which will lift 
to the map of dg cocategories 
$$\bullet: Bar(Hoch(A,B)) \otimes 
Bar(Hoch(B,C)) \to Bar(Hoch(A,C)).$$
The map on underlying structure is defined 
as follows:
%
\begin{align*}
Bar(Hoch(A,B)) \otimes 
Bar(Hoch(B,C)) 
&\to 
Hoch(A,C)\\
%
\textrm{On objects: } f \otimes g
&\mapsto 
g \circ f\\
%
\textrm{On morphisms: }
\xymatrix{
	A 
	\ar@/^5pc/[rr]^{f_0 \in Obj(Bar(Hoch(A,B)))}_{\substack{\\\textcolor{blue}{\Downarrow \phi_1}}}
	\ar@/^2pc/[rr]^{f_1}_{\substack{\\\textcolor{blue}{\Downarrow \phi_2}}}
	\ar@/_1pc/[rr]^{f_2}_{\textcolor{blue}{\vdots}}
	\ar@/_4pc/[rr]^{f_{n-1}}_{\substack{\\\textcolor{blue}{\Downarrow \phi_n}}}
	\ar@/_6pc/[rr]_{f_n \in Obj(Bar(Hoch(A,B)))} 
	&&	B 
	\ar@/^2pc/[rr]^{g_0 \in Obj(Bar(Hoch(B,C)))}_{\substack{\\\\\textcolor{blue}{\Downarrow \psi}}}
	\ar@/_2pc/[rr]_{g_1\in Obj(Bar(Hoch(B,C)))} 
	&& C 
}
&\mapsto
\xymatrix{
	A
	\ar@/^2pc/[rr]^{g_0f_0 \in Obj(Hoch(A,C))}_{\substack{\\\\\\ \textcolor{blue}{\psi\{\phi_1, \dots, \phi_n\}}\\ \textcolor{blue}{\Downarrow}}}
	\ar@/_2pc/[rr]_{g_1f_n \in Obj(Hoch(A,C))}
	&& C
}\\
%
\xymatrix{
	A 
	\ar@/^1pc/[r]^{f_0}_{\substack{\\\\\textcolor{blue}{\Downarrow \phi}}}
	\ar@/_1pc/[r]_{f_n} 
	&	B 
	\ar@/^1pc/[r]^{g_0}_{\substack{\\\\\textcolor{blue}{\Downarrow 1 \in k}}}
	\ar@/_1pc/[r]_{g_1} 
	& C}
&\mapsto	
\xymatrix{
	A
	\ar@/^1pc/[r]^{g_0f_0}_{\substack{\\\\ \textcolor{blue}{\Downarrow \phi}}}
	\ar@/_1pc/[r]_{g_1f_n}
	& C
}\\ 
%
\xymatrix{
	A 
	\ar@/^1pc/[r]^{f_0}_{\substack{\\\\\textcolor{blue}{\Downarrow 1 \in k}}}
	\ar@/_1pc/[r]_{f_n} 
	&	B 
	\ar@/^1pc/[r]^{g_0}_{\substack{\\\\\textcolor{blue}{\Downarrow \psi}}}
	\ar@/_1pc/[r]_{g_1} 
	& 	C}
&\mapsto
\xymatrix{
	A
	\ar@/^1pc/[r]^{g_0f_0}_{\substack{\\\\ \textcolor{blue}{\Downarrow \psi}}}
	\ar@/_1pc/[r]_{g_1f_n}
	& C
}
\end{align*}
All other non-pictured pairings of a morphism from 
$Bar(Hoch(A,B))$ and a morphism from $Bar(Hoch(B,C))$ 
map to zero. The brace operation is given in Reference 
\cite{T}, Equation 4.8, and the fact that it is 
associative follows from References \cite{GJ}, \cite{GV}, \cite{Ka}.


\chapter{A 2-category with a trace functor}
	\section{Motivation of this chapter}
In this chapter, we give a trace functor 
on $\underline{C}$, the 2-category 
introduced in Equation 
\ref{eq:2-cat}. This trace functor enriches 
the categorical structure on algebras by 
incorporating the action on Hochschild 
cohomology ($HH^0$) on Hochschild homology 
($HH_0$). We start with Kaledin's definition 
of a trace functor on a 2-category.

In preparation of Section \ref{sec:}, we 
generalize Kaledin's definition to a 
trace functor on a category in dg cocategories 
in Section \ref{sec:}.
	\section{A trace on C} \label{sec:2cat_trace_eg}
%
\begin{defn} \label{def:trace_functor}
(Kaledin): A trace functor on a 
2-category $\underline{C}$ is:
\begin{itemize}
	\item for each $A \in Obj(\underline{C})$, a functor 
	$TR_A: \underline{C}(A,A) \to k-mod$
	%
	\item for each pair $A, B \in Obj(\underline{C})$, a natural transformation
	$\tau_!(A,B)$:
	$$\xymatrix{
	\underline{C}(A,B) \otimes \underline{C}(B,A)
	\ar[rr]^{\tau = flip}
	\ar[d]_{m}
	&& \underline{C}(B,A) \otimes \underline{C}(A,B)
	\ar[d]_{m} \\
	\underline{C}(A,A)
	\ar[rd]_{TR_A}
	& \textcolor{blue}{\substack{\Rightarrow\\ \tau_!(A,B)}}
	& \underline{C}(B,B) 
	\ar[ld]^{TR_B}\\
	& k-mod
	}$$
	%
	such that, for $A,B,C \in Obj(\underline{C})$, 
	\begin{equation*} 
	\tau_!(B,A) \circ \tau_!(C,B) \circ \tau_!(A,C) = id.
	\end{equation*}
	$$\xymatrixcolsep{0.25pc}
	\xymatrix{
	& \underline{C}(C,A) \otimes \underline{C}(A,C) \otimes \underline{C}(B,C) 
	\ar[rd]^{\tau}
	\ar[dd]_{\textcolor{blue}{
		\substack{\Leftarrow\\ \tau_!(B,A)}}}^{\substack{
		\\\\\\\\\\\\\\\\\\TR_C \circ m^2}}\\
	\underline{C}(A,B) \otimes \underline{C}(B,C) \otimes \underline{C}(C,A) 
	\ar[ru]^{\tau}
	\ar[rd]^{\textcolor{blue}{\quad \quad \quad \substack{\Rightarrow\\ \tau_!(A,C)}}}_{TR_A\circ m^2}
	&& \underline{C}(B,C) \otimes \underline{C}(C,A) \otimes \underline{C}(A,B)
	\ar@/_1pc/[ll]_{\quad \quad \quad \quad \quad \tau}
	\ar[ld]_{\textcolor{blue}{\substack{\Rightarrow\\ \tau_!(C,B)}}\quad \quad\quad \quad}^{TR_B \circ m^2}\\
	& k-mod
	}$$
\end{itemize}
\end{defn}
%
Now, we will give a trace functor on the 
2-category, $\underline{C}$, define in Equation 
\ref{eq:2-cat}. Let $A \in Obj(\underline{C})$ 
be an algebra and $f: A \to A$ a map of algebras. 
Then, we set
$$TR_A(_fA) := \frac{A}{[A, _fA]}
= \frac{A}{(f(a)\cdot a^\prime - a^\prime \cdot a)}.$$
And for morphisms, 
\begin{align*}
\underline{C}(A,A)(f,g) \otimes \frac{A}{[A, _gA]}
\cong Z_A(_fA_g) \otimes \frac{A}{[A, _gA]}
&\to \frac{A}{[A, _fA]}\\
b \otimes a
&\mapsto 
b\cdot a
\end{align*}
is a well-defined map on $k$-modules. 
For algebra maps $f,f^\prime:A \leftrightarrows 
B:g,g^\prime$, 
we define 
the natural transformation $\tau_!(A,B)$ 
as follows:
$$
\xymatrixcolsep{1pc}
\xymatrix{
  _fB \underset{B}{\otimes} { _gA}/
    [A, _fB \underset{B}{\otimes} { _gA}]
  \ar[rrr]^{\tau_!(A,B)(f,g)}
  \ar[ddd]
  &&& _gA \underset{A}{\otimes} { _fB}/
    [B, _gA \underset{A}{\otimes} { _fB}]
  \ar[ddd]\\
  %
  &[b \otimes a] 
  \ar@{|->}[r]
  \ar@{|->}[d]_{(b^\prime \cdot, a^\prime \cdot)}
  & [a \otimes b]
  \ar@{|->}[d]^{(a^\prime \cdot, b^\prime \cdot)}\\
  & [b^\prime \cdot b \otimes a^\prime \cdot a]
  \ar@{|->}[r]
  & [a^\prime \cdot a \otimes b^\prime \cdot b]\\
  %
  _{f^\prime}B \underset{B}{\otimes} { _{g^\prime}A}/
    [A, _{f^\prime}B \underset{B}{\otimes} { _{g^\prime}A}]
  \ar[rrr]_{\tau_!(A,B)(f^\prime,g^\prime)}
  &&& _{g^\prime}A \underset{A}{\otimes} { _{f^\prime}B}/
  [B, _{g^\prime}A \underset{A}{\otimes} { _{f^\prime}B}]\\
}
$$
where $b^\prime \in Z_A(_{f^\prime}B_f)$, 
$a^\prime \in Z_B(_{g^\prime}A_g)$, $a \in A$, 
$b \in B$. Clearly, this flip map $\tau_!$ satisfies 
Equation \ref{eq:trace_cocycle}.






	\section{Redefining the trace functor} \label{sec:redefine_trace}
In this section, we generalize Kaledin's 
definition of a trace functor on a 2-category 
to a trace functor on dg cocategories.
First, we transform the definition from 
the language from functors and natural 
transformations to the language of modules.
%
\begin{defn} \label{def:module_over_cat}
Let $\mathcal{C}$ be a $k$-linear category. 
A left module over $\mathcal{C}$ is a 
$k$-linear functor 
$\mathcal{C} \to k-mods$.
\end{defn}
%
Given the definition above, we can rewrite 
the definition of a trace functor on a 
2-category in the language of modules.
%
\begin{defn} \label{def:trace_module}
(Kaledin, reformulated): Let $\mathcal{C}$ 
be a category in $k$-linear categories. 
A trace functor on $\mathcal{C}$ is:
\begin{itemize}
\item for each $A \in Obj(\mathcal{C})$, a 
left module $T(A)$ over $\mathcal{C}(A,A)$
%
\item for each pair $A, B \in Obj(\mathcal{C})$, a
map of modules over $\mathcal{C}(A,B) \otimes \mathcal{C}(B,A)$
$$\tau_!(A,B): m_{ABA}^*T(A) \to \tau^*m_{BAB}^*T(B)$$ 
where $m_{ABA}$ is the composition functor 
$m_{ABA}: \mathcal{C}(A,B) \otimes \mathcal{C}(B,A) \to 
\mathcal{C}(A,A)$, $\tau$ is a flip functor, and 
pulling back along a functor means pre-composition.
%
\item for $A,B,C \in Obj(\mathcal{C})$, 
$$\tau^{*2} \tau_!(B,A) \circ \tau^* \tau_!(C,B) \circ 
\tau_!(A,C) = id.$$
\end{itemize}
\end{defn}
%
Now, we will translate from modules to dg comodules. 
Reversing the arrows in Definition 
\ref{def:module_over_cat}, we have the following 
definition for a dg comodule over a category 
in dg cocategories.
%
\begin{defn}\label{def:dg_comod} 
Let $\mathcal{C}$ be a dg 
cocategory. A dg comodule over 
$\mathcal{C}$ is: 
for each $f \in Obj(\mathcal{C})$, a complex 
$T^\bullet(f)$ and map of complexes
$$\Delta_f: T^\bullet(f) \to 
\prod\limits_{g \in Obj(\mathcal{C})} 
\mathcal{C}^\bullet(f,g) \otimes T^\bullet(g)
$$
such that the following two maps 
coincide (coassociativity):
$$
\xymatrix{
T^\bullet(f)
\ar[d]_{\Delta(f)}\\
\prod \limits_{g \in Obj(\mathcal{C})}
\mathcal{C}^\bullet(f,g) \otimes
T^\bullet(g)
\ar@/^1pc/[d]^{id \otimes \Delta(g)}
\ar@/_1pc/[d]_{\Delta_{\mathcal{C}(} \otimes id}\\
 \prod \limits_{g, g^\prime \in Obj(\mathcal{C})}
\mathcal{C}^\bullet(f,g) \otimes
\mathcal{C}^\bullet(g,g^\prime) \otimes
T^\bullet(g^\prime)  
}
$$
and the following diagram commutes 
(counitality):
$$
\xymatrixcolsep{5pc}
\xymatrixrowsep{5pc}
\xymatrix{
T^\bullet(f)
\ar[r]^{\Delta(f)}
\ar[rd]_{id}
& \prod \limits_{g \in Obj(\mathcal{C})}
\mathcal{C}^\bullet(f,g) \otimes
T^\bullet(g)
\ar[d]^{\epsilon_{\mathcal{C}}
\otimes id} \\
%
& T^\bullet(f).  
}
$$
\end{defn}
%
Finally, we can rewrite Definition 
\ref{def:trace_module} in terms of dg 
comodules.
%
\begin{defn} \label{def:trace_dg_comodule}
Let $\mathcal{C}$ be a category in 
dg cocategories. A trace functor on 
$\mathcal{C}$ is:
\begin{itemize}
\item for each $A \in Obj(\mathcal{C})$, a 
dg comodule $T(A)$ over $\mathcal{C}(A,A)$
%
\item for each pair $A, B \in Obj(\mathcal{C})$, a
map of dg comodules over $\mathcal{C}(A,B) \otimes \mathcal{C}(B,A)$
$$\tau_!(A,B): m_{ABA}^*T(A) \to \tau^*m_{BAB}^*T(B)$$ 
where $m_{ABA}$ is the composition functor 
$m_{ABA}: \mathcal{C}(A,B) \otimes \mathcal{C}(B,A) \to 
\mathcal{C}(A,A)$, $\tau$ is a flip functor. We 
can take any definition for the pullback that 
is a natural and satisifies 
\begin{equation}
F^*G^* = (GF)^*.
\end{equation}
%
\item for $A,B,C \in Obj(\mathcal{C})$, 
$$\tau^{*2} \tau_!(B,A) \circ \tau^* \tau_!(C,B) \circ 
\tau_!(A,C) = id.$$
\end{itemize}
\end{defn}





















\chapter{Interlude: from a trace functor to a dg functor}
	\section{Motivation of this chapter}
The purpose of this chapter is to 
show that a trace functor $T$ on a 
category $\mathcal{C}$ in dg cocategories 
gives a dg functor $\mathcal{F}: 
\chi(\mathcal{C}) \to \mathcal{D}$ where 
$\chi(\mathcal{C})$ and 
$\mathcal{D}$ are dg categories 
introduced in Sections \ref{sec:} and 
\ref{sec:}, respectively. We switch from 
the trace functor $T$ to the dg functor 
$\mathcal{F}$ so that we can make 
precise the notion of a $``$trace 
functor up to homotopy''. Namely, a 
trace functor on $\mathcal{C}$ up to homotopy 
is an $A_\infty$-functor from 
$\chi(\mathcal{C})$ to $\mathcal{D}$. 
We will then give such an $A_\infty$-functor 
for $\mathcal{C}$ being the category 
given in Equation \ref{eq:cat_in_dgcocat} 
(see Section \ref{sec:}).

% Uncomment the 'singlespace' environment and '\bibsep' command
% if needed - some bibliographic styles overide the definition
% of 'thebibliography' in nuthesis.cls
\begin{singlespace}
%\bibsep 12pt
\clearpage\phantomsection % needed for hyperlikns to work correctly
% The usages of \bibitem and \cite{..} are 
% explained in Section 4.3 (page 73) of % LaTeX manual.
% Or you may use BibTeX.

% (The following suggested by Francisco Iacobelli - 5/11/2010)
% In case, you want to use BibTeX, you should replace (or comment)
% the bibliography environment.
% Instead uncomment the following 3 lines and replace <bib file>
% with your .bib file:
% \renewcommand\refname{\begin{centering}References\end{centering}}
% \bibliography{<bib file>}
% \bibliographystyle{acm} %or another suitable style.


\begin{thebibliography}{}
\bibitem{T} Tsygan, Boris. Noncommutative Calculus and Operads.
\bibitem{F} Faonte, Giovanni. $A_\infty$-Functors and Homotopy Theory of DG-Categories
\end{thebibliography}
\end{singlespace}

\appendix		% Appendix begins here (optional).
%\appendices	% If more than one appendix chapters
\chapter{Connes cyclic category, $\Lambda$}\label{chap:lambda}
Here, we give generators and relations for 
the cyclic category, $\Lambda$. None of this 
is new, but we do it to establish notation 
for the rest of the paper.

$\Lambda$ has objects $\{[n]: n \in \mathbb{N}\}$ 
and generating morphisms:
\begin{equation} \label{eqn:cyclic_generators}
\begin{split}
\textrm{rotations } \tau_n:[n] \to [n], \\ 
\textrm{coboundaries } \delta_{j,n}: [n] \to [n-1], 0 \leq j \leq n-1, \\ 
\textrm{codegeneracies } \sigma_{i,n}:[n] \to [n+1], 1 \leq i \leq n+1
\end{split}
\end{equation}
subject to relations:
\begin{equation}\label{eqn:cyclic_relations}
\begin{split}
\delta_{i,n-1} \delta_{j,n} &= \delta_{j-1,n-1} \delta_{i,n} 
  \quad 1 \leq i \leq j \leq n-1 \\
\sigma_{i,n+1} \sigma_{j,n} &= \sigma_{j+1,n+1} \sigma_{i,n}
  \quad 1 \leq i < j \leq n+1 \\
\delta_{j,n+1}\sigma_{i,n} &= 
  \begin{cases}
    \sigma_{i,n-1}\delta_{j-1,n} 
      & 0 \leq i < j \leq n-1\\
    id & j = i, i-1, 0 \leq j \leq n\\
    \sigma_{i-1,n-1}\delta_{j,n} 
      & \quad 0 \leq j < i \leq n
   \end{cases}\\
\tau_{n+1}\sigma_{i,n} &= \sigma_{i+1,n}\tau_n
  \quad 1 \leq i \leq n\\
\tau_{n+1}^2 \sigma_{n+1,n} &= \sigma_{1,n} \tau_n \\
\tau_{n-1}\delta_{i,n} &= \delta_{i+1,n}\tau_n
  \quad 0 \leq i \leq n-1\\
\delta_{0,n}\tau_n^2 &= \tau_{n-1}\delta_{n-1,n}. 
\end{split}
\end{equation}

Some definitions of $\Lambda$ include an extra 
coboundary $\delta_{n,n}$ and codegeneracy 
$\sigma_{0,n}$ even though the morphisms we have given 
are sufficient to generate $\Lambda$. However, 
we will still refer to these extra 
morphisms, and in terms of our generators, they are 
$\delta_{n,n} := \delta_{0,n}\tau_n$ and 
$\sigma_{0,n} := \tau_{n+1} \sigma_{n+1,n}$.
\chapter{Computations}
In this appendix, we give the 
computational propositions 
needed to establish the 
homotopically sheafy-cyclic structure 
on dg comodules. All the comodules we work 
with will be cofree, 
and we will define maps into them by 
giving maps into cogenerators 
(see Equation \ref{eq:quasicofree}).

\section{Computational notation}
For this section's propositions, we 
establish the following notation:
\begin{align*}
A_0, A_1 
&\textrm{ fixed algebras}\\
(\vec{\phi} | \vec{\psi} | \alpha) 
&:= 
(\phi_1\smdots\phi_n | \psi_1\smdots\psi_m| \alpha)\\
&= 
\xymatrix{
A_0 \ar@/^5pc/[r]^{f_0} 
\ar@/^2pc/[r]^{\big\Downarrow \phi_1}_{f_1} 
\ar@/_2pc/[r]^{\substack{\vdots\\ f_n}}
\ar@/_4pc/[rr]_{id}^{\substack{\alpha \\ \\ \\ \\ }}
& A_1 \ar@/^5pc/[r]^{g_0} 
\ar@/^2pc/[r]^{\big\Downarrow \psi_1}_{g_1} 
\ar@/_2pc/[r]^{\substack{\vdots\\ g_m}}
& A_0
}
\in C(A_0 \to A_1 \to A_0)(g_0f_0)\\
\vec{\phi}_{\{i_1, i_2,\smdots, i_k\}}
&:= 
\phi_{i_1}\phi_{i_2}\smdots\phi_{i_k}
 \quad \textrm{ where $\{i_1, i_2,\smdots, i_k\}$ 
 is an ordered subset of $\{1,\smdots,n\}$}\\
\vec{\phi}_{\{\}}
&:= 
1 \in k \cong Bar_0(C^\bullet(A_0, A_1))\\
\vec{\psi}_{\{\}}
&:= 
1 \in k \cong Bar_0(C^\bullet(A_1, A_0))\\
|I| 
&:=
\textrm{number of elements in a set $I$}\\
I_1I_2 
&:= 
\textrm{concatenation as ordered sets 
of possibly-empty sets $I_1$ and $I_2$}\\
\end{align*}
%
\subsection{Notation for elements of Hochschild chains}
Let $a_0 \otimes a_1 \otimes \cdots \otimes a_n$ 
denote a typical element of $C_{-\bullet}(A,A)$ where 
$A$ is some algebra. At times, we wish to feed a portion 
of $a_0 \otimes a_1 \otimes \smdots \otimes a_n$ to a 
Hochschild cochain (or other map on chains) without 
specifying the degree of the cochain. To do this, 
we will rewrite $a_0 \otimes a_1 \otimes \smdots \otimes a_n 
= a_0 \otimes \mathfrak{a}_1 \otimes \smdots \otimes \mathfrak{a}_r$ 
where each $\mathfrak{a}_i = a_{j_i} \otimes a_{j_i+1} \otimes 
\smdots \otimes a_{j_{i+1}-1}$ and $\mathfrak{a}_i$ 
is an empty chain if $j_i = j_{i+1}$.

For example, if $\phi \in C^2(A,A)$, then we 
rewrite 
$$
\sum \limits_{1\leq i \leq n-1} 
a_0 \otimes a_1 \otimes \smdots a_{i-1} \otimes 
\phi(a_i, a_{i+1}) \otimes a_{i+2} \otimes \smdots \otimes a_n
=
\sum
a_0 \otimes \mathfrak{a}_1 \otimes \phi(\mathfrak{a}_2)
\otimes \mathfrak{a}_3.
$$

\section{Computational Propositions}
% \begin{figure} \label{fig:upsilon}
% \centerline{\xymatrix{
% A_0 \ar@/^5pc/[r]^{f_0} 
% \ar@/^2pc/[r]^{\big\Downarrow \phi_1}_{f_1} 
% \ar@/_2pc/[r]^{\substack{\vdots\\ f_n}}
% \ar@/_4pc/[rr]_{id}^{\substack{\alpha \\ \\ \\ \\ }}
% & A_1 \ar@/^5pc/[r]^{g_0} 
% \ar@/^2pc/[r]^{\big\Downarrow \psi_1}_{g_1} 
% \ar@/_2pc/[r]^{\substack{\vdots\\ g_m}}
% & A_0
% }
% $\overset{\Upsilon}\longrightarrow$
% \xymatrix{
% A_1 \ar@/^5pc/[r]^{g_0} 
% \ar@/^2pc/[r]^{\big\Downarrow \psi_1}_{g_1} 
% \ar@/_2pc/[r]^{\substack{\vdots\\ g_m}}
% \ar@/_4pc/[rr]_{id}^{\substack{\alpha \\ \\ \\ \\ }}
% & A_0 \ar@/^5pc/[r]^{f_0} 
% \ar@/^2pc/[r]^{\big\Downarrow \phi_1}_{f_1} 
% \ar@/_2pc/[r]^{\substack{\vdots\\ f_n}}
% & A_1
% }}
% \caption{A picture of the domain and target of $\Upsilon$}
% \end{figure}

\begin{prop}
\label{prop:c1}
Let $\hat{\tau}_1: 
B(A_0 \to A_1 \to A_0) 
\longrightarrow B(A_1 \to A_0 \to A_1)$ 
be as defined in Section 
\ref{sec:cyclic_B(n)}.
Recall from Example \ref{eg:pb5} that 
$\hat{\tau}_1^*C(A_1 \to A_0 \to A_0)
\cong C(A_1 \to A_0 \to A_1)$ 
as complexes. Define a map 
$$
\Upsilon_{A_0,A_1}: C(A_0 \to A_1 \to A_0)
\to \hat{\tau}_1^*C(A_1 \to A_0 \to A_1)
$$
of comodules over 
$B(A_0 \to A_1 \to A_0)$ by mapping into 
cogenerators as follows:
\begin{align*}
\upsilon^{f_0, g_0}: C(A_0 \to A_1 \to A_0)(f_0,g_0) 
&\to
\hat{\tau}_1^*C(A_1 \to A_0 \to A_1)(g_0,f_0)
\cong 
C(A_1 \to A_0 \to A_1)(g_0,f_0)\\
&\xrightarrow[cogenerators]{\textrm{project onto}}
C_{-\bullet}(A_1, _{f_0g_0}{A_1}_{id})\\
\upsilon_{n,m}^{f_0,g_0} 
(\vec{\phi} | \vec{\psi} | \alpha) = 
& \sum_{\substack{I_1I_2 = \{2,\cdots,n\} \\
                          \textrm{as ordered sets}}}
  \phi_1(\lambda(\vec{\psi})\lambda(\vec{\phi_{I_2}})\cdot \mathfrak{a}_3, a_0, \mathfrak{a}_1) \otimes \lambda(\vec{\phi_{I_1}}) \cdot \mathfrak{a}_2 \\
&\phantom{{}move{}}
\bigg( + f_0a_0 \otimes \lambda(\vec{\phi}) \mathfrak{a}_1 
  \; \; \; \; if \; \; m = 0 \bigg).
\end{align*}
Then, $\Upsilon_{A_0,A_1}: C(A_0 \to A_1 \to A_0)
\to \hat{\tau}^*C(A_1 \to A_0 \to A_1)$ 
is a map of dg comodules over 
$B(A_0 \to A_1 \to A_0)$.
\end{prop}
%
\begin{proof}
We must show: (1) $\Upsilon$ is a map of comodules, and 
(2) $\Upsilon$ commutes with the differentials. (In this 
proof, we drop the subscripts and write 
$\Upsilon := \Upsilon_{A_0, A_1}$.)

(1) This proof is standard for cofree comodules. 
Let ($\vec{\phi} | \vec{\psi} | \alpha$) be as 
in the statement of the proposition. We want to 
show that $\Upsilon$ commutes with the coproducts. 
On one hand,
\begin{align*}
&\phantom{{}={}}
[(id_B \otimes \Upsilon) \circ 
  \Delta_{C(A_0 \to A_1 \to A_0)}] 
  ( \vec{\phi} | \vec{\psi} | \alpha ) \\
&= [id_B \otimes \Upsilon]
	\big( \sum_{\substack{I_1I_2 = \{1,2,\cdots,n\} \, \textrm{and} \\ 
						  J_1J_2 = \{1,2,\cdots,m\} \\
				          \textrm{as ordered sets}}} 
    \pm (\vec{\phi}_{I_1} | \vec{\psi}_{J_1}) \otimes (\vec{\phi}_{I_2} | \vec{\psi}_{J_2} | \alpha) \, \big) \\
&= \sum_{\substack{I_1I_2I_3 = \{1,2,\cdots,n\} \, \textrm{and} \\ 
				   J_1J_2J_3 = \{1,2,\cdots,m\} \\
				   \textrm{as ordered sets}}} 
    \pm (\vec{\phi}_{I_1} | \vec{\psi}_{J_1}) \otimes 
    (\vec{\phi}_{I_2} | \vec{\psi}_{J_2}) \otimes 
    \upsilon_{|I_3|,|J_3|}(\vec{\phi}_{I_3} | \vec{\psi}_{J_3} | \alpha) \\
\end{align*}
where the signs are as in Equation 
\ref{eq:coprod_signs}. On the other hand,
\begin{align*}
&\phantom{{}={}}
[\Delta_{\hat{\tau}^*C(A_1 \to A_0 \to A_1)} 
  \circ \Upsilon ]
  ( \vec{\phi} | \vec{\psi} | \alpha ) \\
&= \Delta_{\hat{\tau}^*C(A_1 \to A_0 \to A_1)}
	\big( \sum_{\substack{I_1I_2 = \{1,2,\cdots,n\} \, \textrm{and} \\ 
						  J_1J_2 = \{1,2,\cdots,m\} \\
				          \textrm{as ordered sets}}}
	\pm (\vec{\phi}_{I_1} | \vec{\psi}_{J_1}) \otimes 
    \upsilon_{|I_2|,|J_2|}(\vec{\phi}_{I_2} | \vec{\psi}_{J_2} | \alpha) \, \big)\\
&= \sum_{\substack{I_1I_2I_3 = \{1,2,\cdots,n\} \, \textrm{and} \\ 
				   J_1J_2J_3 = \{1,2,\cdots,m\} \\
				   \textrm{as ordered sets}}} 
    \pm (\vec{\phi}_{I_1} | \vec{\psi}_{J_1}) \otimes 
    (\vec{\phi}_{I_2} | \vec{\psi}_{J_2}) \otimes 
    \upsilon_{|I_3|,|J_3|}(\vec{\phi}_{I_3} | \vec{\psi}_{J_3} | \alpha).   				          
\end{align*}
where the signs are as in Equation 
\ref{eq:coprod_signs}. Clearly 
$(id_B \otimes \Upsilon) \circ 
\Delta_{C(A_0 \to A_1 \to A_0)} = 
\Delta_{\hat{\tau}^*C(A_1 \to A_0 \to A_1)} 
\circ \Upsilon$.

(2) We will show that $\Upsilon$ commutes with 
the differentials by direct computation. Since 
$\Upsilon$ is a map of cofree comodules, we only 
need to check that $\pi_1 \circ D(\Upsilon) = 0$ 
where $D(\Upsilon)$ is the differential applied 
to $\Upsilon$ as a linear map between complexes 
and $\pi_1$ denotes projection of a comodule 
onto its cogenerators. More explicitly, we want 
to check that
\begin{equation} \label{eq:upsilon}
\begin{aligned}
&\upsilon_{n, m} ( \tilde{\delta}(\vec{\phi}) | \vec{\psi} | \alpha ) \; + 
\upsilon_{n, m} ( \vec{\phi} | \tilde{\delta}(\vec{\psi}) | \alpha ) \; + 
\upsilon_{n-1, m} ( b^\prime(\vec{\phi}) | \vec{\psi} | \alpha ) \; + 
\upsilon_{n, m-1} ( \vec{\phi} | b^\prime(\vec{\psi}) | \alpha ) \; + \\
&\upsilon_{n, m} ( \vec{\phi} | \vec{\psi} | b(\alpha) ) \; + 
b \circ \upsilon_{n, m} ( \vec{\phi} | \vec{\psi} | \alpha ) \; + \\
& \sum \limits_{\substack{
  I_1I_2 = \{1,\smdots,n\}\\ 
  \textrm{as ordered sets}}}
(-1)^{(\sum \limits_{p\in I_2}|\phi_p|+1)(\sum \limits_{1\leq q \leq m-1}|\psi_q|+1)}
  \upsilon_{|I_1|, m-1}(\vec{\phi}_{I_1} | \vec{\psi}_{\{1,\cdots, m-1\}} | \psi_{m} \{\vec{\phi}_{I_2}\} \cdot \alpha ) \; + \\
& \sum \limits_{\substack{
  J_1J_2 = \{1,\smdots,m\}\\ 
  \textrm{as ordered sets}}}
  (-1)^{(\sum \limits_{2 \leq p\leq n}|\phi_p|+1)(\sum \limits_{q\in J_2}|\psi_q|+1)}
\phi_1 \{ \psi_{J_1}\}\cdot \upsilon_{n-1, |J_2|}(\phi_{\{2,\cdots , n\}} | \psi_{J_2} | \alpha ) \; + \\  
& (-1)^{(|\phi_n|+1)(\sum \limits_{1\leq q \leq m}|\psi_q|+1)}
  \upsilon_{n-1, m}(\vec{\phi}_{\{1,\cdots, n-1\}} |\vec{\psi} | \phi_{n}\cdot \alpha) \; + \\
& (-1)^{(|\psi_1|+1)(\sum \limits_{1 \leq p \leq n}|\phi_p|+1)}
\psi_1\cdot \upsilon_{n,m-1} ( \vec{\phi} | \vec{\psi}_{\{2,\cdots, m\}} | \alpha) \\ 
&= 0.
\end{aligned}
\end{equation}
In Equation \ref{eq:upsilon}, we will call the 
terms in rows 1-2 the ``standard terms'', 
and the terms in rows 3-6 the 
``extra terms''.

We compute the sum of the standard terms. 
In Table \ref{table:t1}, the leftmost column 
lists the expressions that don't cancel in 
the sum of the standard terms, the middle 
column gives the standard term from which 
the expression comes, and the rightmost 
column gives the term (extra or standard) 
that cancels the expression. 

All of the terms in Table \ref{table:t1} 
cancel, so $\Upsilon$ is a map of complexes.
\end{proof}
%
\begin{landscape}
\begin{center}
\begin{table}
  \begin{tabular}{ p{3.25in} | p{2in} | p{2.5in} }
    \hline
    Expression & Comes from Standard Term & Cancelling Term \\ \hline
    
    \breakcell{$f_0\psi_1(\lambda(\vec{\phi}_{I_2}) \mathfrak{a}_3 ) \phi_1(\lambda(\vec{\psi}_{\{2,\cdots, m\}} \lambda(\vec{\phi}_{I_3}) \mathfrak{a}_4, a_0, \mathfrak{a}_1)$ \\ $\otimes \lambda(\vec{\phi}_{I_1}) \mathfrak{a}_2$} &
    $\upsilon_{n, m} (\delta(\phi_1)\phi_2 \cdots \phi_n | \vec{\psi} | \alpha)$ & 
    $f_0 \psi_1 \cdot \upsilon_{n,m-1} ( \vec{\phi} | \vec{\psi}_{\{2,\cdots, m\}} | \alpha)$ \\ \hline

    \breakcell{$\phi_1( \lambda(\vec{\psi}_{\{1,\cdots, m-1\}}) \lambda(\vec{\phi}_{I_2}) \mathfrak{a}_3, \psi_{m} ( \lambda(\vec{\phi}_{I_3}) \mathfrak{a}_4) \cdot a_0, \mathfrak{a}_1 )$ \\ $\otimes \lambda(\vec{\phi}_{I_1}) \mathfrak{a}_2$} &
    $\upsilon_{n, m} (\delta(\phi_1)\phi_2 \cdots \phi_n | \vec{\psi} | \alpha)$ &
    $\upsilon_{|I_1|, m-1}(\vec{\phi}_{I_1} | \vec{\psi}_{\{1,\cdots, m-1\}} | \psi_{m} \{\vec{\phi}_{I_2}\}\cdot \alpha )$ \\ \hline

    \breakcell{$\phi_1( \lambda(\vec{\psi}) \lambda(\vec{\phi}_{I_2}) \mathfrak{a}_3, g_m \phi_n(\mathfrak{a}_4) \cdot a_0, \mathfrak{a}_1) \otimes \lambda(\vec{\phi}_{I_1}) \mathfrak{a}_2$} &
    $\upsilon_{n, m} (\delta(\phi_1)\phi_2 \cdots \phi_n | \vec{\psi} | \alpha)$ &
    $\upsilon_{n-1, m}(\vec{\phi}_{\{1, \cdots, n-1\}} | \vec{\psi} | g_m \phi_{n} \cdot \alpha )$ \\ \hline

    \breakcell{$\phi_1( \lambda(\vec{\psi}) \lambda(\vec{\phi}_{I_2}) \mathfrak{a}_2) \cdot f_1(a_0) \otimes \lambda(\vec{\phi}_{I_1}) \mathfrak{a}_1$} &
    $\upsilon_{n, m} (\delta(\phi_1)\phi_2 \cdots \phi_n | \vec{\psi} | \alpha)$ &
    $\phi_1 \cdot \upsilon_{n-1, 0}(\vec{\phi}_{\{2, \cdots, n\}} | \vec{\psi} |\alpha )$ \\ \hline

    \breakcell{$f_0a_0 \cdot \phi_1(\mathfrak{a}_1) \otimes \lambda(\vec{\phi}_{\{1,\cdots,n-1\}}) \mathfrak{a}_2$} &
    \breakcell{$\upsilon_{n, m} (\delta(\phi_1)\phi_2 \cdots \phi_n | \vec{\psi} | \alpha)$ \\ if $\vec{\psi} = 1$} & 
    \breakcell{$b \circ \upsilon_{n, m} (\vec{\phi} | \vec{\psi} | \alpha)$ \\ if $\vec{\psi} =1$} \\ \hline

    \breakcell{$f_0 g_m \phi_n(\mathfrak{a}_2) f_0a_0 \otimes \lambda(\vec{\phi}_{\{1,\cdots,n-1\}}) \mathfrak{a}_1$} &
    \breakcell{$b \circ \upsilon_{n, m} (\vec{\phi} | \vec{\psi} | \alpha)$ \\ if $\vec{\psi} = 1$} &
    \breakcell{$\upsilon_{n-1, m}(\vec{\phi}_{\{1, \cdots, n-1\}} | \vec{\psi} | g_m \phi_{n} \cdot \alpha )$ \\ if $\vec{\psi} = 1$} \\ \hline

    \breakcell{$\phi_1(\lambda(\vec{\psi}) \lambda(\vec{\phi}_{I_2}) \mathfrak{a}_4, a_0, \mathfrak{a}_1) \cdot \phi_2(\mathfrak{a}_2) \otimes \lambda(\vec{\phi}_{I_1}) \mathfrak{a}_3$} &
    $b \circ \upsilon_{n, m} (\vec{\phi} | \vec{\psi} | \alpha)$ &
    $\upsilon_{n-1, m}(\phi_1 \cup \phi_2 \phi_3 \cdots \phi_n | \vec{\psi} | \alpha)$ \\ \hline
    
    \breakcell{$\phi_1(\lambda(\vec{\psi}_{J_1}) \lambda(\vec{\phi}_{I_2}) \mathfrak{a}_3) \phi_2(\lambda(\vec{\psi}_{J_2} \lambda(\vec{\phi}_{I_3}) \mathfrak{a}_3, a_0, \mathfrak{a}_1)$ \\ $\otimes \lambda(\vec{\phi}_{I_1}) \mathfrak{a}_2$} &
    $\upsilon_{n-1, m}(\phi_1 \cup \phi_2 \phi_3 \cdots \phi_n | \vec{\psi} | \alpha)$ &
     $\phi_1 \{ \vec{\psi}_{J_1} \} \cdot \upsilon_{n-1, |J_2|}(\vec{\phi}_{\{2, \cdots, n\}} | \vec{\psi}_{J_2} |\alpha )$\\ \hline

    \breakcell{$f_0 \psi_1(\lambda(\vec{\phi}_{I_2}) \mathfrak{a}_2) \cdot f_0a_0 \otimes \lambda(\vec{\phi}_{I_1}) \mathfrak{a}_1$} &  
    \breakcell{$f_0 \psi_1 \cdot \upsilon_{n, 0}(\vec{\phi} | 1 | \alpha)$ \\ if $\vec{\psi} = \psi_1$} &
    \breakcell{$ \upsilon_{|I_1|, 0} (\vec{\phi}_{I_1} | 1 | \psi_1 \{ \vec{\phi}_{I_2} \} \cdot \alpha )$ \\ if $\vec{\psi} = \psi_1$} \\ \hline
    \hline
  \end{tabular}
\caption{Expansion of terms in Equation \ref{eq:upsilon}}
\label{table:t1}
(Technically, the last term in the middle column is not a standard term, but we include it in the table for convenience.)
\end{table}
\end{center}
\end{landscape}
% \begin{figure} \label{fig:upsilon}
% \centerline{\xymatrix{
% A_0 \ar@/^5pc/[r]^{f_0} 
% \ar@/^2pc/[r]^{\big\Downarrow \phi_1}_{f_1} 
% \ar@/_2pc/[r]^{\substack{\vdots\\ f_n}}
% \ar@/_4pc/[rr]_{id}^{\substack{\alpha \\ \\ \\ \\ }}
% & A_1 \ar@/^5pc/[r]^{g_0} 
% \ar@/^2pc/[r]^{\big\Downarrow \psi_1}_{g_1} 
% \ar@/_2pc/[r]^{\substack{\vdots\\ g_m}}
% & A_0
% }
% $\overset{B}\longrightarrow$
% \xymatrix{
% A_0 \ar@/^5pc/[r]^{f_0} 
% \ar@/^2pc/[r]^{\big\Downarrow \phi_1}_{f_1} 
% \ar@/_2pc/[r]^{\substack{\vdots\\ f_n}}
% \ar@/_4pc/[rr]_{id}^{\substack{\alpha \\ \\ \\ \\ }}
% & A_1 \ar@/^5pc/[r]^{g_0} 
% \ar@/^2pc/[r]^{\big\Downarrow \psi_1}_{g_1} 
% \ar@/_2pc/[r]^{\substack{\vdots\\ g_m}}
% & A_0
% }}
% \caption{A picture of the domain and target of $B$}
% \end{figure}
%
\begin{prop}
\label{prop:c2}
Let 
$B_{A_0,A_1} = B:C(A_0 \to A_1 \to A_0)
\longrightarrow C(A_0 \to A_1 \to A_0)$ 
be the map of cofree comodules defined by 
the following maps to cogenerators:
\begin{equation}
\label{eq:def_sigma}
B_{n, m} (\vec{\phi} | \vec{\psi} | \alpha) 
= 1 \otimes \lambda(\psi)\lambda(\phi) \mathfrak{a}_2 \otimes a_0 \otimes \mathfrak{a}_1.
\end{equation}
Then, $D(B_{A_0,A_1}) = \Upsilon_{A_1,A_0}\Upsilon_{A_0,A_1} - id$ where
$\Upsilon$ is defined in Proposition 
\ref{prop:c1}.
\end{prop}
%
\begin{proof}
We prove the statement by direct computation. 
Since all of the maps are maps of cofree comodules, 
we only need to check that $\pi_1(D(B_{A_0,A_1}) - 
\Upsilon_{A_1,A_0}\Upsilon_{A_0,A_1} - id) = 0$ 
where $\pi_1$ denotes projection of the comodule 
onto cogenerators. More explicitly, for an element 
$(\vec{\phi}|\vec{\psi}|\alpha)$, we want to check that
\begin{equation} \label{eq:upsilon_homotopy}
\begin{aligned}
&B_{n, m} ( \tilde{\delta}(\vec{\phi}) | \vec{\psi} | \alpha ) \; + 
B_{n, m} ( \vec{\phi} | \tilde{\delta}(\vec{\psi}) | \alpha ) \; + 
B_{n-1, m} ( b^\prime(\vec{\phi}) | \vec{\psi} | \alpha ) \; + 
B_{n, m-1} ( \vec{\phi} | b^\prime(\vec{\psi}) | \alpha ) \; + \\
&B_{n, m} ( \vec{\phi} | \vec{\psi} | b(\alpha) ) \; + 
b \circ B_{n, m} ( \vec{\phi} | \vec{\psi} | \alpha ) \; + \\
&B_{|I_1|, m-1}(\vec{\phi}_{I_1} | \vec{\psi}_{\{1,\cdots, m-1\}} | \psi_{m} \{\vec{\phi}_{I_2}\}\cdot \alpha ) \; + 
B_{n-1, m}(\vec{\phi}_{\{1,\cdots, n-1\}} |\vec{\psi}_{m} | \phi_{n} \cdot \alpha) \; + \\
&\phi_1 \{ \psi_{J_1}\} \cdot B_{\vec{\phi}|-1, |J_2|}(\phi_{\{2,\cdots , n\}} | \psi_{J_2} | \alpha ) \; + 
\psi_1 \cdot B_{n,m-1} ( \vec{\phi} | \vec{\psi}_{\{2,\cdots, |\vec{\psi}\}} | \alpha)  - \\ 
&\upsilon_{|J_1|, |I_1|} (\vec{\psi}_{J_1} | \vec{\phi}_{I_1} | \upsilon_{|I_2|, |J_2|} (\vec{\phi}_{I_2} | \vec{\psi}_{J_2} | \alpha ))  - \pi_1(\vec{\phi} | \vec{\psi} | \alpha) \\
&= 0.
\end{aligned}
\end{equation}

We will call the terms in the first two rows the ``standard terms'' in the computaion of $D(B_{A_0,A_1})$, and the terms in the second two rows the ``extra terms'' in the computation of $D(B_{A_0,A_1})$. The fifth row is $\pi_1(\Upsilon_{A_1,A_0}\Upsilon_{A_0,A_1} - id)$. 

We compute the sum of the standard terms. 
In Table \ref{table:t21}, the leftmost column 
lists the expressions that don't cancel in the 
sum of the standard terms, the middle column 
gives the standard term from which the expression 
comes, and the rightmost column gives the extra 
term that cancels the expression. 
Table \ref{table:t22} lists the remaining terms 
from the fifth row that are not already listed in 
Table \ref{table:t21}. In Table \ref{table:t22}, 
the left column lists the remaining expressions 
that don't cancel in the fifth row, and the 
right column gives the extra term that cancels 
the expression.

All of the terms in the tables describing the 
expansion of equation \ref{eq:upsilon_homotopy} 
cancel, so $D(B_{A_0,A_1}) = \Upsilon_{A_1,A_0}
\Upsilon_{A_0,A_1} - id$.
\end{proof}
%
\begin{landscape}
\begin{center}
\begin{table}
  \begin{tabular}{ p{3.25in} | p{2in} | p{2.5in} }
    \hline
    Expression & Comes from Standard Term & Cancels with Extra Term \\ \hline

    $\psi_1(\lambda(\vec{\phi}_{I_1}) \mathfrak{a}_2) \otimes \lambda(\vec{\psi}_{\{2,\cdots,m\}}) \lambda(\vec{\phi}_{I_2}) \mathfrak{a}_3 \otimes a_0 \otimes \mathfrak{a}_1$ &
    $b \circ B_{n,m} (\vec{\phi} | \vec{\psi} | \alpha)$ & 
    $\psi_1 \{ \vec{\phi}_{I_1} \} \cdot B_{|I_2|, m-1} (\vec{\phi}_{I_2} | \vec{\psi}_{\{2, \cdots, m \}} | \alpha)$ \\ \hline

    $g_0\phi_1( \mathfrak{a}_2 ) \otimes \lambda(\vec{\psi}) \lambda(\vec{\phi}_{\{2, \cdots, n\}}) \mathfrak{a}_3 \otimes a_0 \otimes \mathfrak{a}_1$ &
    $b \circ B_{n,m} (\vec{\phi} | \vec{\psi} | \alpha)$ & 
    $\phi_1 \cdot B_{n-1, m} (\vec{\phi}_{\{2, \cdots, n\}} | \vec{\psi} | \alpha)$ \\ \hline

    $1 \otimes \lambda(\vec{\psi}) \lambda(\vec{\phi}_{\{1, \cdots, n-1\}}) \mathfrak{a}_2 \otimes g_m \phi_n(\mathfrak{a}_3 \cdot a_0 \otimes \mathfrak{a}_1$ &
    $b \circ B_{n,m} (\vec{\phi} | \vec{\psi} | \alpha)$ & 
    $B_{n-1, m} (\vec{\phi}_{\{1, \cdots, n-1 \}} | \vec{\psi} | \phi_n \cdot \alpha)$ \\ \hline

    $1 \otimes \lambda(\vec{\psi}_{\{1, \cdots, m-1 \}}) \lambda(\vec{\phi}_{I_1}) \mathfrak{a}_2 \otimes g_m \psi_m( \lambda(\vec{\phi}_{I_2} \mathfrak{a}_3) \cdot a_0 \otimes \mathfrak{a}_1$ &
    $b \circ B_{n,m} (\vec{\phi} | \vec{\psi} | \alpha)$ & 
    $B_{|I_1|, m-1} (\vec{\phi}_{I_2} | \vec{\psi}_{\{1, \cdots, m-1\}} | \psi_m \{ \vec{\phi}_{I_2} \} \cdot \alpha)$ \\ \hline

    $g_0f_0a_0 \otimes \lambda(\vec{\psi}) \lambda(\vec{\phi}) \mathfrak{a}_1$ &
    $b \circ B_{n,m} (\vec{\phi} | \vec{\psi} | \alpha)$ & 
    $\upsilon_{|J_1|, |I_1|} (\vec{\psi}_{J_1} | \vec{\phi}_{I_1} | \upsilon_{|I_2|, |J_2|} (\vec{\phi}_{I_2} | \vec{\psi}_{J_2} | \alpha ))$ \\ \hline

    \hline
  \end{tabular}
\caption{Expansion of $``$standard terms'' in 
Equation \ref{eq:upsilon_homotopy} and the 
$``$extra terms'' that cancel them}
\label{table:t21}  
(Technically, the last term in the right column is not an extra term, but we include it in the table for convenience.)
\end{table} 
\end{center}
%
$\newline
\newline$
\begin{table}
\begin{center}
  \begin{tabular}{ p{6.25in} | p{2.5in} }
    \hline
    Expression from Fifth Row & Cancels with Extra Term \\ \hline

    $\psi_1(\lambda(\vec{\phi}_{I_1}) \lambda(\vec{\psi}_{J_2}) \lambda(\vec{\phi}_{I_4}) \mathfrak{a}_4, \phi_{|I_1|+1} (\lambda(\vec{\psi}_{J_3}) \lambda(\vec{\phi}_{I_5}) \mathfrak{a}_5, a_0, \mathfrak{a}_1), \lambda(\vec{\phi}_{I_2 \backslash |I_1| + 1}) \mathfrak{a}_2) \otimes \lambda(\vec{\psi}_{J_1}) \lambda(\vec{\phi}_{I_3}) \mathfrak{a}_3$ &
     $\psi_1 \{ \vec{\phi}_{I_1} \} \cdot B_{|I_2|, m-1} (\vec{\phi}_{I_2} | \vec{\psi}_{\{2,\cdots,m\}} | \alpha)$ \\ \hline

    $\psi_1(\lambda(\vec{\phi}_{I_1}) \lambda(\vec{\psi}_{J_2}) \lambda(\vec{\phi}_{I_4}) \mathfrak{a}_4, f_{|I_1|+1}a_0, \lambda(\vec{\phi}_{I_2 \backslash |I_1| + 1}) \mathfrak{a}_1) \otimes \lambda(\vec{\psi}_{J_1}) \lambda(\vec{\phi}_{I_3}) \mathfrak{a}_2$ &
    $\phi_1 \cdot B_{n-1, m} (\vec{\phi}_{\{2,\cdots,n\}} | \vec{\psi} | \alpha)$ \\ \hline

    $g_0\phi_1(\lambda(\vec{\psi}_{J_2}) \lambda(\vec{\phi}_{I_2}) \mathfrak{a}_3, a_0, \mathfrak{a}_1) \otimes \lambda(\vec{\psi}_{J_1}) \lambda(\vec{\phi}_{I_1}) \mathfrak{a}_2$ &
    $\psi_1 \{ \vec{\phi}_{I_1} \} \cdot B_{|I_2|, m-1} (\vec{\phi}_{I_2} | \vec{\psi}_{\{2,\cdots,m\}} | \alpha)$ \\ \hline

    \hline
  \end{tabular}
\end{center}
\caption{Expansion of remaining $``$fifth-row terms'' in 
Equation \ref{eq:upsilon_homotopy} and the 
$``$extra terms'' that cancel them}
\label{table:t22}
\end{table}
\end{landscape}
\begin{prop}
\label{prop:c3}
Let $\tau_{1!}(A_0,A_1): 
T(A_0 \to A_1 \to A_0) \longrightarrow
T(A_1 \to A_0 \to A_1)$ and 
$B(A_0,A_1): T(A_0 \to A_1 \to A_0) 
\longrightarrow T(A_0 \to A_1 \to A_0)$ 
be the maps defined in Propositions 
\ref{prop:c1} and \ref{prop:c2} above. 
Then, $$[\tau_{1!}, B] := 
\tau_{1!}(A_0,A_1) \circ B(A_0,A_1) - 
B(A_1,A_0) \circ \tau_{1!}(A_0,A_1) = 0.$$
\end{prop}
%
\begin{proof}
We show that $[\tau_{1!}, B] = 0$ by direct 
computation. Since all of the maps are maps 
of cofree comodules, we only need to check 
that $\pi_1([\tau_{1!}, B]) = 0$ where 
$\pi_1$ denotes projection of the comodule 
onto cogenerators. We check this directly.
%
\begin{align*}
&\phantom{=}
[\pi_1 \circ \tau_{1!}(A_0,A_1) \circ B(A_0,A_1)] 
  (\vec{\phi} | \vec{\psi} | \alpha ) \\
&= \sum \limits_{\substack{
  I_1I_2 = \{1,\smdots,n\}\\
  J_1J_2 = \{1,\smdots,m\}\\
  \textrm{as ordered sets}}}
\epsilon_{I_1,J_2} \cdot
  \tau_{1!}^{|I_1|, |J_1|} (\vec{\phi}_{I_1} | \vec{\psi}_{J_1} | 
    B^{|I_2|, |J_2|} (\vec{\phi}_{I_2} | \vec{\psi}_{J_2} | \alpha)) \\
&= 
\sum \limits_{\substack{
  I_1I_2 = \{1,\smdots,n\}\\
  J_1J_2 = \{1,\smdots,m\}\\
  \textrm{as ordered sets}}}
\begin{array}{l}  
\epsilon_{I_1,J_2} \cdot 
\eta_{\mathfrak{a_1},\mathfrak{a_2}} \cdot\\
\tau_{1!}^{|I_1|, |J_1|} (\vec{\phi}_{I_1} | \vec{\psi}_{J_1} | 
  1 \otimes \lambda(\vec{\psi}_{J_2}) \lambda(\vec{\phi}_{I_2}) 
  \mathfrak{a}_2, a_0, \mathfrak{a}_1)
\end{array} \\
&= 
\sum \limits_{\substack{
  I_1I_2 = \{1,\smdots,n\}\\
  J_1J_2 = \{1,\smdots,m\}\\
  \textrm{as ordered sets}}}
\epsilon_{I_1,J_2} \cdot 
\eta_{\mathfrak{a_1},\mathfrak{a_2}} \cdot
1 \otimes \lambda(\vec{\phi}_{I_1}) \big( 
  \lambda(\vec{\psi}) \lambda(\vec{\phi}_{I_2}) 
  \mathfrak{a}_2, a_0, \mathfrak{a}_1 \big)
\end{align*}
%
\begin{align*}
& \phantom{{}={}}
[\pi_1 \circ B(A_1,A_0) \circ \tau_{1!}(A_0,A_1)] 
  (\vec{\phi} | \vec{\psi} | \alpha ) \\
&=
\sum \limits_{\substack{
  I_1I_2 = \{1,\smdots,n\}\\
  J_1J_2 = \{1,\smdots,m\}\\
  \textrm{as ordered sets}}}
\epsilon_{I_1,J_2} \cdot
B^{|J_1|, |I_1|} (\vec{\psi}_{J_1} | \vec{\phi}_{I_1} | 
  \tau_{1!}^{|I_2|, |J_2|} (\vec{\phi}_{I_2} | \vec{\psi}_{J_2} | \alpha)) \\
&= 
\sum \limits_{\substack{
  I_1I_2 = \{1,\smdots,n\}\\
  J_1J_2 = \{1,\smdots,m\}\\
  \textrm{as ordered sets}}}
\begin{array}{l}{}
\epsilon_{I_1,J_2} \cdot
B^{|J_1|, |I_1|} \big(\vec{\psi}_{J_1} | \vec{\phi}_{I_1} | \phi_{|I_1|+1} (
  \lambda(\vec{\psi}_{J_2}) \lambda(\vec{\phi}_{I_3}) 
  \mathfrak{a}_3, a_0, \mathfrak{a}_1) \otimes 
  \lambda(\vec{\phi}_{I_2 \backslash |I_1| + 1}) 
  \mathfrak{a}_2 \; + \\
\hphantom{{}moveovermoveovermoveover{}} 
  + a_0 \otimes \lambda(\vec{\phi}_{I_2 \backslash |I_1| + 1}) 
  \mathfrak{a}_1 \; \; \; 
  \text{if }J_2 = \emptyset \big)
\end{array} \\
&= 
\sum \limits_{\substack{
  I_1I_2 = \{1,\smdots,n\}\\
  J_1J_2 = \{1,\smdots,m\}\\
  \textrm{as ordered sets}}}
\begin{array}{l}  
\epsilon_{I_1,J_2} \cdot  
\eta_{\mathfrak{a}_2,\mathfrak{a}_3} \cdot  
1 \otimes \lambda(\vec{\phi}_{I_1}) \lambda(\vec{\psi}_{J_1}) 
  \lambda(\vec{\phi}_{I_3}) \mathfrak{a}_3 \otimes 
  \phi_{|I_1|+1} (\lambda(\vec{\psi}_{J_2}) \lambda(\vec{\phi}_{I_4}) 
  \mathfrak{a}_4, a_0, \mathfrak{a}_1) \otimes \\
  \phantom{{}moveovermov{}}\otimes 
  \lambda(\vec{\phi}_{I_2 \backslash |I_1| + 1}) 
  \mathfrak{a}_2 \; + 
\end{array}\\
&\hphantom{{}moveovermove{}} 
  +\epsilon_{I_1,J_2} \cdot  
  \eta_{\mathfrak{a}_1,\mathfrak{a}_2} \cdot  
  1 \otimes \lambda(\vec{\phi}_{I_1}) \lambda(\vec{\psi}) 
  \lambda(\vec{\phi}_{I_3}) \mathfrak{a}_2 \otimes 
  a_0 \otimes \lambda(\vec{\phi}_{I_2}) \mathfrak{a}_1
\end{align*}
%
It's clear that $\pi_1 \circ \tau_{1!}(A_0,A_1) 
\circ B(A_0,A_1) =  \pi_1 \circ B(A_1,A_0) 
\circ \tau_{1!}(A_0,A_1)$: The final expansion of 
$\pi_1 \circ \tau_{1!}(A_0,A_1) \circ B(A_0,A_1)$ 
is the sum of the two terms in the final expansion 
of $\pi_1 \circ B(A_1,A_0) \circ \tau_{1!}(A_0,A_1)$, 
which is the sum of terms in which one of 
the $\phi$'s contains $a_0$ and the terms in which 
none of the $\phi$'s contains $a_0$).
\end{proof}

\section{More notation}
For the next two propositions, we will need 
some more notation. Set
\begin{align*}
A_0, A_1, A_2
&\textrm{ fixed algebras}\\
(\vec{\phi} | \vec{\psi} | \vec{\theta}| \alpha) 
&:= 
(\phi_1\smdots\phi_n | \psi_1\smdots\psi_m| 
  \theta_1 \smdots \theta_r|\alpha)\\
&= 
\xymatrix{
A_0 \ar@/^5pc/[r]^{f_0} 
\ar@/^2pc/[r]^{\big\Downarrow \phi_1}_{f_1} 
\ar@/_2pc/[r]^{\substack{\vdots\\ f_n}}
\ar@/_5pc/[rrr]_{id}^{\substack{\alpha \\ \\ \\ \\ }}
& A_1 \ar@/^5pc/[r]^{g_0} 
\ar@/^2pc/[r]^{\big\Downarrow \psi_1}_{g_1} 
\ar@/_2pc/[r]^{\substack{\vdots\\ g_m}}
& A_2 \ar@/^5pc/[r]^{h_0} 
\ar@/^2pc/[r]^{\big\Downarrow \theta_1}_{h_1} 
\ar@/_2pc/[r]^{\substack{\vdots\\ h_p}}
& A_0
}\\
&\in 
C(A_0 \to A_1 \to A_2 \to A_0)(h_0g_0f_0)\\
\Upsilon_{A_0\bullet A_1, A_2}:
  C(A_0 \to A_1 \to A_2 \to A_0) 
&\to
\hat{\tau}_2^*C(A_2 \to A_0 \to A_1 \to A_2)
  \quad \textrm{ map of dg comodules}\\
(\vec{\phi} | \vec{\psi} | \vec{\theta}| \alpha) 
&\mapsto 
\Upsilon_{A_0,A_2}
  (\vec{\phi} \bullet \vec{\psi} | \vec{\theta}| \alpha)\\
\Upsilon_{A_0,A_1\bullet A_2}:
  C(A_0 \to A_1 \to A_2 \to A_0) 
&\to
\hat{\tau}_2^{*2}C(A_1 \to A_2 \to A_0 \to A_1)
  \quad \textrm{ map of dg comodules}\\
(\vec{\phi} | \vec{\psi} | \vec{\theta}| \alpha) 
&\mapsto 
\Upsilon_{A_0,A_1}
  (\vec{\phi}|\vec{\psi} \bullet \vec{\theta}| \alpha)\\
\end{align*}

\section{More Propositions}
% \begin{figure} \label{fig:upsilon}
% \centerline{\xymatrix{
% A_0 \ar@/^5pc/[r]^{f_0} 
% \ar@/^2pc/[r]^{\big\Downarrow \phi_1}_{f_1} 
% \ar@/_2pc/[r]^{\substack{\vdots\\ f_n}}
% \ar@/_5pc/[rrr]_{id}^{\substack{\alpha \\ \\ \\ \\ }}
% & A_1 \ar@/^5pc/[r]^{g_0} 
% \ar@/^2pc/[r]^{\big\Downarrow \psi_1}_{g_1} 
% \ar@/_2pc/[r]^{\substack{\vdots\\ g_m}}
% & A_2 \ar@/^5pc/[r]^{h_0} 
% \ar@/^2pc/[r]^{\big\Downarrow \theta_1}_{h_1} 
% \ar@/_2pc/[r]^{\substack{\vdots\\ h_p}}
% & A_0
% }
% $\overset{\mathcal{B}}\longrightarrow$
% \xymatrix{
% A_2 \ar@/^5pc/[r]^{h_0} 
% \ar@/^2pc/[r]^{\big\Downarrow \theta_1}_{h_1} 
% \ar@/_2pc/[r]^{\substack{\vdots\\ h_p}}
% \ar@/_5pc/[rrr]_{id}^{\substack{\alpha \\ \\ \\ \\ }}
% & A_0 \ar@/^5pc/[r]^{f_0} 
% \ar@/^2pc/[r]^{\big\Downarrow \phi_1}_{f_1} 
% \ar@/_2pc/[r]^{\substack{\vdots\\ f_n}}
% & A_1 \ar@/^5pc/[r]^{g_0} 
% \ar@/^2pc/[r]^{\big\Downarrow \psi_1}_{g_1} 
% \ar@/_2pc/[r]^{\substack{\vdots\\ g_m}}
% & A_2
% }}
% \caption{A picture of the domain and target of $\mathcal{B}$}
% \end{figure}
%
\begin{prop} \label{prop:c4}
Let 
$$
\mathcal{B}: C(A_0 \to A_1 \to A_2 \to A_0)
\to \hat{\tau}_2^{*2}C(A_1 \to A_2 \to A_0 \to A_1)
$$ 
be a map of comodules over 
$B(A_0 \to A_1 \to A_2 \to A_0)$ 
determined by the following maps to 
cogenerators:
\begin{align*}
\mathcal{B}^{f_0, g_0,h_0}: 
  C(A_0 \to A_1 \to A_0)(h_0g_0f_0) 
&\to
\hat{\tau}_2^{*2}C(A_1 \to A_2 \to A_0 \to A_1)
  (f_0h_0g_0)\\
&\xrightarrow[cogenerators]{\textrm{project onto}}
C_{-\bullet}(A_1, _{f_0h_0g_0}{A_1}_{id})\\
\mathcal{B}_{n, m, p} (\vec{\phi} | \vec{\psi} | \vec{\theta} | \alpha) 
= & \sum_{\substack{I_1I_2 = \{1,2,\cdots,n\} \\
                          \textrm{as ordered sets}}}
  1 \otimes \lambda(\vec{\phi}_{I_1})\big( \lambda(\vec{\theta}) \lambda(\vec{\psi}) \lambda(\vec{\phi}_{I_2})
  \mathfrak{a}_2 \otimes a_0 \otimes \mathfrak{a}_1 \big)               
\end{align*}
Then, 
\begin{equation} \label{eq:prop4}
D(\mathcal{B}) = 
  \Upsilon_{A_2\bullet A_0, A_1} \circ
  \Upsilon_{A_0\bullet A_1, A_2} 
   - \Upsilon_{A_0, A_1\bullet A_2}.
\end{equation}
\end{prop}

\begin{proof}
We will show that Equation \ref{eq:prop4} 
holds by direct computation. Since all of 
the maps are maps of cofree comodules, we 
only need to check that $\pi_1($
Equation \ref{eq:prop4}$)$ holds where 
$\pi_1$ denotes projection of the comodule 
onto cogenerators. More explicitly, we 
want to check that
\begin{equation} \label{eq:prop4_expand}
\begin{aligned}
%hochschild cochain delta
\mathcal{B}_{n, m, p} ( \tilde{\delta}(\vec{\phi}) | \vec{\psi} | \vec{\theta} | \alpha ) \; + 
\mathcal{B}_{n, m, p} ( \vec{\phi} | \tilde{\delta}(\vec{\psi}) | \vec{\theta} | \alpha ) \; + 
\mathcal{B}_{n, m, p} ( \vec{\phi} | \vec{\psi} | \tilde{\delta}(\vec{\theta}) | \alpha ) \; + \\
%cochain b prime
\mathcal{B}_{n-1, m, p} ( b^\prime(\vec{\phi}) | \vec{\psi} | \vec{\theta} | \alpha ) \; + 
\mathcal{B}_{n, m-1, p} ( \vec{\phi} | b^\prime(\vec{\psi}) | \vec{\theta} | \alpha ) \; + 
\mathcal{B}_{n, m, p-1} ( \vec{\phi} | \vec{\psi} | b^\prime(\vec{\theta}) | \alpha ) \; + \\
%chain b
\mathcal{B}_{n, m, p} ( \vec{\phi} | \vec{\psi} | \vec{\theta} | b(\alpha) ) \; + 
b \circ \mathcal{B}_{n, m, p} ( \vec{\phi} | \vec{\psi} | \vec{\theta} | \alpha ) \; + \\
%twist before
 \mathcal{B}_{|I_1|, |J_1|, p-1}(\vec{\phi}_{I_1} | \vec{\psi}_{J_1} | \vec{\theta}_{\{1,\cdots, p-1\}} |
     \theta_{p} \{\vec{\psi}_{J_2}\} \{\vec{\phi}_{I_2}\} \cdot \alpha ) \; + \\
 \mathcal{B}_{|I_1|, m-1, p}(\vec{\phi}_{I_1} | \vec{\psi}_{\{1,\cdots, m-1\}} | \vec{\theta} |
     \psi_{m} \{\vec{\phi}_{I_2}\}\cdot \alpha ) \; + 
\mathcal{B}_{n-1, m, p}(\vec{\phi}_{\{1,\cdots, n-1\}} |\vec{\psi}_{m} | \vec{\theta} | 
     \phi_{n} \cdot \alpha) \; + \\
%twist after
\phi_1 \{\vec{\theta}_{K_1}\} \{\vec{\psi}_{J_1}\} \cdot
     \mathcal{B}_{n-1, |J_2|, |K_2|}
     (\vec{\phi}_{\{2,\cdots,n\}} | \vec{\psi}_{J_2} | \vec{\theta}_{K_2} | \alpha) \; + \\
\theta_1 \{\vec{\psi}_{J_1}\} \cdot
     \mathcal{B}_{n, |J_2|, p-1}
     (\vec{\phi} | \vec{\psi}_{J_2} | \vec{\theta}_{\{2,\cdots,p\}} | \alpha) \; +
\psi_1 \cdot
     \mathcal{B}_{n, m-1, p}
     (\vec{\phi} | \vec{\psi}_{\{2,\cdots,m\}} | \vec{\theta} | \alpha) \; + \\
%prop4
\upsilon_{n, p \leq * \leq m+p}(\vec{\phi} | \vec{\psi} \bullet \vec{\theta} | \alpha ) \; + \\
\upsilon_{|I_1| \leq * \leq |I_1| + |K_1|,|J_1|}(\vec{\theta}_{K_1} \bullet \vec{\phi}_{I_1}, \vec{\psi}_{J_1}, 
    \upsilon_{|J_2| \leq * \leq |I_2| + |J_2|,|K_2|}(\vec{\phi}_{I_2} \bullet \vec{\psi}_{J_2} | \vec{\theta}_{K_2} | \alpha )) \\
%
=0.
\end{aligned}
\end{equation}
In Equation \ref{eq:prop4_expand} above, 
we call the terms in rows 1-3 the 
``standard terms'' in the computation of 
$D(\mathcal{B})$, and the terms in rows 
4-7 the ``extra terms'' in the computation 
of $D(\mathcal{B})$. The terms in rows 8-9 
are $\pi_1$ of the righthand side of Equation 
\ref{eq:prop4}; we will call these the 
``8$^{th}$- and 9$^{th}$-row terms''.

We compute the sum of the standard terms. 
In the chart below, the leftmost column lists 
the expressions that don't cancel in the sum 
of the standard terms, the middle column gives 
the standard term from which the expression comes, 
and the rightmost column gives the term that 
cancels the expression. 
\newpage
\begin{landscape}
\begin{center}
  \begin{tabular}{ p{3.25in} | p{1.75in} | p{2.75in} }
    \hline
    Expression & Comes from Standard Term & Cancelling Term \\ \hline
    %
    % twist before
    $1 \otimes \lambda(\vec{\phi}_{I_1}) [
    \lambda(\vec{\theta}_{\{1,\cdots,p-1\}}
    \lambda(\vec{\psi}_{J_1})
    \lambda(\vec{\phi}_{I_2})
    \mathfrak{a}_2 \otimes 
    \theta_p(\lambda(\vec{\psi}_{J_2}) \lambda(\vec{\phi}_{I_3}) \mathfrak{a}_3) \cdot a_0 \otimes
    \mathfrak{a}_1 ]$ & 
    $b \circ \mathcal{B}_{n,m,p} (\vec{\phi} | \vec{\psi} | \vec{\theta} | \alpha)$ & 
    $\mathcal{B}_{|I_1|, |J_1|, p-1}(\vec{\phi}_{I_1} | \vec{\psi}_{J_1} | \vec{\theta}_{\{1,\cdots, p-1\}} |
     \theta_{p} \{\vec{\psi}_{J_2}\} \{\vec{\phi}_{I_2}\} \cdot \alpha )$ \\ \hline

    $1 \otimes \lambda(\vec{\phi}_{I_1}) [
    \lambda(\vec{\theta}
    \lambda(\vec{\psi}_{\{1,\cdots,m-1\}})
    \lambda(\vec{\phi}_{I_2})
    \mathfrak{a}_2 \otimes 
    \psi_m(\lambda(\vec{\phi}_{I_3}) \mathfrak{a}_3) \cdot a_0 \otimes
    \mathfrak{a}_1 ]$ & 
    $b \circ \mathcal{B}_{n,m,p} (\vec{\phi} | \vec{\psi} | \vec{\theta} | \alpha)$ & 
    $\mathcal{B}_{|I_1|, m-1, p}(\vec{\phi}_{I_1} | \vec{\psi}_{\{1,\cdots, m-1\}} | \vec{\theta} |
     \psi_m \{\vec{\phi}_{I_2}\} \cdot \alpha )$ \\ \hline

    $1 \otimes \lambda(\vec{\phi}_{I_1}) [
    \lambda(\vec{\theta}
    \lambda(\vec{\psi}
    \lambda(\vec{\phi}_{\{1,\cdots,n-1\}})
    \mathfrak{a}_2 \otimes 
    \psi_n(\mathfrak{a}_3) \cdot a_0 \otimes
    \mathfrak{a}_1 ]$ & 
    $b \circ \mathcal{B}_{n,m,p} (\vec{\phi} | \vec{\psi} | \vec{\theta} | \alpha)$ & 
    $\mathcal{B}_{n-1, m, p}(\vec{\phi}_{\{1,\cdots, n-1\}} | \vec{\psi} | \vec{\theta} |
     \phi_n \cdot \alpha )$ \\ \hline
    %
    %twist after
    $\phi_1(\lambda(\vec{\theta}_{K_1}) \lambda(\vec{\psi}_{J_1}) \lambda(\vec{\phi}_{I_2}) \mathfrak{a}_2)
    \otimes \lambda(\vec{\phi}_{I_1\backslash 1})[
    \lambda(\vec{\theta}_{K_2}) \lambda(\vec{\psi}_{J_3}) \lambda(\vec{\phi}_{I_3}) \mathfrak{a}_3
    \otimes a_0 \otimes \mathfrak{a}_1]$ &
    $b \circ \mathcal{B}_{n,m,p} (\vec{\phi} | \vec{\psi} | \vec{\theta} | \alpha)$ & 
    $\phi_1 \{\vec{\theta}_{K_1}\} \{\vec{\psi}_{J_1}\} \cdot
     \mathcal{B}_{n-1, |J_2|, |K_2|}
     (\vec{\phi}_{\{2,\cdots,n\}} | \vec{\psi}_{J_2} | \vec{\theta}_{K_2} | \alpha)$ \\ \hline

    $f_0\theta_1( \lambda(\vec{\psi}_{J_1}) \lambda(\vec{\phi}_{I_2}) \mathfrak{a}_2)
    \otimes \lambda(\vec{\phi}_{I_1})[
    \lambda(\vec{\theta}_{\{2,\cdots,p\}}) \lambda(\vec{\psi}_{J_2}) \lambda(\vec{\phi}_{I_3}) \mathfrak{a}_3
    \otimes a_0 \otimes \mathfrak{a}_1]$ &
    $b \circ \mathcal{B}_{n,m,p} (\vec{\phi} | \vec{\psi} | \vec{\theta} | \alpha)$ & 
    $\theta_1 \{\vec{\psi}_{J_1}\} \cdot
     \mathcal{B}_{n, |J_2|, p-1}
     (\vec{\phi} | \vec{\psi}_{J_2} | \vec{\theta}_{\{2,\cdots,p\}} | \alpha)$ \\ \hline

    $f_0h_0\psi_1( \lambda(\vec{\phi}_{I_2}) \mathfrak{a}_2)
    \otimes \lambda(\vec{\phi}_{I_1})[
    \lambda(\vec{\theta}) \lambda(\vec{\psi}_{\{2,\cdots,m\}}) \lambda(\vec{\phi}_{I_3}) \mathfrak{a}_3
    \otimes a_0 \otimes \mathfrak{a}_1]$ &
    $b \circ \mathcal{B}_{n,m,p} (\vec{\phi} | \vec{\psi} | \vec{\theta} | \alpha)$ & 
    $\psi_1 \cdot
     \mathcal{B}_{n, m-1, p}
     (\vec{\phi} | \vec{\psi}_{\{2,\cdots,m\}} | \vec{\theta} | \alpha)$ \\ \hline

    %wrap around
    $f_0h_0g_0 \phi_{i_1} ( \lambda(\vec{\theta}_{K_2}) \lambda(\vec{\psi}_{J_2}) \lambda(\vec{\phi}_{I_3})
       \mathfrak{a}_3 \otimes a_0 \otimes \mathfrak{a}_1) \otimes
       \lambda(\vec{\phi}_{I_1}) \lambda(\vec{\theta}_{K_1}) \lambda(\vec{\psi}_{J_1}) 
       \lambda(\vec{\phi}_{I_2 \backslash i_1}) \mathfrak{a}_2$ &
    $b \circ \mathcal{B}_{n,m,p} (\vec{\phi} | \vec{\psi} | \vec{\theta} | \alpha)$ & 
    9$^{th}$ row \\ \hline

    $f_0h_0g_0f_{i_1}a_0 \otimes \lambda(\vec{\phi}_{I_1}) \lambda(\vec{\theta}) 
       \lambda(\vec{\psi}_{J_1}) \lambda(\vec{\phi}_{I_2}) \mathfrak{a}_1$ &
    $b \circ \mathcal{B}_{n,m,p} (\vec{\phi} | \vec{\psi} | \vec{\theta} | \alpha)$ & 
    9$^{th}$ row \\ \hline

    $\phi_1( \lambda(\vec{\phi}_{I_1}) \lambda(\vec{\theta}) 
       \lambda(\vec{\psi}_{J_1}) \lambda(\vec{\phi}_{I_2}) \mathfrak{a}_3, a_0, \mathfrak{a}_1 ) \otimes
       \lambda(\vec{\phi}_{I_1 \backslash 1}) \mathfrak{a}_2$ &
    $b \circ \mathcal{B}_{n,m,p} (\vec{\phi} | \vec{\psi} | \vec{\theta} | \alpha)$ & 
    8$^{th}$ row \\ \hline

  \end{tabular}
\end{center}
\end{landscape}

\newpage
\begin{landscape}

Now, we compute the remaining terms from the ninth row. In the chart below, the left column lists the remaining expressions that don't cancel in the ninth row, and the right column gives the extra term that cancels the expression. 

\begin{center}
  \begin{tabular}{ p{6.25in} | p{2.5in} }
    \hline
    Expression from ninth Row & Cancels with Extra Term \\ \hline
    %twist after
    $\phi_1(\lambda(\vec{\theta}_{K_1}) \lambda(\vec{\psi}_{J_1}) \lambda(\vec{\phi}_{I_2}) [
      \lambda(\vec{\theta}_{K_3}) \lambda(\vec{\psi}_{J_4}) \lambda(\vec{\phi}_{I_5})
      \mathfrak{a}_3, a_0, \mathfrak{a}_1])
      \otimes \lambda(\vec{\phi}_{I_1\backslash 1}) \lambda(\vec{\theta}_{K_2}) 
      \lambda(\vec{\psi}_{J_3}) \lambda(\vec{\phi}_{I_4}) \mathfrak{a}_2$ &
    $\phi_1 \{\vec{\theta}_{K_1}\} \{\vec{\psi}_{J_1}\} \cdot
     \mathcal{B}_{n-1, |J_2|, |K_2|}
     (\vec{\phi}_{\{2,\cdots,n\}} | \vec{\psi}_{J_2} | \vec{\theta}_{K_2} | \alpha)$ \\ \hline

    $f_0\theta_1( \lambda(\vec{\psi}_{J_1}) \lambda(\vec{\phi}_{I_2}) [
      \lambda(\vec{\theta}_{K_2}) \lambda(\vec{\psi}_{J_3}) \lambda(\vec{\phi}_{I_4})
      \mathfrak{a}_3, a_0, \mathfrak{a}_1])
      \otimes \lambda(\vec{\phi}_{I_1}) \lambda(\vec{\theta}_{K_1 \backslash 1}) 
      \lambda(\vec{\psi}_{J_2}) \lambda(\vec{\phi}_{I_3}) \mathfrak{a}_2$ &
    $\theta_1 \{\vec{\psi}_{J_1}\} \cdot
     \mathcal{B}_{n, |J_2|, p-1}
     (\vec{\phi} | \vec{\psi}_{J_2} | \vec{\theta}_{\{2,\cdots,p\}} | \alpha)$ \\ \hline

    $f_0h_0\psi_1( \lambda(\vec{\phi}_{I_2}) [
      \lambda(\vec{\theta}_{K_2}) \lambda(\vec{\psi}_{J_2}) \lambda(\vec{\phi}_{I_4})
      \mathfrak{a}_3, a_0, \mathfrak{a}_1])
      \otimes \lambda(\vec{\phi}_{I_1}) \lambda(\vec{\theta}_{K_1}) 
      \lambda(\vec{\psi}_{J_1 \backslash 1}) \lambda(\vec{\phi}_{I_3}) \mathfrak{a}_2$ & 
    $\psi_1 \cdot
     \mathcal{B}_{n, m-1, p}
     (\vec{\phi} | \vec{\psi}_{\{2,\cdots,m\}} | \vec{\theta} | \alpha)$ \\ \hline

    \hline
  \end{tabular}
\end{center}
\end{landscape}

All of the terms in the table describing the expansion of Equation \ref{eq:prop4_expand} cancel, so we're done.
\end{proof}
% \begin{figure} \label{fig:upsilon}
% \centerline{\xymatrix{
% A_0 \ar@/^5pc/[r]^{f_0} 
% \ar@/^2pc/[r]^{\big\Downarrow \phi_1}_{f_1} 
% \ar@/_2pc/[r]^{\substack{\vdots\\ f_n}}
% \ar@/_5pc/[rrr]_{id}^{\substack{\alpha \\ \\ \\ \\ }}
% & A_1 \ar@/^5pc/[r]^{g_0} 
% \ar@/^2pc/[r]^{\big\Downarrow \psi_1}_{g_1} 
% \ar@/_2pc/[r]^{\substack{\vdots\\ g_m}}
% & A_2 \ar@/^5pc/[r]^{h_0} 
% \ar@/^2pc/[r]^{\big\Downarrow \theta_1}_{h_1} 
% \ar@/_2pc/[r]^{\substack{\vdots\\ h_p}}
% & A_0
% }
% $\overset{[d_0^* \Upsilon, \mathcal{B}]}\longrightarrow$
% \xymatrix{
% A_0 \ar@/^5pc/[r]^{f_0} 
% \ar@/^2pc/[r]^{\big\Downarrow \phi_1}_{f_1} 
% \ar@/_2pc/[r]^{\substack{\vdots\\ f_n}}
% \ar@/_5pc/[rrr]_{id}^{\substack{\alpha \\ \\ \\ \\ }}
% & A_1 \ar@/^5pc/[r]^{g_0} 
% \ar@/^2pc/[r]^{\big\Downarrow \psi_1}_{g_1} 
% \ar@/_2pc/[r]^{\substack{\vdots\\ g_m}}
% & A_2 \ar@/^5pc/[r]^{h_0} 
% \ar@/^2pc/[r]^{\big\Downarrow \theta_1}_{h_1} 
% \ar@/_2pc/[r]^{\substack{\vdots\\ h_p}}
% & A_0
% }}
% \caption{A picture of the domain and target of $[d_0^* \Upsilon, \mathcal{B}]$}
% \end{figure}

\begin{prop}
\label{prop:c5}
Let $\Upsilon$ and $\mathcal{B}$ be as 
defined in the previous propositions. 
Then, $[\Upsilon, \mathcal{B}] := 
\Upsilon_{A_1\bullet A_2, A_0} 
\mathcal{B}_{A_0,A_1,A_2} - 
\mathcal{B}_{A_2, A_0, A_1} 
\Upsilon_{A_0 \bullet A_1, A_2} = 0$. 
(Note that $[\Upsilon, \mathcal{B}]$ is 
a map from $C(A_0 \to A_1 \to A_2 \to A_0)$ 
to itself.)
\end{prop}
%
\begin{proof}
We show the proposition by direct computation. 
Since all of the maps are maps of cofree 
comodules, we only need to check that 
$\pi_1([\Upsilon, \mathcal{B}]) = 0$ where 
$\pi_1$ denotes projection of the comodule 
onto cogenerators. We check this directly.
%
\begin{equation*}
\begin{aligned}
&\phantom{{}={}}
\pi_1 \circ \Upsilon_{A_1\bullet A_2, A_0} 
  \mathcal{B}_{A_0,A_1,A_2} 
  (\vec{\phi} | \vec{\psi} | \vec{\theta} | \alpha ) \\
&= 
\sum \limits_{\substack{
  I_1I_2 = \{1,\smdots,n\}\\
  J_1J_2 = \{1,\smdots,m\}\\
  K_1K_2 = \{1,\smdots,p\}\\
  \textrm{as ordered sets}}}
\epsilon_{I_1,J_1,K_1}\cdot
\upsilon_{|K_1| \leq * \leq |K_1|+|J_1|, |I_1|} (
   \vec{\psi}_{J_1} \bullet \vec{\theta}_{K_1} | 
   \vec{\phi}_{I_1} | 
   \mathcal{B}_{|I_2|, |J_2|, |K_2|} (
   \vec{\phi}_{I_2} | \vec{\psi}_{J_2} | \vec{\theta}_{K_2} | \alpha)) \\
&= 
\sum \limits_{\substack{
  I_1I_2 = \{1,\smdots,n\}\\
  J_1J_2 = \{1,\smdots,m\}\\
  K_1K_2 = \{1,\smdots,p\}\\
  \textrm{as ordered sets}}}
\epsilon_{I_1,J_1,K_1}\cdot
\eta_{\mathfrak{a}_1, \mathfrak{a}_2} \cdot
\upsilon_{|K_1| \leq * \leq |K_1|+|J_1|, |I_1|} (
   \vec{\psi}_{J_1} \bullet \vec{\theta}_{K_1} | 
   \vec{\phi}_{I_1} |
   1 \otimes \lambda(\vec{\phi}_{I_2})[
     \lambda(\vec{\theta}_{K_2}) \lambda(\vec{\psi}_{J_2}) 
     \lambda(\vec{\phi}_{I_3}) 
     \mathfrak{a}_2, a_0, \mathfrak{a}_1] \\
&= 
\sum \limits_{\substack{
  I_1I_2 = \{1,\smdots,n\}\\
  J_1J_2 = \{1,\smdots,m\}\\
  K_1K_2 = \{1,\smdots,p\}\\
  \textrm{as ordered sets}}}
\epsilon_{I_1,J_1,K_1}\cdot
\eta_{\mathfrak{a}_1, \mathfrak{a}_2} \cdot
1 \otimes \lambda(\vec{\theta}_{K_1}) \lambda(\vec{\psi}_{J_1}) 
  \lambda(\vec{\phi}_{I_1})[
     \lambda(\vec{\theta}_{K_2}) \lambda(\vec{\psi}_{J_2}) 
     \lambda(\vec{\phi}_{I_2})
     \mathfrak{a}_2, a_0, \mathfrak{a}_1]    
\end{aligned}
\end{equation*}
%
\begin{align*}
& \phantom{{}={}}
\pi_1 \circ \mathcal{B}_{A_2, A_0, A_1} 
  \Upsilon_{A_0 \bullet A_1, A_2} 
  (\vec{\phi} | \vec{\psi} | \vec{\theta} | \alpha ) \\
&= 
\sum \limits_{\substack{
  I_1I_2 = \{1,\smdots,n\}\\
  J_1J_2 = \{1,\smdots,m\}\\
  K_1K_2 = \{1,\smdots,p\}\\
  \textrm{as ordered sets}}}
\epsilon_{I_1,J_1,K_1}\cdot
B_{|K_1|, |I_1|, |J_1|} 
   (\vec{\theta}_{K_1} | \vec{\phi}_{I_1} | \vec{\psi}_{J_1} | 
   \upsilon_{|J_2| \leq * \leq |I_2|+|J_2|, |K_2|} 
   (\vec{\phi}_{I_2} \bullet \vec{\psi}_{J_2} | \vec{\theta}_{K_2} | \alpha)) \\
&= 
\sum \limits_{\substack{
  I_1I_2 = \{1,\smdots,n\}\\
  J_1J_2 = \{1,\smdots,m\}\\
  K_1K_2 = \{1,\smdots,p\}\\
  \textrm{as ordered sets}}}
\epsilon_{I_1,J_1,K_1}\cdot
\eta_{\mathfrak{a}_1, \mathfrak{a}_2} \cdot
1 \otimes \lambda(\vec{\theta}_{K_1}) \lambda(\vec{\psi}_{J_1}) 
  \lambda(\vec{\phi}_{I_1})[
     \lambda(\vec{\theta}_{K_2}) \lambda(\vec{\psi}_{J_2}) 
     \lambda(\vec{\phi}_{I_2})
     \mathfrak{a}_2, a_0, \mathfrak{a}_1]  
\end{align*}
where $\epsilon_{I_1,J_1,K_1} = 
(-1)^{(\sum \limits_{r \in I_2}|\phi_r|+1)
  ((\sum \limits_{s \in J_1}|\psi_s|+1) + 
  (\sum \limits_{t \in K_1}|\theta_t|+1)) + 
  (\sum \limits_{s \in J_2}|\psi_s|+1)
  (\sum \limits_{t \in K_1}|\theta_t|+1)}$
and $\eta_{\mathfrak{a}_1, \mathfrak{a}_2} = 
(-1)^{|\mathfrak{a}_1|(|\mathfrak{a}_1+ 
\mathfrak{a}_2|)}$.
It's clear that $\pi_1([\Upsilon, \mathcal{B}]) = 0$.
\end{proof}
\chapter{Background on Hochschild chains and cochains}
In this section, we give some known 
constructions on Hochschild chains and 
cochains for the reader's convenience. 
%
\section{Standard constructions and notation}
Let $k$ be a field of characteristic zero, 
$A$ a flat unital $k$-algebra, and $M$ be an 
$A$-$A$-bimodule. Then, we can take 
$(C_\bullet(A,M), b)$, the 
(reduced or standard) Hochschild chain 
complex of $A$ 
with coefficients in $M$ (see Reference 
\cite{T}, Equation 2.1). When $M = B$ is also 
an algebra over $k$ with left and right 
module structure given by two maps of algebras 
$f:A \to B$ and $g:A \to B$, respectively, 
we may write $C_\bullet(A,_fB_g)$ to clarify 
the module structure.

Let $k, A, M$ be as above. We can also 
take $(C^\bullet(A,M), \delta)$, the 
(reduced) Hochschild cochain complex of $A$ 
with coefficients in $M$ (see Reference 
\cite{T}, Equation 2.12). When $M=B$ is 
an algebra, $(C^\bullet(A,B), \delta, \cup)$ 
is a dga where the cup product $\cup$ is 
given in Reference \cite{T}, Equation 2.14.

\subsection{Bar complexes}
% \tilde{\delta} 
% &:= 
% \substack{\textrm{extension of the Hochschild 
% cochain differential}\\\textrm{to a differential on 
% a bar complex}}\\
% b^\prime 
% &:= 
% \sum_{i=0}^{r-1} (-1)^{i}b_i, 
%   \textrm{ for appropriate r}\\
% b_i 
% &:= \textrm{cup product on Hochschild cochains 
% between the $i^{th}$ and $i+1^{th}$ terms}\\

%\lambda()
% , $\iota$ is 
% defined in Equation \ref{eq:action_term}, and 
% the braces $\cdot \{ \cdot \}$ are defined in ....



\section{Brace operation on Hochschild cochains}
\label{sec:def_braces}
Fix algebras $A_0, A_1$ and maps of algebras 
$f_0,f_n: A_0 \rightrightarrows A_1$, 
$A_0 \leftleftarrows A_1: g_0, g_m$. We will 
define a map of complexes 
\begin{align*}
&\phantom{{}={}}
C(A_0 \to A_1 \to A_0)((f_0,g_0), (f_n,g_m))\\
&:=
\Big( \bigoplus \limits_{\substack{
  i \in \mathbb{N}\\
  f_1,\smdots,f_i \textrm{ maps of algebras}\\
  f_{i+1} = f_n}}
  C^\bullet(A_0, _{f_0}{A_1}_{f_1}) \otimes \smdots \otimes 
  C^\bullet(A_0, _{f_i}{A_1}_{f_{i+1}}) \Big) \otimes \\
&\phantom{{}={}}  
\Big( \bigoplus \limits_{\substack{
	j \in \mathbb{N}\\
	g_1,\smdots,g_j \textrm{ maps of algebras}\\
	g_{j+1} = g_m}}
  C^\bullet(A_1, _{g_0}{A_0}_{g_1}) \otimes \smdots \otimes 
  C^\bullet(A_1, _{g_j}{A_0}_{g_{j+1}}) \Big)\\
&\phantom{{}moveovermoveovermoveover{}}
  \Big\downarrow -\bullet -\\
&\phantom{{}={}}
C(A_0 \to A_0)(g_0f_0, g_mf_n)\\
&:=
\bigoplus \limits_{\substack{
  i \in \mathbb{N}\\
  h_1,\smdots,h_i \textrm{ maps of algebras}\\
  h_{i+1} = g_mf_n}}
  C^\bullet(A_0, _{g_0f_0}{A_0}_{h_1}) \otimes \smdots \otimes 
  C^\bullet(A_0, _{h_i}{A_0}_{h_{i+1}}).
\end{align*}
First, for 
$$
(\phi_1\smdots\phi_n | 1) = 
\xymatrix{
A_0 
\ar@/^5pc/[r]^{f_0} 
\ar@/^2pc/[r]^{\big\Downarrow \phi_1}_{f_1} 
\ar@/_2pc/[r]^{\substack{\vdots\\ f_n}}
& A_0 
\ar@/^2pc/[r]^{}^{id}_{\big\Downarrow 1}
\ar@/_2pc/[r]_{id}
& A_0
}
\quad \textrm{and} \quad
(1|\phi_1\smdots\phi_n) = 
\xymatrix{
A_0
\ar@/^2pc/[r]^{}^{id}_{\big\Downarrow 1}
\ar@/_2pc/[r]_{id}
& A_0 
\ar@/^5pc/[r]^{f_0} 
\ar@/^2pc/[r]^{\big\Downarrow \phi_1}_{f_1} 
\ar@/_2pc/[r]^{\substack{\vdots\\ f_n}}
& A_0,
}$$
define $(\phi_1\smdots\phi_n | 1) 
\overset{\bullet}{\mapsto} 
(\phi_1\smdots\phi_n) \bullet 1 = 
(\phi_1\smdots\phi_n)$ and 
$(1| \phi_1\smdots\phi_n) 
\overset{\bullet}{\mapsto} 
1 \bullet (\phi_1\smdots\phi_n) = 
(\phi_1\smdots\phi_n)$.
Then, for $n,m\geq 1$, let 
$$
(\phi_1\smdots\phi_n | \psi_1\smdots\psi_m)
=
\xymatrix{
A_0 \ar@/^5pc/[r]^{f_0} 
\ar@/^2pc/[r]^{\big\Downarrow \phi_1}_{f_1} 
\ar@/_2pc/[r]^{\substack{\vdots\\ f_n}}
& A_1 \ar@/^5pc/[r]^{g_0} 
\ar@/^2pc/[r]^{\big\Downarrow \psi_1}_{g_1} 
\ar@/_2pc/[r]^{\substack{\vdots\\ g_m}}
& A_0
}
$$
and define $(\phi_1\smdots\phi_n) \bullet
(\psi_1\smdots\psi_m) \in C(A_0 \to A_0)
(g_0f_0, g_mf_n)$ as follows:
\begin{align*}
&\phantom{{}={}}
(\phi_1\smdots\phi_n) \bullet
(\psi_1\smdots\psi_m)\\
&= 
\sum \limits_{
  0 \leq i_1 \leq \smdots
  \leq i_{2m} \leq n
}
  g_0\phi_1 \otimes \smdots \otimes g_0\phi_{i_1} \otimes 
  \psi_1\{\phi_{i_1+1}\smdots \phi_{i_2}\} \otimes\\
&\phantom{{}move{}}  
  \otimes g_1\phi_{i_2+1} \otimes \smdots \otimes g_1\phi_{i_3} 
  \otimes \psi_2\{\phi_{i_3+1}\smdots \phi_{i_4}\} \otimes \\
&\phantom{{}move{}}    
  \otimes \dots \otimes
  \psi_m\{\phi_{i_{2m-1}+1}\smdots \phi_{i_{2m}}\} \otimes \\
&\phantom{{}move{}}    
  \otimes g_m\phi_{i_{2m}+1} \otimes \smdots \otimes g_m\phi_n \\
&= \sum  
\xymatrix{
A_0 
\ar@/^14pc/[rrrrrrrrrr]^{g_0f_0}_{\big\Downarrow g_0\phi_1}
\ar@{.>}@/^12pc/[rrrrrrrrrr]_{\substack{
  g_0f_1\\ \vdots\\\\g_0f_{i_1-1}}} 
\ar@{.>}@/^8pc/[rrrrrrrrrr]
\ar@{.>}@/^6pc/[rrrrrrrrrr]^{\big\Downarrow g_0\phi_{i_1}}_{g_0f_{i_1}}
\ar@{.>}@/^3pc/[rrrrrrrrrr]^{\big\Downarrow 
  \psi_1\{\phi_{i_1+1}\smdots \phi_{i_2}\}}_{\substack{
  g_1f_{i_2}\\ \vdots \\\\ g_{m-1}f_{i_{2m-1}}}}
\ar@{.>}@/_1pc/[rrrrrrrrrr]
\ar@{.>}@/_3pc/[rrrrrrrrrr]^{\big\Downarrow 
 \psi_m\{\phi_{i_{2m-1}+1}\smdots \phi_{i_{2m}}\}}_{g_m
 f_{i_{2m}}}
\ar@{.>}@/_6pc/[rrrrrrrrrr]^{\big\Downarrow g_m
  \phi_{i_{2m}+1}}_{\substack{
  g_mf_{i_{2m}+1}\\ \vdots\\\\g_mf_{n-1} }} 
\ar@{.>}@/_10pc/[rrrrrrrrrr]_{\big\Downarrow g_m\phi_n}
\ar@/_12pc/[rrrrrrrrrr]_{g_mf_n}
&&&&&&&&&& A_0  
}
\end{align*}
where $\psi_j\{\phi_{i_r+1}\smdots \phi_{i_r+s}\}
\in C^\bullet(A_0, _{g_{j-1}f_{i_r}} {A_0}_{g_jf_{i_r+s}})$ 
is the following cochain: 
\begin{align*}
&\phantom{{}={}}
\psi_j\{\phi_{i_r+1}\smdots \phi_{i_r+s}\}
  (b_0 \otimes a_1 \otimes \smdots \otimes a_p)\\
&= 
\sum \limits_{
  0 \leq k_1 \leq \smdots \leq k_{2s} \leq p}
\psi_j \Bigg( b_0, f_{i_r}a_1, \smdots, f_{i_r}a_{k_1},
  \phi_{i_r+1}(a_{k_1+1}, \smdots, a_{k_2}),\\
&\phantom{{}moveovermoveover{}}  
  f_{i_r+1}a_{k_2+1}, \smdots, f_{i_r+1}a_{k_3},
  \phi_{i_r+2}(a_{k_3+1}, \smdots, a_{k_4}), \\
&\phantom{{}moveovermoveover{}}   
  ,\smdots, 
  \phi_{i_r+s}(a_{k_{2s-1}}, \smdots, a_{k_{2s}}), \\
&\phantom{{}moveovermoveover{}} 
  f_{i_r+s}a_{k_{2s}+1}, \smdots, f_{i_r+s}a_p \Bigg)
\end{align*}
for $b_0 \in A_1$, $a_1,\smdots,a_p \in A_0$, 
$\psi_j \in C^\bullet(A_1, _{g_{j-1}}{A_0}_{g_j})$, 
$\phi_i \in C^\bullet(A_), _{f_{i-1}}{A_1}_{f_i})$.
% -reduced hochschild chains
% -reference for braces and signs
% -lambda(\vec\phi) notation


% -computational props:
% - dot notation = iota notation
% - lambda(\phi) notation

\end{document}

