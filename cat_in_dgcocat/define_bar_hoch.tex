\section{Dg cocategories $Bar(Hoch(A,B))$}
Let $A$, $B$ be $k$-algebras. We define 
a dg category, $Hoch(A,B)$, as follows:
\begin{align*}
	&\textrm{Objects: algebra maps } f:A \to B\\
	&\textrm{Morphisms: } 
	  Hoch(A)(f,g) = (C^\bullet(A,\, _fB_g),\, _f\delta_g)\\
	&\textrm{Composition: cup product on cochains.}
\end{align*}
(See Appendix \ref{chap:hochschild} for notation and 
standard operations on Hochschild complexes.) 
The cup product is an associative map of complexes, 
so $Hoch(A,B)$ is a dg category.

Now, we will take $Bar(-)$ of $Hoch(A,B)$, which 
is a categorified bar construction:
$$Bar: DGCat \to DGCocat.$$
$Bar(Hoch(A,B))$ has the same objects as 
$Hoch(A,B)$. A morphism in $Bar(Hoch(A,B))$ 
from object $f_0$ to object $f_n$ 
is a sequence of composable morphisms in 
$Hoch(A,B)$ starting at $f_0$ and ending at 
$f_n$. We can picture such a morphism as 
follows:
%
\begin{figure}[H]
\centerline{\xymatrix{
	A 
	\ar@/^5pc/[rrr]^{f_0}_{\substack{\\\textcolor{blue}{\Downarrow \phi_1}}}
	\ar@/^2pc/[rrr]^{f_1}_{\substack{\\\textcolor{blue}{\Downarrow \phi_2}}}
	\ar@/_1pc/[rrr]^{f_2}_{\textcolor{blue}{\vdots}}
	\ar@/_4pc/[rrr]^{f_{n-1}}_{\substack{\\\textcolor{blue}{\Downarrow \phi_n}}}
	\ar@/_6pc/[rrr]_{f_n} 
	&&&	B}}
\caption{A morphism in $Bar(Hoch(A,B))(f_0,f_n)$}
\end{figure}
%
As a complex, 
\begin{align*}
&Bar(Hoch(A,B))^\bullet(f,g) 
=\\
&\quad = \underbrace{k[0]}_{\textrm{counit}} \oplus
\bigoplus \limits_{\substack{
	n \geq 0,\\
	f_i \in Obj(Hoch(A,B))}}
{\scriptstyle Hoch(A,B)^\bullet[1](f,f_1) \otimes 
Hoch(A,B)^\bullet[1](f_1,f_2) \otimes \dots \otimes 
Hoch(A,B)^\bullet[1](f_n,g)}\\
%
& d_{Bar(Hoch(A,B))} = 
\tilde{d}_{Hoch(A,B)} + d_\cup\\
%
& \tilde{d}_{Hoch(A,B)} =
\textrm{extension of $d_{Hoch(A,B)}$ 
to a differential on $Bar$}\\
%
& d_\cup = 
\textrm{signed sum over composing (cup-producting) 
two consecutive $\phi_i$'s}
\end{align*}
with cocomposition
$$\Delta(\phi_1\dots \phi_n) = 
\sum \limits_{0 \leq i \leq n}
\pm (\phi_1 \dots \phi_i) \otimes (\phi_{i+1}\dots \phi_n).$$
%
For more precise details and explicit signs, 
see Reference \cite{T}, Section 4.6.