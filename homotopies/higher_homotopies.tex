\section{Higher Homotopies}
In this section, we show 
that no higher homotopies are needed. 
First, we will summarize the 
maps of comodules that we have 
already given. Let 
$\mathcal{A} \in Obj(\chi)$ and 
$\lambda$ be a generating morphism in $\Lambda$ 
that induces a morphism in $\chi$ with 
source $\mathcal{A}$. We have 
\begin{align*}
\lambda_!: C(\mathcal{A})
&\to 
\hat{\lambda}^*C(\lambda\mathcal{A})
  \quad \textrm{maps of dg comodules}\\
\sigma_{\mathcal{A}!}:
  C(\mathcal{A}) 
&\to 
\tau^{*2}C(\tau^2\mathcal{A})
  \quad \textrm{deg -1 map of comodules.}
\end{align*}
where
\begin{align*}
\sigma_{(A_0 \to A_1 \to A_0)!}
&= 
B \textrm{ given in Appendix 
  Proposition \ref{prop:c2}}\\
\sigma_{(A_0 \to A_1 \to A_2 \to A_0)!}
&= 
\mathcal{B} \textrm{ given in Appendix 
  Proposition \ref{prop:c4}}\\
\sigma_{(A_0 \to \smdots A_n \to A_0)!}
&= 
(\widehat{\delta_{0,3}\smdots\delta_{0,n}})^*
  \mathcal{B} \textrm{ for } n>2.\\  
\end{align*}
(We will write $\sigma_!$ instead of 
$\sigma_{\mathcal{A}!}$ when the source 
is clear or doing so unnecessarily 
encumbers the exposition.) 
Using the constructions we've given, 
a typical map between comodules is 
one that is freely generated by composable 
pullbacks of $\lambda_!$'s and $\sigma_!$'s. 
First, we will establish that there are 
no such maps of degree $\geq2$. Suppose 
we have a map $\eta_!$ of degree 
$\geq2$. Then, $\eta_!$ must contain at 
least two (pullbacks of) some $\sigma_!$'s. 
Each $\sigma_!$ involves inserting a 1 
into the first slot of the Hochschild 
chains component (see Equations 
\ref{eq:def_sigma}, \ref{eq:def_sigma2}). 
However, since we are working with 
reduced chains, any chain with two or 
more 1's is equal to zero. So, 
$\eta_! = 0$.

Since there are no maps of degree 
$\geq2$, we know from the classical 
theory of $A_\infty$ algebras that 
the only need for higher homotopies 
will arise from the following 
situation: For $n\geq2$, 
We have two maps of dg comodules
$$
\xymatrixrowsep{5pc}
\xymatrix{
C(A_0 \to \smdots \to A_n \to A_0) 
 \ar@/^1pc/[d]^{\substack{
   (\widehat{\delta_{n-2,n-1}\delta_{n-1,n}})^*
   \tau_{n-2!}\\\\\\
   \textrm{``brace together the last 3 algebras,}\\
   \textrm{then apply $\tau_{n-2!}$ once''}}}
 \ar@/_1pc/[d]_{\substack{
   \hat{\tau}_n^{*2} \tau_{n!}\circ
   \hat{\tau}_n^*\tau_{n!}\circ \tau_{n!}\\\\\\
   \textrm{``apply $\tau_{n!}$ 3 times''}}}\\
C(A_{n-2} \to A_{n-1} \to A_n \to 
A_0 \to \smdots \to A_{n-2}).
}
$$
These two maps are homotopic via 
two homotopies: 
$\hat{\delta}_{n-1,n}^*
\sigma_{A_0 \to \smdots \to A_{n-1}\to A_0!}
+ \tau_n^{*2}\tau_{n!} \circ 
\sigma_{A_0 \to \smdots \to A_n\to A_0!}$ 
and 
$\hat{\delta}_{n-2,n}^*
\sigma_{A_0 \to \smdots \to A_{n-2}\to A_0 \to A_0!} +
\hat{\tau}_n^*\sigma_{A_n \to A_0 \smdots \to A_n!}
\circ \tau_{n!}$ (see Figure 
\ref{fig:two_homotopies}). If the 
two homotopies were different, then 
their difference would be closed and
we would desire a higher homotopy (i.e., 
a degree -2 map of comodules) between 
them. However, we will establish that 
higher homotopies are not necessary 
by showing that the two homotopies 
are the same.
%
\begin{figure}
\xymatrixrowsep{3pc}
\xymatrixcolsep{5pc}
\centerline{\xymatrix{
\underset{\substack{\\\\
  \textrm{``brace together $A_{n-2}, A_{n-1}, A_n$,}\\
  \textrm{then apply $\tau_{n-2!}$''}
         }}
         {(\widehat{\delta_{n-2,n-1}
         \delta_{n-1,n}})^*\tau_{n-2!}
         }
 \ar[r]^{\hat{\delta}_{n-1,n}^*\sigma_!}
 \ar[d]_{\substack{\\\\\\
         \hat{\delta}_{n-2,n}^*\sigma_!
         }}
& \underset{\substack{ \\\\
    \textrm{``brace together $A_{n-1}, A_n$ and apply $\tau_{n-1!}$,}\\
    \textrm{then apply $\tau_{n!}$''}
           }}
           {\hat{\tau}_n^{*2}\tau_{n!} \circ 
            \hat{\delta}_{n-1,n}^*\tau_{n-1!}
           } 
  \ar[d]^{\substack{\\\\\\
          \tau_n^{*2}\tau_{n!} \circ \sigma_!
         }}\\
\underset{\substack{\\\\
          \textrm{``apply $\tau_{n!}$,}\\
          \textrm{then brace together $A_{n-1}, A_{n-2}$ and apply $\tau_{n-1!}$''}
         }}
         {(\widehat{\delta_{n-1,n}\tau_n})^* 
         \tau_{n-1!} \circ \tau_{n!}
         }
  \ar[r]^{\hat{\tau}_n^*\sigma_! \circ \tau_{n!}}
& \underset{\substack{\\\\
  \textrm{``apply $\tau_{n!}$ three times''}
  }}
  {\hat{\tau}_n^{*2} \tau_{n!}\circ
   \hat{\tau}_n^*\tau_{n!}\circ \tau_{n!}}\\
}}
\caption{Two homotopies between 
$(\widehat{\delta_{n-2,n-1}\delta_{n-1,n}})^*
\tau_{n-2!}$ and $\hat{\tau}_n^{*2} \tau_{n!}\circ
\hat{\tau}_n^*\tau_{n!}\circ \tau_{n!}$;
vertices are maps of dg comodules and 
arrows are chain homotopies}
\label{fig:two_homotopies}
\end{figure}

For $n=1$, the situation reduces to: 
We have two maps of dg comodules
$$
\xymatrix{
C(A_0 \to A_1 \to A_0) 
 \ar@/^1pc/[d]^{\tau_{1!}}
 \ar@/_1pc/[d]_{\hat{\tau}_1^{*2} \tau_{1!}\circ
   \hat{\tau}_1^*\tau_{1!}\circ \tau_{1!}}\\
C(A_1 \to A_0 \to A_1).
}
$$
These two maps are homotopic via 
two homotopies: 
$\tau_{1!} \circ \sigma_{A_0 \to A_1 \to A_0!}$ 
and 
$\sigma_{A_1 \to A_0 \to A_1!} \circ \tau_{1!}$ 
(see Figure \ref{fig:two_homotopies_1}). Again, we will 
show that these two homotopies are the same. 
%
\begin{figure}
\xymatrixcolsep{5pc}
\centerline{\xymatrix{
id \circ \tau_{1!} = \tau_{1!} = \tau_{1!} \circ id
 \ar@/^1pc/[d]^{\tau_{1!} \circ \sigma_{A_0 \to A_1 \to A_0!}}
 \ar@/_1pc/[d]_{\sigma_{A_1 \to A_0 \to A_1!} \circ \tau_{1!}}\\
\big( \hat{\tau}_1^{*2} \tau_{1!}\circ
   \hat{\tau}_1^*\tau_{1!}  \big)  \circ \tau_{1!} 
=   
\hat{\tau}_1^{*2} \tau_{1!}\circ
   \big( \hat{\tau}_1^*\tau_{1!}\circ \tau_{1!} \big)
}}
\caption{Two homotopies between 
$\tau_{1!}$ and $\hat{\tau}_1^{*2} \tau_{1!}\circ
\hat{\tau}_1^*\tau_{1!}\circ \tau_{1!}$;
vertices are maps of dg comodules and 
arrows are chain homotopies}
\label{fig:two_homotopies_1}
\end{figure}
