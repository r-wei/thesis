\section{Higher Homotopies}
In this section, we show 
that no higher homotopies are needed. 
First, we will summarize the 
maps of comodules that we have 
already given. Let $\lambda$ be a 
generating morphism in $\Lambda$ 
that induces a morphism in $\chi$ with 
source $\mathcal{A}\in Obj(\chi)$. We have 
\begin{align*}
\lambda_!: C(\mathcal{A})
&\to 
\hat{\lambda}^*C(\lambda\mathcal{A})
  \quad \textrm{maps of dg comodules}\\
\sigma_!:
  C(\mathcal{A}) 
&\to 
\tau^{*2}C(\tau^2\mathcal{A})
  \quad \textrm{deg -1 map of comodules.}
\end{align*}
where
\begin{align*}
\sigma_!
=
\begin{cases}
B \textrm{ given in Appendix Proposition 
  \ref{prop:c2}}
  & \textrm{if } 
  \mathcal{A} = (A_0 \to A_1 \to A_0)\\
\mathcal{B} \textrm{ given in Appendix 
  Proposition \ref{prop:c4}}
  &  \textrm{if } 
  \mathcal{A} = (A_0 \to A_1 \to A_2 \to A_0)\\
(\widehat{\delta_{0,3}\smdots\delta_{0,n}})^*
  \mathcal{B}
  & \textrm{if } 
  \mathcal{A} = (A_0 \to \smdots \to A_n \to A_0), 
  n>2.
\end{cases}
\end{align*}
Using the constructions we've given, 
a typical map between comodules is 
one that is freely generated by composable 
pullbacks of $\lambda_!$'s and $\sigma_!$'s. 
First, we will establish that there are 
no such maps of degree $\geq2$. Suppose 
we have a map $\eta_!$ of degree 
$\geq2$. Then, $\eta_!$ must contain at 
least two (pullbacks of) some $\sigma_!$'s. 
Each $\sigma_!$ involves inserting a 1 
into the first slot of the Hochschild 
chains component (see Equations 
\ref{eq:def_sigma}, \ref{eq:def_sigma2}). 
However, since we are working with 
reduced chains, any chain with two or 
more 1's is equal to zero. So, 
$\eta_! = 0$.

Since there are no maps of degree 
$\geq2$, we know from the classical 
theory of $A_\infty$ algebras that 
the only need for higher homotopies 
will arise from the following 
situation: For $n\geq2$, 
We have two maps of dg comodules
\begin{equation}
\label{eq:two_maps}
\xymatrixrowsep{5pc}
\xymatrix{
C(A_0 \to \smdots \to A_n \to A_0) 
 \ar@/^1pc/[d]^{\substack{
   (\widehat{\delta_{n-2,n-1}\delta_{n-1,n}})^*
   \tau_{n-2!}\\\\\\
   \textrm{$``$brace together the last 3 algebras,}\\
   \textrm{then apply $\tau_{n-2!}$ once''}}}
 \ar@/_1pc/[d]_{\substack{
   \hat{\tau}_n^{*2} \tau_{n!}\circ
   \hat{\tau}_n^*\tau_{n!}\circ \tau_{n!}\\\\\\
   \textrm{$``$apply $\tau_{n!}$ 3 times''}}}\\
C(A_{n-2} \to A_{n-1} \to A_n \to 
A_0 \to \smdots \to A_{n-2}).
}
\end{equation}
These two maps are homotopic via 
two homotopies: 
$\hat{\delta}_{n-1,n}^*\sigma_!
+ \tau_n^{*2}\tau_{n!} \circ \sigma_!$ 
and 
$\hat{\delta}_{n-2,n}^*\sigma_! +
\hat{\tau}_n^*\sigma_!
\circ \tau_{n!}$ (see Figure 
\ref{fig:two_homotopies}). If the 
two homotopies were different, then 
their difference would be closed and
we would desire a higher homotopy (i.e., 
a degree -2 map of comodules) between 
them. However, we will show the 
two homotopies are the same, so that 
no higher homotopies are needed. 

First, we show that $\hat{\delta}_{n-1,n}^*
\sigma_! = \hat{\delta}_{n-2,n}^*\sigma_!$. 
We have
\begin{align*}
\hat{\delta}_{n-1,n}^*\sigma_!
=
\mathcal{B}_{A_0 \bullet \smdots \bullet A_{n-2},
  A_{n-1}\bullet A_n}
= \mathcal{B}_{A_0 \bullet \smdots \bullet A_{n-1},A_n} 
= \hat{\delta}_{n-2,n}^*\sigma_!
\end{align*}
where the second equality holds by definition 
of $\mathcal{B}$ in Appendix Equation 
\ref{eq:def_sigma2}.

Second, we show that $\tau_n^{*2}\tau_{n!} 
\circ \sigma_! = \hat{\tau}_n^*\sigma_!
\circ \tau_{n!}$ in Appendix Proposition 
\ref{prop:c5}. In the appendix, $\tau_n^{*2}
\tau_{n!} \circ \sigma_! = \Upsilon_{A_1\bullet A_2,A_0} 
\mathcal{B}_{A_0,A_1,A_2}$ and $\hat{\tau}_n^*
\sigma_!\circ \tau_{n!} = \mathcal{B}_{A_2,A_0,A_1} 
\Upsilon_{A_0 \bullet A_1, A_2}$.
%
\begin{figure}%[H]
\centerline{\xymatrix{
\underset{\substack{\\\\
  \textrm{$``$brace together $A_{n-2}, A_{n-1}, A_n$,}\\
  \textrm{then apply $\tau_{n-2!}$''}}}
  {(\widehat{\delta_{n-2,n-1}
   \delta_{n-1,n}})^*\tau_{n-2!}}
\ar[r]^{\cong}
\ar[d]_{\cong}   
& \hat{\delta}_{n-1,n}^*(
  \hat{\delta}_{n-2,n-1}^*\tau_{n-2!})   
\ar[r]^{\hat{\delta}_{n-1,n}^*\sigma_!}
& \hat{\delta}_{n-1,n}^*(
  \hat{\tau}_{n-1}^*\tau_{n-1!} \circ \tau_{n-1!})
\ar[d]^{\cong}  \\
(\widehat{\delta_{n-2,n-1}\delta_{n-2,n}})^*
  \tau_{n-2!}  
\ar[d]_{\hat{\delta}_{n-2,n}^*\sigma_!}
&& \underset{\substack{ \\\\
    \textrm{$``$brace together $A_{n-1}, A_n$}\\
    \textrm{and apply $\tau_{n-1!}$,}\\
    \textrm{then apply $\tau_{n!}$''}}}
    {\hat{\tau}_n^{*2}\tau_{n!} \circ 
    \hat{\delta}_{n-1,n}^*\tau_{n-1!}} 
\ar[d]^{\substack{\\\\\\\\
        \tau_n^{*2}\tau_{n!} \circ \sigma_!}}\\
\hat{\delta}_{n-2,n}^*(
  \hat{\tau}_{n-1}^*\tau_{n-1!} \circ \tau_{n-1!})
\ar[r]_{\cong}                   
& \underset{\substack{\\\\
          \textrm{$``$apply $\tau_{n!}$,}\\
          \textrm{then brace together $A_{n-1}, A_{n-2}$}\\
          \textrm{and apply $\tau_{n-1!}$''}}}
         {\hat{\tau}_n^*(\hat{\delta}_{n-1,n}^*
          \tau_{n-1!}) \circ \tau_{n!}}
\ar[r]^{\hat{\tau}_n^*\sigma_! \circ \tau_{n!}}
& \underset{\substack{\\\\
    \textrm{$``$apply $\tau_{n!}$ three times''}}}
    {\hat{\tau}_n^{*2} \tau_{n!}\circ
     \hat{\tau}_n^*\tau_{n!}\circ \tau_{n!}}
}}
\caption{Two homotopies between 
$(\widehat{\delta_{n-2,n-1}\delta_{n-1,n}})^*
\tau_{n-2!}$ and $\hat{\tau}_n^{*2} \tau_{n!}\circ
\hat{\tau}_n^*\tau_{n!}\circ \tau_{n!}$}
\label{fig:two_homotopies}
Vertices are maps of dg comodules and 
arrows are chain homotopies.
\end{figure}

For $n=1$, the situation in Equation 
\ref{eq:two_maps} reduces to: 
We have two maps of dg comodules
$$
\xymatrix{
C(A_0 \to A_1 \to A_0) 
 \ar@/^1pc/[d]^{\tau_{1!}}
 \ar@/_1pc/[d]_{\hat{\tau}_1^{*2} \tau_{1!}\circ
   \hat{\tau}_1^*\tau_{1!}\circ \tau_{1!}}\\
C(A_1 \to A_0 \to A_1).
}
$$
These two maps are homotopic via 
two homotopies: 
$\tau_{1!} \circ \sigma_{A_0 \to A_1 \to A_0!}$ 
and 
$\sigma_{A_1 \to A_0 \to A_1!} \circ \tau_{1!}$ 
(see Figure \ref{fig:two_homotopies_1}; 
for clarity, here, we indicate the sources 
of the $\sigma_!$'s). We show that these two 
homotopies are the same in Appendix Proposition 
\ref{prop:c3}, so no higher homotopies are 
needed. In the appendix, $\tau_{1!} \circ 
\sigma_{A_0 \to A_1 \to A_0!} = \Upsilon_{A_0,A_1} 
\circ B_{A_0,A_1}$ and 
$\sigma_{A_1 \to A_0 \to A_1!} \circ 
\tau_{1!} = B_{A_1,A_0} \circ \Upsilon_{A_0,A_1}$.
%
\begin{figure}%[H]
\xymatrixrowsep{4pc}
\centerline{\xymatrix{
id \circ \tau_{1!} = \tau_{1!} = \tau_{1!} \circ id
 \ar@/^1pc/[d]^{\tau_{1!} \circ \sigma_{A_0 \to A_1 \to A_0!}}
 \ar@/_1pc/[d]_{\substack{
 \sigma_{A_1 \to A_0 \to A_1!} \circ \tau_{1!}\\
 = \hat{\tau}_1^*(\sigma_{A_0\to A_1\to A_0!})
 \circ \tau_{1!}}}\\
\big( \hat{\tau}_1^{*2} \tau_{1!}\circ
   \hat{\tau}_1^*\tau_{1!}  \big)  \circ \tau_{1!} 
=   
\hat{\tau}_1^{*2} \tau_{1!}\circ
   \big( \hat{\tau}_1^*\tau_{1!}\circ \tau_{1!} \big)
}}
\caption{Two homotopies between 
$\tau_{1!}$ and $\hat{\tau}_1^{*2} \tau_{1!}\circ
\hat{\tau}_1^*\tau_{1!}\circ \tau_{1!}$}
\label{fig:two_homotopies_1}
Vertices are maps of dg comodules and 
arrows are chain homotopies.
\end{figure}
