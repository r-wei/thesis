\section{Homotopies}
Here, we will show that the maps 
of dg comodules given in Section 
\ref{sec:shriek_maps} satisfy the 
relations in $\Lambda$ (Equation 
\ref{eqn:cyclic_relations}) up to 
homotopy. More precisely, we will 
show that
\begin{subequations}\label{eq:strict}
\begin{align}
\begin{split}\label{eq:strict_1}
\hat{\delta}_{j,n}^*(\delta_{i,n-1!}) \circ \delta_{j,n!} 
&= 
\hat{\delta}_{i,n}^*(\delta_{j-1,n-1!}) \circ \delta_{i,n!} 
  \quad 0 \leq i < j \leq n-1 \\
\hat{\sigma}_{j,n}^*(\sigma_{i,n+1!}) \circ \sigma_{j,n!} 
&= 
\hat{\sigma}_{i,n}^*(\sigma_{j+1,n+1!}) \circ \sigma_{i,n!}
  \quad 0 \leq i \leq j \leq n \\
\hat{\sigma}_{i,n}^*(\delta_{j,n+1!}) \circ \sigma_{i,n!} 
&= 
  \begin{cases}
    \hat{\delta}_{j-1,n}^*(\sigma_{i,n-1!}) \circ \delta_{j-1,n!} 
      & 0 \leq i < j \leq n\\
    id & j = i, i-1\\
    \hat{\delta}_{j,n}^*(\sigma_{i-1,n-1!}) \circ \delta_{j,n!} 
      & \quad 0 \leq j < i-1 \leq n-1
   \end{cases}
\end{split}\\
\begin{split}\label{eq:strict_2}
\hat{\sigma}_{i,n}^*(\tau_{n+1!}) \circ \sigma_{i,n!} 
&= 
\hat{\tau}_n^*(\sigma_{i+1,n!}) \circ \tau_{n!}
  \quad 0 \leq i \leq n-1\\
\hat{\delta}_{j,n}^*(\tau_{n-1!}) \circ \delta_{j,n!} 
&= 
\hat{\tau}_n^*(\delta_{j+1,n!}) \circ \tau_{n!}
  \quad 0 \leq j \leq n-1
\end{split}\\
\begin{split}\label{eq:strict_3}
(\widehat{\tau_{1}\sigma_{0,0}})^*
  (\delta_{0,1!}) \circ
  \hat{\sigma}_{0,0}^*(\tau_{1!}) \circ 
  \sigma_{0,0!}
= id
\end{split} 
\end{align}
\end{subequations}
and 
\begin{subequations} \label{eq:weak}
\begin{equation} \label{eq:weak_delta}
\hat{\tau}_n^{*2}(\delta_{0,n!}) \circ 
  \hat{\tau}_n^*(\tau_{n!}) \circ \tau_{n!} 
\simeq 
\hat{\delta}_{n-1,n}^*(\tau_{n-1!}) \circ \delta_{n-1,n!}
\end{equation}
\begin{equation} \label{eq:weak_tau}
\hat{\tau}_n^{*n}(\tau_{n!}) \circ \smdots 
  \circ \hat{\tau}_n^*(\tau_{n!}) \circ \tau_{n!}
\simeq id
\end{equation}
\begin{align*} \label{eq:weak_sigma}
&\phantom{{}\simeq{}}
\hat{\sigma}_{n,n}^*(\tau_{n+1!}) \circ \sigma_{n,n!}\\
&\simeq
(\widehat{\tau_{n+1}^n\sigma_{0,n}\tau_{n}})^*(\tau_{n+1!})
  \circ \smdots \circ 
  (\widehat{\tau_{n+1}\sigma_{0,n}\tau_{n}})^*(\tau_{n+1!}) \circ
  (\widehat{\sigma_{0,n}\tau_{n}})^*(\tau_{n+1!}) \circ
  \hat{\tau}_n^*(\sigma_{0,n!}) \circ \tau_{n!}
\end{align*}
\end{subequations}
%
\subsection{Aside on composing $\lambda_!$'s}
In this section and the next, we will use 
the following convention for composing 
$\lambda_!$'s, which we first illustrate with 
an example. Suppose we have 3 composable 
generating morphisms in $\Lambda$: 
$\lambda_1 \in \Lambda([m],[n])$, 
$\lambda_2 \in \Lambda([n],[p])$, and 
$\lambda_3 \in \Lambda([p],[q])$. 
Fix a sequence of algebras $A_0, \smdots, A_m$. 
We can construct the following 
composition of morphisms of dg 
comodules over $B(\mathcal{A}):=
B(A_0 \to \smdots \to A_m \to A_0)$:
\begin{align*}
C(\mathcal{A})
\xrightarrow{\lambda_{1, \mathcal{A}!}}
%
\hat{\lambda}_1^*C(\lambda_1\mathcal{A})
\xrightarrow{\hat{\lambda}_1^*(
  \lambda_{2, \lambda_1\mathcal{A}!})}
%
\hat{\lambda}_1^*\hat{\lambda}_2^*C(\lambda_2\lambda_1\mathcal{A})
\xrightarrow{\hat{\lambda}_1^*\hat{\lambda}_2^*
  (\lambda_{3, \lambda_2\lambda_1\mathcal{A}!})}
%
\hat{\lambda}_1^*\hat{\lambda}_2^*\hat{\lambda}_3^*
  C(\lambda_3\lambda_2\lambda_1\mathcal{A}).
\end{align*}
To simplify the notation, we will write 
$$
\hat{\lambda}_1^*\hat{\lambda}_2^*(\lambda_{3!}) 
\circ \hat{\lambda}_1^*(\lambda_{2!})
\circ \lambda_{1!}
\quad \textrm{instead of} \quad
\hat{\lambda}_1^*\hat{\lambda}_2^*
  (\lambda_{3, \lambda_2\lambda_1\mathcal{A}!}) 
\circ \hat{\lambda}_1^*(\lambda_{2, \lambda_1\mathcal{A}!}) 
\circ \lambda_{1, \mathcal{A}!}.
$$

More generally, when the reader sees 
$\hat{\lambda}_1^* \smdots \hat{\lambda}_{r-1}^*
(\lambda_{r!}) \circ \smdots \circ 
\hat{\lambda}_1^*(\lambda_{2!}) 
\circ \lambda_{1!}$
for a sequence of composable generating morphisms 
$\lambda_r \circ \smdots \circ \lambda_1$ in $\Lambda$, 
s/he may decode this notation by choosing a sequence 
of algebras $A_0, \smdots, A_{m=\textrm{source of }
\lambda_1}$, setting $\mathcal{A} = (A_0 \to \smdots 
\to A_m \to A_0)$, and letting the composition of 
$\lambda_!$'s denote
$\hat{\lambda}_1^* \smdots \hat{\lambda}_{r-1}^*
  (\lambda_{r, \lambda_{r-1} \smdots \lambda_1 \mathcal{A}!}) 
\circ \smdots \circ  
\hat{\lambda}_1^*(\lambda_{2, \lambda_1\mathcal{A}!}) 
\circ \lambda_{1, \mathcal{A}!}$.
%
\subsection{Strict relations: showing Equations \ref{eq:strict} hold}
Equation \ref{eq:strict_1} has three relations. 
All of the $\sigma_!$'s and $\delta_!$'s in Equation 
\ref{eq:strict_1} are identity maps, so it's clear that 
these relations hold.

Equation \ref{eq:strict_2} has two relations. 
To show that the first one holds, we have
\begin{align*}
\hat{\sigma}_{i,n}^*(\tau_{n+1!}) \circ \sigma_{i,n!} 
&= 
\hat{\sigma}_{i,n}^*(
  (\widehat{\delta_{0,2}\smdots\delta_{0,n+1}})^*
  (\tau_{1!})) \circ \sigma_{i,n!}
  \quad \textrm{definition of }\tau_{n+1!}\\
&= 
(\widehat{\delta_{0,2}\smdots\delta_{0,n+1}\sigma_{i,n}})^*
  (\tau_{1!}) \circ \sigma_{i,n!}
  \quad \textrm{Proposition } \ref{prop:pbs_compose}\\
&= 
(\widehat{\delta_{0,2}\smdots\delta_{0,n}})^*
  (\tau_{1!}) \circ \sigma_{i,n!}\\
&= 
\tau_{n!} \circ \sigma_{i,n!}
  \quad \textrm{definition of }\tau_{n!}\\
&= 
\tau_{n!} \circ id  
  = id \circ \tau_{n!} \\
&= 
\hat{\tau}_n^*(\sigma_{i+1,n!}) \circ \tau_{n!}.
\end{align*}
To show that the second relation holds, the 
reasoning is the same as above. We have
\begin{align*}
\hat{\delta}_{j,n}^*(\tau_{n-1!}) \circ \delta_{j,n!} 
&= 
\hat{\delta}_{j,n}^*
  ((\widehat{\delta_{0,2}\smdots\delta_{0,n-1}})^*
  (\tau_{1!})) \circ \delta_{j,n!} \\
&=
(\widehat{\delta_{0,2}\smdots\delta_{0,n-1}\delta_{j,n}})^*
  (\tau_{1!}) \circ \delta_{j,n!} \\
&=
\tau_{n!} \circ \delta_{j,n!}\\
&= 
\tau_{n!} \circ id   
  = id \circ \tau_{n!}\\
&=
\hat{\tau}_n^*(\delta_{j+1,n!}) \circ \tau_{n!}. 
\end{align*} 

Equation \ref{eq:strict_3} has one relation. 
The only map in this relation that is not 
defined to be an identity map is $\hat{\sigma}_{0,0}^*
(\tau_{1!})$. We will compute this map and 
show that it is also an identity. 
Let $(\phi_{0,1}\smdots \phi_{0,k_0}|\alpha)
\in C(A_0 \to A_0)$ (see Figure \ref{fig:phi|alpha} 
for notation). By Proposition \ref{prop:compute_pb},
\begin{align*}
C(A_0 \to A_0)
&\xrightarrow{\cong} 
\hat{\sigma}_{0,0}^*C(A_0 \to A_0 \to A_0)\\
(\phi_{0,1}\smdots \phi_{0,k_0}|\alpha)
&\mapsto
\sum \limits_{0 \leq r_0 \leq k_0}
  (\phi_{0,1}\smdots \phi_{0,r_0}) \otimes 
  (1 | \phi_{0,r_0+1}\smdots \phi_{0,k_0}|\alpha).
\end{align*}
Applying $\hat{\sigma}_{0,0}^*(\tau_{1!})$
to the righthand side, we have
\begin{align*}
\hat{\sigma}_{0,0}^*C(A_0 \to A_0 \to A_0)
&\xrightarrow{\hat{\sigma}_{0,0}^*(\tau_{1!})}
\hat{\sigma}_{0,0}^*\hat{\tau}_1^*C(A_0 \to A_0 \to A_0)\\
\sum \limits_{0 \leq r_0 \leq k_0}
  (\phi_{0,1}\smdots \phi_{0,r_0}) \otimes 
  (1 | \phi_{0,r_0+1}\smdots \phi_{0,k_0}|\alpha)
&\mapsto
\sum \limits_{0 \leq r_0 \leq s_0 \leq k_0}
  (\phi_{0,1}\smdots \phi_{0,r_0}) \otimes \\
&\phantom{{}\mapsto \sum{}}  
  (\phi_{0,r_0+1}\smdots \phi_{0,s_0}|1|
  \tau_1(1|\phi_{0,s_0+1}\smdots\phi_{0,k_0}|\alpha)).
\end{align*}
The righthand side above is equal to
\begin{align*}
&\phantom{{}={}}
\sum \limits_{0 \leq r_0 \leq s_0 \leq k_0}
  (\phi_{0,1}\smdots \phi_{0,r_0}) \otimes 
  (\phi_{0,r_0+1}\smdots \phi_{0,s_0}|1|
  \tau_1(1|\phi_{0,s_0+1}\smdots\phi_{0,k_0}|\alpha))\\
&=  
\sum \limits_{0 \leq r_0 \leq s_0 \leq k_0}
  (\phi_{0,1}\smdots \phi_{0,r_0}) \otimes 
  (\phi_{0,r_0+1}\smdots \phi_{0,s_0}|1|
  \upsilon_{0,k_0-s_0}(1|\phi_{0,s_0+1}\smdots\phi_{0,k_0}|\alpha))\\
&\phantom{{}=\sum\sum{}}  
  \quad \textrm{(see Proposition \ref{prop:c1} for }
  \upsilon_{\cdot,\cdot})\\
&=  
\sum \limits_{0 \leq r_0 \leq k_0}
  (\phi_{0,1}\smdots \phi_{0,r_0}) \otimes 
  (\phi_{0,r_0+1}\smdots \phi_{0,k_0}|1|\alpha)
  \quad \quad (\upsilon_{0,>0}=0)\\
&\in
\hat{\sigma}_{0,0}^*\hat{\tau}_1^*C(A_0 \to A_0 \to A_0).
\end{align*}
Finally, applying Proposition 
\ref{prop:compute_pb} again, we have
\begin{align*}
\hat{\sigma}_{0,0}^*\hat{\tau}_1^*C(A_0 \to A_0 \to A_0)
&\xrightarrow[\cong]{\textrm{project onto cogenerators}} 
C(A_0 \to A_0)\\
\sum \limits_{0 \leq r_0 \leq k_0}
  (\phi_{0,1}\smdots \phi_{0,r_0}) \otimes 
  (\phi_{0,r_0+1}\smdots \phi_{0,k_0}|1|\alpha)
&\mapsto
(\phi_{0,1}\smdots \phi_{0,k_0}|\alpha).
\end{align*}
So, we've shown 
$$
C(A_0 \to A_0) \cong 
\hat{\sigma}_{0,0}^*C(A_0 \to A_0 \to A_0)
\xrightarrow{\hat{\sigma}_{0,0}^*(\tau_{1!})}
\hat{\sigma}_{0,0}^*\hat{\tau}_1^*C(A_0 \to A_0 \to A_0)
\cong C(A_0 \to A_0)
$$
is the identity map.
% 
\subsection{Weak relations: showing Equations \ref{eq:weak} hold} \label{sec:weak_relations}
\subsubsection{Showing Equation \ref{eq:weak_delta} holds}\label{sec:weak_relations_delta}
For $n=1$, eliminating the identity 
maps reduces Equation \ref{eq:weak_delta} to:
\begin{align*} 
\hat{\tau}_1^*(\tau_{1!}) \circ \tau_{1!} 
\simeq id.
\end{align*}
We prove the above in Appendix Proposition 
\ref{prop:c2}. (In the appendix, 
$\tau_{1!} = \Upsilon_{A_0,A_1}$, 
$\hat{\tau}_1^*(\tau_{1!}) = \Upsilon_{A_1,A_0}$, 
and the homotopy is denoted $B$.)

For $n=2$, eliminating the identity 
maps and writing $\tau_{2!}$ in terms 
of $\tau_{1!}$ reduces Equation \ref{eq:weak_delta} to:
\begin{align*} 
(\widehat{\delta_{0,2}\tau_2})^*(\tau_{1!}) \circ 
  \hat{\delta}_{0,2}^*(\tau_{1!})
\simeq \hat{\delta}_{1,2}^*(\tau_{1!}).
\end{align*}
We prove the above in Appendix Proposition 
\ref{prop:c4}. (In the appendix, 
$\hat{\delta}_{0,2}^*(\tau_{1!}) = 
\Upsilon_{A_0\bullet A_1, A_2}$, 
$(\widehat{\delta_{0,2}\tau_2})^*(\tau_{1!}) 
= \Upsilon_{A_2\bullet A_0,A_1}$, 
$\hat{\delta}_{1,2}^*(\tau_{1!}) =
\Upsilon_{A_0, A_1\bullet A_2}$, 
and the homotopy is denoted 
$\mathcal{B}$.)

For $n>2$, we reduce Equation 
\ref{eq:weak_delta} to the case when 
$n=2$. We have
\begin{align*}
\textrm{Lefthand side of Equation 
\ref{eq:weak_delta}}
&= 
\hat{\tau}_n^{*2}(\delta_{0,n!}) \circ 
  \hat{\tau}_n^*(\tau_{n!}) \circ \tau_{n!} \\  
&=  
id \circ 
  \hat{\tau}_n^*(
  (\widehat{\delta_{0,2}\smdots\delta_{0,n}})^*(\tau_{1!})) 
  \circ \tau_{n!} \\
&=   
(\widehat{\delta_{0,2}\smdots\delta_{0,n}\tau_n})^*(\tau_{1!}) 
  \circ \tau_{n!} \\
&=
(\widehat{\delta_{0,2}\tau_2\delta_{0,3}\smdots\delta_{0,n}})^*(\tau_{1!}) 
  \circ \tau_{n!} \\
&=
(\widehat{\delta_{0,2}\tau_2\delta_{0,3}\smdots\delta_{0,n}})^*(\tau_{1!}) 
  \circ (\widehat{\delta_{0,2}\smdots\delta_{0,n}\tau_n})^*(\tau_{1!}) \\
&=
(\widehat{\delta_{0,3}\smdots\delta_{0,n}})^*(
  (\widehat{\delta_{0,2}\tau_2})^*(\tau_{1!}) 
  \circ \hat{\delta}_{0,2}^*\tau_{1!}) \\ 
%
\textrm{Righthand side of Equation 
\ref{eq:weak_delta}}
&=  
\hat{\delta}_{n-1,n}^*(\tau_{n-1!}) 
  \circ \delta_{n-1,n!}\\
&=  
\hat{\delta}_{n-1,n}^*(
  (\widehat{\delta_{0,2}\smdots\delta_{0,n-1}})^*
  (\tau_{1!})) \circ id\\
&=  
(\widehat{\delta_{0,2}\smdots\delta_{0,n-1}
  \delta_{n-1,n}})^*(\tau_{1!}))\\  
&=  
(\widehat{\delta_{1,2}\delta_{0,3}\smdots\delta_{0,n}})^*
  (\tau_{1!})\\ 
&=  
(\widehat{\delta_{0,3}\smdots\delta_{0,n}})^*
  (\hat{\delta}_{1,2}^*(\tau_{1!})).
\end{align*}
So, Equation \ref{eq:weak_delta} = 
$(\widehat{\delta_{0,3}\smdots\delta_{0,n}})^*$(
Equation \ref{eq:weak_delta}, $n=2$). 
If $\mathcal{B}$ is a homotopy giving 
Equation \ref{eq:weak_delta} for $n=2$, then 
$(\widehat{\delta_{0,3}\smdots\delta_{0,n}})^*
\mathcal{B}$ is a homotopy giving Equation 
\ref{eq:weak_delta} for $n>2$.
%
\subsubsection{Showing Equation \ref{eq:weak_tau} holds}\label{sec:weak_relations_tau}
We prove this by induction on $n$. For $n=1$, 
Equation \ref{eq:weak_tau} is the same as 
Equation \ref{eq:weak_delta}, which we 
established in the previous section. Now, 
assume that Equation \ref{eq:weak_tau} holds 
for $N=n-1$. We show that Equation 
\ref{eq:weak_tau} holds for $N=n$ below:
\begin{align*}
\hat{\tau}_n^{*n}(\tau_{n!}) \circ \smdots 
  \circ \hat{\tau}_n^*(\tau_{n!}) \circ \tau_{n!}
&= 
\hat{\tau}_n^{*n-1}(
  \hat{\tau}_n^*\tau_{n!}\circ \tau_{n!}) \circ
  \hat{\tau}_n^{*n-2}\tau_{n!}\circ
  \smdots \circ \tau_{n!}\\
&\simeq 
\hat{\tau}_n^{*n-1}(
  \hat{\delta}_{n-1,n}^*\tau_{n-1!}) \circ
  \hat{\tau}_n^{*n-2}\tau_{n!}\circ
  \smdots \circ \tau_{n!}
  \quad \quad \textrm{(Equation \ref{eq:weak_delta})}\\ 
&= 
(\widehat{\tau_{n-1}^{n-1}\delta_{0,n}})^*
  \tau_{n-1!} \circ \\
&\phantom{{}={}}  
  \circ \big(
  \hat{\tau}_n^{*n-2} \hat{\delta}_{n-2,n}^*
  \tau_{n-1!}\circ \smdots \circ 
  \hat{\tau}_n^* \hat{\delta}_{1,n}^* \tau_{n-1!} 
  \circ \hat{\delta}_{0,n}^* \tau_{n-1!} 
  \big)\\
&= 
(\widehat{\tau_{n-1}^{n-1}\delta_{0,n}})^*
  \tau_{n-1!} \circ \hat{\delta}_{0,n}^* \big(
  \hat{\tau}_{n-1}^{*n-2} \tau_{n-1!}\circ \smdots \circ 
  \hat{\tau}_{n-1}^* \tau_{n-1!} 
  \circ \tau_{n-1!} \big)  \\
&= 
\hat{\delta}_{0,n}^* \big(
  \hat{\tau}_{n-1}^{*n-1} \tau_{n-1!} \circ 
  \smdots \circ \tau_{n-1!} \big)  \\ 
&\simeq 
\hat{\delta}_{0,n}^* \big(
  id \big)
  \quad \quad \quad 
  \textrm{(Inductive hypothesis)}\\
&= id.
\end{align*}
%
\subsubsection{Showing Equation \ref{eq:weak_sigma} holds}\label{sec:weak_relations_sigma}
By manipulating morphisms in $\Lambda$, 
we have
\begin{align*}
\textrm{Righthand side of Equation \ref{eq:weak_sigma}}
&= 
\hat{\tau}_n^{*n+1}\tau_{n!} \circ 
  \hat{\tau}_n^{*n}\tau_{n!} \circ 
  \smdots \circ 
  \hat{\tau}_n^*\tau_{n!} \circ
  \hat{\tau}_n^{*n+1} id \circ 
  \tau_{n!}\\
&= 
\tau_{n!} \circ \big(
  \hat{\tau}_n^{*n}\tau_{n!} \circ 
  \smdots \circ 
  \hat{\tau}_n^*\tau_{n!} \circ
  \tau_{n!} \big)\\
&\simeq
\tau_{n!} \circ \big(id)
  \quad \textrm{Equation \ref{eq:weak_tau}.}\\  
\end{align*}
On the other hand, we have 
\begin{align*}
\textrm{Lefthand side of Equation \ref{eq:weak_sigma}}
&=
\hat{\sigma}_{n,n}^*(\tau_{n+1!}) \circ id\\
&=
\hat{\sigma}_{n,n}^*(
  \hat{\delta}_{n,n+1}^*(\tau_{n+1!}))\\
&=
(\widehat{\delta_{n,n+1}\sigma_{n,n}})^*
  (\tau_{n!})\\  
&=
id^*(\tau_{n!}).   
\end{align*}
So, Equation \ref{eq:weak_sigma} holds.