\section{A functor to dg cocategories}
The previous section \ref{sec:rectify}
gave a dg functor from $U(\chi_\infty)$ 
to $\mathcal{D}_\infty = ``$dg cocategories 
and dg comodules''. We would like to have our 
functor land in the dg category, 
$\mathcal{E} = ``$dg categories and dg 
modules''. To do so, we will first give a 
dg functor $\mathcal{D}_\infty \to 
\mathcal{D}_1$, 
which makes use of the adjunction in 
Proposition \ref{prop:adjunction}. Then, 
we will give a dg functor $\Omega: 
\mathcal{D}_1 \to \mathcal{E}$.

\subsection{Using the adjunction}
Let $\mathcal{D}_1$ be the 
dg category with the same objects 
as $\mathcal{D}$ and morphisms
\begin{align*}
\mathcal{D}_1^\bullet(
  (B_1, C_1), (B_0,C_0))
= \big\{ \big(
& F: 
B_1 \to B_0 \quad \textrm{dg functor},\\
& F_!:
F_{\#}C_1 \to C_0 \quad \textrm{map of 
comodules of degree $\bullet$}
\big) \big\}\\
d_{\mathcal{D}_\infty}(F,F_!)
&=
(F,\; d_{C_0} \circ F_! - (-1)^{|F_!|} 
F_! \circ d_{F_\# C_1})
\end{align*}
with composition
\begin{align*}  
\mathcal{D}_1^\bullet(
  (B_2, C_2), (B_1, C_1)) \otimes  
  \mathcal{D}_1^\bullet(
  (B_1, C_1), (B_0, C_0))
&\to
\mathcal{D}_1^\bullet(
  (B_2, C_2), (B_0, C_0))\\
(f,f_!) \otimes (g, g_!)
&\mapsto
(gf, g_!\circ g_\#(f_!)).
\end{align*}
This composition is well-defined because 
we can apply the formulas from $g_\#$ to 
(not necessarily graded) morphisms of 
comodules. The composition is associative 
because of the following easy-to-check 
fact: $g_\#f_\#C = (gf)_\#C$ for 
$B_2 \xrightarrow{f} B_1 \xrightarrow{g} 
B_0$ functors of dg cocategories and $C$ 
a dg comodule over $B_2$.

Now, we define a dg functor 
\begin{align*}
Adj: \mathcal{D}_\infty 
&\to 
\mathcal{D}_1\\
\textrm{on objects: }(B,C)
& \mapsto
(B,C)\\
\textrm{on morphisms: }\bigg((B_1, C_1) 
  \xrightarrow{(F, F_!)} (B_0,C_0)\bigg)
& \mapsto
\bigg((B_1, C_1) 
  \xrightarrow{(F, \Phi^{-1}_FF)} 
  (B_0,C_0)\bigg)
\end{align*}
where 
$\Phi^{-1}_F:Hom_{\substack{
  \textrm{dg comodules}\\\textrm{over $B_1$}}}
  (C,F^*D) 
\to 
Hom_{\substack{
  \textrm{dg comodules}\\\textrm{over $B_0$}}}
  (F_\#C,D)$
is defined in the proof of Proposition 
\ref{prop:adjunction} and makes sense as 
a function on (not necessarily graded) 
maps of comodules. To check that $Adj$ 
commutes with the differentials and 
respects composition, we need
\begin{align*}
\Phi^{-1}_F \circ d_{Hom_{B_2}(C_2, F^*C_1)}
&= 
d_{Hom_{B_1}(F_\#C_2, C_1)} \circ \Phi^{-1}_F\\
\Phi^{-1}_{GF}(F^*G_! \circ F_!)
&=
\Phi^{-1}_G(G_!) \circ G_\#(\Phi^{-1}_F(F_!))\\
\textrm{where }
(B_2, C_2) 
  \xrightarrow{(F, F_!)} &(B_1,C_1)
  \xrightarrow{(G, G_!)} (B_0,C_0)
  \textrm{ in }\mathcal{D}_\infty.
\end{align*}
The equations above follow straight-forwardly 
from the definitions.
%
%
\subsection{Applying $Cobar$}
In this section, we will use the notion of 
a dg module over a dg category. This is 
dual to a dg comodule over 
a dg cocategory (Definition \ref{def:dg_comod}). 
Given a dg functor between dg categories 
$F:A_1 \to A_0$, we define $``$restriction 
of scalars'', $F^*$, a functor from 
the category of dg comodules over $A_0$ 
to the category of dg comodules over $A_1$. 
For $M_0$ a dg comodule over $A_0$ and $f \in 
Obj(B_1)$, $F^*M_0(f):=M_0(Ff)$.

Let $\mathcal{E}$ be the dg category defined 
below:
\begin{align*}
Obj(\mathcal{E}) 
&= 
\{(A,M) |
  \textrm{A is a dg category, 
  M is a dg module over A}\} \\
\mathcal{E}^p((A_1, M_1), (A_0, M_0))  
&= 
\{(f, f_!) | f:A_1 \to A_0 
  \textrm{ is a dg functor,}\\
& \phantom{{}=[(f, f_!)]{}}  
  f_!:M_1 \to f^*M_0 
  \textrm{ is a degree-$p$ map of modules over }
  A_1\}\\
d_{\mathcal{E}}(f,f_!)
&=
(f,\; d_{f^*M_0} \circ f_! - (-1)^{|f_!|} 
f_! \circ d_{C_1})  
\end{align*}
\begin{align*}  
\mathcal{E}^\bullet((A_2, M_2), (A_1, M_1)) \times  
\mathcal{E}^\bullet((A_1, M_1), (A_0, M_0))
&\xrightarrow{composition}
\mathcal{E}^\bullet((A_2, M_2), (A_0, M_0))\\
(f,f_!) \times (g, g_!)
&\mapsto
(gf, f^*(g_!)\circ f_!).
\end{align*}

We will define a dg functor $\Omega: 
\mathcal{D}_1 \to \mathcal{E}$. 
On objects, $\Omega(B,C) := 
(Cobar(B), Cobar(B,C))$ where the first 
Cobar is a dg functor from the category of 
dg cocategories to the category of dg 
categories, and the second Cobar sends 
dg comodules over $B$ to dg modules over 
$Cobar(B)$ (see \cite{T}, Section 4.6). 
On morphisms, 
\begin{align*}
\bigg(\substack{
	B_1 \xrightarrow{F} B_0\\
	C_1 \xrightarrow{F_!} F^*C_0}\bigg)
& \mapsto
\bigg(\substack{
  Cobar(B_1)\xrightarrow{Cobar(F)}Cobar(B_0)\\
  Cobar(B_1,C_1)}) \xrightarrow{\Omega(F_!)} 
  (Cobar(F))^*Cobar(B_0,C_0)\bigg)\\
\textrm{where } 
\Omega(F_!): Cobar(B_1,C_1)
& \to
(Cobar(F))^*Cobar(B_0,C_0)\\
(b_1|\smdots|b_n|c)
&\mapsto
(Fb_1|\smdots|Fb_n|F_!c)\\
\textrm{for } 
  b_i \in B_1^\bullet(f_{i-1},f_i),
  c \in C_1^\bullet(f_n), 
  \textrm{ and } f_i \in 
&Obj(B_1), 0 \leq i \leq n.
\end{align*}
It's straightforward from the definitions 
to check that $\Omega$ commutes with 
the differentials and respects composition.


