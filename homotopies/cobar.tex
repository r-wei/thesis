\section{A functor to dg cocategories}
The previous section \ref{sec:rectify}
gave a dg functor from $U(\chi_\infty)$ 
to $\mathcal{D}_\infty = ``$dg cocategories 
and dg comodules''. We would like to have our 
functor land in the dg category, 
$\mathcal{E} = ``$dg categories and dg 
modules''. To do so, we will first give a 
dg functor $\mathcal{D}_\infty \to 
\mathcal{D}_1$, 
which makes use of the adjunction in 
Proposition \ref{prop:adjunction}. Then, 
we will give a dg functor $Cobar: 
\mathcal{D}_1 \to \mathcal{E}$.

\subsection{Using the adjunction}
Let $\mathcal{D}_1$ be the 
dg category with the same objects 
as $\mathcal{D}$ and morphisms
\begin{align*}
\mathcal{D}_1^\bullet(
  (B_1, C_1), (B_0,C_0))
= \big\{ \big(
& F: 
B_1 \to B_0 \quad \textrm{dg functor},\\
& F_!:
F_{\#}C_1 \to C_0 \quad \textrm{map of 
comodules of degree $\bullet$}
\big) \big\}\\
d_{\mathcal{D}_\infty}(F,F_!)
&=
(F,\; d_{C_0} \circ F_! - (-1)^{|F_!|} 
F_! \circ d_{F_\# C_1})
\end{align*}
with composition
\begin{align*}  
\mathcal{D}_1^\bullet(
  (B_2, C_2), (B_1, C_1)) \otimes  
  \mathcal{D}_1^\bullet(
  (B_1, C_1), (B_0, C_0))
&\to
\mathcal{D}_1^\bullet(
  (B_2, C_2), (B_0, C_0))\\
(f,f_!) \otimes (g, g_!)
&\mapsto
(gf, g_!\circ g_\#(f_!)).
\end{align*}
This composition is well-defined because 
we can apply the formulas from $g_\#$ to 
(not necessarily graded) morphisms of 
comodules. The composition is associative 
because of the following easy-to-check 
fact: $g_\#f_\#C = (gf)_\#C$ for 
$B_2 \xrightarrow{f} B_1 \xrightarrow{g} 
B_0$ functors of dg cocategories and $C$ 
a dg comodule over $B_2$.

Now, we define a dg functor 
\begin{align*}
Adj: \mathcal{D}_\infty 
&\to 
\mathcal{D}_1\\
\textrm{on objects: }(B,C)
& \mapsto
(B,C)\\
\textrm{on morphisms: }\bigg((B_1, C_1) 
  \xrightarrow{(F, F_!)} (B_0,C_0)\bigg)
& \mapsto
\bigg((B_1, C_1) 
  \xrightarrow{(F, \Phi^{-1}_FF)} 
  (B_0,C_0)\bigg)
\end{align*}
where 
$\Phi^{-1}_F:Hom_{\substack{
  \textrm{dg comodules}\\\textrm{over $B_1$}}}
  (C,F^*D) 
\to 
Hom_{\substack{
  \textrm{dg comodules}\\\textrm{over $B_0$}}}
  (F_\#C,D)$
is defined in the proof of Proposition 
\ref{prop:adjunction} and makes sense as 
a function on (not necessarily graded) 
maps of comodules. To check that $Adj$ 
commutes with the differentials and 
respects composition, we need
\begin{align*}
\Phi^{-1}_F \circ d_{Hom_{B_2}(C_2, F^*C_1)}
&= 
d_{Hom_{B_1}(F_\#C_2, C_1)} \circ \Phi^{-1}_F\\
\Phi^{-1}_{GF}(F^*G_! \circ F_!)
&=
\Phi^{-1}_G(G_!) \circ G_\#(\Phi^{-1}_F(F_!))\\
\textrm{where }
(B_2, C_2) 
  \xrightarrow{(F, F_!)} &(B_1,C_1)
  \xrightarrow{(G, G_!)} (B_0,C_0)
  \textrm{ in }\mathcal{D}_\infty.
\end{align*}
The equations above follow straight-forwardly 
from the definitions.
%
%
\subsection{Applying $Cobar$}




