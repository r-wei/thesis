\section{An $A_\infty$-functor}
We can repackage the work of the previous 
sections into a concise statement: We have 
constructed an $A_\infty$-functor from 
$\chi$ to $\mathcal{D}$. This section is 
devoted to making that concise statement 
more rigorous. We refer to Reference 
\cite{F}, Appendix A, Definitions 
A.6 and A.8 for the notation and 
definition of an 
$A_\infty$-category and  
$A_\infty$-functor.

First, we must think of $\chi$ and 
$\mathcal{D}$ as $A_\infty$-categories. 
Let $\chi_\infty$ be the (usual) category 
with the same objects as $\chi$ and 
morphisms linear combinations over $k$ 
of morphisms in $\chi$. We will think of 
the morphisms in $\chi_\infty$ as 
complexes concentrated in degree zero. 
$\chi_\infty$ is an $A_\infty$-category 
with $m_2$ = (usual) composition of 
morphisms in $\chi_\infty$, $m_1 = 
m_{\geq 3} = 0$. The relations for an 
$A_\infty$-category are satisfied 
because $m_2$ is associative.

Let $\mathcal{D}_\infty$ be the 
dg category with the same objects 
as $\mathcal{D}$ and morphisms
\begin{align*}
\mathcal{D}_\infty^\bullet((B_1, C_1), (B_0,C_0))
= \big\{ \big(
& F: 
B_1 \to B_0 \quad \textrm{dg functor},\\
& F_!:
C_1 \to F^*C_0 \quad \textrm{map of 
comodules of degree $\bullet$}
\big) \big\}\\
d_{\mathcal{D}_\infty}(F,F_!)
&=
(F,\; d_{F^*C_0} \circ F_! - (-1)^{|F_!|} 
F_! \circ d_{C_1}).
\end{align*}
We can think of $\mathcal{D}_\infty$ 
as an $A_\infty$-category with $m_1 = 
d_{\mathcal{D}_\infty}$, $m_2$ = (usual) 
composition of morphisms, $m_{\geq 3} = 0$. 
For $\mathcal{D}_\infty$, the relations for an 
$A_\infty$-category are precisely that 
(1) the differentials square to zero, 
(2) composition is a map of 
complexes, and (3) composition is 
associative.

Now, we will show that the constructions 
given in the previous sections constitute 
an $A_\infty$-functor $\mathcal{F}: 
\chi_\infty \to \mathcal{D}_\infty$. Still using 
the notation in Reference \cite{F}, 
Definition A.8, we define $\mathcal{F}$ 
as follows:
\begin{align*}
f: Obj(\chi_\infty) 
&\to 
Obj(\mathcal{D}_\infty)
\quad \textrm{map of sets}\\
\mathcal{A}
&\mapsto
(B(\mathcal{A}), C(\mathcal{A}))
\textrm{ defined in Chapter 1}\\
%
f_1: \chi_\infty^\bullet(x_0, x_1)
&\to
\mathcal{D}_\infty^\bullet(fx_0, fx_1)
\quad \textrm{map of graded vector spaces}\\
\lambda
&\mapsto 
(\hat{\lambda}, \lambda_!)
\textrm{ defined in Sections \ref{sec:cyclic_B(n)}, \ref{sec:shriek_maps}}\\
&\phantom{{}moveover{}}
  \textrm{for $\lambda$ a generating morphism in $\Lambda$}\\
id_{\mathcal{A}}
&\mapsto
(id_{B(\mathcal{A})}, id_{C(\mathcal{A})})\\
%  
f_2: \chi_\infty^\bullet(x_0, x_1)
  \otimes \chi_\infty^\bullet(x_1, x_2)
&\to
\mathcal{D}_\infty^\bullet(fx_0, fx_2)
\quad \textrm{degree $-1$ map of vector spaces}\\
%
f_{i\geq 3}: \chi_\infty^\bullet(x_0, x_1)
  \otimes \smdots \otimes 
  \chi_\infty^\bullet(x_{i-1}, x_i)
&\to 
\mathcal{D}_\infty^\bullet(fx_0, fx_i)
\quad \textrm{degree $1-i$ map of vector spaces}\\
\mu_1 \otimes \smdots \otimes \mu_i
&\mapsto
0 \textrm{ for all morphisms $\mu$}
\end{align*}
We will show that the $f_i$'s satisfy 
the relations in Reference \cite{F}, 
Definition A.8.

First, we finish defining $\mathcal{F}$, 
namely, we must define $f_1(\mu)$ for 
$\mu$ not a generating morphism in $\Lambda$ 
as well as for linear combinations of 
morphisms. We also still need to define $f_2$.

Let $\mu$ be a non-generating morphism 
in $\Lambda$ that induces a morphsim in 
$\chi$ with source $\mathcal{A}$. Choose 
(i.e., fix once and 
for all) a presentation of $\mu$ as a 
composition of generating morphisms. 
Within the chosen presentation, in the 
following order, (1) replace 
all occurrences of $\tau_{n-1} \delta_{n-1,n}$ 
with $\delta_{0,n}\tau_n^2$, (2) replace all 
$\tau_{n+1}\sigma_{n,n}$ with 
$\tau_{n+1}^{n+1}\sigma_{0,n}\tau_n$, 
(3) replace all decompositions of identity 
maps with identity maps, (4) remove all 
identity maps if $\mu \neq id$, (5) call 
this new presentation 
$``$the presentation corresponding to $\mu$'', 
denoted $\mu = \lambda_{\mu,k_\mu}\smdots 
\lambda_{\mu,1}$. The presentation 
corresponding to $\mu$ is not unique (i.e., 
still depends on the original, chosen 
presentation). However, letting 
\begin{align*}
f_1(\mu) := \big(
&\hat{\mu}: 
B(\mathcal{A}) \to 
  B(\mu\mathcal{A})\\
&
\hat{\lambda}_{\mu,1}^*\smdots
  \hat{\lambda}_{\mu,k_\mu-1}^*
  (\lambda_{\mu,k_\mu!})
  \circ \smdots \circ
  \hat{\lambda}_{\mu,1}^*(\lambda_{\mu,2!})
  \circ \lambda_{\mu,1!}: C(\mathcal{A})
  \to \hat{\mu}^*C(\mu\mathcal{A}) \big)
\end{align*}
is well-defined because we have made 
consistent choices for all of the 
relations among $\lambda_!$'s that 
only hold up to homotopy (see Equations 
\ref{eq:weak}). More explicitly, 
$f_1(\mu)$ would have been well-defined 
for any choice of presentation if 
Equations \ref{eq:weak} were equalities 
rather than homotopies. Instead, 
we choose to define $f_1(\mu)$ via a 
presentation that only uses the lefthand 
side of Equation \ref{eq:weak_delta} and 
only uses the righthand sides of Equations 
\ref{eq:weak_sigma} and \ref{eq:weak_tau}.
Finally, extend $f_1$ linearly over $k$ to 
define $f_1$ for all linear combinations of 
morphisms in $\chi$.

Before defining $f_2$, let's take a look 
at an $A_\infty$ relation we expect 
$f_2$ to satisfy: For $\cdot 
\xrightarrow{\mu_1} \cdot 
\xrightarrow{\mu_2} \cdot$ 
composable morphisms in $\chi$, 
we expect
\begin{align}
\label{eq:A_2}
f_1(\mu_2\circ \mu_1) 
= 
f_1(\mu_2) \circ f_1(\mu_1) + 
d_{\mathcal{D}_\infty} \circ f_2(\mu_1, \mu_2).
\end{align}
Given the definition of $f_1$ above, 
we require a non-zero $f_2$ only if: 
(Condition H) 
the presentation corresponding to 
$\mu_2$ composed with the presentation 
corresponding to $\mu_1$ contains, after 
removing (decompositions of) identity 
maps except for $\tau_{n}^{n+1}$, one 
or more of the following terms:
$\tau_{n-1} \delta_{n-1,n}$, 
$\tau_{n+1}\sigma_{n,n}$, 
$\tau_{n}^{n+1}$. If $\mu_1, \mu_2$ 
satisfy Condition H, homotopies given 
in Section \ref{sec:weak_relations} 
can be used to define $f_2$. If 
$\mu_1, \mu_2$ do not satisfy Condition 
H, let $f_2(\mu_1, \mu_2) = 0$.

We will give some instructive examples 
of non-zero $f_2$ that satisfy Equation 
\ref{eq:A_2}.
\begin{eg}
Let $\mu_1 = \delta_{n-1,n}$, $\mu_2 = 
\tau_{n-1}$. Then, the presentation 
corresponding to $\mu_2\mu_1$ is 
$\delta_{0,n}\tau_n^2$. Let 
$f_2(\mu_1, \mu_2)$ be the homotopy given 
in Section \ref{sec:weak_relations_delta} 
(also given in Appendix Proposition 
\ref{prop:c2} or \ref{prop:c4}). 
Then, Equation \ref{eq:A_2} 
is equivalent to Equation \ref{eq:weak_delta}.
\end{eg}
%
\begin{eg}
Let $\mu_1 = \sigma_{0,n-1} \delta_{n-1,n}$, 
$\mu_2 = \tau_{n-1} \delta_{0,n}$. To 
form the presentation corresponding to 
$\mu_2\mu_1$, we follow these steps: 
$$
\tau_{n-1} \delta_{0,n} \sigma_{0,n-1} 
\delta_{n-1,n} 
\xrightarrow[\textrm{of identities}]{\textrm{remove decompositions}}
\tau_{n-1} \delta_{n-1,n}
\xrightarrow{\textrm{replace}}
\delta_{0,n}\tau_n^2.
$$
On the other hand, 
\begin{align*}
f_1(\mu_2)f_1(\mu_1) 
&= 
(\widehat{\delta_{0,n}\sigma_{0,n-1}
  \delta_{n-1,n}})^*(\tau_{n-1!}) \circ
  (\widehat{\sigma_{0,n-1}\delta_{n-1,n}})^*
  (\delta_{0,n!}) \circ 
  \hat{\delta}_{n-1,n}^*(\sigma_{0,n-1!}) 
  \circ \delta_{n-1,n!}\\
&= 
\hat{\delta}_{n-1,n}^*(\tau_{n-1!}) \circ
  id \circ \delta_{n-1,n!}.
\end{align*}
So, we can let $f_2(\mu_1, \mu_2)$ be the 
homotopy given in Section 
\ref{sec:weak_relations_delta}, and Equation 
\ref{eq:A_2} is equivalent to Equation 
\ref{eq:weak_delta}.
\end{eg}
%
\begin{eg}
Let $(\mu_1, \mu_2) \in \{ (\tau_{n+1}, 
\sigma_{n,n}), (\tau_{n}^{n+1-j}, 
\tau_n^j): 1\leq j \leq n, n \in \mathbb{N}
\}$. Let $f_2(\mu_1, \mu_2)$ be 
the homotopy given in 
\ref{sec:weak_relations_sigma} if $\mu_2 = 
\sigma_{n,n}$ and the homotopy given in 
\ref{sec:weak_relations_tau} if $\mu_2 \neq 
\sigma_{n,n}$. Then, Equation \ref{eq:A_2} 
is equivalent to either Equation 
\ref{eq:weak_sigma} ($\mu_2 = 
\sigma_{n,n}$) or Equation \ref{eq:weak_tau} 
($\mu_2 \neq \sigma_{n,n}$).
\end{eg}
%
\begin{eg}
Let $\mu_1 = \sigma_{n-1,n-1} \delta_{n-1,n}$, 
$\mu_2 = \tau_n$. To 
form the presentation corresponding to 
$\mu_2\mu_1$, we follow these steps: 
$$
(\tau_n \sigma_{0,n-1}) 
\delta_{n-1,n} 
\xrightarrow{\textrm{replace $(\cdot)$}}
\tau_n^n \sigma_{0,n-1}(\tau_{n-1}\delta_{n-1,n})
\xrightarrow{\textrm{replace $(\cdot)$}}
\tau_n^n \sigma_{0,n-1}\delta_{0,n}\tau_n^2.
$$
Let $f_2(\mu_1, \mu_2) = g_1 + g_2$ where 
$g_1 = \hat{\delta}_{n-1,n}^*( 
\textrm{homotopy in Section 
\ref{sec:weak_relations_sigma}})\circ
\delta_{n-1,n!}$ and $g_2 = 
(\widehat{\tau_{n-1}\delta_{n-1,n}})^*\big(
(\widehat{\tau_n^{n-1}\sigma_{0,n-1}})^*
  (\tau_{n!}) \circ \smdots \circ 
  \hat{\sigma}_{0,n-1}^*(\tau_{n!}) \circ 
  \sigma_{0,n-1!} \big) \circ 
\textrm{(homotopy in Section 
\ref{sec:weak_relations_delta})}$. 
Then, Equation \ref{eq:A_2} 
reduces to $\delta_{n-1,n}^*$(Equation 
\ref{eq:weak_sigma}) and Equation 
\ref{eq:weak_delta}.
\end{eg}
%
Now, we will check that the $f_i$'s 
we gave satisfy the rest of the 
relations for an $A_\infty$-functor 
from Reference \cite{F}, Definition A.8:
For $\cdot 
\xrightarrow{\mu_1} \cdot 
\xrightarrow{\mu_2} \cdot
\xrightarrow{\mu_3} \cdot 
\xrightarrow{\mu_4} \cdot $ 
composable morphisms in $\chi$, 
we expect
\begin{align} 
0
&= 
d_{\mathcal{D}_\infty} \circ f_1(\mu_1)
\label{eq:A_1}\\
f_2(\mu_3, \mu_2 \circ \mu_1) - 
  f_2(\mu_3 \circ \mu_2, \mu_1)
&= 
f_2(\mu_3, \mu_2) \circ f_1(\mu_1) - 
  f_1(\mu_3) \circ f_2(\mu_2, \mu_1)
\label{eq:A_3}\\  
0
&= 
f_2(\mu_4, \mu_3) \circ f_2(\mu_2, \mu_1).  
\label{eq:A_4}
\end{align}
Equation \ref{eq:A_1} is satisfied 
since the $\lambda_!$'s we defined 
in Section \ref{sec:shriek_maps} are maps 
of complexes. Equation \ref{eq:A_4} is 
satisfied since composing two of our 
degree $-1$ homotopies is always equal to 
zero (see Section 
\ref{sec:higher_homotopies}, first 
paragraph). Finally, Equation \ref{eq:A_3} 
boils down to showing that the two 
homotopies in Figure \ref{fig:two_homotopies} 
and in Figure \ref{fig:two_homotopies_1} are 
the same (see Section 
\ref{sec:higher_homotopies}).




