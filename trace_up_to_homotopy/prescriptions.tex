\section{Prescriptions for 
$\mathcal{F}(\mu_1, \dots, \mu_n)$} 

\subsection{Prescription for $\mathcal{F}(\mu)$} \label{sec:F_of_mu}
Now, we will define $\mathcal{F}(\mu)$ for 
$\mu$ not a generating morphism in $\Lambda$. 
(A general morphism in $\chi(\mathcal{C})$ is a 
linear combination of morphisms in $\Lambda$, so 
we extend $\mathcal{F}$ linearly to define 
$\mathcal{F}$ on any morphism in $\chi(\mathcal{C})$, 
see Definition \ref{def:chi}.)

Let $\mu$ be a non-generating morphism 
in $\Lambda$ that induces a morphsim in 
$\chi(\mathcal{C})$ with source $\mathcal{A} :=
(A_0 \to \dots \to A_n \to A_0)$ for some 
algebras $A_i$, $0 \leq i \leq n$, $n \geq 0$. 
Choose (i.e., fix once and 
for all) a presentation of $\mu$ as a 
composition of generating morphisms. 
Within the chosen presentation, in the 
following order, (1) replace 
all occurrences of $\tau_{n-1} \delta_{n-1,n}$ 
with $\delta_{0,n}\tau_n^2$, (2) replace all 
$\tau_{n+1}\sigma_{n,n}$ with 
$\tau_{n+1}^{n+1}\sigma_{0,n}\tau_n$, 
(3) replace all decompositions of identity 
maps with identity maps, (4) remove all 
identity maps if $\mu \neq id$, (5) call 
this new presentation 
$``$the presentation corresponding to $\mu$'', 
denoted $\mu = \lambda_{\mu,k_\mu}\smdots 
\lambda_{\mu,1}$. The presentation 
corresponding to $\mu$ is not unique (i.e., 
still depends on the original chosen 
presentation). However, letting $\mathcal{F}(\mu)$ 
act on comodules via
\begin{align*}
\mathcal{F}(\mu) := 
\hat{\lambda}_{\mu,1}^*\smdots
  \hat{\lambda}_{\mu,k_\mu-1}^*
  (\lambda_{\mu,k_\mu!})
  \circ \smdots \circ
  \hat{\lambda}_{\mu,1}^*(\lambda_{\mu,2!})
  \circ \lambda_{\mu,1!}: 
  T(\mathcal{A})
  \to \hat{\mu}^*T(\mu \mathcal{A})
\end{align*}
is well-defined because we have made 
consistent choices. More explicitly, 
we show in Section 
\ref{sec:composition_relations} that
the choices we have made for 
$\mathcal{F}(\{ 
\textrm{generating morphisms}\})$ 
respect all of the relations in $\Lambda$ 
(Equation \ref{eqn:cyclic_relations}) 
except for Equations \ref{eq:weak}. 
The above steps ensure that the 
presentation corresponding 
to $\mu$ only uses the lefthand 
side of Equation \ref{eq:weak_delta} and 
the righthand sides of Equations 
\ref{eq:weak_sigma} and \ref{eq:weak_tau}.
%
\subsection{Prescription for $\mathcal{F}(\mu_1, \mu_2)$} \label{sec:F_of_mu_2}
Before defining $\mathcal{F}$ on pairs of 
composable morphisms, let's take a look 
at an $A_\infty$ relation we expect 
$\mathcal{F}$ to satisfy: For $\cdot 
\xrightarrow{\mu_1} \cdot 
\xrightarrow{\mu_2} \cdot$ 
composable morphisms in $\chi(\mathcal{C})$, 
we expect
\begin{align}
\label{eq:A_2}
\mathcal{F}(\mu_2\circ \mu_1) 
= 
\mathcal{F}(\mu_2) \circ \mathcal{F}(\mu_1) + 
d_{\mathcal{D}_\infty} \circ \mathcal{F}(\mu_1, \mu_2).
\end{align}
Given the definition of $\mathcal{F}(\mu)$ above, 
we require a non-zero $\mathcal{F}(\mu_1, \mu_2)$ 
if and only if: 
(Condition H) 
the presentation corresponding to 
$\mu_2$ composed with the presentation 
corresponding to $\mu_1$ contains, after 
removing (decompositions of) identity 
maps except for $\tau_{n}^{n+1}$, one 
or more of the following terms:
$\tau_{n-1} \delta_{n-1,n}$, 
$\tau_{n+1}\sigma_{n,n}$, 
$\tau_{n}^{n+1}$. If $\mu_1, \mu_2$ 
satisfy Condition H, homotopies given 
in Section \ref{sec:weak_relations} 
can be used to define $\mathcal{F}(\mu_1, \mu_2)$. 
If $\mu_1, \mu_2$ do not satisfy Condition 
H, let $\mathcal{F}(\mu_1, \mu_2) = 0$ 
on comodules.

We will give some instructive examples 
of non-zero $\mathcal{F}(\mu_1, \mu_2)$ that satisfy Equation 
\ref{eq:A_2}.
\begin{eg}
Let $\mu_1 = \delta_{n-1,n}$, $\mu_2 = 
\tau_{n-1}$. Then, the presentation 
corresponding to $\mu_2\mu_1$ is 
$\delta_{0,n}\tau_n^2$. Let 
$\mathcal{F}(\mu_1, \mu_2)$ be the homotopy given 
in Section \ref{sec:weak_relations_delta}. 
Then, Equation \ref{eq:A_2} holds because it 
is equivalent to Equation \ref{eq:weak_delta}.
\end{eg}
%
\begin{eg}
Let $\mu_1 = \sigma_{0,n-1} \delta_{n-1,n}$, 
$\mu_2 = \tau_{n-1} \delta_{0,n}$. To 
form the presentation corresponding to 
$\mu_2\mu_1$, we follow these steps: 
$$
\tau_{n-1} \delta_{0,n} \sigma_{0,n-1} 
\delta_{n-1,n} 
\xrightarrow[\textrm{of identities}]{\textrm{remove decompositions}}
\tau_{n-1} \delta_{n-1,n}
\xrightarrow{\textrm{replace}}
\delta_{0,n}\tau_n^2.
$$
On the other hand, 
\begin{align*}
\mathcal{F}(\mu_2)\mathcal{F}(\mu_1) 
&= 
(\widehat{\delta_{0,n}\sigma_{0,n-1}
  \delta_{n-1,n}})^*(\tau_{n-1!}) \circ
  (\widehat{\sigma_{0,n-1}\delta_{n-1,n}})^*
  (\delta_{0,n!}) \circ 
  \hat{\delta}_{n-1,n}^*(\sigma_{0,n-1!}) 
  \circ \delta_{n-1,n!}\\
&= 
\hat{\delta}_{n-1,n}^*(\tau_{n-1!}) \circ
  id \circ \delta_{n-1,n!}.
\end{align*}
So, we can let $\mathcal{F}(\mu_1, \mu_2)$ be the 
homotopy given in Section 
\ref{sec:weak_relations_delta}, and Equation 
\ref{eq:A_2} holds because it is equivalent to Equation 
\ref{eq:weak_delta}.
\end{eg}
%
\begin{eg}
Let $(\mu_1, \mu_2) \in \{ (\tau_{n+1}, 
\sigma_{n,n}), (\tau_{n}^{n+1-j}, 
\tau_n^j): 1\leq j \leq n, n \in \mathbb{N}
\}$. Let $\mathcal{F}(\mu_1, \mu_2)$ be 
the homotopy given in 
\ref{sec:weak_relations_sigma} if $\mu_2 = 
\sigma_{n,n}$ and the homotopy given in 
\ref{sec:weak_relations_tau} if $\mu_2 \neq 
\sigma_{n,n}$. Then, Equation \ref{eq:A_2} 
holds because it 
is equivalent to either Equation 
\ref{eq:weak_sigma} ($\mu_2 = 
\sigma_{n,n}$) or Equation \ref{eq:weak_tau} 
($\mu_2 \neq \sigma_{n,n}$).
\end{eg}
%
\begin{eg}
Let $\mu_1 = \sigma_{n-1,n-1} \delta_{n-1,n}$, 
$\mu_2 = \tau_n$. To 
form the presentation corresponding to 
$\mu_2\mu_1$, we follow these steps: 
$$
(\tau_n \sigma_{0,n-1}) 
\delta_{n-1,n} 
\xrightarrow{\textrm{replace $(\cdot)$}}
\tau_n^n \sigma_{0,n-1}(\tau_{n-1}\delta_{n-1,n})
\xrightarrow{\textrm{replace $(\cdot)$}}
\tau_n^n \sigma_{0,n-1}\delta_{0,n}\tau_n^2.
$$
Let $\mathcal{F}(\mu_1, \mu_2) = g_1 + g_2$ where 
$g_1 = \hat{\delta}_{n-1,n}^*( 
\textrm{homotopy in Section 
\ref{sec:weak_relations_sigma}})\circ
\delta_{n-1,n!}$ and $g_2 = 
(\widehat{\tau_{n-1}\delta_{n-1,n}})^*\big(
(\widehat{\tau_n^{n-1}\sigma_{0,n-1}})^*
  (\tau_{n!}) \circ \smdots \circ 
  \hat{\sigma}_{0,n-1}^*(\tau_{n!}) \circ 
  \sigma_{0,n-1!} \big) \circ 
\textrm{(homotopy in Section 
\ref{sec:weak_relations_delta})}$. 
Then, Equation \ref{eq:A_2} holds because it 
reduces to $\delta_{n-1,n}^*$(Equation 
\ref{eq:weak_sigma}) and Equation 
\ref{eq:weak_delta}.
\end{eg}
%