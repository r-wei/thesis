
\section{Verification of $A_\infty$ relations} \label{sec:verify_A_relations}
Now, we will check that our choices 
for $\mathcal{F}$ satisfy the rest of the 
relations for an $A_\infty$-functor 
from Reference \cite{F}, Definition A.8:
For $\cdot 
\xrightarrow{\mu_1} \cdot 
\xrightarrow{\mu_2} \cdot
\xrightarrow{\mu_3} \cdot 
\xrightarrow{\mu_4} \cdot $ 
composable morphisms in $\chi(\mathcal{C})$, 
we expect
\begin{align} 
0
&= 
d_{\mathcal{D}} \circ \mathcal{F}(\mu_1)
\label{eq:A_1}\\
\mathcal{F}(\mu_3, \mu_2 \circ \mu_1) - 
  \mathcal{F}(\mu_3 \circ \mu_2, \mu_1)
&= 
\mathcal{F}(\mu_3, \mu_2) \circ \mathcal{F}(\mu_1) - 
  \mathcal{F}(\mu_3) \circ \mathcal{F}(\mu_2, \mu_1)
\label{eq:A_3}\\  
0
&= 
\mathcal{F}(\mu_4, \mu_3) \circ \mathcal{F}(\mu_2, \mu_1).  
\label{eq:A_4}
\end{align}
Equation \ref{eq:A_1} is satisfied 
since, for $\lambda \in \Lambda$ 
a generating morphism, the $\lambda_!$'s we 
gave at the beginning of Section 
\ref{sec:composition_relations} are maps 
of complexes. Equation \ref{eq:A_4} requires 
that composing two of our 
degree $-1$ homotopies is always equal to 
zero. This is true because we use 
reduced Hochschild chains (Section 
\ref{chap:hochschild}) and each homotopy 
(Equations \ref{eq:def_sigma}, \ref{eq:def_sigma2}) 
inserts a 1 into the first slot of the 
Hochschild chains component. 

We check that Equation \ref{eq:A_3} holds for 
$n=1$ and $n \geq 2$ separately. For $n\geq2$, 
checking Equation \ref{eq:A_3} boils down 
to the following situation:
We have two maps of dg comodules
\begin{equation}
\label{eq:two_maps}
\xymatrixrowsep{5pc}
\xymatrix{
T(A_0 \to \smdots \to A_n \to A_0) 
 \ar@/^1pc/[d]^{\substack{
   (\widehat{\delta_{n-2,n-1}\delta_{n-1,n}})^*
   \tau_{n-2!}\\\\\\
   \textrm{$``$brace together the last 3 algebras,}\\
   \textrm{then apply $\tau_{n-2!}$ once''}}}
 \ar@/_1pc/[d]_{\substack{
   \hat{\tau}_n^{*2} \tau_{n!}\circ
   \hat{\tau}_n^*\tau_{n!}\circ \tau_{n!}\\\\\\
   \textrm{$``$apply $\tau_{n!}$ 3 times''}}}\\
T(A_{n-2} \to A_{n-1} \to A_n \to 
A_0 \to \smdots \to A_{n-2}).
}
\end{equation}
These two maps are homotopic via 
two homotopies: 
$\hat{\delta}_{n-1,n}^*
  \mathcal{B}(A_0 \bullet \dots \bullet A_{n-3}, 
  A_{n-2}, A_{n-1}\bullet A_n)
+ \tau_n^{*2}\tau_{n!} \circ
  \mathcal{B}(A_0 \bullet \dots \bullet A_{n-2}, A_{n-1}, A_n)$ 
and 
$\hat{\delta}_{n-2,n}^*\mathcal{B}(A_0 \bullet \dots \bullet A_{n-3}, 
  A_{n-2} \bullet A_{n-1}, A_n)
+ \hat{\tau}_n^*\mathcal{B}(A_n \bullet \dots \bullet A_{n-3}, 
  A_{n-2}, A_{n-1}) \circ \tau_{n!}$ (see Figure 
\ref{fig:two_homotopies}). If the 
two homotopies were different, then 
their difference would be closed and
we would desire a higher homotopy (i.e., 
a degree -2 map of comodules) between 
them. However, we will show the 
two homotopies are the same, so that 
no higher homotopies are needed. 

First, it follows directly from the definition 
of $\mathcal{B}$ (Appendix Equation 
\ref{eq:def_sigma2}) that
$$\hat{\delta}_{n-1,n}^*
  \mathcal{B}(A_0 \bullet \dots \bullet A_{n-3}, 
  A_{n-2}, A_{n-1}\bullet A_n)
= \hat{\delta}_{n-2,n}^*\mathcal{B}(A_0 \bullet \dots \bullet A_{n-3}, 
  A_{n-2} \bullet A_{n-1}, A_n).$$

Second, for $n=2$, we show that 
\begin{equation} \label{eq:n_2_homotopy}
\tau_2^{*2}\tau_{2!} \circ
  \mathcal{B}(A_0, A_1, A_2)
  = \hat{\tau}_2^*\mathcal{B}(A_2, A_0, A_1) 
  \circ \tau_{2!}
\end{equation} 
in Appendix Proposition \ref{prop:c5}. 
(In the appendix, 
$\tau_2^{*2}\tau_{2!} = \tau_{1!}(A_1 \bullet A_2, A_0)$ 
and $\tau_{21} = \tau_{1!}(A_0 \bullet A_1, A_2)$.) 
For $n>2$, the equation $\tau_n^{*2}\tau_{n!} \circ
  \mathcal{B}(A_0 \bullet \dots \bullet A_{n-2}, A_{n-1}, A_n)
  = \hat{\tau}_n^*\mathcal{B}(A_n \bullet \dots \bullet A_{n-3}, 
  A_{n-2}, A_{n-1}) \circ \tau_{n!}$ 
is a pullback along $\hat{\delta}_{0}$'s of 
Equation \ref{eq:n_2_homotopy}.
%
\begin{figure}%[H]
\centerline{
\xymatrixrowsep{3pc}
\xymatrix{
\underset{\substack{\\\\
  \textrm{$``$brace together $A_{n-2}, A_{n-1}, A_n$,}\\
  \textrm{then apply $\tau_{n-2!}$''}}}
  {(\widehat{\delta_{n-2,n-1}
   \delta_{n-1,n}})^*\tau_{n-2!}}
\ar[r]^{\cong}
\ar[d]_{\cong}   
& \hat{\delta}_{n-1,n}^*(
  \hat{\delta}_{n-2,n-1}^*\tau_{n-2!})   
\ar[r]_{\substack{\\\\
  \hat{\delta}_{n-1,n}^*
  \mathcal{B}(A_0 \bullet \dots \bullet A_{n-3}, 
  A_{n-2}, A_{n-1}\bullet A_n)}}
& \hat{\delta}_{n-1,n}^*(
  \hat{\tau}_{n-1}^*\tau_{n-1!} \circ \tau_{n-1!})
\ar[d]^{\cong}  \\
(\widehat{\delta_{n-2,n-1}\delta_{n-2,n}})^*
  \tau_{n-2!}  
\ar[d]^{\hat{\delta}_{n-2,n}^*\mathcal{B}(A_0 \bullet \dots \bullet A_{n-3}, 
A_{n-2} \bullet A_{n-1}, A_n)}
&& \underset{\substack{ \\\\
    \textrm{$``$brace together $A_{n-1}, A_n$}\\
    \textrm{and apply $\tau_{n-1!}$,}\\
    \textrm{then apply $\tau_{n!}$''}}}
    {\hat{\tau}_n^{*2}\tau_{n!} \circ 
    \hat{\delta}_{n-1,n}^*\tau_{n-1!}} 
\ar[d]_{\substack{\\\\
       \tau_n^{*2}\tau_{n!} \circ
       \mathcal{B}(A_0 \bullet \dots \bullet A_{n-2}, A_{n-1}, A_n)}}\\
\hat{\delta}_{n-2,n}^*(
  \hat{\tau}_{n-1}^*\tau_{n-1!} \circ \tau_{n-1!})
\ar[r]_{\cong}                   
& \underset{\substack{\\\\
          \textrm{$``$apply $\tau_{n!}$,}\\
          \textrm{then brace together $A_{n-1}, A_{n-2}$}\\
          \textrm{and apply $\tau_{n-1!}$''}}}
         {\hat{\tau}_n^*(\hat{\delta}_{n-1,n}^*
          \tau_{n-1!}) \circ \tau_{n!}}
\ar[r]^{\substack{
  \hat{\tau}_n^*\mathcal{B}(A_n \bullet \dots \bullet A_{n-3}, 
  A_{n-2}, A_{n-1})\\ \circ \tau_{n!}\\}}
& \underset{\substack{\\\\
    \textrm{$``$apply $\tau_{n!}$ three times''}}}
    {\hat{\tau}_n^{*2} \tau_{n!}\circ
      \hat{\tau}_n^*\tau_{n!}\circ \tau_{n!}}
}}
\caption{Two homotopies between 
$(\widehat{\delta_{n-2,n-1}\delta_{n-1,n}})^*
\tau_{n-2!}$ and $\hat{\tau}_n^{*2} \tau_{n!}\circ
\hat{\tau}_n^*\tau_{n!}\circ \tau_{n!}$}
\label{fig:two_homotopies}
Vertices are maps of dg comodules and 
arrows are chain homotopies.
\end{figure}

For $n=1$, the situation in Equation 
\ref{eq:two_maps} reduces to: 
We have two maps of dg comodules
$$
\xymatrix{
T(A_0 \to A_1 \to A_0) 
 \ar@/^1pc/[d]^{\tau_{1!}}
 \ar@/_1pc/[d]_{\hat{\tau}_1^{*2} \tau_{1!}\circ
   \hat{\tau}_1^*\tau_{1!}\circ \tau_{1!}}\\
T(A_1 \to A_0 \to A_1).
}
$$
These two maps are homotopic via 
two homotopies: 
$\tau_{1!}(A_0,A_1) \circ B(A_0, A_1)$
and 
$B(A_1, A_0) \circ \tau_{1!}(A_0,A_1)$ 
(see Figure \ref{fig:two_homotopies_1}). 
We show that these two 
homotopies are the same in Appendix Proposition 
\ref{prop:c3}, so no higher homotopies are 
needed.
%
\begin{figure}%[H]
\xymatrixrowsep{4pc}
\centerline{\xymatrix{
id \circ \tau_{1!} = \tau_{1!} = \tau_{1!} \circ id
 \ar@/^1pc/[d]^{\tau_{1!}(A_0,A_1) \circ B(A_0, A_1)}
 \ar@/_1pc/[d]_{B(A_1, A_0) \circ \tau_{1!}(A_0,A_1)}\\
\big( \hat{\tau}_1^{*2} \tau_{1!}\circ
   \hat{\tau}_1^*\tau_{1!}  \big)  \circ \tau_{1!} 
=   
\hat{\tau}_1^{*2} \tau_{1!}\circ
   \big( \hat{\tau}_1^*\tau_{1!}\circ \tau_{1!} \big)
}}
\caption{Two homotopies between 
$\tau_{1!}$ and $\hat{\tau}_1^{*2} \tau_{1!}\circ
\hat{\tau}_1^*\tau_{1!}\circ \tau_{1!}$}
\label{fig:two_homotopies_1}
Vertices are maps of dg comodules and 
arrows are chain homotopies.
\end{figure}