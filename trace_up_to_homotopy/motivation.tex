\section{Motivation of this chapter}
In this chapter, we give a trace functor 
up to homotopy on the category $\mathcal{C}$ 
defined in Equation \ref{eq:cat_in_dgcocat}. 
To do so, we give an $A_\infty$-functor 
$\mathcal{F}: \chi(\mathcal{C}) \to 
\mathcal{D}$ satisfying 
certain requirements (see Definition 
\ref{def:trace_up_to_homotopy}). Applying 
the definition of an $A_\infty$-functor 
(from Reference \cite{F}, 
Appendix A, Definition A.8), the 
only choices we need to make to define 
$\mathcal{F}$ are:
\begin{enumerate}
	\item for each algebra $A$, 
	a dg comodule $T(A)$ over 
	$\mathcal{C}(A, A)$, 
	\item for a functor of dg cocategories 
	$F: C_1 \to C_0$ and a dg comodule $T_0$ over 
	$C_0$, a definition of a pullback 
	$F^*T_0$ that is natural in $T_0$ and satisfies 
	Equation \ref{eq:pullback_yoga},
	\item for each pair of algebras $A,B$, 
	a map of dg comodules over 
	$\mathcal{C}(A,B) \otimes \mathcal{C}(B,A)$
	$$\tau_{1!}(A,B):T(A) \to \hat{\tau}_1^* T(B)$$
	where $\hat{\tau}_1: \mathcal{C}(A,B) \otimes 
	\mathcal{C}(B,A) \to \mathcal{C}(B,A) \otimes 
	\mathcal{C}(A,B)$ is rotation,
	\item for each non-generating 
	morphism $\mu \in \chi(\mathcal{C})$, 
	a map of dg comodules $\mathcal{F}(\mu) \in 
	\mathcal{D}$,
	\item for each pair of morphisms $\mu_1, \mu_2 
	\in \chi(\mathcal{C})$, 
	a degree-1 map of comodules 
	$\mathcal{F}(\mu_1, \mu_2) \in 
	\mathcal{D}$, 
	\item for each sequence of morphisms 
	$\mu_1, \dots, \mu_n \in \chi(\mathcal{C})$ 
	where $n>2$, a degree-(n-1) map of comodules 
	$\mathcal{F}(\mu_1, \dots, \mu_n) \in 
	\mathcal{D}$.
\end{enumerate}
In Section \ref{sec:dg_comod}, we define item (1), the dg 
comodule $T(A)$, which is a (categorified) bar 
construction of the module $C_{\bullet}(A,A)$ over 
the algebra $C^\bullet(A,A)$ acting via contraction. 
In Appendix \ref{chap:pullbacks}, we give item (2) as well as 
compute some examples of pullbacks for later use. 
In Proposition \ref{prop:c1}, we define item (3) by 
adapting known equations for the Lie derivative of a 
Hochschild cochain against a chain. 
In Section \ref{sec:F_of_mu}, we give a prescription for 
defining item (4). We see that $\mathcal{F}$ 
respects composition except for a few cases 
(Section \ref{sec:composition_relations}), and 
we give a prescription for 
defining the few non-zero $\mathcal{F}(\mu_1, \mu_2)$'s 
in item (5) (Section \ref{sec:F_of_mu_2}). 
Finally, for item (6), we set 
$\mathcal{F}(\mu_1, \dots, \mu_n) = $ (zero map 
on comodules) for all composable $mu_1, \dots, \mu_n$, 
$n>2$: this is the claim that we have no higher 
homotopies, justified in Section \ref{sec:verify_A_relations}.