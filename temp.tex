
\section{The functors $\lambda^*$}
Let $\lambda:[n] \to [m]$ be a morphism in $\Lambda$. 
We will define a functor $\lambda^*$ from the category of 
dg comodules over $B(m)$ to dg comodules over $B(n)$. 
It is enough to define $\lambda^*$ for the generators of 
the cyclic category, i.e., for $\lambda \in \{
\tau_n, \delta_{j,n}, \sigma_{i,n}\}$ (see Equation 
\ref{eqn:cyclic_generators}). (We will not worry about 
relations in $\Lambda$ for now--that will be addressed 
in the following chapters.)

\subsection{Category-theoretic definition of $\lambda^*$}
For $\lambda:[n] \to [m]$ in $\Lambda$, we have a given a 
functor, which we will also call $\lambda$, from 
$B(n) \to B(m)$ (see \ref{sec:cyclic_B(n)}). Let $C$ be an 
arbitrary dg comodule over $B(m)$. We define 
$\lambda^* C$ as follows:
$$
\lambda^*C := 
coker \big( B(n)\otimes C 
\mathrel{\mathop{\rightrightarrows}^
{\mathrm{1\otimes \Delta_{C}}}_
{\mathrm{(1\otimes \lambda \otimes 1)\circ (\Delta_{B(n)}\otimes 1})}}
B(n) \otimes B(m)\otimes C \big).
$$

Let $f:C\to D$ be a map of dg comodules over $B(m)$. 
By the universal property of $\lambda^*C$, we can define a 
morphism $\lambda^*f:\lambda^*C \to \lambda^*D$ by giving a 
morphism from $(\lambda^*f)^\prime:B(n) \otimes B(m) \otimes C 
\to \lambda^*D$ such that the two maps 
\begin{equation}\label{eqn:coker}
(\lambda^*f)^\prime \circ (1\otimes \Delta_C), \>
(\lambda^*f)^\prime \circ (1\otimes \lambda \otimes 1)\circ 
(\Delta_{B(n)}\otimes 1): 
B(n) \otimes C \rightrightarrows 
B(n) \otimes B(m) \otimes C \to \lambda^*D
\end{equation}
coincide. We define $(\lambda^*f)^\prime$ as follows:
$$
(\lambda^*f)^\prime:
B(n) \otimes B(m) \otimes C 
\xrightarrow{1\otimes 1\otimes f}
B(n) \otimes B(m) \otimes D
\xrightarrow[projection]{canonical}
\lambda^*D
$$
It's easy to check that this map is well-defined (i.e., 
that the two maps in Equation \ref{eqn:coker} coincide): 
Let $b\otimes c$ be an arbitrary element of $B(n)\otimes C$. 
Then,
\begin{equation*}
\begin{split}
[(\lambda^*f)^\prime \circ (1\otimes \Delta_C)] (b\otimes c) = 
\Sigma_i b \otimes c^i_{(1)}\otimes fc^i_{(2)}\\
[(\lambda^*f)^\prime \circ (1\otimes \lambda \otimes 1)\circ 
(\Delta_{B(n)}\otimes 1)] (b \otimes c) =
\Sigma_i b^i_{(1)}\otimes \lambda b^i_{(2)} \otimes fc.
\end{split}
\end{equation*}
And the righthand sides are equal in the quotient, $\lambda^*D$,
via the image of $b\otimes fc \in B(n)\otimes D$. More explicitly,
\begin{align*}
(1\otimes \Delta_D) (b\otimes fc) 
&= [(1\otimes \Delta_D) \circ (1\otimes f)] (b\otimes c) \\
&= [(1 \otimes 1 \otimes f) \circ (1\otimes \Delta_C)]
  (b\otimes c) \quad \textrm{(f is a map of comodules)}\\
&= \Sigma_i b \otimes c^i_{(1)}\otimes fc^i_{(2)}\\
[(1\otimes \lambda \otimes 1)\circ 
(\Delta_{B(n)}\otimes 1)] (b \otimes fc)
&= \Sigma_i b^i_{(1)}\otimes \lambda b^i_{(2)} \otimes fc\\
\end{align*}
where is $\Delta(b) = \Sigma_i b^i_{(1)}\otimes b^i_{(2)}$
is Sweedler notation.
So, $\lambda^*f$ is well-defined. In summary, we have commuting 
diagrams:
\begin{equation}\label{cd:lambda^*f}
\begin{CD}
B(n)\otimes B(m)\otimes C  
@>1\otimes 1\otimes f =\textrm{"lift" of }\lambda^*f>>  
B(n)\otimes B(m)\otimes D \\
@V\substack{\textrm{canonical}\\ \textrm{projection}}VV
@V\substack{\textrm{canonical}\\ \textrm{projection}}VV \\
\lambda^*C @>\lambda^*f>> \lambda^*D.
\end{CD}
\end{equation}
Finally, it is straightforward to see that $\lambda^*$ 
is a functor, i.e., that $\lambda^*$ preserves composition
of morphisms: Let $C \overset{f}{\to} D \overset{g}{\to} E$ be 
composable morphisms of dg comodules over $B(m)$. The lifts of
$\lambda^*f$, $\lambda^*g$ and $\lambda^*(g\circ f)$ are 
$1\otimes 1\otimes f$, $1\otimes 1\otimes g$ and 
$1\otimes 1\otimes gf$, respectively. The lifts 
respect composition--$(1\otimes 1\otimes g) \circ 
(1\otimes 1\otimes f) = 1\otimes 1\otimes gf$--and by 
the commuting diagrams \ref{cd:lambda^*f}, the 
functor $\lambda^*$ does as well.

\subsection{Explicit description of $\lambda^*C(m)$}
In this section, we describe $\lambda^*C(m)$ explicitly as 
complexes. Recall from \ref{sec:comod_strre} that $C(m)$ is 
quasi-cofree, and let $T(m)$ denote the cogenerators of 
$C(m)$.


\section{The functors $\lambda_*$}
\section{Adjunction between $\lambda^*$ and $\lambda_*$}
\section{Maps $\lambda_!$}