% The usages of \bibitem and \cite{..} are 
% explained in Section 4.3 (page 73) of % LaTeX manual.
% Or you may use BibTeX.

% (The following suggested by Francisco Iacobelli - 5/11/2010)
% In case, you want to use BibTeX, you should replace (or comment)
% the bibliography environment.
% Instead uncomment the following 3 lines and replace <bib file>
% with your .bib file:
% \renewcommand\refname{\begin{centering}References\end{centering}}
% \bibliography{<bib file>}
% \bibliographystyle{acm} %or another suitable style.


\begin{thebibliography}{}

\bibitem{DTT} Dolgushev, V. A., Tamarkin, D. E., Tsygan, B. L. (2008). Formality of the homotopy algebra of Hochschild (co)chains. Retrieved from arxiv.org/pdf/0807.5117v1.pdf

\bibitem{F} Faonte, G. (2014). $A_\infty$-Functors and Homotopy Theory of DG-Categories. Retrieved from arxiv.org/pdf/1412.1255.pdf

\bibitem{HKR} Hochschild, G. P., Kostant, B., $\&$ Rosenberg, A. L. (1962). Differential forms on regular affine algebras. \textit{Transactions AMS, 102}(3), 383–408. doi:10.2307/1993614.

\bibitem{K} Kontsevich, M. L. (2003). Deformation quantization of Poisson manifolds. \textit{Lett. Math. Phys., 66}, 157-216.

\bibitem{Tam} Tamarkin, D. E. (1998). Another proof of M. Kontsevich formality theorem. Retrieved from arxiv.org/pdf/math/9803025v4.pdf

\bibitem{T} Tsygan, B. L. (2012). Noncommutative Calculus and Operads. Retrieved from arxiv.org/pdf/1210.5249v1.pdf

\end{thebibliography}