\documentclass[12pt]{nuthesis}	

\usepackage{bm} 	%bold in any mode
\usepackage{amscd}        % For writing commutative diagrams.
\usepackage{eucal}        % Euler fonts.
\usepackage{verbatim}     % Quotations
\usepackage{amsthm}	  % newtheorems
\usepackage{lscape} % landscape
\usepackage{amssymb} % arrows
\usepackage{mathtools} %extensible arrows; underbrace
\usepackage[all]{xy} %diagrams
\usepackage{color} %colored arrows
\usepackage{float} %figures appear here [H]


%%%%%%%%%%%%%%%%%%%%%%% Some Math support %%%%%%%%%%%%%%%%%%%%

%% \theoremstyle{plain} %% This is the default
\newtheorem{thm}{Theorem}[chapter]
\newtheorem{cor}[thm]{Corollary}
\newtheorem{lem}[thm]{Lemma}
\newtheorem{prop}[thm]{Proposition}
\newtheorem{ax}{Axiom}
\theoremstyle{definition}
\newtheorem{defn}{Definition}[section]
\theoremstyle{remark}
\newtheorem{rem}{Remark}[section]
\newtheorem*{notation}{Notation}
\theoremstyle{example}
\newtheorem{eg}{Example}[section]
\newcommand{\breakcell}[2][c]{% break cell in tabular
  \begin{tabular}[#1]{@{}c@{}}#2\end{tabular}}
\newcommand \smdots{\makebox[1em][c]{.\hfil.\hfil.}} %small \dots
\newcommand{\leftmapsto}{\leftarrow\!\shortmid} %\left mapsto

\newenvironment{paren_enclose}
  {\left( \left.\begin{aligned}\end{aligned}\right\r)}
%%%%%%%%%%%%%%%%%%%%%%%%%%%%%%%%%%%
% DATA OF AUTHOR AND DISSERTATION %
%%%%%%%%%%%%%%%%%%%%%%%%%%%%%%%%%%%

\author{Ann Rebecca Wei}

\title{Title of the Dissertation}

%\degree{DOCTOR OF PHILOSOPHY}  % Default: DOCTOR OF PHILOSOPHY

%\field{Mathematics}            % Default: Mathematics

%\graduationmonth{June}         % The default is June or December
                                % depending on current date.

%\graduationyear{2003}          % Default: current year.


% Use \includeonly to select the 
\includeonly{notation, 
appendices/lambda, appendices/computations, 
Bn_Cn/motivation, Bn_Cn/dg_cocats, Bn_Cn/dg_comods,
pb_pf_adj/cyclic_object_Bn, pb_pf_adj/motivation, 
pb_pf_adj/pullbacks, pb_pf_adj/pullback_examples,
pb_pf_adj/adjunction, pb_pf_adj/maps_lambda_shriek, 
homotopies/motivation, homotopies/homotopies, 
homotopies/higher_homotopies}	

\begin{document}
%%%%%%%%%%%%%%%%%%%%%%
% Some initial stuff %
%%%%%%%%%%%%%%%%%%%%%%

\frontmatter		% Preliminary pages start here.

\maketitle		% Produces the title page.

%\copyrightpage		% Creates the copyright page.


\abstract		% Abstract.

This is the abstract.

\acknowledgements	% Acknowledgements (optional).

Text for acknowledgments.

%\preface		% Preface (optional).

%% A few more optional pages (uncomment if needed)
%
\listofabbreviations 

This is the list of abbreviations (optional).

%\glossary
%
%This is the glossary (optional).
%
%\nomenclature
%
%This is the nomenclature (optional).
%
%% Note that the dedication text must be passed as an argument
%% of the \dedication command
%\dedication{This is the dedication (optional).}
%

\clearpage\phantomsection % needed for the hyperlinks to work correctly
\tableofcontents	% Table of Contents will be automatically
			% generated and placed here.

\clearpage\phantomsection % needed for the hyperlinks to work correctly
\listoftables		% List of Tables and List of Figures will be placed

\clearpage\phantomsection % needed for the hyperlinks to work correctly
\listoffigures		% here, if applicable (optional).



\mainmatter             % Actual text starts here.
% If there is an introduction it must be the first chapter
%
\chapter{$B(n)$ and $C(n)$}
	\section{Motivation of this chapter}
In this chapter, we introduce the main 
characters/objects of study, $B(n)$ and 
$C(n)$, $n \in \mathbb{N}$. The $B(n)$'s are 
dg cocategories constructed using 
Hochschild cochains. Each $C(n)$ is a dg 
comodule over $B(n)$ constructed using 
an action of Hochschild cochains on 
Hochschild chains. 
We start with definitions for the 
less-widely-used concepts, and show 
that the main characters are 
conilpotent.
	\section{Dg cocategories: $B(n)$}\label{sec:B(n)}
%
\subsection{Background on dg cocategories}
\begin{defn} A \textbf{dg cocategory} is a 
cocategory enriched over chain complexes. 
More explicitly, a dg cocategory $B$ consists 
of the following data:
\begin{itemize}
\item A collection of objects 
  denoted $Obj(B)$;
\item For each pair of objects, $x,z \in 
  Obj(B)$, a complex $B^\bullet(x,z)$ and 
  a morphism of complexes
  $$
  \Delta_B(x,z): B^\bullet(x,z) \to 
  \prod \limits_{y \in Obj(B)}
  B^\bullet(x,y) \otimes
  B^\bullet(y,z)
  $$
  such that the following diagrams 
  commute (coassociativity):
  $$
  \xymatrixcolsep{5pc}
  \xymatrixrowsep{5pc}
  \xymatrix{
  B^\bullet(x,z)
  \ar[r]^{\Delta_B(x,z)}
  \ar[d]_{\Delta_B(x,z)}
  & \prod \limits_{y \in Obj(B)}
    B^\bullet(x,y) \otimes
    B^\bullet(y,z)
  \ar[d]^{\prod\limits_y 
    id_{B(x,y)} \otimes \Delta_B(y,z)}\\
  \prod \limits_{y \in Obj(B)}
    B^\bullet(x,y) \otimes
    B^\bullet(y,z)
  \ar[r]^{\prod\limits_y 
    \Delta_B(x,y) \otimes id_{B(y,z)}}
  & \prod \limits_{y, y^\prime \in Obj(B)}
    B^\bullet(x,y) \otimes
    B^\bullet(y,y^\prime) \otimes
    B^\bullet(y^\prime,z)  
  }
  $$
\item For each pair of objects, $x,z \in
  Obj(B)$, a morphism of complexes
  $$
  \epsilon_B(x,z): B^\bullet(x,z) \to k
  $$
  where $k$ is the ground field considered 
  as a chain complex concentrated in degree 
  0 and $\epsilon_B(x,z) = 0$ if $x\neq z$, 
  such that the following diagrams commute 
  (counitality):
  $$
  \xymatrixcolsep{5pc}
  \xymatrixrowsep{5pc}
  \xymatrix{
  B^\bullet(x,z)
  \ar[r]^{\Delta_B(x,z)}
  \ar[d]_{\Delta_B(x,z)}
  \ar[rd]_{id}
  & \prod \limits_{y \in Obj(B)} 
    B^\bullet(x,y) \otimes
    B^\bullet(y,z)
  \ar[d]^{\prod \limits_y
    \epsilon_B(x,y)
    \otimes id_{B(y,z)}} \\
  %
  \prod \limits_{y \in Obj(B)} 
    B^\bullet(x,y) \otimes
    B^\bullet(y,z)
  \ar[r]_{\prod \limits_y
    id_{B(x,y)} \otimes 
    \epsilon_B(y,z)}
  & B^\bullet(x,z).  
  }
  $$
\end{itemize}
\end{defn}
We will denote a dg cocategory with its 
cocomposition and counit as $(B, \Delta_B, 
\epsilon_B)$. To make the notation more 
readable, when the meaning is clear, 
we will omit references to the 
objects and write $\Delta_B$ instead of 
$\Delta_B(x,z)$, $\epsilon_B$ instead of 
$\epsilon_B(x,z)$, and for the differentials 
on morphisms, $d_B$ instead of $d_B(x,z)$.
%
\begin{defn} A \textbf{(dg) functor} $F: A \to B$ 
between two dg cocategories is a functor 
between the cocategories satisfying 
$d_B\circ F(f) = F\circ d_A(f)$ for all 
morphisms $f$ in $A$.
\end{defn}
%
\begin{defn} A \textbf{conilpotent} dg 
cocategory is a dg cocategory $(B, 
\Delta_B, \epsilon_B)$ satisfying: for each 
morphism $f:x\to y$ in $B$, there exists 
$n_f \in \mathbb{N}$ such that 
$\bar{\Delta}_B^{n_f}(f) = 0$
where
\begin{align*}
\bar{\Delta}_B(x,z): B^\bullet(x,z) 
&\to 
  \prod \limits_{y \in Obj(B)} 
  B^\bullet(x,y) \otimes
  B^\bullet(y,z)\\
f
&\mapsto
\Delta_B(f)
- \sum \limits_{e_x \in 
  \epsilon_B(x,x)^{-1}(1)}
  e_x \otimes f
- \sum \limits_{e_z \in 
  \epsilon_B(z,z)^{-1}(1)}
  f \otimes e_z. 
\end{align*}
\end{defn}
%
\textbf{Fact (needs reference?):} 
If $B$ is a conilpotent dg cocategory, 
then for all $x \in Obj(B)$, 
$\epsilon_B(x,x)^{-1}(1)$ has exactly 
one element, which we will denote $e_x$.
%
\subsection{Structure of $B(n)$}
For each sequence of algebras, 
$A_0, A_1, \cdots, A_n$, 
we will define a conilpotent dg cocategory, 
$B(A_0 \to A_1 \to \cdots \to A_n \to A_0)$. 
In this chapter, we fix the sequence of 
algebras, and abbreviate 
$$
B(n):= B(A_0 \to A_1 \to \cdots \to A_n \to A_0).
$$
%
\subsubsection{Objects}
$B(n)$ has objects tuples $(f_0, f_1, \cdots, f_n)$ where $f_i : A_i \to A_{i+1 \textrm{ (mod n+1)}}$, $0 \leq i \leq n$, are maps of algebras. We can picture an object in $B(n)$ as follows:

\begin{equation*}
\xymatrix{
A_0 \ar[r]^{f_0}
& A_1 \ar[r]^{f_1}
& A_2 \ar[r]^{f_2}
& \cdots \ar[r]^{f_{n-1}}
& A_n \ar[r]^{f_n}
& A_0
}
\end{equation*}

\subsubsection{Morphisms}
The graded vector space of morphisms in 
$B(n)$ between two objects, 
$(f_0, \smdots, f_n)$ and 
$(g_0, \smdots, g_n)$, is 
%
\begin{equation*}
Bar(C^\bullet(A_0, _{f_0}{A_1}_{g_0})) \otimes Bar(C^\bullet(A_1, _{f_1}{A_2}_{g_1})) \otimes \cdots
\otimes Bar(C^\bullet(A_n, _{f_n}{A_0}_{g_n}))
\end{equation*}
%
where $Bar(C^\bullet(A, _fB_g))$ is 
the following complex: 
%
\begin{align*}
Bar(C^\bullet(A, _fB_g)) 
&:= Bar_0(C^\bullet(A, _fB_g)) \oplus 
  \bigoplus_{\substack{m \geq 1}} 
    Bar_m(C^\bullet(A, _fB_g)) \\
Bar_0(C^\bullet(A, _fB_g))    
&:= k\\ 
Bar_m(C^\bullet(A, _fB_g))
&:= 
  \bigoplus \limits_{\substack{
  h_0 = f, \\
  h_m = g, \\
  h_1, \dots, h_{m-1}\\ 
	\textrm{algebra maps}}} 
  \begin{array}{l}
  C^\bullet(A, \, _{h_0}B_{h_1})[1] \otimes 
  C^\bullet(A, \, _{h_1}B_{h_2})[1] 
  \otimes \dots \otimes \\
  \phantom{{}move{}} \otimes\;
  C^\bullet(A, \, _{h_{m-1}}B_{h_m})[1]
  \end{array}\\
(C^\bullet(A, _{h_i}B_{h_j}), \;
  _{h_i}\delta_{h_j})
&=
\textrm{Hochschild cochain complex, 
see Appendix \ref{chap:hochschild}}
\end{align*}
\begin{align*}
d_{Bar(C^\bullet(A, _fB_g))} 
&= 
\tilde{\delta} + b^\prime\\
\tilde{\delta}(\phi_1 \otimes \smdots \otimes \phi_m)
&=
\sum \limits_{1 \leq i \leq m}
  (-1)^{1+ \sum \limits_{j<i} |\phi_i|+1}
  \phi_1 \otimes \smdots \otimes 
  [_{h_{i-1}}\delta_{h_i}](\phi_i)
  \otimes \smdots \otimes \phi_m\\
b^\prime(\phi_1 \otimes \smdots \otimes \phi_m)
&=
\sum \limits_{1 \leq i \leq m-1}
  (-1)^{\sum \limits_{j\leq i} |\phi_i|+1}
  \phi_1 \otimes \smdots \otimes 
  \phi_i \cup \phi_{i+1}
  \otimes \smdots \otimes \phi_m\\  
\cup
&= 
\textrm{cup product on Hochschild cochains, 
see Appendix \ref{chap:hochschild}.}  
\end{align*}
(This sign convention is consistent with 
Reference \cite{T}, Section 4.6.)
%
\begin{figure}
\centerline{\xymatrix{
A_0 \ar@/^5pc/[r]^{f_{0,0}} 
\ar@/^2pc/[r]^{\big\Downarrow \phi_{0,1}}_{f_{0,1}} 
\ar@/_2pc/[r]^{\substack{\vdots\\ f_{0,k_0-1}\\ \\}}
\ar@/_5pc/[r]^{\substack{\phi_{0,k_0} \\ \big\Downarrow}}_{f_{0,k_0}}
& A_1 \ar@/^5pc/[r]^{f_{1,0}} 
\ar@/^2pc/[r]^{\big\Downarrow \phi_{1,1}}_{f_{1,1}} 
\ar@/_2pc/[r]^{\substack{\vdots\\ f_{1,k_1-1}\\ \\}}
\ar@/_5pc/[r]^{\substack{\phi_{1,k_1} \\ \big\Downarrow}}_{f_{1,k_1}}
& A_2 \ar@{.>}@/^5pc/[r] 
\ar@{.>}@/^2pc/[r]_{\substack{\\ \\ \vdots}} 
\ar@{.>}@/_2pc/[r]
\ar@{.>}@/_5pc/[r]
& \cdots \ar@{.>}@/^5pc/[r] 
\ar@{.>}@/^2pc/[r]_{\substack{\\ \\ \vdots}} 
\ar@{.>}@/_2pc/[r]
\ar@{.>}@/_5pc/[r]
& A_n \ar@/^5pc/[r]^{f_{n,0}} 
\ar@/^2pc/[r]^{\big\Downarrow \phi_{n,1}}_{f_{n,1}} 
\ar@/_2pc/[r]^{\substack{\vdots\\ f_{n,k_n-1}\\ \\}}
\ar@/_5pc/[r]^{\substack{\phi_{n,k_n} \\ \big\Downarrow}}_{f_{n,k_n}}
& A_0 
}}
\caption{A morphism in $B(n)$
from $(f_{0,0}, f_{1,0}, \cdots, f_{n,0})$ to  
$(f_{0,k_0}, f_{1,k_1}, \cdots, f_{n,k_n})$ where
$\phi_{i,j} \in C^\bullet(A_i, _{f_{j-1}}{A_{i+1 \textrm{ (mod n+1)}}}_{f_j})$}
 \label{fig:phi}
\end{figure}
%
\subsubsection{Aside on notation} \label{sec:phi_notation}
When referring to an arbitrary morphism in $B(n)$,
we will assume it is a morphism from object $(f_{0,0}, f_{1,0}, \cdots, f_{n,0})$ 
to object $(f_{0,k_0}, f_{1,k_1}, \cdots, f_{n,k_n})$.
We will denote the morphism 
$$\phi_{0,1}\smdots\phi_{0,k_0} | 
\phi_{1,1}\smdots\phi_{1,k_1} | \dots |
\phi_{n,1}\smdots\phi_{n,k_n}$$ 
where $\phi_{i,j} \in C^\bullet(A_i, _{f_{j-1}}
{A_{i+1 \textrm{ (mod n+1)}}}_{f_j})$. 
See Figure \ref{fig:phi} for a picture of 
this morphism.
%
\subsubsection{Differential on $B(n)$}
Putting everything together, 
the differential on \newline $B(n)(
(f_{0,0}, \smdots, f_{n,0}), 
(f_{0,k_0}, \smdots, f_{n,k_n}))$ is
\begin{align*}
&\phantom{{}={}}
d_{B(n)} \big(
  \phi_{0,1}\smdots\phi_{0,k_0} | \smdots |
  \phi_{n,1}\smdots\phi_{n,k_n} \big)\\
&= 
\sum \limits_{0 \leq i \leq n}
  (-1)^{\sum \limits_{p<i;q}
  |\phi_{p,q}|+1}
  \phi_{0,1}\smdots | \smdots |
  d_{Bar(C^\bullet(A_i, A_{i+1}))} 
  (\phi_{i,1}\smdots\phi_{i,k_i}) | \smdots |
  \smdots\phi_{n,k_n} 
\end{align*}
%
\subsubsection{Counit}
Define
\begin{align*}
\epsilon_{B(n)}((f_{0,0},&\smdots,f_{n,0}), 
  (f_{0,k_0},\smdots,f_{n,k_n})):
B(n)((f_{0,0},\smdots,f_{n,0}), 
  (f_{0,k_0},\smdots,f_{n,k_n})) =\\
&=
Bar(C^\bullet(A_0, _{f_{0,0}}{A_1}_{f_{0,k_0}})) 
  \otimes \cdots \otimes 
  Bar(C^\bullet(A_0, _{f_{n,0}}{A_1}_{f_{n,k_n}}))
  \to \\
&\xrightarrow{project}
  Bar_0(C^\bullet(A_0, _{f_{0,0}}{A_1}_{f_{0,k_0}})) 
  \otimes \cdots \otimes 
  Bar_0(C^\bullet(A_0, _{f_{n,0}}{A_1}_{f_{n,k_n}}))
\cong k.
\end{align*}
%
\subsubsection{Cocomposition}
We have a coassociative map of complexes
%
\begin{align*}
\Delta_{A, _fB_g}: Bar(C^\bullet(A, _fB_g)) 
&\to \bigoplus\limits_{h: A \to B} Bar(C^\bullet(A, _fB_h)) \otimes Bar(C^\bullet(A, _hB_g)) \\
\phi_1 \cdots \phi_k 
&\mapsto \sum\limits_{1\leq i \leq k-1} 
\phi_1 \cdots \phi_i \otimes 
  \phi_{i+1} \cdots \phi_k\\
&\phantom{{}\mapsto \sum{}}
  + e_f\otimes \phi_1 \dots \phi_k + 
  \phi_1 \dots \phi_k \otimes e_g  
\end{align*}
%
where 
$e_f = 1$ in $Bar_0(C^\bullet(A, _fB_f)) \cong k$.
Extend $\Delta_{A, _fB_g}$ to a 
cocomposition on $B(n)$ by taking 
(up to signs)
$$
\Delta_{B(n)}((f_{0,0},\smdots,f_{n,0}), 
  (f_{0,k_0},\smdots,f_{n,k_n})) :=
\Delta_{A_0, _{f_{0,0}}{A_1}_{f_{0,k_0}}} 
\otimes \dots \otimes
\Delta_{A_0, _{f_{n,0}}{A_1}_{f_{n,k_n}}}.
$$
The sign on the term
$
(\phi_{0,1}\smdots \phi_{0,i_0} | \smdots |
\phi_{n,1}\smdots \phi_{n,i_n}) \otimes
(\phi_{0,i_0+1}\smdots \phi_{0,k_0} | \smdots |
\phi_{n,i_n+1}\smdots \phi_{n,k_n}) 
$
in the cocomposition is:
\begin{equation}
\label{eq:coprod_signs}
(-1)^{\sum \limits_{\substack{
  1 \leq p \leq n\\ 1 \leq r<p}}\;
(\sum \limits_{1\leq q \leq i_p} |\phi_{p,q}|+1)
(\sum \limits_{i_r+1 \leq s \leq k_r} |\phi_{r,s}|+1)}
\end{equation}
%
In other words, moving $\phi_{i,j}$ past 
$\phi_{p,q}$ introduces a factor of 
$(-1)^{(|\phi_{i,j}|+1)(|\phi_{p,q}|+1)}$.
It's clear from the definitions that
$(B(n), \Delta_{B(n)}, \epsilon_{B(n)})$ 
satisfy the diagrams needed to form a dg cocategory. 
We also see that $B(n)$ is conilpotent: 
$$
\bar{\Delta}_{B(n)}^{min(k_0,\smdots,k_n)}
(\phi_{0,1}\smdots\phi_{0,k_0}|\smdots|
\phi_{n,1}\smdots\phi_{n,k_n}) = 0.
$$
	\section{Dg comodules: $C(n)$}
%
\subsection{Background on dg comodules}
\begin{defn} A \textbf{dg comodule} $C$ 
over a dg cocategory $B$ consists
of the following data:
\begin{itemize}
\item for each object $f \in B$, a complex $C^\bullet(f)$, and
\item maps of complexes
\begin{equation*}
\Delta_C(f): C^\bullet(f) \to 
\prod\limits_{g \in Obj(B)} B^\bullet(f,g) \otimes C^\bullet(g).
\end{equation*}
\end{itemize}
such that the following diagrams 
for coassociativity and 
counitality commute:
$$
\xymatrixcolsep{5pc}
\xymatrixrowsep{5pc}
\xymatrix{
C^\bullet(f)
\ar[r]^{\Delta_C(f)}
\ar[d]_{\Delta_C(f)}
& \prod \limits_{g \in Obj(B)}
B^\bullet(f,g) \otimes
C^\bullet(g)
\ar[d]^{\prod\limits_g 
id_{B(f,g)} \otimes \Delta_C(g)}\\
\prod \limits_{g \in Obj(B)}
B^\bullet(f,g) \otimes
C^\bullet(g)
\ar[r]^{\prod\limits_g 
\Delta_B(f,g) \otimes id_{C(g)}}
& \prod \limits_{g, g^\prime \in Obj(B)}
B^\bullet(f,g) \otimes
B^\bullet(g,g^\prime) \otimes
C^\bullet(g^\prime)  
}
$$
$$
\xymatrixcolsep{5pc}
\xymatrixrowsep{5pc}
\xymatrix{
C^\bullet(f)
\ar[r]^{\Delta_C(f)}
\ar[rd]_{id}
& \prod \limits_{g \in Obj(B)}
B^\bullet(f,g) \otimes
C^\bullet(g)
\ar[d]^{\prod \limits_g
\epsilon_B(f,g)
\otimes id_{C(g)}} \\
%
& C^\bullet(f).  
}
$$
\end{defn}
To simplify notation, we will 
write $\Delta_C$ instead of 
$\Delta_C(f)$ when the meaning 
is clear.
%
\begin{eg}
A dg comodule over a dg cocategory $B$ with one object, $*$, is 
a dg comodule over the counital dg coalgebra $B^\bullet(*,*)$.
\end{eg}
%
\begin{defn} A \textbf{morphism of dg comodules}
$H: C \to D$
over a dg category $B$ consists of 
maps of complexes
$\big( H_f: C^\bullet(f) \to D^\bullet(f)
\big)_{f \in Obj(B)}$ such that
for each $f\in Obj(B)$, the
following diagram commutes:
\begin{equation*}
\begin{CD}
C^\bullet(f)  @>H_f>>  D^\bullet(f) \\
@VV\Delta_{C}V  @VV\Delta_{D}V \\
\prod\limits_{g \in Obj(B)} B^\bullet(f,g) \otimes C^\bullet(g) 
@>\prod\limits_g id_B \otimes H_g>> 
\prod\limits_{g \in Obj(B)} B^\bullet(f,g)) \otimes D^\bullet(g).
\end{CD}
\end{equation*}
\end{defn}
Again, when the meaning is clear, 
we may write $H$ instead of $H_f$.
%
\begin{defn} A \textbf{conilpotent} dg 
comodule over a dg cocategory $B$ 
is a dg comodule $(C, \Delta_C)$ 
over $B$ satisfying: for each 
$f \in Obj(B)$ and each element 
$\alpha \in C^\bullet(f)$, there exists 
$n_\alpha \in \mathbb{N}$ such that 
$\bar{\Delta}_f^{n_\alpha}(\alpha) = 0$
where
\begin{align*}
\bar{\Delta}_C(f): C^\bullet(f) 
&\to 
  \prod \limits_{g \in Obj(B)} 
  B^\bullet(f,g) \otimes
  C^\bullet(g)\\
\alpha
&\mapsto
\Delta_B(\alpha) 
  - \sum \limits_{e_f \in 
  \epsilon_B(f,f)^{-1}(1)}
  e_f \otimes f.
\end{align*}
\end{defn}
%

\subsection{Structure of $C(n)$}
\textbf{Reminder:} In this chapter, we fix algebras $A_0, A_1, \cdots, A_n$.
$C(n)$ and $B(n)$ are short for $C(A_0 \to A_1 \to \cdots \to A_n \to A_0)$
and $B(A_0 \to A_1 \to \cdots \to A_n \to A_0)$, respectively.
\newline
\newline
We now give dg comodules $C(n)$ over $B(n)$.
First, we will describe the graded comodule structure;
then, we will describe the differentials.
For an object $f = (f_0, f_1, \cdots, f_n) \in B(n)$, we have
%
\begin{equation}\label{eq:C(f)}
C(n)^\bullet(f) 
= \bigoplus\limits_{g \in Obj(B(n))} B(n)^\bullet(f,g) \otimes 
C_{-\bullet}(A_0, _{comp(g)}{A_0}_{id})
\end{equation}
%
where, for $g = (g_0, g_1, \cdots, g_n)$, 
we write $comp(g) = g_n \circ g_{n-1} \circ \cdots \circ g_0$,
and $C_\bullet(A,B)$ denotes Hochschild chains.
We will denote a typical element of $C(n)^\bullet(f)$ 
as $$\phi_{0,1}\smdots\phi_{0,k_0} | 
\phi_{1,1}\smdots\phi_{1,k_1} | \dots |
\phi_{n,1}\smdots\phi_{n,k_n} | \alpha$$
where $\phi_{0,1}\cdots \phi_{n,k_n}$ is a morphism
in $B(n)$ (see Section \ref{sec:phi_notation}) and
$\alpha \in C_{-\bullet}(A_0, _{f_{k_n}\cdots f_{k_0}}{A_0}_{id})$.
See Figure \ref{fig:phi|alpha} for a picture of 
a typical element of $C(n)^\bullet(f)$.
% long form of equation
% \begin{align*}
% C^\bullet(f) 
% = C_{-\bullet}(A_0, _{f_n\cdots f_1 f_0}{A_0}_{id}) \oplus \\
%   \bigoplus_{\substack{f_{i,j} \textrm{ maps of algebras}, \\
% 					   \textrm{and } f_{r,0} = f_r, \\
% 					   0\leq r\leq n}} 
% & C^\bullet(A_0, \, _{f_{0,0}}{A_1}_{f_{0,1}}) \otimes 
%   C^\bullet(A_0, \, _{f_{0,1}}{A_1}_{f_{0,2}}) \otimes \cdots \otimes 
%   C^\bullet(A_0, \, _{f_{0,k_0-1}}{A_1}_{f_{0,k_0}}) \otimes \\ \otimes
% & C^\bullet(A_1, \, _{f_{1,0}}{A_2}_{f_{1,1}}) \otimes 
%   C^\bullet(A_1, \, _{f_{1,1}}{A_2}_{f_{1,2}}) \otimes \cdots \otimes 
%   C^\bullet(A_1, \, _{f_{1,k_1-1}}{A_2}_{f_{1,k_1}}) \otimes \\ \otimes
% & \cdots \otimes \\ \otimes
% & C^\bullet(A_n, \, _{f_{n,0}}{A_0}_{f_{n,1}}) \otimes 
%   C^\bullet(A_n, \, _{f_{n,1}}{A_0}_{f_{n,2}}) \otimes \cdots \otimes 
%   C^\bullet(A_n, \, _{f_{n,k_n-1}}{A_0}_{f_{n,k_n}}) \otimes \\ \otimes
% & C_{-\bullet}(A_0, _{f_{n,k_n}\cdots f_{1, k_1} f_{0, k_0}}{A_0}_{id})
% \end{align*}
\begin{figure}
\centerline{\xymatrix{
A_0 \ar@/^5pc/[r]^{f_{0,0}} 
\ar@/^2pc/[r]^{\bm{\big\Downarrow \phi_{0,1}}}_{f_{0,1}} 
\ar@/_2pc/[r]^{\substack{\vdots\\ f_{0,k_0}}}
\ar@/_9pc/[rrrrr]_{id}^{\substack{\alpha \\ \\ \\ \\ }}
& A_1 \ar@/^5pc/[r]^{f_{1,0}} 
\ar@/^2pc/[r]^{\bm{\big\Downarrow \phi_{1,1}}}_{f_{1,1}} 
\ar@/_2pc/[r]^{\substack{\vdots\\ f_{1,k_1}}}
& A_2 \ar@{.>}@/^5pc/[r] 
\ar@{.>}@/^2pc/[r]_{\substack{\\ \\ \vdots}} 
\ar@{.>}@/_2pc/[r]
& \cdots \ar@{.>}@/^5pc/[r] 
\ar@{.>}@/^2pc/[r]_{\substack{\\ \\ \vdots}} 
\ar@{.>}@/_2pc/[r]
& A_n \ar@/^5pc/[r]^{f_{n,0}} 
\ar@/^2pc/[r]^{\bm{\big\Downarrow \phi_{n,1}}}_{f_{n,1}} 
\ar@/_2pc/[r]^{\substack{\vdots\\ f_{n,k_n}}}
& A_0 
}}
\caption{Pictoral representation of an element of
$C(n)^\bullet(f)$ where 
$f = (f_0 = f_{0,0}, f_1 = f_{1,0}, \cdots, f_n = f_{n,0})$, 
$\phi_{i,j} \in C^\bullet(A_i, _{f_{j-1}}{A_{i+1}}_{f_j})$, and 
$\alpha \in C_{-\bullet}(A_0, _{f_n\cdots f_1 f_0}{A_0}_{id})$}
 \label{fig:phi|alpha}
\end{figure}
%
\subsubsection{Comodule structure}\label{sec:comod_strre}
The comodule maps on $C(n)^\bullet(f)$ are given by the
cocomposition maps in $B(n)$: 
%
\begin{equation*}
\begin{CD}
C(n)^\bullet(f) @>\Delta_C>>
\bigoplus\limits_{h \in Obj(B(n))} B(n)^\bullet(f,h) \otimes C(n)^\bullet(h) \\
@|  @| \\
\bigoplus\limits_{g \in Obj(B(n))} B(n)^\bullet(f,g) \otimes C_{-\bullet}(A_0, _g{A_0}_{id})
@>\Delta_{B(n)}\otimes 1_{C_{-\bullet}}>>
\bigoplus\limits_{g,h\in Obj(B(n))} \substack{B(n)^\bullet(f,h)\otimes B(n)^\bullet(h,g) \\ \otimes 
C_{-\bullet}(A_0, _g{A_0}_{id})}
\end{CD} 
\end{equation*}
%
Because $\Delta_{C(n)}$ is induced by 
$\Delta_{B(n)}$, we have that $\Delta_{C(n)}$ 
satisfies coassocitivity and counitality 
and is conilpotent. 

$C(n)$ is quasi-cofree 
(i.e., cofree as a comodule) in the sense that 
a morphism to $C(n)$ is determined by projections
to its Hochschild-chains component.
More precisely, there is a one-to-one correspondence
\begin{equation}\label{eq:quasicofree}
\bigg\{ \let\scriptstyle\textstyle
\substack{\textrm{maps of comodules}\\
  D \to C(n)\textrm{ over } B(n)}
\bigg\}
\overset{1:1}\longleftrightarrow
\bigg\{ \bigg( \let\scriptstyle\textstyle
\substack{\textrm{maps of graded vector spaces}\\
  D^\bullet(f) \to C_{-\bullet}(A_0, _f{A_0}_{id})}
\bigg)_{f \in Obj(B(n))} \bigg\}.
\end{equation}
%
\begin{defn}\label{def:cogenerators}
We will call elements of 
$T(A_0 \to \smdots A_n \to A_0)(f) :=
C_{-\bullet}(A_0, _f{A_0}_{id})$ 
the \textbf{cogenerators} of 
$C(A_0 \to \smdots A_n \to A_0)(f)$. 
More generally, we will refer to 
the set $T(A_0 \to \smdots A_n \to A_0) 
= \{T(A_0 \to \smdots A_n \to A_0)(f) | 
f \in Obj(B(A_0 \to \smdots A_n \to A_0))\}$ 
as the \textbf{cogenerators} of 
$C(A_0 \to \smdots A_n \to A_0)$. When we 
have fixed a sequence of algebras, 
$A_0, \smdots A_n$, we will 
use $T(n)$ to denote 
$T(A_0 \to \smdots A_n \to A_0)$.
\end{defn}
%
\subsubsection{Differential}
The differential $d_{C(n)}(f)$ on $C(n)^\bullet(f)$ is: 
%
\begin{equation} \label{eq:C(n)_differential}
d_{C(n)}(f) = 
\sum\limits_{g \in Obj(B(n))} 
(d_{B(n)} \otimes id_{C_{-\bullet}}
+ id_{ B(n)}\otimes b_g)
+ \mathcal{I}
\end{equation}
%
where $d_{B(n)}$ is the differential on $B(n)$, 
$b_g$ is the Hochschild-chain differential on 
$C_{-\bullet}(A_0, _g{A_0}_{id})$, 
and $\mathcal{I}$ is a term that captures the action of cochains on chains 
described by the equations below:
%
\begin{equation}\label{eq:action_term}
\begin{split}
\mathcal{I} 
&= 
(id_{B(n)} \otimes \eta_{C(n)})
  \circ \Delta_{C(n)}\\
\eta_{C(n)}(
\phi_{0,1}\smdots\phi_{0,k_0} | \dots |
\phi_{n,1}\smdots\phi_{n,k_n} | \alpha)
&= 
\iota_{C(0)}( \pi_{B(0)}( 
(\phi_{0,1}\smdots \phi_{0,k_0}) \bullet \dots \bullet
(\phi_{n,1}\smdots \phi_{n,k_n})), \alpha)\\
\bullet
&= 
\textrm{brace operation on Hochschild cochains see ...}\\
\pi_{B(0)}: 
B(0)^\bullet
(f_{n,1}\smdots f_{1,1} f_{0,1}, 
f_{n,k_n}\smdots f_{1,k_1} f_{0,k_0})
&\xrightarrow[component]{project\;onto}
C^\bullet(A_0, _{f_{n,1}\smdots f_{0,1}}{A_0}_{f_{n,k_n}\smdots f_{0,k_0}})\\
\iota_{C(0)}: C^p(A, _fA_g) \bigotimes C_{-q}(A, _gA_h) 
&\longrightarrow 
C_{-(q-p)}(A, _fA_h) \\
\phi \bigotimes a_0\otimes \dots \otimes a_q 
&\mapsto 
\phi(a_{q-p+1},\dots,a_q)\cdot a_0 
\otimes a_1 \otimes \dots \otimes a_{q-p}.
\end{split}
\end{equation}
%
Given Equation \ref{eq:C(n)_differential}, 
it's easy to check that we can promote 
Equation \ref{eq:quasicofree} to a dg statement:
%
\begin{equation*}
\bigg\{ \let\scriptstyle\textstyle
\substack{\textrm{maps of dg comodules}\\
  D \to C(n)\textrm{ over }B(n)} 
\bigg\}
\overset{1:1}{\longleftrightarrow}
\bigg\{ \bigg( \let\scriptstyle\textstyle
\substack{\textrm{maps of complexes}\\
  D^\bullet(f) \to C_{-\bullet}(A_0, _f{A_0}_{id})}
\bigg)_{f \in Obj(B(n))} \bigg\}.
\end{equation*}
%
\chapter{Pullbacks, Pushforwards and Adjunctions}
	\section{A sheafy-cyclic object in dg cocategories}\label{sec:cyclic_B(n)}
We would like to say that we have a functor 
from Connes cyclic category $\Lambda$ 
(see Appendix \ref{chap:lambda} for 
generators and relations) to the category of dg 
cocategories where $[n] \mapsto B(n)$, 
but defining $B(n)$ involved choosing a 
sequence of algebras $A_0, \dots, A_n$. 

Instead, we have the following: 
Let $X: \Lambda \to Set$ be the functor 
that sends $[n]$ to the set of diagrams 
$A_0 \to A_1 \to \dots \to A_n \to A_0$ 
where the $A_i$'s are algebras. On 
generating morphisms in $\Lambda$, 
$X$ acts as follows: Let $\mathcal{A} = 
(A_0 \to \smdots \to A_n \to A_0)
\in X([n])$. 
\begin{align*}
X(\tau_n): \mathcal{A}
  &\mapsto (A_n \to A_0 \to \smdots \to A_{n-1} \to A_n)\\
X(\delta_{j,n}): \mathcal{A}
  &\mapsto 
    (A_0 \to \smdots \to A_j\longrightarrow A_{j+2 
    \textrm{ (mod n+1)}} \to \smdots \to A_n \to A_0) \\
X(\sigma_{i,n}): \mathcal{A}
  &\mapsto 
  \begin{cases}
    (A_0 \to \smdots \to 
  		   A_i \to A_i \to \smdots A_n \to A_0)
    & 1 \leq i \leq n\\
    (A_0 \to \smdots \to A_n \to A_0 \to A_0)
    & i=n+1	
  \end{cases}	   
\end{align*}
It's straightforward to 
check that $X$ respects composition of morphisms. 
Now, let $\chi$ be the category with objects given by
diagrams $A_0 \to \smdots \to A_n \to A_0$ 
where the $A_i$'s are algebras and $n \in \mathbb{N}$. 
Morphisms in $\chi$ are the pointwise images of 
$X$. In other words, the set of morphisms in
$\chi$ is $\{ X(\lambda)|_x: \lambda \in 
\Lambda([n],[m]), x \in X([n]) \}$. We will give 
a functor, $\mathcal{G}$, from $\chi$ to the 
category of dg 
cocategories; (this is our sheafy-cyclic object, 
i.e., a \textbf{sheafy-cyclic} object in a 
category $\mathcal{C}$ is a functor $\chi \to 
\mathcal{C}$).
%
\subsection{Aside on notation:} 
Fix $\lambda:[n] \to [m]$ in $\Lambda$ 
and $x \in X([n])$. To define $\mathcal{G}$, 
we will need to define a functor $\mathcal{G}
\big(X(\lambda)|_x \big): B(x) \to B(X(\lambda)(x))$. 
To simplify notation, we will denote 
$\hat{\lambda} := \mathcal{G}
\big(X(\lambda)|_x \big)$ and write 
$\hat{\lambda}: B(x) \to B(\lambda x)$. 
Technically, we are losing information 
about the $x$ when we write 
$\hat{\lambda}$ instead of $\mathcal{G}
\big(X(\lambda)|_x \big)$, but 
we will be clear 
about the source and target when needed.
%
\subsection{Definition of $\mathcal{G}$}
\label{sec:def_G}
Now, we will define $\mathcal{G}$. On objects, 
\begin{align*}
\mathcal{G}: (A_0 \to \smdots \to A_n \to A_0)
&\mapsto 
B(A_0 \to \smdots \to A_n \to A_0)\\
&\phantom{{}\mapsto{}}
\textrm{(see Section \ref{sec:B(n)} for 
definition of $B(\cdot)$)}
\end{align*}
On generating morphisms in $\chi$, set $\mathcal{A} = 
(A_0 \to \smdots \to A_n \to A_0) \in Obj(\chi)$, and 
define 
\begin{align*}
\hat{\tau}_n
  &\mapsto 
  \begin{cases}
  B(\mathcal{A}) \longrightarrow 
  B(A_n \to A_0 \to \smdots \to A_{n-1} \to A_n) \\
  \textrm{objects: } (f_0,f_1, \smdots, f_n) \mapsto 
  (f_n, f_0, \smdots, f_{n-1}) \\
  \textrm{morphisms: } \phi_{0,1}\smdots \phi_{0,k_0} | \smdots |
	\phi_{n,1}\smdots\phi_{n,k_n} \mapsto 
	\phi_{n,1}\smdots\phi_{n,k_n} | \smdots |
	\phi_{n-1,1}\smdots\phi_{n-1,k_{n-1}}
  \end{cases}\\	
%
\hat{\delta}_{j,n}
  &\mapsto 
  \begin{cases}
  B(\mathcal{A}) \longrightarrow 
  B(A_0 \to \smdots \to A_j\to A_{j+2 \textrm{ (mod n+1)}} 
      \to \smdots \to A_0) \\
  \textrm{objects: } (f_0,f_1, \smdots, f_n) \mapsto 
  (f_0, \smdots,f_{j+1}\circ f_j, \smdots, f_n) \\
  \textrm{morphisms: } \phi_{0,1}\smdots \phi_{0,k_0} | \smdots |
	\phi_{n,1}\smdots\phi_{n,k_n} \mapsto \\
  \phantom{{}morphisms: {}} 
	\phi_{0,1}\smdots \phi_{0,k_0} | \smdots |
	(\phi_{j,1}\smdots \phi_{j,k_j}) \bullet
	(\phi_{j+1,1}\smdots \phi_{j+1,k_{j+1}}) | \smdots |
	\phi_{n,1}\smdots\phi_{n,k_n} 
  \end{cases}\\
%
\underset{1 \leq i \leq n}{\hat{\sigma}_{i,n}}
  &\mapsto 
  \begin{cases}
  B(\mathcal{A}) \longrightarrow 
  B(A_0 \to \smdots \to A_i\to A_i
      \to \smdots \to A_0) \\
  \textrm{objects: } (f_0,f_1, \smdots, f_n) \mapsto 
  (f_0, \smdots,f_{i-1}, id_{A_i}, f_i, \smdots, f_n) \\
  \textrm{morphisms: } \phi_{0,1}\smdots \phi_{0,k_0} | \smdots |
	\phi_{n,1}\smdots\phi_{n,k_n} \mapsto \\
  \phantom{{}morphisms: {}} 
	\phi_{0,1}\smdots \phi_{0,k_0} | \smdots |
	\phi_{i-1,1}\smdots \phi_{i-1,k_{i-1}} | 1 |
	\phi_{i,1}\smdots \phi_{i,k_i} | \smdots |
	\phi_{n,1}\smdots\phi_{n,k_n} \\
  \phantom{{}morphisms: {}}	
  	1 \in k =\textrm{degree 0 component of }
  	Bar(C^\bullet(A_i, A_i))
  \end{cases}\\    
%
\hat{\sigma}_{n+1,n}
  &\mapsto 
  \begin{cases}
  B(\mathcal{A}) \longrightarrow 
  B(A_0 \to \smdots \to A_n\to A_0 \to A_0) \\
  \textrm{objects: } (f_0,f_1, \smdots, f_n) \mapsto 
  (f_0, \smdots, f_n, id_{A_0}) \\
  \textrm{morphisms: } \phi_{0,1}\smdots \phi_{0,k_0} | \smdots |
  \phi_{n,1}\smdots\phi_{n,k_n} \mapsto \\
  \phantom{{}morphisms: {}} 
  \phi_{0,1}\smdots \phi_{0,k_0} | \smdots |
  \phi_{n,1}\smdots\phi_{n,k_n} | 1 \\
  \phantom{{}morphisms: {}} 
    1 \in k =\textrm{degree 0 component of }
    Bar(C^\bullet(A_0, A_0))  
  \end{cases}    
\end{align*}
It's straightforward to check that $\mathcal{G}$ is a 
functor (i.e., that composition of morphisms and the relations 
are preserved). The only facts we need are that $\bullet$ is 
an associative map of complexes and that 
$1 \bullet (\phi_0 \smdots \phi_k) = 
(\phi_0 \smdots \phi_k) \bullet 1 = (\phi_0 \smdots \phi_k)$ 
where the 1's are in the degree 0 components of
$Bar(C^\bullet(A_i, A_i))$ for the appropriate $A_i$ 
(see Appendix Section \ref{sec:def_braces}).
	\section{Motivation of this chapter}
We would like to extend the sheafy-cyclic 
structure in \ref{sec:cyclic_B(n)} from 
$B(\mathcal{A})$ to the pair $(B(\mathcal{A}), 
C(\mathcal{A}))$ where $\mathcal{A} = 
(A_0 \to \smdots \to A_n \to A_0) \in Obj(\chi)$. 
However, this presents some complications 
as $C(\mathcal{A})$ and $C(\mathcal{A^\prime})$ 
are comodules over different 
cocategories. Instead, we will use 
the functors $\hat{\lambda}: B(\mathcal{A}) 
\to B(\mathcal{A}^\prime)$ from Section 
\ref{sec:def_G} to
define pullbacks $\hat{\lambda}^*C(\mathcal{A}^\prime)$ 
and maps $\hat{\lambda}_!: C(\mathcal{A}) 
\to \hat{\lambda}^*C(\mathcal{A}^\prime)$ 
of dg comodules over $B(\mathcal{A})$. 

First, we will define functors $\hat{\lambda}^*$ 
from the category of 
conilpotent dg comodules over $B(\mathcal{A})$ 
to the category of conilpotent dg comodules 
over $B(\mathcal{A}^\prime)$. 
Second, we will give $\hat{\lambda}_\#$, the left adjoint 
to $\hat{\lambda}^*$. Finally, we will 
give explicit maps of dg comodules 
$\hat{\lambda}_!: C(\mathcal{A}) \to 
\hat{\lambda}^*C(\mathcal{A}^\prime)$, 
and apply the adjunction 
to these maps. The following chapters will formalize 
the relations between the $\hat{\lambda}_!$'s.
	\section{Pullbacks of dg comodules--theory}
Let $\lambda: B_1 \to B_0$ be a 
functor between conilpotent 
dg cocategories. 
In this section, we will define 
a functor $\lambda^*$ from the 
category of conilpotent dg 
comodules over 
$B_0$ to the category of 
conilpotent dg 
comodules over $B_1$. We 
call $\lambda^*$
``co-extension of scalars''.
%
\subsection{Category-theoretic definition of $\lambda^*$}
\label{sec:pb_defn}
Let $\lambda$ be as above, and let $C$ 
be a conilpotent dg 
comodule over $B_0$. We define 
$\lambda^* C$ as follows:
\begin{equation} \label{eq:pb_defn}
\lambda^*C := 
ker \big( B_1\otimes_\lambda C 
\mathrel{\mathop{\rightrightarrows}^
{\mathrm{id_{B_1}\otimes \Delta_{C}}}_
{\mathrm{(id_{B_1}\otimes \lambda \otimes id_C)
  \circ (\Delta_{B_1}\otimes id_C})}}
B_1 \otimes_\lambda B_0\otimes C \big)
\end{equation}
where $B_1\otimes_\lambda C$ and 
$B_1\otimes_\lambda B_0 \otimes C$ are dg comodules
over $B_1$ defined below. 
For $f \in Obj(B_1)$,
\begin{align*}
[B_1 \otimes_\lambda C](f) 
:&= 
\big(
\bigoplus \limits_{h \in Obj(B_1)} B_1^\bullet(f, h) 
\otimes C^\bullet(\lambda h), 
\Delta(f) = 
\bigoplus \limits_{h} \Delta_{B_1(f, h)} 
\otimes id_{C(\lambda h)}
\big) \\
[B_1 \otimes_\lambda B_0 \otimes C](f) 
:&= 
\big(
\bigoplus \limits_{\substack{h_1 \in Obj(B_1),\\ h_2 \in Obj(B_0)}}
B_1^\bullet(f, h_1) \otimes 
B_0^\bullet(\lambda h_1, h_2) \otimes
C^\bullet(h_2), \\
& \phantom{{}:=\big( {}}
\Delta(f) = 
\bigoplus \limits_{h_1,\,h_2} \Delta_{B_1(f, h_1)} 
\otimes id_{B_0(\lambda h_1, h_2)} 
\otimes id_{C(h_2)}
\big).
\end{align*}
The names of the maps in Equation \ref{eq:pb_defn} 
are also meant to be suggestive. In full detail, 
for $f \in Obj(B_1)$,
$$
{[id_{B_1}\otimes \Delta_{C}]}(f) 
:=
\bigoplus \limits_{h} id_{B_1(f, h)} \otimes \Delta_C(\lambda h)
$$
and 
\begin{align*}
[B_1 \otimes_\lambda C](f)
&\xrightarrow{
{[\Delta_{B_1}\otimes id_C]}(f) :=
\bigoplus \limits_{h} \Delta_{B_1}(f, h) 
  \otimes id_{C(\lambda h)}}
%
\bigoplus \limits_{h_1,h_2 \in Obj(B_1)}
B_1(f, h_1) \otimes B_1(h_1, h_2) 
\otimes C(\lambda h_2) \\
%
&\xrightarrow{
{[id_{B_1}\otimes \lambda \otimes id_C]}(f) := 
\bigoplus \limits_{h_1,\,h_2} id_{B_1(f, h_1)} \otimes \lambda(h_1,h_2)
\otimes id_{C(\lambda h)}}
[B_1 \otimes_\lambda B_0 \otimes C](f). 
\end{align*}
That the kernel is well-defined follows formally from the 
abelianness of the category of chain complexes, but it is also 
easy to check that the induced differentials from 
$[B_1\otimes_\lambda C](f)$ 
on the kernel are well-defined. 
Since $\Delta_{\lambda^*C}$ is 
induced by $\Delta_{B_1}$, we have
that $\Delta_{\lambda^*C}$ also satisfies 
coassociativity, counitality and 
conilpotency.

Next, we will define $\lambda^*$ on morphisms.
Let $F:C\to D$ be a map of conilpotent 
dg comodules over $B_0$. 
By the universal property of $\lambda^*D$, we can define a 
morphism $\lambda^*F:\lambda^*C \to \lambda^*D$ by giving a 
morphism from $(\lambda^*F)^\prime:\lambda^*C \to 
B_1 \otimes_\lambda D$ such that the two maps 
\begin{equation}\label{eqn:coker}
(id_{B_1}\otimes \Delta_D) \circ (\lambda^*F)^\prime, \>
(id_{B_1}\otimes \lambda \otimes id_D)\circ 
(\Delta_{B_1}\otimes id_D) \circ (\lambda^*F)^\prime: 
\lambda^*C \to B_1 \otimes_\lambda D 
\rightrightarrows
B_1 \otimes_\lambda B_0 \otimes D
\end{equation}
coincide. We define $(\lambda^*F)^\prime$ as follows:
$$
(\lambda^*F)^\prime:
\lambda^*C
\xrightarrow[inclusion]{canonical}
B_1 \otimes_\lambda C 
\xrightarrow{id_{B_1}\otimes F}
B_1 \otimes_\lambda D
$$
It's easy to check that the two maps in Equation 
\ref{eqn:coker} coincide: 
Let $b\otimes c$ be an arbitrary element of $\lambda^*C(f)
\hookrightarrow [B_1\otimes_\lambda C](f)$. 
Then,
\begin{align*}
[(id_{B_1}\otimes \Delta_D) \circ (\lambda^*F)^\prime] (b\otimes c) 
&= \sum \limits_{(Fc)} b \otimes (Fc)_{(1)} \otimes (Fc)_{(2)} \\
&= \sum \limits_{(c)} b \otimes Fc_{(1)} \otimes Fc_{(2)} 
\quad \textrm{(F is a map of comodules)}\\
&= [(id_{B_1}\otimes F \otimes F) \circ (id_{B_1}\otimes \Delta_C)] 
(b\otimes c) \\
&= [(id_{B_1}\otimes F\otimes F) \circ 
(id_{B_1}\otimes \lambda \otimes id_C)\circ 
(\Delta_{B_1}\otimes id_C)] (b\otimes c)\\
&\quad \textrm{($b\otimes c$ is in the kernel)}\\
&= \sum \limits_{(b)} b_{(1)}\otimes \lambda b_{(2)} \otimes Fc \\
&= [(id_{B_1}\otimes \lambda \otimes id_D)\circ 
(\Delta_{B_1}\otimes id_D) \circ (\lambda^*F)^\prime] 
(b \otimes c).
\end{align*}
So, $\lambda^*F$ is well-defined. In summary, we have commuting 
diagrams:
\begin{equation}\label{cd:lambda^*f}
\begin{CD}
\lambda^*C
@>\substack{\textrm{canonical}\\ \textrm{inclusion}}>>
B_1\otimes_\lambda C \\
@V\lambda^*FVV 
@VVid_{B_1}\otimes F = \textrm{ map inducing } \lambda^*FV \\
\lambda^*D
@>\substack{\textrm{canonical}\\ \textrm{inclusion}}>>
B_1\otimes_\lambda D
\end{CD}
\end{equation}
Finally, it is straightforward to see that $\lambda^*$ 
is a functor, i.e., that $\lambda^*$ preserves composition
of morphisms: Let $C \overset{F}{\to} D \overset{G}{\to} E$ be 
composable morphisms of dg comodules over $B_0$. 
The maps inducing 
$\lambda^*F$, $\lambda^*G$ and $\lambda^*(G\circ F)$ are 
$id_{B_1}\otimes F$, $id_{B_1}\otimes G$ and 
$id_{B_1}\otimes GF$, respectively. The inducing maps
respect composition--$(id_{B_1}\otimes G) \circ 
(id_{B_1}\otimes F) = id_{B_1}\otimes GF$--and by 
the commuting diagrams \ref{cd:lambda^*f}, the 
functor $\lambda^*$ does as well.
%
\begin{prop} \label{prop:pbs_compose}
Let $F:B_2 \to B_1$ and $G: B_1 \to B_0$ 
be functors between dg cocategories $B_2$, $B_1$ and 
$B_0$. Let $M$ be a dg comodule over $B_0$. Then,
$$
(GF)^*M \cong F^*G^*M.
$$
\end{prop}
%
\begin{proof}
We will prove the proposition by showing that 
$F^*G^*M$ satisfies the universal property of 
$(GF)^*M$. First, let $N$ be a dg comodule over $B_2$ 
and $H: N \to B_2 \otimes_{GF} M$ be a map of 
dg comodules such that the two maps 
\begin{equation}\label{eq:H_reln}
(id_{B_2} \otimes GF \otimes id_M) \circ 
(\Delta_{B_2}\otimes id_M) \circ H, \>
(id_{B_2} \otimes \Delta_M) \circ H: 
N \to B_2 \otimes_{GF} M
\rightrightarrows 
B_2 \otimes_{GF} \otimes B_0 \otimes M
\end{equation}
coincide. We will show that $H$ determines a 
map of dg comodules $\tilde{H}: N \to F^*G^*M$. 
Let $x \in Obj(B_2)$. Define 
\begin{align*}
H^\prime_x: N(x)
&\xrightarrow{H_x}
\bigoplus \limits_{y \in Obj(B_2)}
  B_2(x,y) \otimes M(GFy)\\
&\xrightarrow{F \otimes id_M}
\bigoplus \limits_{y \in Obj(B_2)}
  B_1(Fx,Fy) \otimes M(GFy)\\
&\subset
[B_1 \otimes_G M](Fx).  
\end{align*}
The image of $H^\prime_x$ lands in 
$G^*M(Fx)$, a subcomplex of 
$[B_1 \otimes_G M](Fx)$; checking 
this is straightforward using the 
universal property of $G^*M$, the 
fact that $F$ commutes with the 
coproducts, and Equation \ref{eq:H_reln}. 
So, for 
each $x \in Obj(B_2)$, we have a map  
of complexes $H^\prime_x: N(x) \to 
G^*M(Fx)$. Now define $\tilde{H}$ as 
follows:
\begin{align*}
\tilde{H}_x: N(x) 
&\xrightarrow{\Delta_N}
\bigoplus \limits_{y \in Obj(B_2)}
  B_2(x,y) \otimes N(y)\\
&\xrightarrow{\prod id_{B_2} \otimes H^\prime_y}  
\bigoplus \limits_{y \in Obj(B_2)}
  B_2(x,y) \otimes G^*M(Fy)\\
&\subset
[B_2 \otimes_F G^*M](x).
\end{align*}
Showing that $\tilde{H}$ lands in 
$G^*F^*M$, a subcomodule of 
$[B_2 \otimes_F G^*M]$, is also 
straightforward; we only need that 
$F$ and $H$ commute with the appropriate 
coproducts, and that the cocomposition 
on $B_2$ is coassociative. So, for each 
$x \in Obj(B_2)$, we have a map 
$\tilde{H}_x: N(x) \to G^*F^*M(x)$. 
It's clear that $\tilde{H}$ is a map 
of dg comodules since all of the maps 
used to construct $\tilde{H}$ are maps 
of dg comodules.

Now, let $\tilde{H}: N \to F^*G^*M$ 
be a map of dg comodules over $B_2$. 
We will show 
that $\tilde{H}$ determines a map of dg 
comodules $H: N \to B_2 \otimes_GF M$ 
satisfying Equation \ref{eq:H_reln}. 
For $x \in Obj(B_2)$, let $H$ be 
defined as follows:
\begin{align*}
H_x: N(x)
&\xrightarrow{\tilde{H}_x}
F^*G^*M(x)\\
&\xrightarrow[inclusion]{canonical}
\bigoplus \limits_{\substack{
	y \in Obj(B_2)\\
	z_1 \in Obj(B_1)}}
  B_2(x,y) \otimes B_1(Fy, z_1) \otimes 
  M(Gz_1)\\
&\xrightarrow{id_{B_2} \otimes \epsilon_{B_1} 
   \otimes id_M}  
\bigoplus \limits_{y \in Obj(B_2)}
  B_2(x,y) \otimes M(GFy).
\end{align*}
The universal property of $G^*M$ implies 
that $(id_{B_2} \otimes \Delta_M) \circ H$ 
is equal to:
\begin{align*}
N(x)
&\xrightarrow{\tilde{H}_x} 
\bigoplus \limits_{\substack{
	y \in Obj(B_2)\\
	z_1 \in Obj(B_1)}}
  B_2(x,y) \otimes B_1(Fy, z_1) \otimes 
  M(Gz_1)\\
&\xrightarrow[(id_{B_2} \otimes \Delta_{B_1} 
  \otimes id_M)]{(id_{B_2} \otimes id_{B_1}
  \otimes G \otimes id_M)\circ}
\bigoplus \limits_{\substack{
	y \in Obj(B_2)\\
	y_1, z_1 \in Obj(B_1)}}
  B_2(x,y) \otimes B_1(Fy, y_1) \otimes 
  B_0(Gy_1, Gz_1) \otimes M(Gz_1) \\
&\xrightarrow{id_{B_2} \otimes \epsilon_{B_1}
  \otimes id_{B_0} \otimes id_M}
\bigoplus \limits_{\substack{
	y \in Obj(B_2)\\
	z_1 \in Obj(B_1)}}
  B_2(x,y) \otimes B_0(GFy, Gz_1) \otimes M(Gz_1).   
\end{align*}
On the other hand, the universal property 
of $F^*$ implies that $(id_{B_2} \otimes GF 
\otimes id_M) \circ (\Delta_{B_2}\otimes id_M) 
\circ H$ is equal to:
\begin{align*}
N(x)
&\xrightarrow{\tilde{H}_x} 
\bigoplus \limits_{\substack{
	y \in Obj(B_2)\\
	z_1 \in Obj(B_1)}}
  B_2(x,y) \otimes B_1(Fy, z_1) \otimes 
  M(Gz_1)\\
&\xrightarrow[(id_{B_2} \otimes \Delta_{B_1} 
  \otimes id_M)]{(id_{B_2} \otimes G
  \otimes id_{B_1} \otimes id_M)\circ}
\bigoplus \limits_{\substack{
	y \in Obj(B_2)\\
	y_1, z_1 \in Obj(B_1)}}
  B_2(x,y) \otimes B_0(GFy,Gy_1) \otimes
  B_1(y_1, z_1) \otimes M(Gz_1) \\
&\xrightarrow{id_{B_2} \otimes id_{B_0} 
  \otimes \epsilon_{B_1} \otimes id_M}
\bigoplus \limits_{\substack{
	y \in Obj(B_2)\\
	z_1 \in Obj(B_1)}}
  B_2(x,y) \otimes B_0(GFy, Gz_1) \otimes M(Gz_1).   
\end{align*}
So, the difference between the two maps 
in Equation \ref{eq:H_reln} comes down 
to the difference between 
$(\epsilon_{B_1} \otimes G) \circ \Delta_{B_1}$ 
and $(G \otimes \epsilon_{B_1}) \circ \Delta_{B_1}$. 
However, by the counitality of $B_1$, both of 
these maps are equal to $G$. So, $H$ satisfies 
Equation \ref{eq:H_reln}.
\end{proof}
%
%
\begin{prop}\label{prop:compute_pb}
Let $\lambda:B_1 \to B_0$ be a functor 
between conilpotent dg cocategories and 
$C$ a conilpotent quasi-cofree dg comodule 
over $B_0$. 
Then, as comodules,
\begin{equation}\label{eq:compute_pb}
\lambda^*C \cong 
B_1 \otimes_{\lambda} T 
\end{equation}
where righthand side is the following 
cofree comodule over $B_1$:
\begin{align*}
[B_1 \otimes_{\lambda} T](f)
&:= 
\bigoplus \limits_{h \in Obj(B_0)}
B_1(f, h) \otimes T(\lambda h)\\
T(\lambda h)
&=
\textrm{cogenerators of }C(\lambda h)
\textrm{ (see Section \ref{def:cogenerators}).}
\end{align*}
\end{prop}
%
\begin{proof}[Proof of Proposition \ref{prop:compute_pb}]
To prove the proposition, 
we will give maps
$$
F: \lambda^*C \rightleftarrows B_1 \otimes_{\lambda} T:G
$$
and show that $F\circ G = id_{B_1 \otimes_{\lambda} T}$ 
and $G \circ F = id_{\lambda^*C}$.
We define $F$ as follows: 
$$
F:\lambda^*C 
\xrightarrow[inclusion]{canonical}
B_1 \otimes_{\lambda} C
\xrightarrow[cogenerators]{project\;onto}
B_1 \otimes_{\lambda} T.
$$
To define $G$, we will give a map 
$G^\prime: B_1 \otimes_{\lambda} T \to 
B_1 \otimes_{\lambda} C$, and show that the image of 
$G^\prime$ lands in $\lambda^*C$. We define 
$G^\prime$ as follows:
$$
G^\prime(b \otimes t) = 
\sum \limits_{(b)} b_{(1)} \otimes \lambda b_{(2)} \cdot t
$$
where $b \otimes t \in B_1 \otimes_{\lambda} T$ and $\lambda 
b_{(2)} \cdot t$ are elements of the 
appropriate components of $C$ 
written in terms of cogenerators.

To prove that the image of $G^\prime$ lands in 
$\lambda^*C$, we need to show that the two maps 
$$
(id_{B_1}\otimes \Delta_{C}) \circ G^\prime, \>
(id_{B_1}\otimes \lambda \otimes id_C)\circ 
(\Delta_{B_1}\otimes id_C) \circ G^\prime: 
B_1 \otimes_{\lambda} T
\to B_1 \otimes_{\lambda} C
\rightrightarrows
B_1 \otimes_{\lambda} B_0 \otimes C
$$
coincide. We have
\begin{align*}
[(1\otimes \Delta_{C}) \circ G^\prime](b \otimes t) 
&= 
\sum \limits_{(b),\, (\lambda b)} b_{(1)} \otimes 
(\lambda b_{(2)})_{(1)} \otimes 
(\lambda b_{(2)})_{(2)} \cdot t \\
&= 
\sum \limits_{(b)} b_{(1)} \otimes 
\lambda b_{(2)} \otimes 
\lambda b_{(3)} \cdot t \\
&= 
[(id_{B_1}\otimes \lambda \otimes id_C)\circ 
(\Delta_{B_1}\otimes id_C) \circ G^\prime]
(b\otimes t)
\end{align*}
where the second equality holds since $\lambda$ 
is a map of cocategories
and $\Delta_{B_1}$ is coassociative.

It's clear from the definitions that $F$ and $G$ are 
maps of comodules and that 
$F\circ G = id_{B_1\otimes_{\lambda} T}$. All that remains 
is to show that $G \circ F = id_{\lambda^*C}$. 
Let $\kappa = \Sigma_i b_i \otimes \beta_i \cdot t_i$ be an 
arbitrary element of $\lambda^*C \hookrightarrow 
B_1 \otimes_{\lambda} C$ where $\beta_i \cdot t_i$ are elements 
of $C$ written in terms of cogenerators. 
Then, 
\begin{equation*}
GF(\kappa) = 
GF(\Sigma_i b_i \otimes \beta_i \cdot t_i) = 
\sum \limits_{\substack{i, \\ \beta_i = 1, \\ (b_i)}} 
{b_i}_{(1)} \otimes \lambda {b_i}_{(2)} 
\cdot t_i.
\end{equation*}
We can divide the terms in $\kappa$ into two groups: 
(a) terms in which $\beta_i = 1 \in k$ and (b) terms in which $\beta_i
\neq 1 \in k$. Likewise, we can divide the terms in 
$GF(\kappa)$ into (a) terms in which $\lambda {b_i}_{(2)} = 1$ 
and (b) terms in which $\lambda {b_i}_{(2)} \neq 1$. 
From the definitions of $F$ and $G$, it's clear that 
the Group A terms in $\kappa$ are exactly the Group A 
terms in $GF(\kappa)$. 

To show that the Group B terms 
are the same, let $b_i \otimes \beta_i \cdot t_i$ be 
an arbitrary Group B term in $\kappa$.
Then, there is a term $b_i \otimes \beta_i \otimes t_i$ 
in $(id_{B_1} \otimes \Delta_{C}) \kappa$. Since 
$(id_{B_1} \otimes \Delta_{C}) \kappa = 
(id_{B_1} \otimes \lambda \otimes id_C) \circ 
(\Delta_{B_1} \otimes id_C) \kappa$, 
there must be a Group A term, $b_{j_i} \otimes t_{j_i}$, 
in $\kappa$ such that $b_i \otimes \beta_i \otimes t_i$ 
is one of the terms in the sum 
$[(id_{B_1} \otimes \lambda \otimes id_C) \circ 
(\Delta_{B_1} \otimes id_C)] (b_{j_i} \otimes t_{j_i}) = 
\sum \limits_{(b_{j_i})} {b_{j_i}}_{(1)} \otimes 
\lambda {b_{j_i}}_{(2)} \otimes t_{j_i}$. Thus, 
$b_i \otimes \beta_i \cdot t_i$ is a Group B term in 
$GF(\kappa)$. 

Now let ${b_i}_{(1)} \otimes \lambda {b_i}_{(2)} 
\cdot t_i$ be an arbitrary Group B term in $GF(\kappa)$. 
Then, ${b_i}_{(1)} \otimes \lambda {b_i}_{(2)} \otimes t_i$ 
is a term in 
$(id_{B_1} \otimes \lambda \otimes id_C) \circ 
(\Delta_{B_1} \otimes id_C) \kappa = 
(id_{B_1} \otimes \Delta_{C}) \kappa 
$. So, there is a Group B term, $b_{j_i} \otimes 
\beta_{j_i} \cdot t_{j_i}$, in $\kappa$ such that 
${b_i}_{(1)} \otimes \lambda {b_i}_{(2)} 
\otimes t_i$ is one of the terms in the sum 
$(id_{B_1} \otimes \Delta_{C}) 
(b_{j_i} \otimes \beta_{j_i} \cdot t_{j_i}) = 
\sum \limits_{(\beta_{j_i})}
b_{j_i} \otimes {\beta_{j_i}}_{(1)} 
\otimes {\beta_{j_i}}_{(2)} \cdot t_{j_i}$. 
Since $t_i$ is a cogenerator, the only term in the sum 
that could be equal to ${b_i}_{(1)} \otimes 
\lambda {b_i}_{(2)} \otimes t_i$ is 
$b_{j_i} \otimes \beta_{j_i} \otimes t_{j_i}$. 
Thus, ${b_i}_{(1)} \otimes \lambda {b_i}_{(2)} 
\cdot t_i$ is a Group B term in $\kappa$.
\end{proof}
%\subsection{Explicit description of 
$\hat{\lambda}^*C(\mathcal{A}^\prime)$}
Let $\lambda$ be a morphism in $\Lambda$ 
that induces a morphism in $\chi$ with source 
$\mathcal{A} \in Obj(\chi)$. 
Recall that from Section \ref{sec:def_G}
that we have a functor $\hat{\lambda}: 
B(\mathcal{A}) \to B(\lambda \mathcal{A})$. 
Applying the constructions in 
Section \ref{sec:pb_defn} to
$\hat{\lambda}$, we get a functor 
$\hat{\lambda}^*$ from the 
category of conilpotent 
dg comodules over $B(\lambda \mathcal{A})$ 
to the category 
of conilpotent dg comodules 
over $B(\mathcal{A})$. 
Below, we compute explicitly the complexes 
$[\hat{\lambda}^*C(\lambda \mathcal{A})](f)$ 
for $f \in Obj(B(\mathcal{A}))$.
%
\begin{prop}\label{prop:compute_pb}
Let $\lambda$ be a morphism in $\Lambda$ 
that induces a morphism in $\chi$ with source 
$\mathcal{A} \in Obj(\chi)$. 
Fix $f_0 \in Obj(B(\mathcal{A}))$. As comodules,
\begin{equation}\label{eq:compute_pb}
[\hat{\lambda}^*C(\lambda \mathcal{A})](f_0) \cong 
[B(\mathcal{A}) \otimes_{\hat{\lambda}} T(\lambda \mathcal{A})](f_0) := 
\bigoplus \limits_{h \in Obj(B(\mathcal{A}))}
B(\mathcal{A})(f_0, h) \otimes T(\lambda \mathcal{A})(\hat{\lambda} h)
\end{equation}
where $T(\lambda \mathcal{A})(\hat{\lambda} h)$ are the cogenerators of 
$C(\lambda \mathcal{A})(\hat{\lambda} h)$ 
(see Section \ref{def:cogenerators}).
\end{prop}
%
\begin{rem}
Proposition \ref{prop:compute_pb} holds 
for any quasi-cofree comodule over 
$B(\lambda \mathcal{A})$. The proof is the same.
\end{rem}
%
\begin{proof}[Proof of Proposition \ref{prop:compute_pb}]
To simplify notation in this proof, 
we will drop all references to $f_0$ and,
when unambiguous, references to $\lambda \mathcal{A}$. 
In other words, in this proof only,
\begin{align*}
C := C(\lambda \mathcal{A}) 
&\textrm{ will denote } C(\lambda \mathcal{A})(f_0), \\
\hat{\lambda}^*C := \hat{\lambda}^*C(\lambda \mathcal{A}) 
&\textrm{ will denote } [\hat{\lambda}^*C(\lambda \mathcal{A})](f_0), \\
B(\mathcal{A}) \otimes_{\hat{\lambda}} T 
&\textrm{ will denote } [B(\mathcal{A}) \otimes_{\hat{\lambda}} T(\lambda \mathcal{A})](f_0), \\
B(\mathcal{A}) \otimes_{\hat{\lambda}} C 
&\textrm{ will denote } [B(\mathcal{A}) \otimes_{\hat{\lambda}} C(\lambda \mathcal{A})](f_0), \\
B(\mathcal{A}) \otimes_{\hat{\lambda}} B(\lambda \mathcal{A}) \otimes C 
&\textrm{ will denote } 
[B(\mathcal{A}) \otimes_{\hat{\lambda}} B(\lambda \mathcal{A}) \otimes C(\lambda \mathcal{A})](f_0).
\end{align*}
%
To prove the proposition, 
we will give maps
$$
F: \hat{\lambda}^*C \rightleftarrows B(\mathcal{A}) \otimes_{\hat{\lambda}} T:G
$$
and show that $F\circ G = id_{B(\mathcal{A}) \otimes_{\hat{\lambda}} T}$ 
and $G \circ F = id_{\hat{\lambda}^*C}$.
We define $F$ as follows: 
$$
F:\hat{\lambda}^*C 
\xrightarrow[inclusion]{canonical}
B(\mathcal{A}) \otimes_{\hat{\lambda}} C
\xrightarrow[cogenerators]{project\;onto}
B(\mathcal{A}) \otimes_{\hat{\lambda}} T.
$$
To define $G$, we will give a map 
$G^\prime: B(\mathcal{A}) \otimes_{\hat{\lambda}} T \to 
B(\mathcal{A}) \otimes_{\hat{\lambda}} C$, and show that the image of 
$G^\prime$ lands in $\hat{\lambda}^*C$. We define 
$G^\prime$ as follows:
$$
G^\prime(b \otimes t) = 
\sum \limits_{(b)} b_{(1)} \otimes \hat{\lambda} b_{(2)} \cdot t
$$
where $b \otimes t \in B(\mathcal{A}) \otimes_{\hat{\lambda}} T$ and $\hat{\lambda} 
b_{(2)} \cdot t$ are elements of $C(\lambda \mathcal{A})(\hat{\lambda} h)$ 
written in terms of cogenerators.

To prove that the image of $G^\prime$ lands in 
$\hat{\lambda}^*C$, we need to show that the two maps 
$$
(id_{B(\mathcal{A})}\otimes \Delta_{C}) \circ G^\prime, \>
(id_{B(\mathcal{A})}\otimes \hat{\lambda} \otimes id_C)\circ 
(\Delta_{B(\mathcal{A})}\otimes id_C) \circ G^\prime: 
B(\mathcal{A}) \otimes_{\hat{\lambda}} T
\to B(\mathcal{A}) \otimes_{\hat{\lambda}} C
\rightrightarrows
B(\mathcal{A}) \otimes_{\hat{\lambda}} B(\lambda \mathcal{A}) \otimes C
$$
coincide. We have
\begin{align*}
[(1\otimes \Delta_{C}) \circ G^\prime](b \otimes t) 
&= 
\sum \limits_{(b),\, (\hat{\lambda} b)} b_{(1)} \otimes 
(\hat{\lambda} b_{(2)})_{(1)} \otimes 
(\hat{\lambda} b_{(2)})_{(2)} \cdot t \\
&= 
\sum \limits_{(b)} b_{(1)} \otimes 
\hat{\lambda} b_{(2)} \otimes 
\hat{\lambda} b_{(3)} \cdot t \\
&= 
[(id_{B(\mathcal{A})}\otimes \hat{\lambda} \otimes id_C)\circ 
(\Delta_{B(\mathcal{A})}\otimes id_C) \circ G^\prime]
(b\otimes t)
\end{align*}
where the second equality holds since $\hat{\lambda}$ 
is a map of cocategories
and $\Delta_{B(\mathcal{A})}$ is coassociative.

It's clear from the definitions that $F$ and $G$ are 
maps of comodules and that 
$F\circ G = id_{B(\mathcal{A})\otimes_{\hat{\lambda}} T}$. All that remains 
is to show that $G \circ F = id_{\hat{\lambda}^*C}$. 
Let $\kappa = \Sigma_i b_i \otimes \beta_i \cdot t_i$ be an 
arbitrary element of $\hat{\lambda}^*C \hookrightarrow 
B(\mathcal{A}) \otimes_{\hat{\lambda}} C$ where $\beta_i \cdot t_i$ are elements 
of $C(\lambda \mathcal{A})(\hat{\lambda} h)$ written in terms of cogenerators. 
Then, 
\begin{equation*}
GF(\kappa) = 
GF(\Sigma_i b_i \otimes \beta_i \cdot t_i) = 
\sum \limits_{\substack{i, \\ \beta_i = 1, \\ (b_i)}} 
{b_i}_{(1)} \otimes \hat{\lambda} {b_i}_{(2)} 
\cdot t_i.
\end{equation*}
We can divide the terms in $\kappa$ into two groups: 
(a) terms in which $\beta_i = 1 \in k$ and (b) terms in which $\beta_i
\neq 1 \in k$. Likewise, we can divide the terms in 
$GF(\kappa)$ into (a) terms in which $\hat{\lambda} {b_i}_{(2)} = 1$ 
and (b) terms in which $\hat{\lambda} {b_i}_{(2)} \neq 1$. 
From the definitions of $F$ and $G$, it's clear that 
the Group A terms in $\kappa$ are exactly the Group A 
terms in $GF(\kappa)$. 

To show that the Group B terms 
are the same, let $b_i \otimes \beta_i \cdot t_i$ be 
an arbitrary Group B term in $\kappa$.
Then, there is a term $b_i \otimes \beta_i \otimes t_i$ 
in $(id_{B(\mathcal{A})} \otimes \Delta_{C}) \kappa$. Since 
$(id_{B(\mathcal{A})} \otimes \Delta_{C}) \kappa = 
(id_{B(\mathcal{A})} \otimes \hat{\lambda} \otimes id_C) \circ 
(\Delta_{B(\mathcal{A})} \otimes id_C) \kappa$, 
there must be a Group A term, $b_{j_i} \otimes t_{j_i}$, 
in $\kappa$ such that $b_i \otimes \beta_i \otimes t_i$ 
is one of the terms in the sum 
$[(id_{B(\mathcal{A})} \otimes \hat{\lambda} \otimes id_C) \circ 
(\Delta_{B(\mathcal{A})} \otimes id_C)] (b_{j_i} \otimes t_{j_i}) = 
\sum \limits_{(b_{j_i})} {b_{j_i}}_{(1)} \otimes 
\hat{\lambda} {b_{j_i}}_{(2)} \otimes t_{j_i}$. Thus, 
$b_i \otimes \beta_i \cdot t_i$ is a Group B term in 
$GF(\kappa)$. 

Now let ${b_i}_{(1)} \otimes \hat{\lambda} {b_i}_{(2)} 
\cdot t_i$ be an arbitrary Group B term in $GF(\kappa)$. 
Then, ${b_i}_{(1)} \otimes \hat{\lambda} {b_i}_{(2)} \otimes t_i$ 
is a term in 
$(id_{B(\mathcal{A})} \otimes \hat{\lambda} \otimes id_C) \circ 
(\Delta_{B(\mathcal{A})} \otimes id_C) \kappa = 
(id_{B(\mathcal{A})} \otimes \Delta_{C}) \kappa 
$. So, there is a Group B term, $b_{j_i} \otimes 
\beta_{j_i} \cdot t_{j_i}$, in $\kappa$ such that 
${b_i}_{(1)} \otimes \hat{\lambda} {b_i}_{(2)} 
\otimes t_i$ is one of the terms in the sum 
$(id_{B(\mathcal{A})} \otimes \Delta_{C}) 
(b_{j_i} \otimes \beta_{j_i} \cdot t_{j_i}) = 
\sum \limits_{(\beta_{j_i})}
b_{j_i} \otimes {\beta_{j_i}}_{(1)} 
\otimes {\beta_{j_i}}_{(2)} \cdot t_{j_i}$. 
Since $t_i$ is a cogenerator, the only term in the sum 
that could be equal to ${b_i}_{(1)} \otimes 
\hat{\lambda} {b_i}_{(2)} \otimes t_i$ is 
$b_{j_i} \otimes \beta_{j_i} \otimes t_{j_i}$. 
Thus, ${b_i}_{(1)} \otimes \hat{\lambda} {b_i}_{(2)} 
\cdot t_i$ is a Group B term in $\kappa$.
\end{proof}
	\section{Pullbacks of dg comodules--examples}
%
\begin{eg}[Another definition of C(1)] \label{eg:pb}
Using $F$ and $G$ from Proposition 
\ref{prop:compute_pb}, we can induce 
differentials on $B(\mathcal{A}) \otimes_{\hat{\lambda}} T$ from
$\hat{\lambda}^*C$. We will compute this differential 
for a particular choice of $\lambda$. 
Let $\lambda = \delta_{0,1} \in \Lambda([1], [0])$. 
Fix algebras $A_0$ and $A_1$, and set 
\begin{align*}
\hat{\lambda}: B(1) 
&:= B(A_0 \to A_1 \to A_0) \to 
  B(A_0 \to A_0) =: B(0)\\
C(1) 
&:= C(A_0 \to A_1 \to A_0)\\
C(0) 
&:= C(A_0 \to A_0)
\end{align*}
($\hat{\lambda}$ is given by braces, 
see \ref{sec:def_G}.)

Note that 
$\hat{\delta}_{0,1}^* C(0) \cong
[B(1) \otimes_{\hat{\lambda}} 
  T(0)](f_{0,0},f_{1,0}) \cong
C(1)(f_{0,0},f_{1,0})$ 
as comodules where 
$(f_{0,0},f_{1,0}) \in Obj(B(1))$. 
Let 
$\phi_{0,1} \smdots \phi_{0,k_0} |
\phi_{1,1} \smdots \phi_{1,k_1} |
t$ be a typical element of 
$[B(1) \otimes_{\hat{\lambda}} 
  T(0)](f_{0,0},f_{1,0})$
(see \ref{fig:phi|alpha} for notational 
conventions). Then,
\begin{align*}
& \phantom{{}={}}
d_{B(1) \otimes_{\hat{\lambda}} T(0)}
(\phi_{0,1} \smdots \phi_{0,k_0} |
\phi_{1,1} \smdots \phi_{1,k_1} | t)\\
&=
Fd_{\hat{\lambda}^*C(0)}G
(\phi_{0,1} \smdots \phi_{0,k_0} |
\phi_{1,1} \smdots \phi_{1,k_1} | t) \\
&=
[F \circ (d_{B(1)}\otimes id_{C(0)} + 
  id_{B(1)} \otimes d_{C(0)})] \\
& \phantom{{}=  {}}
\big( \sum \limits_{\substack{1 \leq r_0 \leq k_0+1 \\ 
1 \leq r_1 \leq k_1+1}}
(\phi_{0,1} \smdots \phi_{0,r_0-1} |
\phi_{1,1} \smdots \phi_{1,r_1-1} ) \otimes 
((\phi_{0,r_0}\smdots \phi_{0,k_0}) \bullet 
(\phi_{1,r_1}\smdots \phi_{1,k_1}) | t) \big) \\
&= 
d_{C(1)(f_{0,0},f_{1,0})}
(\phi_{0,1} \smdots \phi_{0,k_0} |
\phi_{1,1} \smdots \phi_{1,k_1} | t)
\end{align*}
where the last equality holds by looking at which 
terms from $d_{B(1)}\otimes id_{C(0)} + 
id_{B(1)} \otimes d_{C(0)}$ 
are non-zero after projecting to cogenerators, and 
seeing that those are the same terms as in 
$d_{C(1)}$. So, $\hat{\delta}_{0,1}^*C(0) \cong C(1)$ as 
dg comodules. 
\end{eg}
%
\begin{eg}[Another definition of $C(n)$] 
  \label{eg:pb2}
Let $\lambda = \delta_{0,n} \in 
\Lambda([n],[n-1])$. Fix algebras 
$A_0, \dots A_n$, and set
\begin{align*}
\hat{\lambda}: B(n)
&:= B(A_0 \to A_1 \to \dots \to A_n \to A_0) 
  \to B(A_0 \to A_2 \to A_3 \to \dots \to A_n \to A_0) 
  =: B(n^\prime)\\
C(n) 
&:= C(A_0 \to A_1 \to \dots \to A_n \to A_0)\\
C(n^\prime) 
&:= C(A_0 \to A_2 \to A_3 \to \dots \to A_n \to A_0)
\end{align*}
($\hat{\lambda}$ is given by bracing the 
first and second terms, see \ref{sec:def_G}. 
We choose the notation $n^\prime$ instead of $n-1$ 
since, for the algebras fixed above, 
$B(n-1)$ would denote 
$B(A_0 \to A_1 \to A_2 \to \dots \to A_{n-1} \to A_0)$.)

Example \ref{eg:pb} shows
\begin{equation} \label{eq:base_case}
C(1) \cong \hat{\delta}_{0,1}^*C(0)
\end{equation}
as dg comodules. Given Equation 
\ref{eq:base_case} as a base 
case, we can show by induction that 
$$
C(n) \cong \hat{\delta}_{0,n}^*\dots\hat{\delta}_{0,1}^*C(0)
$$ 
as dg comodules. Suppose that 
$C(W_0 \to \dots \to W_{n-1} \to W_0) 
\cong \hat{\delta}_{0,n-1}^* \dots \hat{\delta}_{0,1}^*
C(W_0 \to W_0)$ for any 
choice of algebras $W_0, \dots W_{n-1}$.
(inductive hypothesis). Then, as 
comodules, we know
\begin{align*}
\hat{\delta}_{0,n}^*\dots\hat{\delta}_{0,1}^*C(0)
& \cong
  \hat{\delta}_{0,n}^*C(n^\prime)
  \quad \textrm{(inductive hypothesis applied to algebras } 
  A_0, A_2,\dots,A_n)\\
& \cong 
  B(n) \otimes_{\hat{\delta}_{0,n}} T(n^\prime)
  \quad \textrm{(Proposition 
  \ref{prop:compute_pb})}\\
& \cong 
  B(n) \otimes_{\hat{\delta}_{0,n}} T(0)
  \quad \textrm{(Definition of $T$)} \\
& \cong 
  C(n)
  \quad \textrm{(Definition of $C(n)$)}
\end{align*}
where $T(n)$ are the cogenerators of 
$C(n)$ (see Definition 
\ref{def:cogenerators}).

To show that the differentials coincide, 
we compute 
\begin{align*}
& \phantom{{}={}}
Fd_{\hat{\delta}_{0,n}^*\dots\hat{\delta}_{0,1}^*C(0)}G
  (\phi_{0,1} \smdots \phi_{0,k_0} | \smdots |
  \phi_{n,1} \smdots \phi_{n,k_n} | t) \\
&=
[F \circ (d_{B(n)}\otimes 
  id_{\hat{\delta}_{0,n-1}^*\dots\hat{\delta}_{0,1}^*C(0)} + 
  id_{B(n)} \otimes 
  d_{\hat{\delta}_{0,n-1}^*\dots\hat{\delta}_{0,1}^*C(0)})] \\
& \phantom{{}=  {}}
\big( \sum \limits_{\substack{1 \leq j \leq n\\
  1 \leq r_j \leq k_j+1}}
  (\phi_{0,1} \smdots \phi_{0,r_0-1} | \smdots |
  \phi_{n,1} \smdots \phi_{n,r_1-1} ) \otimes \\
&\phantom{{}= \big( \sum \sum {}}
  ((\phi_{0,r_0}\smdots \phi_{0,k_0}) \bullet 
  (\phi_{1,r_1}\smdots \phi_{1,k_1}) |
  \phi_{2,r_1}\smdots \phi_{2,k_2} | \smdots |
  \phi_{n,r_1}\smdots \phi_{n,k_n} | t) \big) \\
&=
[F \circ (d_{B(n)}\otimes id_{C(n^\prime)} + 
  id_{B(n)} \otimes d_{C(n^\prime)})] \\
& \phantom{{}=  {}}
\big( \sum \limits_{\substack{1 \leq j \leq n\\
  1 \leq r_j \leq k_j+1}}
  (\phi_{0,1} \smdots \phi_{0,r_0-1} | \smdots |
  \phi_{n,1} \smdots \phi_{n,r_1-1}) \otimes \\
&\phantom{{}= \big( \sum \sum {}}
  ((\phi_{0,r_0}\smdots \phi_{0,k_0}) \bullet 
  (\phi_{1,r_1}\smdots \phi_{1,k_1}) |
  \phi_{2,r_1}\smdots \phi_{2,k_2} | \smdots |
  \phi_{n,r_1}\smdots \phi_{n,k_n} | t) \big)
\end{align*}
where the last equality holds by the 
inductive hypothesis. The terms from 
$d_{B(n)}\otimes id_{C(n^\prime)} + 
id_{B(n)} \otimes d_{C(n^\prime)}$ that
are non-zero after projecting to cogenerators 
are exactly the terms in $d_{C(n)}$. 
So, $C(n) \cong \hat{\delta}_{0,n}^*\dots
\hat{\delta}_{0,1}^*C(0)$ as dg comodules. 
\end{eg}
%
\begin{eg}[Yet another description of C(n)] 
\label{eg:pb3}
Fix algebras $A_0, \dots A_n$, $n>0$ 
and choose a sequence of coboundaries 
$\delta_{i_1,1}, \dots, \delta_{i_n,n}$ with 
$0 \leq i_j \leq j-1$, $1 \leq j \leq n$. 
Then,
$$
\delta_{i_1,1} \circ \dots \circ \delta_{i_n,n} 
= \delta_{0,1} \circ \dots \circ \delta_{0,n}
= \textrm{unique map in } \Delta([n],[0])
  \subset \Lambda([n],[0]).
$$
This implies that, as functors 
on categories of comodules, 
\begin{align*}
\hat{\delta}_{i_n,n}^* \dots \hat{\delta}_{i_1,1}^*
&=
\widehat{(\delta_{i_1,1} \circ \dots \circ 
  \delta_{i_n,n})}^*
  \quad \textrm{(Proposition \ref{prop:pbs_compose})}\\
&= 
\widehat{(\delta_{0,1} \circ \dots \circ 
  \delta_{0,n})}^*
  \quad \textrm{(Computation above)}\\
&= 
\hat{\delta}_{0,n}^* \dots \hat{\delta}_{0,1}^* 
  \quad \textrm{(Proposition \ref{prop:pbs_compose}).} 
\end{align*}
Since braces are associative, the 
differentials on $\hat{\delta}_{i_n,n}^* 
\dots \hat{\delta}_{i_1,1}^*C(0)$ and 
$\hat{\delta}_{0,n}^* 
\dots \hat{\delta}_{0,1}^*C(0)$ 
coincide. So, $\hat{\delta}_{i_n,n}^* 
\dots \hat{\delta}_{i_1,1}^*C(0)
\cong \hat{\delta}_{0,n}^* 
\dots \hat{\delta}_{0,1}^*C(0)$ 
as dg comodules.
\end{eg}
%
\begin{eg}[Pullbacks along codegeneracies]
\label{eg:pb4}
Fix algebras $A_0, \dots, A_n$ and let 
$\sigma_{i,n} \in \Lambda([n],[n+1])$, 
$1 \leq i \leq n+1$ be a generating 
codegeneracy (see Appendix 
\ref{chap:lambda}). Set
$$
\hat{\sigma}_{i,n}: B(n) \to 
B(n^\prime)
$$
where 
\begin{align*}
B(n) 
&:= 
B(A_0 \to \dots \to A_n \to A_0)\\
B(n^\prime) 
&:= 
\begin{cases}
  B(A_0 \to \dots \to A_i \to A_i 
    \to \dots A_n \to A_0)
  & 1 \leq i \leq n \\
  B(A_0 \to \dots \to A_n \to A_0 \to A_0)
  & i = n+1
\end{cases}\\
C(n) 
&:= 
C(A_0 \to \dots \to A_n \to A_0)\\
C(n^\prime) 
&:= 
\begin{cases}
  C(A_0 \to \dots \to A_i \to A_i 
    \to \dots A_n \to A_0)
  & 1 \leq i \leq n \\
  C(A_0 \to \dots \to A_n \to A_0 \to A_0)
  & i = n+1.
\end{cases}
\end{align*}
From Proposition \ref{prop:compute_pb}, we 
know that $\hat{\sigma}_{i,n}^*C(n^\prime) 
\cong B(n) \otimes_{\hat{\sigma}_{i,n}} 
T(n^\prime) \cong C(n)$ as comodules.
To show that the differentials coincide, 
we compute 
\begin{align*}
& \phantom{{}={}}
Fd_{\hat{\sigma}_{i,n}^*C(n^\prime)}G
  (\phi_{0,1} \smdots \phi_{0,k_0} | \smdots |
  \phi_{n,1} \smdots \phi_{n,k_n} | t) \\
&=
[F \circ (d_{B(n)}\otimes 
  id_{\hat{\sigma}_{i,n}^*C(n^\prime)} + 
  id_{B(n)} \otimes 
  d_{\hat{\sigma}_{i,n}^*C(n^\prime)})] \\
& \phantom{{}=  {}}
\big( \sum \limits_{\substack{1 \leq j \leq n\\
  1 \leq r_j \leq k_j+1}}
  (\phi_{0,1} \smdots \phi_{0,r_0-1} | \smdots |
  \phi_{n,1} \smdots \phi_{n,r_1-1} ) \otimes \\
&\phantom{{}= \big( \sum \sum {}}
  (\phi_{0,r_0}\smdots \phi_{0,k_0}| \smdots |
  \phi_{0,r_{i-1}}\smdots \phi_{0,k_{i-1}}| 1 |
  \phi_{0,r_i}\smdots \phi_{0,k_i}| \smdots |
  \phi_{n,r_1}\smdots \phi_{n,k_n} | t) \big).
\end{align*}
Since 1 is a unit for braces, the terms from 
$d_{B(n)} \otimes id_{\hat{\sigma}_{i,n}^*C(n^\prime)} 
+ id_{B(n)} \otimes d_{\hat{\sigma}_{i,n}^*C(n^\prime)}$ 
thatare non-zero after projecting to cogenerators 
are exactly the terms in $d_{C(n)}$. 
So, $\hat{\sigma}_{i,n}^*C(n^\prime) \cong C(n)$ 
as dg comodules.
\end{eg}
%
\begin{eg}[Pullbacks along rotations]
\label{eg:pb5}
Fix algebras $A_0, \dots, A_n$ and let 
$\tau_n \in \Lambda([n],[n])$ 
be a generating rotation (see Appendix 
\ref{chap:lambda}). Set
$$
\hat{\tau}_n: B(n) \to 
B(n^\prime)
$$
where 
\begin{align*}
B(n) &:= 
B(A_0 \to \dots \to A_n \to A_0)\\
B(n^\prime) &:= 
B(A_n \to A_0 \to \dots \to A_{n-1} \to A_n)\\
C(n) &:= 
C(A_0 \to \dots \to A_n \to A_0)\\
C(n^\prime) &:= 
C(A_n \to A_0 \to \dots \to A_{n-1} \to A_n).
\end{align*}
From Proposition \ref{prop:compute_pb}, we 
know that $\hat{\tau}_n^*C(n^\prime) 
\cong B(n) \otimes_{\hat{\tau}_n} 
T(n^\prime)$ as comodules. Unpacking the 
righthand side, we see that $B(n) 
\otimes_{\hat{\tau}_n} T(n^\prime) 
\cong C(n^\prime)$ as complexes--the 
isomorphism is given by $\hat{\tau}_n 
\otimes id_{T}$.
\end{eg}
	\section{Adjunction between $\lambda^*$ and $\lambda_\#$}
In this section, we define $\lambda_\#$, 
the left adjoint to $\lambda^*$. 
More precisely, for any functor, $\lambda:
B_1 \to B_0$ between 
conilpotent dg cocategories, 
we define a functor 
$\lambda_\#$ from the category of 
conilpotent dg comodules 
over $B_1$ to the category of 
conilpotent dg 
comodules over $B_0$.
The adjunction will be used to show 
that structures we've established 
for $(B(n), C(n))$ still exist 
after we pass from cocategories and 
comodules to categories and modules
by applying (a categorified) $Cobar(-)$ 
to $(B(n), C(n))$ (see Section 
\ref{sec:cobar}). If a lesson 
of this thesis is that working with
cocategories is more tractable 
than with categories, then the reader 
may skip this section or save it until 
s/he is ready for Section 
\ref{sec:use_adjunction}.
%
\subsection{The functors $\lambda_\#$}
\label{sec:pf_defn}
Let $\lambda:B_1 \to B_0$ be a functor 
between conilpotent dg 
cocategories. Let $C$ be a 
conilpotent dg comodule over $B_1$.
We define $\lambda_\# C$ as follows: 
for $f \in Obj(B_0)$,
\begin{align*}
\lambda_\# C(f) := 
\big( \bigoplus \limits_{f^\prime \in \lambda^{-1}f}
C^\bullet(f^\prime),\\
\phantom{{}\lambda_\# C(f):=\big({}}
\Delta_{\lambda_\#C}(f):
\bigoplus \limits_{f^\prime \in \lambda^{-1}f}
C^\bullet(f^\prime)
&
\xrightarrow{\bigoplus \limits_{f^\prime} 
  \Delta_{C^\bullet}(f^\prime)}
\bigoplus \limits_{
  \substack{f^\prime \in \lambda^{-1}f\\h^\prime \in Obj(B_1)}}
 B_1^\bullet(f^\prime,h^\prime) \otimes C^\bullet(h^\prime) \\
&
\xrightarrow{\bigoplus \limits_{h^\prime,f^\prime} \lambda 
\otimes id_{C^\bullet(h^\prime)}}
\bigoplus \limits_{h^\prime \in Obj(B_1)}
  B_0^\bullet(f,\lambda h^\prime)
  \otimes C^\bullet(h^\prime) \\
&
\xrightarrow{include}
\bigoplus \limits_{h \in Obj(B_0)}
  B_0^\bullet(f,h) \otimes 
  (\bigoplus \limits_{h^\prime \in \lambda^{-1}h}
  C^\bullet(h^\prime))   
\big).
\end{align*}
To check that $\Delta_{\lambda_\#C}$ is well-defined, 
we need that the image of the first map, 
$\bigoplus \limits_{f^\prime} 
  \Delta_{C^\bullet}(f^\prime)$, 
is a finite sum. This is true since 
$C$ being conilpotent implies that 
the image of $\Delta_{C^\bullet}(f^\prime)$ 
is a finite sum for each $f^\prime \in
Obj(B_1)$. If $\lambda^{-1}f$ is empty, 
we set $\lambda_\# C(f) := 0$.
It is straightforward to check 
that $(\lambda_\#C, \Delta_{\lambda_\#C})$ 
is coassociative, conilpotent
and coaugmented. 
We will call $\lambda_\#$ 
``co-restriction of scalars''.

Let $F:C \to D$ be map of dg comodules 
over $B_1$. We define $\lambda_\# F$ as follows:
$$
(\lambda_\# F)_f : 
\lambda_\# C (f) = 
\bigoplus \limits_{f^\prime \in \lambda^{-1}f}
C^\bullet(f^\prime)
\xrightarrow{\bigoplus \limits_{f^\prime \in \lambda^{-1}f}
F_{f^\prime}} 
\bigoplus \limits_{f^\prime \in \lambda^{-1}f}
D^\bullet(f^\prime)
= \lambda_\# D (f).
$$
It's straightforward to check that $\lambda_\#$ 
is a functor (i.e., respects composition of 
morphisms).

%
\subsection{Adjunction}
\begin{prop}\label{prop:adjunction}
Given a functor between 
conilpotent dg cocategories, 
$\lambda: B_1 \to B_0$, let
\begin{equation*}
\lambda^*:
\let\scriptstyle\textstyle
\substack{
  \textrm{Category of}\\
  \textrm{conilpotent}\\
  \textrm{dg comodules over }B_0}
\leftrightarrows
\let\scriptstyle\textstyle
\substack{
  \textrm{Category of}\\
  \textrm{conilpotent}\\
  \textrm{dg comodules over }B_1}
: \lambda_\#
\end{equation*}
be the functors defined in Sections 
\ref{sec:pb_defn} and \ref{sec:pf_defn}. Then,
$\lambda_\#$ is left adjoint to $\lambda^*$.
\end{prop}
%
\begin{rem} \label{rem:adjunction}
Proposition \ref{prop:adjunction} 
is a categorified co-version of the 
adjunction between extension of scalars 
(left) and restriction of scalars (right) 
for modules over algebras.
\end{rem}
%
\begin{proof}[Proof of Proposition \ref{prop:adjunction}]
Let $C$ be a conilpotent dg comodule 
over $B_1$ and $D$ be a dg 
conilpotent dg comodule over $B_0$.
We want to show that
$$
Hom_{B_1}(C,\, \lambda^*D) = 
Hom_{B_0}(\lambda_\# C,\, D)
$$
as sets.

We will give maps 
$$
\Phi: Hom_{B_0}
(\lambda_\# C,\, D) \leftrightarrows Hom_{B_1}
(C,\, \lambda^*D): \Phi^{-1}
$$ 
satisfying 
$\Phi \circ \Phi^{-1} = id$ and 
$\Phi^{-1} \circ \Phi = id$. 

First, we define 
$\Phi$. Let 
$F$ be a morphism from $\lambda_\# C$ to $D$. 
By defintion, for $f \in Obj(B_0)$, we 
have maps of complexes
$$
F_f: 
\bigoplus \limits_{f^\prime \in \lambda^{-1}f}
C^\bullet(f^\prime)
\to D^\bullet(f).
$$
Define $\Phi F \in Hom_{B_1}(C,\, \lambda^*D)$ 
as follows: for $f^\prime \in Obj(B_1)$,
\begin{equation}\label{eqn:Phi_F}
\begin{split}
\Phi F_{f^\prime}:
C^\bullet(f^\prime)
&
\xrightarrow{\Delta_C}
\bigoplus \limits_{h^\prime \in Obj(B_1)}
  B_1^\bullet(f^\prime, h^\prime) 
  \otimes C^\bullet(h^\prime) \\
&
\xrightarrow{\bigoplus 
  \limits_{h^\prime}
  id_{B_1}\otimes 
  F_{\lambda h^\prime}|_{h^\prime}}
\bigoplus \limits_{h^\prime \in Obj(B_1)}
  B_1^\bullet(f^\prime, h^\prime) 
  \otimes D^\bullet(\lambda h^\prime) \\
&
\xrightarrow{include}
[B_1 \otimes_\lambda D](f^\prime).
\end{split}
\end{equation}
By the universal property of $\lambda^*D$, 
this defines a morphism $C \to \lambda^*D$ 
if the two maps
$$
(id_{B_1}\otimes \Delta_D) \circ \Phi F, \;
(id_{B_1}\otimes \lambda \otimes id_D)\circ 
  (\Delta_{B_1}\otimes id_D) \circ 
  \Phi F: 
C \rightrightarrows
B_1 \otimes_\lambda B_0\otimes D  
$$
coincide. In fact, on $f^\prime \in Obj(B_1)$,
both maps are equal to:
\begin{align*}
C^\bullet(f^\prime)
& \xrightarrow{\Delta_C}
\bigoplus \limits_{h^\prime \in Obj(B_1)}
  B_1^\bullet(f^\prime, h^\prime)
  \otimes C^\bullet(h^\prime)\\
& \xrightarrow{\bigoplus \limits_{h^\prime} 
  id_{B_1} \otimes \Delta_C}
\bigoplus \limits_{g^\prime, h^\prime \in Obj(B_1)}
  B_1^\bullet(f^\prime, g^\prime)
  \otimes B_1^\bullet(g^\prime, h^\prime)
  \otimes C^\bullet(h^\prime)\\
& \xrightarrow{\bigoplus \limits_{h^\prime,g^\prime}
  id_{B_1}\otimes \lambda \otimes 1_{C}}
\bigoplus \limits_{g^\prime, h^\prime \in Obj(B_1)}
  B_1^\bullet(f^\prime, g^\prime) \otimes
  B_0^\bullet(\lambda g^\prime, \lambda h^\prime)
  \otimes C^\bullet(h^\prime)\\
& \xrightarrow{\bigoplus \limits_{h^\prime,g^\prime}
  id_{B_1}\otimes id_{B_0} \otimes 
  F_{\lambda h^\prime}|_{h^\prime}}
\bigoplus \limits_{g^\prime, h^\prime \in Obj(B_1)}
  B_1^\bullet(f^\prime, g^\prime) \otimes
  B_0^\bullet(\lambda g^\prime, \lambda h^\prime)
  \otimes D^\bullet(\lambda h^\prime).
\end{align*}
This fact follows from $F$ being a map of comodules.
It's also clear that $\Phi F$ commutes with 
coproducts and differentials. So, we've 
shown $\Phi F \in Hom_{B_1}(C,\, \lambda^*D)$.

Second, we define $\Phi^{-1}$. Now, let 
$F \in Hom_{B_1}(C, \lambda^*D)$. 
For $f \in Obj(B_0)$, define 
\begin{align*}
\Phi^{-1}F_f : 
\bigoplus \limits_{f^\prime \in \lambda^{-1}f}
  C^\bullet(f^\prime)
&\xrightarrow{\bigoplus \limits_{f^\prime} 
  F_{f^\prime}}
\bigoplus \limits_{\substack{f^\prime \in \lambda^{-1}f,\\
  h^\prime \in Obj(B_1)}}
  B_1^\bullet(f^\prime, h^\prime)
  \otimes D^\bullet(\lambda h^\prime)\\    
&\xrightarrow{\bigoplus \limits_{f^\prime, h^\prime}
  \lambda \otimes id_D}
\bigoplus \limits_{h \in Obj(B_0)}
  B_0^\bullet(f, h)
  \otimes D^\bullet(h) \\
&\xrightarrow{\bigoplus \limits_h
  \epsilon_{B_0} \otimes id_D}
D^\bullet(f).
\end{align*}
It's clear that $\Phi^{-1}F$ commutes with the 
differentials. We will show that
$\Phi^{-1}F$ is a map of comodules.
Figure \ref{fig:delta_phi-1} gives a diagram 
showing that 
\begin{equation}\label{eq:delta_phi-1}
\Delta_D \circ \Phi^{-1}F_f = 
(\bigoplus \limits_{f^\prime, h^\prime, r^\prime}
  \epsilon_{B_0} \lambda \otimes \lambda 
  \otimes id_{D}) \circ 
(\bigoplus \limits_{f^\prime, h^\prime}
  \Delta_{B_1} \otimes id_D) \circ  
(\bigoplus \limits_{f^\prime}
 F_{f^\prime}).
\end{equation}
On the other hand, Figure 
\ref{fig:phi-1_delta} gives a diagram 
showing that 
\begin{equation} \label{eq:phi-1_delta}
(id_{B_1}\otimes 
\Phi^{-1}F) \circ \Delta_{\lambda_\# C} = 
(\bigoplus \limits_{f^\prime, h^\prime, r^\prime}
  \lambda \otimes \epsilon_{B_0} \lambda 
  \otimes id_{D}) \circ
(\bigoplus \limits_{f^\prime, h^\prime}
  \Delta_{B_1} \otimes id_D) \circ  
(\bigoplus \limits_{f^\prime}
 F_{f^\prime}).
\end{equation}
We see that the righthand sides of 
Equations \ref{eq:delta_phi-1} and 
\ref{eq:phi-1_delta} are the same 
except for the $B_0$ factor on which 
$\epsilon_{B_0}$ acts. However, in general,
for $\lambda: B_1 \to B_0$ a map 
of dg cocategories, we have
\begin{align*}
(\lambda \otimes \epsilon_{B_0} \lambda)
  \circ \Delta_{B_1}
&=
(id_{B_0} \otimes \epsilon_{B_0}) \circ
  \Delta_{B_0} \circ \lambda
\quad \textrm{($\lambda$ commutes with coproduct)}\\
&= 
id_{B_0} \circ \lambda  
\quad \textrm{(definition of cocategory)}\\
&= 
(\epsilon_{B_0} \otimes id_{B_0}) \circ
  (\Delta_{B_0}) \circ \lambda
\quad \textrm{(definition of cocategory)}\\
&=
(\epsilon_{B_0} \lambda \otimes \lambda)
  \circ \Delta_{B_1}
\quad \textrm{($\lambda$ commutes with coproduct).}  
\end{align*}
So, $(id_{B_1}\otimes \Phi^{-1}F) \circ 
\Delta_{\lambda_\# C} = \Delta_D \circ \Phi^{-1}F$, 
and $\Phi^{-1}F \in Hom_{B_0}(\lambda_\#C, D)$.

For $F: C \to \lambda^*D$ a map of 
dg comodules and $f^\prime 
\in B_1$, Figure \ref{fig:phi_phi-1} 
shows that $\Phi\Phi^{-1}F_{f^\prime}
 = F_{f^\prime}$.
For $F: \lambda_\# C \to D$ a map of 
dg comodules and $f 
\in B_0$, Figure \ref{fig:phi-1_phi} 
shows that $\Phi^{-1}\Phi F_f= F_f$. 
Thus, we have $\Phi\Phi^{-1} = id$ 
and $\Phi^{-1}\Phi = id$.
\end{proof}
%
\begin{landscape}
\begin{figure}
\centerline{\xymatrixrowsep{5pc}
\xymatrix{
\bigoplus \limits_{f^\prime \in \lambda^{-1}f}
  C^\bullet(f^\prime)
\ar@[red][d]^{\bigoplus \limits_{f^\prime} F_{f^\prime}}\\
%
\bigoplus \limits_{\substack{
  f^\prime \in \lambda^{-1}f\\
  h^\prime \in Obj(B_1)}}
  B_1^\bullet(f^\prime, h^\prime)
  \otimes D^\bullet(\lambda h^\prime)
\ar@[red][r]^{\bigoplus \limits_{f^\prime, h^\prime} 
  \lambda \otimes id_{D}}  
\ar@/_/[d]_{\bigoplus \limits_{f^\prime, h^\prime, r^\prime}
  \substack{
  (\Delta_{B_1} \otimes id_D) \circ \\
  (id_{B_1} \otimes \lambda \otimes id_D)}}
\ar@/^/[d]^{\bigoplus \limits_{f^\prime, h^\prime} 
  id_{B_1} \otimes \Delta_{D}}
& \bigoplus \limits_{h \in Obj(B_0)}
  B_0^\bullet(f,h) \otimes D^\bullet(h)
\ar@[red][r]^{\bigoplus \limits_h 
  \epsilon_{B_0} \otimes id_D} 
& D^\bullet(f)
\ar@[red][d]^{\Delta_D} \\
%
\bigoplus \limits{\substack{
  f^\prime \in \lambda^{-1}f,\\
  h^\prime \in Obj(B_1),\\
  r \in Obj(B_0)}}
  B_0^\bullet(f^\prime, h^\prime) \otimes 
  B_1^\bullet(\lambda h^\prime, r) 
  \otimes D^\bullet(r)
\ar[rr]_{\bigoplus \limits_{f^\prime, h^\prime, r}
  \epsilon_{B_0} \lambda \otimes id_{B_1} 
  \otimes id_D}
& & \bigoplus \limits_{h \in Obj(B_0)}
  B_0^\bullet(f,h) \otimes D^\bullet(h) 
}}
\caption{Commuting diagram 
involving $\Delta_D \circ 
\Phi^{-1}F$} \label{fig:delta_phi-1}
$\Delta_D \circ 
\Phi^{-1}F$ = composition 
of red arrows. 
The fact that $F: C \to \lambda^*D$ and 
the universal property of $\lambda^*D$ 
imply that the diagram commutes.  
\end{figure}
%
\begin{figure}
\centerline{\xymatrixrowsep{5pc}
\xymatrix{
\bigoplus \limits_{f^\prime \in \lambda^{-1}f}
  C^\bullet(f^\prime)
\ar[d]^{\bigoplus \limits_{f^\prime} F_{f^\prime}}
\ar@[red][r]^{\bigoplus \limits_{f^\prime} \Delta_C}
& \bigoplus \limits_{\substack{
  f^\prime \in \lambda^{-1}f\\
  h^\prime \in Obj(B_1)}}
  B_1^\bullet(f^\prime, h^\prime)
  \otimes C^\bullet(h^\prime)
\ar@[red][r]^{\bigoplus \limits_{f^\prime, h^\prime} 
  \lambda \otimes id_{C}}  
\ar[d]^{\bigoplus \limits_{f^\prime, h^\prime}
  id_{B_1} \otimes 
  F_{\lambda h^\prime}|_{h^\prime}}
& \bigoplus \limits_{h^\prime \in Obj(B_1)}
  B_0^\bullet(f,\lambda h^\prime) 
  \otimes C^\bullet(h^\prime)
\ar@[red][d]^{\bigoplus \limits_{h^\prime}
  id_{B_0} \otimes 
  F_{\lambda h^\prime}|_{h^\prime}}\\
%  
\bigoplus \limits{\substack{
  f^\prime \in \lambda^{-1}f,\\
  r^\prime \in Obj(B_1)}}
  B_1^\bullet(f^\prime, r^\prime) 
  \otimes D^\bullet(\lambda r^\prime)
\ar[r]^{\substack{\Delta_{\lambda^*D} =\\
  \bigoplus \limits_{f^\prime, r^\prime}
  \Delta_{B_1} \otimes id_D}}
& \bigoplus \limits{\substack{
  f^\prime \in \lambda^{-1}f,\\
  h^\prime, r^\prime \in Obj(B_1)}}
  B_1^\bullet(f^\prime, h^\prime) \otimes
  B_1^\bullet(h^\prime, r^\prime) 
  \otimes D^\bullet(\lambda r^\prime)
\ar[r]^{\bigoplus 
  \limits_{f^\prime, h^\prime, r^\prime}
  \lambda \otimes id_{B_0} \otimes id_D}
& \bigoplus \limits_{h^\prime, r^\prime 
  \in Obj(B_1)}
  B_0^\bullet(f,\lambda h^\prime) \otimes
  B_1^\bullet(h^\prime, r^\prime)
  \otimes D^\bullet(\lambda r^\prime)  
\ar@[red][d]_{\bigoplus \limits_{h^\prime, r^\prime}
  id_{B_0} \otimes 
  \epsilon_{B_0} \lambda \otimes id_{D}}\\
%
& & \bigoplus \limits_{h\in Obj(B_0)}
  B_0^\bullet(f,h) \otimes 
  D^\bullet(h)
}}
\caption{Commuting diagram
involving $(id_{B_1}\otimes 
\Phi^{-1}F) \circ \Delta_{\lambda_\# C}$} \label{fig:phi-1_delta}
$(id_{B_1}\otimes 
\Phi^{-1}F) \circ \Delta_{\lambda_\# C}$
= composition of red arrows. 
The fact that $F$ respects coproducts  
implies that the left square commutes. 
\end{figure}
%
\begin{figure}
\centerline{\xymatrixrowsep{5pc}
\xymatrix{
C^\bullet(f^\prime)
\ar@[red][r]^{\Delta_C}
\ar[d]^{F_{f^\prime}}
& \bigoplus \limits_{g^\prime \in Obj(B_1)}
  B_1^\bullet(f^\prime, g^\prime)
  \otimes C^\bullet(g^\prime)
\ar@[red][d]^{\bigoplus \limits_{g^\prime}
  id_{B_1}\otimes F_{g^\prime}}\\
%
\bigoplus \limits_{h^\prime \in Obj(B_1)}  
  B_1^\bullet(f^\prime, h^\prime)
  \otimes D^\bullet(\lambda h^\prime)
\ar[r]^{\Delta_{\lambda^* D} = \bigoplus
  \limits_{h^\prime} \Delta_{B_1} \otimes id_D}
\ar@/_3pc/[rr]^{id}  
& \bigoplus \limits_{g^\prime, h^\prime \in Obj(B_1)}
  B_1^\bullet(f^\prime, g^\prime)
  \otimes B_1^\bullet(g^\prime, h^\prime)
  \otimes D^\bullet(\lambda h^\prime)
\ar@[red][r]^{\bigoplus \limits_{g^\prime, h^\prime}
  id_{B_1} \otimes 
  (\epsilon_{B_0}\lambda = \epsilon_{B_1})
  \otimes id_D}
& \bigoplus \limits_{g^\prime \in Obj(B_1)}
  B_1^\bullet(f^\prime, g^\prime)
  \otimes D^\bullet(\lambda g^\prime)
}}
\caption{Commuting diagram 
involving $\Phi\Phi^{-1}F_{f^\prime}$}  \label{fig:phi_phi-1}
$\Phi\Phi^{-1}F_{f^\prime}$
= composition of red arrows. 
The square commutes because $F$ 
respects coproducts; 
the composition of the bottom row 
of horizontal arrows is equal to 
the identity because $\lambda_\#D$ 
satisfies counitality.
\end{figure}
%
\begin{figure}
\centerline{
\xymatrixrowsep{5pc}
\xymatrixcolsep{5pc}
\xymatrix{
\bigoplus \limits_{f^\prime \in \lambda^{-1}f}
  C^\bullet(f^\prime)
\ar@[red][r]^{\bigoplus \limits_{f^\prime}
  \Delta_C}
\ar[dd]_{F_f}
& \bigoplus \limits_{\substack{
  f^\prime \in \lambda^{-1}f,\\
  g^\prime \in Obj(B_1)}}
  B_1^\bullet(f^\prime, g^\prime)
  \otimes C^\bullet(g^\prime)
\ar@[red][r]^{\bigoplus \limits_{
  f^\prime, g^\prime} 
  id_{B_1} \otimes 
  F_{\lambda g^\prime}|_{g^\prime}}
\ar[d]^{\bigoplus \limits_{
  f^\prime, g^\prime} 
  \lambda \otimes id_C}
& \bigoplus \limits_{\substack{
  f^\prime \in \lambda^{-1}f,\\
  g^\prime \in Obj(B_1)}} 
  B_1^\bullet(f^\prime, g^\prime)
  \otimes D\bullet(\lambda g^\prime)
\ar@[red][d]^{\bigoplus \limits_{
  f^\prime, g^\prime} 
  \lambda \otimes id_D}\\
%
& \bigoplus \limits_{g^\prime \in Obj(B_1)}
  B_0^\bullet(f, \lambda g^\prime)
  \otimes C^\bullet(g^\prime)
\ar[r]^{\bigoplus \limits_{g^\prime}
  id_{B_0} \otimes 
  F_{\lambda g^\prime}|_{g^\prime}}
& \bigoplus \limits_{g \in Obj(B_0)}
  B_0^\bullet(f, g)
  \otimes D^\bullet(g)
\ar@[red][r]^{\bigoplus \limits_g
  \epsilon_{B_0} \otimes id_D}
& D^\bullet(f)\\
%
D^\bullet(f)
\ar[rru]_{\Delta_D}  
\ar@/_3pc/[rrru]_{id}
}}
\caption{Commuting diagram 
involving $\Phi^{-1}\Phi F_f$}  \label{fig:phi-1_phi}
$\Phi^{-1}\Phi F_f$ = composition of red arrows. 
The concave pentagon on the left 
side commutes because $F$ 
respects coproducts; the triangle in the 
bottom right corner commutes 
because $D$ satisfies counitality.
\end{figure}
\end{landscape}
	\section{Maps $\lambda_!$}
\label{sec:shriek_maps}
In this section, 
we give maps $\lambda_!: 
C(\mathcal{A}) \to 
\hat{\lambda}^*C(\lambda \mathcal{A})$ 
of dg comodules 
over $B(\mathcal{A})$ where 
$\lambda$ is a generating morphism 
in $\Lambda$ (see Appendix \ref{chap:lambda}) 
that induces a morphsim in $\chi$ 
with source $\mathcal{A} \in Obj(\chi)$. 
Showing that the 
$\lambda_!$'s satisfy cyclic relations 
up to homotopy is the 
computational heart of this thesis, 
and will be done in the next chapter. 
For now, we introduce the $\lambda_!$'s.

Technically, we should write 
$\lambda_{\mathcal{A}!}$ instead of 
$\lambda_!$, but we will be clear 
about the source when needed. 
In this section, we 
fix algebras $A_0, \smdots A_{n+1}$ and set 
$B(n) := B(A_0 \to \smdots \to A_n \to A_0)$, 
$C(n) := C(A_0 \to \smdots \to A_n \to A_0)$. 
We define maps $\lambda_!: 
C(A_0 \to \smdots \to A_n \to A_0) \to 
\hat{\lambda}^*C(\lambda(A_0 \to 
\smdots \to A_n \to A_0))$.
%
\subsection{Generating coboundaries 
  ${\delta_{j,n}}_!$ for $n \in \mathbb{N}$, 
  $0 \leq j \leq n-1$}
From Example \ref{eg:pb3}, we know that $C(n) \cong 
\hat{\delta}_{j,n}^*C(A_0 \to \smdots \to A_j 
\longrightarrow A_{j+2} \to \smdots \to A_0)$. 
So, define
$\delta_{j,n!}:C(n)
\xrightarrow{id} C(n) \cong 
\hat{\delta}_{j,n}^*C(A_0 \to \smdots \to A_j 
\longrightarrow A_{j+2} \to \smdots \to A_0)$.
%
\subsection{Generating codegeneracies 
  ${\sigma_{i,n}}_!$ for $n \in \mathbb{N}$, 
  $0 \leq i \leq n$}
From Example \ref{eg:pb4}, we know that $C(n) 
\cong \hat{\sigma}_{i,n}^*C(A_0 \to \smdots \to A_i 
\to A_i \to \smdots \to A_0)$. 
So, define
$\sigma_{i,n!}:C(n)
\xrightarrow{id} C(n) \cong 
\hat{\sigma}_{i,n}^*C(A_0 \to \smdots \to A_i 
\to A_i \to \smdots \to A_0)$.
%
\subsection{Generating rotations ${\tau_n}_!$}
\subsubsection{$n=0$}
Let ${\tau_0}_! = id: C(0) 
\xrightarrow{id} C(0) \cong 
id^*C(0) \cong \hat{\tau}_0^*C(0)$.
%
\subsubsection{$n=1$}
We want to define a map of dg comodules over 
$B(1)$
$$
{\tau_1}_!: 
C(1):= C(A_0 \to A_1 \to A_0) \to 
\hat{\tau}_1^*C(A_1 \to A_0 \to A_1).
$$
Example \ref{eg:pb5} describes the structure of 
$\hat{\tau}_1^*C(A_1 \to A_0 \to A_1)$, 
which is quasi-cofree over $B(1)$.
So, we can define ${\tau_1}_!$ by 
giving maps from $C(1)$ to the cogenerators of 
$\hat{\tau}_1^*C(A_1 \to A_0 \to A_1)$ and 
checking that the corresponding map of comodules 
commutes with the differentials. 

More explicitly, 
for $f = (f_{0,0}, f_{1,0}) \in Obj(B(1))$, we will 
give $k$-linear maps
\begin{align*}
\upsilon^f: C(1)^\bullet(f) 
& \to 
C_{-\bullet}(A_1, _{f_{0,0}f_{1,0}} {A_1}_{id})\\
(\phi_{0,1}\smdots \phi_{0,n_0}|
 \phi_{1,1}\smdots \phi_{0,n_1}| \alpha )
&\mapsto
\upsilon^f_{n_0,n_1}(\phi_{0,1}\smdots \phi_{0,n_0}|
 \phi_{1,1}\smdots \phi_{1,n_1}| \alpha ).
\end{align*}
Then, we lift $\{\upsilon_f | f \in Obj(B(1))\}$ 
to a map of comodules in the standard way: 
\begin{equation}
\label{eq:colift}
\begin{split}
& \phantom{{}={}}
{\tau_{1!}}_f 
  (\phi_{0,1}\smdots \phi_{0,n_0}|
  \phi_{1,1}\smdots \phi_{1,n_1}| \alpha )\\
&= \sum \limits_{\substack{
    0 \leq k_0 \leq n_0\\
    0 \leq k_1 \leq n_1}}
\phi_{0,1}\smdots \phi_{0,k_1}|\phi_{1,1} \smdots \phi_{1,k_0}|
\upsilon^{f_{0,k_1}, f_{1,k_0}}_{n_0-k_0,n_1-k_1} 
  (\phi_{0,k_0+1}\smdots \phi_{0,n_0}|
  \phi_{1,k_1+1}\smdots \phi_{1,n_1}| \alpha )
\end{split}
\end{equation}
(see Figure \ref{fig:phi|alpha} for notation). 
Finally, we will check by direct computation that 
${\tau_1}_!$ defined as such commutes with the 
differentials. To make the exposition smooth, 
all of this is done in Appendix Proposition 
\ref{prop:c1}.

\subsubsection{$n>1$}
For $n>1$, we define ${\tau_n}_!$ by 
pulling back ${\tau_1}_!$ along $\delta_{0,*}$'s 
as follows:
\begin{align*}
{\tau_n}_!: C(n) 
&\cong 
(\widehat{\delta_{0,2}\circ\smdots\circ\delta_{0,n}})^*
  C(A_0 \to A_n \to A_0)\\
&\xrightarrow{
  (\widehat{\delta_{0,2}\circ\smdots\circ\delta_{0,n}})^*
  {\tau_1}_!}
(\widehat{\delta_{0,2}\circ\smdots\circ\delta_{0,n}})^*
  \hat{\tau}_1^*C(A_n \to A_0 \to A_n)\\
&\cong
(\widehat{\tau_1\circ\delta_{0,2}\circ\smdots\circ
  \delta_{0,n}})^*C(A_n \to A_0 \to A_n)\\
&\cong
(\widehat{\delta_{1,2}\circ\smdots\circ\delta_{1,n}
  \circ \tau_n})^*C(A_n \to A_0 \to A_n)\\
&\cong
\hat{\tau}_n^*
  (\widehat{\delta_{1,2}\circ\smdots\circ\delta_{1,n}})^*
  C(A_n \to A_0 \to A_n)\\
&\cong 
\hat{\tau}_n^*C(A_n \to A_0 \to \smdots \to A_n).  
\end{align*}
%
\subsection{Extra coboundary 
  ${\delta_{n,n}}_!$ for 
  $n \in \mathbb{N}$}
In $\Lambda$, we have $\delta_{n,n} = 
\delta_{0,n}\tau_n$, so we define 
\begin{align*}
{{\delta}_{n,n}}_! : C(n)
& \xrightarrow{{\tau_n}_!}
\hat{\tau}_n^*
  C(A_n \to A_0 \to \smdots \to A_n)\\
& \xrightarrow{\hat{\tau}_n^*{\delta_{0,n}}_!}
\hat{\tau}_n^*\hat{\delta}_{0,n}^*
  C(A_n \longrightarrow A_1 \to \smdots \to A_n) \\
&\cong
(\widehat{\delta_{0,n}\tau_n})^*
  C(A_n \longrightarrow A_1 \to \smdots \to A_n)\\
&\cong
\hat{\delta}_{n,n}^*
C(A_n \longrightarrow A_1 \to \smdots \to A_n).
\end{align*}
%
\subsection{Extra codegeneracy 
  ${\sigma_{n+1,n}}_!$ for 
  $n \in \mathbb{N}$}
In $\Lambda$, we have $\sigma_{n+1,n} = 
\tau_{n+1}^{n+1} \sigma_{0,n}$, so we define 
\begin{align*}
{{\sigma}_{n+1,n}}_! : C(n)
& \xrightarrow{\sigma_{0,n!} = id}
\hat{\sigma}_{0,n}^*
  C(A_0 \to A_0 \to \smdots \to A_n \to A_0)\\
& \xrightarrow{\hat{\sigma}_{0,n}^*(\tau_{n+1!})}
\hat{\sigma}_{0,n}^*\hat{\tau}_{n+1}^*
  C(A_n \to A_0 \to A_0 \to \smdots \to A_n) \\
& \to \smdots \\
& \xrightarrow{\hat{\sigma}_{0,n}^*
  \hat{\tau}_{n+1}^{*n}(\tau_{n+1!})}
  \hat{\sigma}_{0,n}^*\hat{\tau}_{n+1}^{*n+1}
  C(A_0 \to \smdots \to A_n \to A_0 \to A_0) \\
&\cong
(\widehat{\tau_{n+1}^{n+1} \sigma_{0,n}})^*
  C(A_0 \to \smdots \to A_n \to A_0 \to A_0)\\
&\cong
\hat{\sigma}_{n+1,n}^*
C(A_0 \to \smdots \to A_n \to A_0 \to A_0).
\end{align*}

%
\chapter{A homotopically sheafy-cyclic object}
	\section{Motivation of this chapter}
In Section \ref{sec:cyclic_B(n)}, we gave a 
sheafy-cyclic object in dg cocategories. 
We would like to extend that construction 
to a sheafy-cyclic object in the 
category of dg cocategories with a dg 
comodule. Namely, we would like to give 
a functor from $\chi$ to $\mathcal{D}$ 
where $\mathcal{D}$ the following 
category:
\begin{align*}
Obj(\mathcal{D}) 
&= 
\{(B,C) |
  \textrm{B is a dg cocategory, 
  C is a dg comodule over B}\} \\
\mathcal{D}((B_1, C_1), (B_0, C_0))  
&= 
\{(f, f_!) | f:B_1 \to B_0 
  \textrm{ is a functor,}\\
& \phantom{{}=[(f, f_!)]{}}  
  f_!:C_1 \to f^*C_0 
  \textrm{ is a map of dg comodules over }
  B_1\}
\end{align*}
\begin{align*}  
\mathcal{D}((B_2, C_2), (B_1, C_1)) \times  
\mathcal{D}((B_1, C_1), (B_0, C_0))
&\xrightarrow{composition}
\mathcal{D}((B_2, C_2), (B_0, C_0))\\
(f,f_!) \times (g, g_!)
&\mapsto
(gf, f^*(g_!)\circ f_!)
\end{align*}
(Proposition \ref{prop:pbs_compose} implies 
that composition in $\mathcal{D}$ is 
associative.)

In Section \ref{sec:shriek_maps}, we 
gave maps $\lambda_!: C(\mathcal{A}) 
\to \hat{\lambda}^*C(\lambda \mathcal{A})$ 
where $\lambda$ is a generating 
morphism in $\Lambda$ that induces a 
morphism in $\chi$ with source $\mathcal{A} 
\in Obj(\chi)$. Ideally,  
$(\hat{\lambda}, \lambda_!)$ would give a functor 
$\chi \to \mathcal{D}$, however, 
the $\lambda_!$'s only respect 
composition up to homotopy. Fortunately, 
most of the compositions are respected 
and, for the ones that are not, we 
have explicit homotopies. Our homotopies 
commute with the composable 
$\lambda_!$'s so that no higher homotopies 
are needed.

In this chapter, we will show which compositions 
are respected on the nose and which ones 
need homotopies. We will then give these homotopies 
and show that no higher homotopies are 
needed. These sections are the computational 
heart of this thesis. Finally, we will 
repackage this ``functor up to homotopy'' 
in more abstract terms.
	\section{Homotopies}
Here, we will show that the maps 
of dg comodules given in Section 
\ref{sec:shriek_maps} satisfy the 
relations in $\Lambda$ (Equation 
\ref{eqn:cyclic_relations}) up to 
homotopy. More precisely, we will 
show that
\begin{subequations}\label{eq:strict}
\begin{align}
\begin{split}\label{eq:strict_1}
\hat{\delta}_{j,n}^*(\delta_{i,n-1!}) \circ \delta_{j,n!} 
&= 
\hat{\delta}_{i,n}^*(\delta_{j-1,n-1!}) \circ \delta_{i,n!} 
  \quad 0 \leq i < j \leq n-1 \\
\hat{\sigma}_{j,n}^*(\sigma_{i,n+1!}) \circ \sigma_{j,n!} 
&= 
\hat{\sigma}_{i,n}^*(\sigma_{j+1,n+1!}) \circ \sigma_{i,n!}
  \quad 0 \leq i \leq j \leq n \\
\hat{\sigma}_{i,n}^*(\delta_{j,n+1!}) \circ \sigma_{i,n!} 
&= 
  \begin{cases}
    \hat{\delta}_{j-1,n}^*(\sigma_{i,n-1!}) \circ \delta_{j-1,n!} 
      & 0 \leq i < j \leq n\\
    id & j = i, i-1\\
    \hat{\delta}_{j,n}^*(\sigma_{i-1,n-1!}) \circ \delta_{j,n!} 
      & \quad 0 \leq j < i-1 \leq n-1
   \end{cases}
\end{split}\\
\begin{split}\label{eq:strict_2}
\hat{\sigma}_{i,n}^*(\tau_{n+1!}) \circ \sigma_{i,n!} 
&= 
\hat{\tau}_n^*(\sigma_{i+1,n!}) \circ \tau_{n!}
  \quad 0 \leq i \leq n-1\\
\hat{\delta}_{j,n}^*(\tau_{n-1!}) \circ \delta_{j,n!} 
&= 
\hat{\tau}_n^*(\delta_{j+1,n!}) \circ \tau_{n!}
  \quad 0 \leq j \leq n-1
\end{split}\\
\begin{split}\label{eq:strict_3}
(\widehat{\tau_{1}\sigma_{0,0}})^*
  (\delta_{0,1!}) \circ
  \hat{\sigma}_{0,0}^*(\tau_{1!}) \circ 
  \sigma_{0,0!}
= id
\end{split} 
\end{align}
\end{subequations}
and 
\begin{subequations} \label{eq:weak}
\begin{equation} \label{eq:weak_delta}
\hat{\tau}_n^{*2}(\delta_{0,n!}) \circ 
  \hat{\tau}_n^*(\tau_{n!}) \circ \tau_{n!} 
\simeq 
\hat{\delta}_{n-1,n}^*(\tau_{n-1!}) \circ \delta_{n-1,n!}
\end{equation}
\begin{equation} \label{eq:weak_tau}
\hat{\tau}_n^{*n}(\tau_{n!}) \circ \smdots 
  \circ \hat{\tau}_n^*(\tau_{n!}) \circ \tau_{n!}
\simeq id
\end{equation}
\begin{equation} \label{eq:weak_sigma}
\hat{\sigma}_{n,n}^*(\tau_{n+1!}) \circ \sigma_{n,n!}
\simeq
(\widehat{\tau_{n+1}^n\sigma_{0,n}\tau_{n}})^*(\tau_{n+1!})
  \circ \smdots \circ 
  (\widehat{\tau_{n+1}\sigma_{0,n}\tau_{n}})^*(\tau_{n+1!}) \circ
  (\widehat{\sigma_{0,n}\tau_{n}})^*(\tau_{n+1!}) \circ
  \hat{\tau}_n^*(\sigma_{0,n!}) \circ \tau_{n!}
\end{equation}
\end{subequations}
%
\subsection{Showing Equations \ref{eq:strict} hold}
Equation \ref{eq:strict_1} has three relations. 
All of the $\sigma_!$'s and $\delta_!$'s in Equation 
\ref{eq:strict_1} are identity maps, so it's clear that 
these relations hold.

Equation \ref{eq:strict_2} has two relations. 
To show that the first one holds, we have
\begin{align*}
\hat{\sigma}_{i,n}^*(\tau_{n+1!}) \circ \sigma_{i,n!} 
&= 
\hat{\sigma}_{i,n}^*(
  (\widehat{\delta_{0,2}\smdots\delta_{0,n+1}})^*
  (\tau_{1!})) \circ \sigma_{i,n!}
  \quad \textrm{definition of }\tau_{n+1!}\\
&= 
(\widehat{\delta_{0,2}\smdots\delta_{0,n+1}\sigma_{i,n}})^*
  (\tau_{1!}) \circ \sigma_{i,n!}
  \quad \textrm{Proposition } \ref{prop:pbs_compose}\\
&= 
(\widehat{\delta_{0,2}\smdots\delta_{0,n}})^*
  (\tau_{1!}) \circ \sigma_{i,n!}\\
&= 
\tau_{n!} \circ \sigma_{i,n!}
  \quad \textrm{definition of }\tau_{n!}\\
&= 
\tau_{n!} \circ id  
  = id \circ \tau_{n!} \\
&= 
\hat{\tau}_n^*(\sigma_{i+1,n!}) \circ \tau_{n!}.
\end{align*}
To show that the second relation holds, the 
reasoning is the same as above. We have
\begin{align*}
\hat{\delta}_{j,n}^*(\tau_{n-1!}) \circ \delta_{j,n!} 
&= 
\hat{\delta}_{j,n}^*
  ((\widehat{\delta_{0,2}\smdots\delta_{0,n-1}})^*
  (\tau_{1!})) \circ \delta_{j,n!} \\
&=
(\widehat{\delta_{0,2}\smdots\delta_{0,n-1}\delta_{j,n}})^*
  (\tau_{1!}) \circ \delta_{j,n!} \\
&=
\tau_{n!} \circ \delta_{j,n!}\\
&= 
\tau_{n!} \circ id   
  = id \circ \tau_{n!}\\
&=
\hat{\tau}_n^*(\delta_{j+1,n!}) \circ \tau_{n!}. 
\end{align*} 

Equation \ref{eq:strict_3} has one relation. 
The only map in this relation that is not 
defined to be an identity map is $\hat{\sigma}_{0,0}^*
(\tau_{1!})$. We will compute this map and 
show that it is also an identity. 
Let $(\phi_{0,1}\smdots \phi_{0,k_0}|\alpha)
\in C(A_0 \to A_0)$ (see Figure \ref{fig:phi|alpha} 
for notation). By Proposition \ref{prop:compute_pb},
\begin{align*}
C(A_0 \to A_0)
&\xrightarrow{\cong} 
\hat{\sigma}_{0,0}^*C(A_0 \to A_0 \to A_0)\\
(\phi_{0,1}\smdots \phi_{0,k_0}|\alpha)
&\mapsto
\sum \limits_{0 \leq r_0 \leq k_0}
  (\phi_{0,1}\smdots \phi_{0,r_0}) \otimes 
  (1 | \phi_{0,r_0+1}\smdots \phi_{0,k_0}|\alpha).
\end{align*}
Applying $\hat{\sigma}_{0,0}^*(\tau_{1!})$
to the righthand side, we have
\begin{align*}
\hat{\sigma}_{0,0}^*C(A_0 \to A_0 \to A_0)
&\xrightarrow{\hat{\sigma}_{0,0}^*(\tau_{1!})}
\hat{\sigma}_{0,0}^*\hat{\tau}_1^*C(A_0 \to A_0 \to A_0)\\
\sum \limits_{0 \leq r_0 \leq k_0}
  (\phi_{0,1}\smdots \phi_{0,r_0}) \otimes 
  (1 | \phi_{0,r_0+1}\smdots \phi_{0,k_0}|\alpha)
&\mapsto
\sum \limits_{0 \leq r_0 \leq s_0 \leq k_0}
  (\phi_{0,1}\smdots \phi_{0,r_0}) \otimes \\
&\phantom{{}\mapsto \sum{}}  
  (\phi_{0,r_0+1}\smdots \phi_{0,s_0}|1|
  \tau_1(1|\phi_{0,s_0+1}\smdots\phi_{0,k_0}|\alpha)).
\end{align*}
The righthand side above is equal to
\begin{align*}
&\phantom{{}={}}
\sum \limits_{0 \leq r_0 \leq s_0 \leq k_0}
  (\phi_{0,1}\smdots \phi_{0,r_0}) \otimes 
  (\phi_{0,r_0+1}\smdots \phi_{0,s_0}|1|
  \tau_1(1|\phi_{0,s_0+1}\smdots\phi_{0,k_0}|\alpha))\\
&=  
\sum \limits_{0 \leq r_0 \leq s_0 \leq k_0}
  (\phi_{0,1}\smdots \phi_{0,r_0}) \otimes 
  (\phi_{0,r_0+1}\smdots \phi_{0,s_0}|1|
  \upsilon_{0,k_0-s_0}(1|\phi_{0,s_0+1}\smdots\phi_{0,k_0}|\alpha))\\
&\phantom{{}=\sum\sum{}}  
  \quad \textrm{(see Proposition \ref{prop:c1} for }
  \upsilon_{\cdot,\cdot})\\
&=  
\sum \limits_{0 \leq r_0 \leq k_0}
  (\phi_{0,1}\smdots \phi_{0,r_0}) \otimes 
  (\phi_{0,r_0+1}\smdots \phi_{0,k_0}|1|\alpha)
  \quad \quad (\upsilon_{0,>0}=0)\\
&\in
\hat{\sigma}_{0,0}^*\hat{\tau}_1^*C(A_0 \to A_0 \to A_0).
\end{align*}
Finally, applying Proposition 
\ref{prop:compute_pb} again, we have
\begin{align*}
\hat{\sigma}_{0,0}^*\hat{\tau}_1^*C(A_0 \to A_0 \to A_0)
&\xrightarrow[\cong]{\textrm{project onto cogenerators}} 
C(A_0 \to A_0)\\
\sum \limits_{0 \leq r_0 \leq k_0}
  (\phi_{0,1}\smdots \phi_{0,r_0}) \otimes 
  (\phi_{0,r_0+1}\smdots \phi_{0,k_0}|1|\alpha)
&\mapsto
(\phi_{0,1}\smdots \phi_{0,k_0}|\alpha).
\end{align*}
So, we've shown 
$$
C(A_0 \to A_0) \cong 
\hat{\sigma}_{0,0}^*C(A_0 \to A_0 \to A_0)
\xrightarrow{\hat{\sigma}_{0,0}^*(\tau_{1!})}
\hat{\sigma}_{0,0}^*\hat{\tau}_1^*C(A_0 \to A_0 \to A_0)
\cong C(A_0 \to A_0)
$$
is the identity map.
% 
\subsection{Showing Equations \ref{eq:weak} hold}
First, we will show that Equation 
\ref{eq:weak_delta} holds. Then, we will 
show that Equation 
\ref{eq:weak_delta} implies Equation 
\ref{eq:weak_tau}, and Equation \ref{eq:weak_tau} 
implies Equation \ref{eq:weak_sigma}.

\subsubsection{Showing Equation \ref{eq:weak_delta} holds}
For $n=1$, eliminating the identity 
maps reduces Equation \ref{eq:weak_delta} to:
\begin{align*} 
\hat{\tau}_1^*(\tau_{1!}) \circ \tau_{1!} 
\simeq id.
\end{align*}
We prove the above in Appendix Proposition 
\ref{prop:c2}. (In the appendix, 
$\tau_{1!} = \Upsilon_{A_0,A_1}$, 
$\hat{\tau}_1^*(\tau_{1!}) = \Upsilon_{A_1,A_0}$, 
and the homotopy is denoted $B$.)

For $n=2$, eliminating the identity 
maps reduces Equation \ref{eq:weak_delta} to:
\begin{align*} 
\hat{\tau}_1^*(\tau_{1!}) \circ \tau_{1!} 
\simeq id.
\end{align*}
We prove the above in Appendix Proposition 
\ref{prop:c4}. (In the appendix,...)

Now, we will reduce Equation 
\ref{eq:weak_delta} for $n\geq 2$ to the case 
$n=2$. We have
\begin{align*}
\hat{\tau}_n^{*2}(\delta_{0,n!}) \circ 
  \hat{\tau}_n^*(\tau_{n!}) \circ \tau_{n!}   
&=  
\hat{\tau}_n^{*2}(\delta_{0,n!}) \circ 
  \hat{\tau}_n^*(
  (\widehat{\delta_{0,2}\smdots\delta_{0,n}})^*(\tau_{1!})) 
  \circ \tau_{n!} \\
&=   
\hat{\tau}_n^{*2}(\delta_{0,n!}) \circ 
  (\widehat{\delta_{0,2}\smdots\delta_{0,n}\tau_n})^*(\tau_{1!}) 
  \circ \tau_{n!} \\
&=
\hat{\tau}_n^{*2}(\delta_{0,n!}) \circ 
  (\widehat{\delta_{0,2}\tau_2\delta_{0,3}\smdots\delta_{0,n}})^*(\tau_{1!}) 
  \circ \tau_{n!} \\
\hat{\tau}_n^{*2}(\delta_{0,n!}) \circ 
  \hat{\tau}_n^*(\tau_{n!}) \circ \tau_{n!}   
\simeq 
\hat{\delta}_{n-1,n}^*(\tau_{n-1!}) \circ \delta_{n-1,n!}
\end{align*}
%
\subsubsection{Showing Equation \ref{eq:weak_tau} holds}
We prove this by induction on $n$. For $n=1$, 
Equation \ref{eq:weak_tau} is the same as 
Equation \ref{eq:weak_delta}, which we 
established in the previous section. Now, 
assume that Equation \ref{eq:weak_tau} holds 
for $N=n-1$. We want to show that Equation 
\ref{eq:weak_tau} holds for $N=n$.
%
\subsubsection{Showing Equation \ref{eq:weak_sigma} holds}
	\section{Higher Homotopies}
In this section, we show 
that no higher homotopies are needed. 
First, we will summarize the 
maps of comodules that we have 
already given. Let 
$\mathcal{A} \in Obj(\chi)$ and 
$\lambda$ be a generating morphism in $\Lambda$ 
that induces a morphism in $\chi$ with 
source $\mathcal{A}$. We have 
\begin{align*}
\lambda_!: C(\mathcal{A})
&\to 
\hat{\lambda}^*C(\lambda\mathcal{A})
  \quad \textrm{maps of dg comodules}\\
\sigma_{\mathcal{A}!}:
  C(\mathcal{A}) 
&\to 
\tau^{*2}C(\tau^2\mathcal{A})
  \quad \textrm{deg -1 map of comodules.}
\end{align*}
where
\begin{align*}
\sigma_{(A_0 \to A_1 \to A_0)!}
&= 
B \textrm{ given in Appendix 
  Proposition \ref{prop:c2}}\\
\sigma_{(A_0 \to A_1 \to A_2 \to A_0)!}
&= 
\mathcal{B} \textrm{ given in Appendix 
  Proposition \ref{prop:c4}}\\
\sigma_{(A_0 \to \smdots A_n \to A_0)!}
&= 
(\widehat{\delta_{0,3}\smdots\delta_{0,n}})^*
  \mathcal{B} \textrm{ for } n>2.\\  
\end{align*}
(We will write $\sigma_!$ instead of 
$\sigma_{\mathcal{A}!}$ when the source 
is clear or doing so unnecessarily 
encumbers the exposition.) 
Using the constructions we've given, 
a typical map between comodules is 
one that is freely generated by composable 
pullbacks of $\lambda_!$'s and $\sigma_!$'s. 
First, we will establish that there are 
no such maps of degree $\geq2$. Suppose 
we have a map $\eta_!$ of degree 
$\geq2$. Then, $\eta_!$ must contain at 
least two (pullbacks of) some $\sigma_!$'s. 
Each $\sigma_!$ involves inserting a 1 
into the first slot of the Hochschild 
chains component (see Equations 
\ref{eq:def_sigma}, \ref{eq:def_sigma2}). 
However, since we are working with 
reduced chains, any chain with two or 
more 1's is equal to zero. So, 
$\eta_! = 0$.

Since there are no maps of degree 
$\geq2$, we know from the classical 
theory of $A_\infty$ algebras that 
the only need for higher homotopies 
will arise from the following 
situation: We have two degree 0 
maps (for $n\geq2$)
$$
\xymatrixcolsep{5pc}
\xymatrix{
C(A_0 \to \smdots \to A_n \to A_0) 
 \ar@/^1pc/[d]^{(\widehat{\delta_{n-2,n-1}\delta_{n-1,n}})^*
   \tau_{n-2!}}
 \ar@/_1pc/[d]_{\hat{\tau}_n^{*2} \tau_{n!}\circ
   \hat{\tau}_n^*\tau_{n!}\circ \tau_{n!}}\\
C(A_{n-2} \to A_{n-1} \to A_n \to 
A_0 \to \smdots \to A_{n-2}).
}
$$

$$
\xymatrix{
C(A_0 \to \smdots \to A_n \to A_0)
  \ar[r]^{\tau_{n!}}
  \ar[d]_{\hat{\delta}_{n-1}^*\tau_{n-1!}} 
  \ar[rd]^{\hat{\tau}_n^{*2} \tau_{n!}\circ 
  \hat{\tau}_n^*\tau_{n!}\circ \tau_{n!}}
& C(A_n \to A_0 \to \smdots \to A_n)
  \ar[d]^{(\widehat{\delta_{n-1}\tau_n})^*
  \tau_{n-1!}VV}\\
C(A_{n-1} \to A_n \to A_0 \to \smdots \to A_{n-1})
  \ar[r]_{\hat{\tau}_n^{*2}\tau_{n!}} 
&C(A_{n-2} \to A_{n-1} \to A_n \to A_0 \to \smdots \to A_{n-2})
}
$$
In the diagram above, the diagonal map, 
$\hat{\tau}_n^{*2} \tau_{n!}\circ 
\hat{\tau}_n^*\tau_{n!}\circ \tau_{n!}$, is 
homotopic to the upper right map, 
$(\widehat{\delta_{n-1}\tau_n})^* \tau_{n-1!}
\circ \tau_{n!}$, as well as to
the bottom left map, 
$\hat{\tau}_n^{*2}\tau_{n!} 
\circ \hat{\delta}_{n-1}^*\tau_{n-1!}$. 
These two homotopies give two 
degree -1 maps from 
$C(A_0 \to \smdots \to A_n \to A_0)$ 
to $C(A_{n-2} \to A_{n-1} \to A_n \to 
A_0 \to \smdots \to A_{n-2})$. 
If these two homotopies are not 
equal, then 

via $\tau_n^*
\sigma_{A_0 \to \smdots \to A_0!}\circ \tau_{n!}$,

, 
via $\tau_n^{*2} \tau_{n!} \circ 
\sigma_{A_0 \to \smdots \to A_0!}$


\chapter{Take Cobar: A homotopically cyclic object in $\infty$-categories in dg categories
with a trace functor}


% The command \includeonly above allows to include a selected 
% set of chapters only.

% Uncomment the 'singlespace' environment and '\bibsep' command
% if needed - some bibliographic styles overide the definition
% of 'thebibliography' in nuthesis.cls
%
\begin{singlespace}
%\bibsep 12pt
\clearpage\phantomsection % needed for hyperlikns to work correctly
\begin{thebibliography}{xxx}

\bibitem{label1} A bibliographic item.  A bibliographic item.  A
bibliographic item.  A bibliographic item.
\bibitem{label2} Another bibliographic item.  
\bibitem{label3} Yet another bibliographic item.  
% The usages of \bibitem and \cite{..} are 
% explained in Section 4.3 (page 73) of % LaTeX manual.
% Or you may use BibTeX.
\end{thebibliography}
\end{singlespace}

% (The following suggested by Francisco Iacobelli - 5/11/2010)
% In case, you want to use BibTeX, you should replace (or comment)
% the bibliography environment.
% Instead uncomment the following 3 lines and replace <bib file>
% with your .bib file:
% \renewcommand\refname{\begin{centering}References\end{centering}}
% \bibliography{<bib file>}
% \bibliographystyle{acm} %or another suitable style.



\appendix		% Appendix begins here (optional).

%\appendices	        % If more than one appendix chapters,
				% use appendices instead of appendix
\chapter{Connes cyclic category, $\Lambda$}\label{chap:lambda}
Here, we give generators and relations for 
the cyclic category, $\Lambda$. None of this 
is new, but we do it to establish notation 
for the rest of the paper.

$\Lambda$ has objects $\{[n]: n \in \mathbb{N}\}$ 
and generating morphisms:
\begin{equation} \label{eqn:cyclic_generators}
\begin{split}
\textrm{rotations } \tau_n:[n] \to [n], \\ 
\textrm{coboundaries } \delta_{j,n}: [n] \to [n-1], 0 \leq j \leq n-1, \\ 
\textrm{codegeneracies } \sigma_{i,n}:[n] \to [n+1], 1 \leq i \leq n+1
\end{split}
\end{equation}
subject to relations:
\begin{equation}\label{eqn:cyclic_relations}
\begin{split}
\delta_{i,n-1} \delta_{j,n} &= \delta_{j-1,n-1} \delta_{i,n} 
  \quad 1 \leq i \leq j \leq n-1 \\
\sigma_{i,n+1} \sigma_{j,n} &= \sigma_{j+1,n+1} \sigma_{i,n}
  \quad 1 \leq i < j \leq n+1 \\
\delta_{j,n+1}\sigma_{i,n} &= 
  \begin{cases}
    \sigma_{i,n-1}\delta_{j-1,n} 
      & 0 \leq i < j \leq n-1\\
    id & j = i, i-1, 0 \leq j \leq n\\
    \sigma_{i-1,n-1}\delta_{j,n} 
      & \quad 0 \leq j < i \leq n
   \end{cases}\\
\tau_{n+1}\sigma_{i,n} &= \sigma_{i+1,n}\tau_n
  \quad 1 \leq i \leq n\\
\tau_{n+1}^2 \sigma_{n+1,n} &= \sigma_{1,n} \tau_n \\
\tau_{n-1}\delta_{i,n} &= \delta_{i+1,n}\tau_n
  \quad 0 \leq i \leq n-1\\
\delta_{0,n}\tau_n^2 &= \tau_{n-1}\delta_{n-1,n}. 
\end{split}
\end{equation}

Some definitions of $\Lambda$ include an extra 
coboundary $\delta_{n,n}$ and codegeneracy 
$\sigma_{0,n}$ even though the morphisms we have given 
are sufficient to generate $\Lambda$. However, 
we will still refer to these extra 
morphisms, and in terms of our generators, they are 
$\delta_{n,n} := \delta_{0,n}\tau_n$ and 
$\sigma_{0,n} := \tau_{n+1} \sigma_{n+1,n}$.
\chapter{Computations}
In this appendix, we give the 
computational propositions 
needed to establish the 
homotopically cyclic structure 
on dg comodules. 
% \begin{figure} \label{fig:upsilon}
% \centerline{\xymatrix{
% A_0 \ar@/^5pc/[r]^{f_0} 
% \ar@/^2pc/[r]^{\big\Downarrow \phi_1}_{f_1} 
% \ar@/_2pc/[r]^{\substack{\vdots\\ f_n}}
% \ar@/_4pc/[rr]_{id}^{\substack{\alpha \\ \\ \\ \\ }}
% & A_1 \ar@/^5pc/[r]^{g_0} 
% \ar@/^2pc/[r]^{\big\Downarrow \psi_1}_{g_1} 
% \ar@/_2pc/[r]^{\substack{\vdots\\ g_m}}
% & A_0
% }
% $\overset{\Upsilon}\longrightarrow$
% \xymatrix{
% A_1 \ar@/^5pc/[r]^{g_0} 
% \ar@/^2pc/[r]^{\big\Downarrow \psi_1}_{g_1} 
% \ar@/_2pc/[r]^{\substack{\vdots\\ g_m}}
% \ar@/_4pc/[rr]_{id}^{\substack{\alpha \\ \\ \\ \\ }}
% & A_0 \ar@/^5pc/[r]^{f_0} 
% \ar@/^2pc/[r]^{\big\Downarrow \phi_1}_{f_1} 
% \ar@/_2pc/[r]^{\substack{\vdots\\ f_n}}
% & A_1
% }}
% \caption{A picture of the domain and target of $\Upsilon$}
% \end{figure}

\begin{prop}
\label{prop:c1}
Let $\hat{\tau}_1: 
B(A_0 \to A_1 \to A_0) 
\longrightarrow B(A_1 \to A_0 \to A_1)$ 
be as defined in Section 
\ref{sec:cyclic_B(n)}.
Recall from Example \ref{eg:pb5} that 
$\hat{\tau}_1^*C(A_1 \to A_0 \to A_0)
\cong C(A_1 \to A_0 \to A_1)$ 
as complexes. Define a map 
$$
\Upsilon_{A_0,A_1}: C(A_0 \to A_1 \to A_0)
\to \hat{\tau}_1^*C(A_1 \to A_0 \to A_1)
$$
of comodules over 
$B(A_0 \to A_1 \to A_0)$ by mapping into 
cogenerators as follows:
\begin{align}\label{eq:define_upsilon}
\upsilon^{f_0, g_0}: C(A_0 \to A_1 \to A_0)(f_0,g_0) 
&\to
\hat{\tau}_1^*C(A_1 \to A_0 \to A_1)(g_0,f_0)\\
&\cong 
C(A_1 \to A_0 \to A_1)(g_0,f_0)\\
&\xrightarrow[cogenerators]{\textrm{project onto}}
C_{-\bullet}(A_1, _{f_0g_0}{A_1}_{id})\\
\upsilon_{n,m}^{f_0,g_0} 
(\vec{\phi} | \vec{\psi} | \alpha) = 
& \sum_{\substack{I_1I_2 = \{2,\cdots,n\} \\
                          \textrm{as ordered sets}}}
  \phi_1(\lambda(\vec{\psi})\lambda(\vec{\phi_{I_2}})\cdot \mathfrak{a}_3, a_0, \mathfrak{a}_1) \otimes \lambda(\vec{\phi_{I_1}}) \cdot \mathfrak{a}_2 \\
&\phantom{{}move{}}
\bigg( + f_0a_0 \otimes \lambda(\vec{\phi}) \mathfrak{a}_1 
  \; \; \; \; if \; \; m = 0 \bigg).
\end{align}
Then, $\Upsilon_{A_0,A_1}: C(A_0 \to A_1 \to A_0)
\to \hat{\tau}^*C(A_1 \to A_0 \to A_1)$ 
is a map of dg comodules over 
$B(A_0 \to A_1 \to A_0)$.
\end{prop}
%
\begin{proof}
We must show: (1) $\Upsilon$ is a map of comodules, and 
(2) $\Upsilon$ commutes with the differentials. (In this 
proof, we drop the subscripts and write 
$\Upsilon := \Upsilon_{A_0, A_1}$.)

(1) This proof is standard for cofree comodules. 
Let ($\vec{\phi} | \vec{\psi} | \alpha$) be as 
in the statement of the proposition. We want to 
show that $\Upsilon$ commutes with the coproducts. 
On one hand,
\begin{align*}
&\phantom{{}={}}
[(id_B \otimes \Upsilon) \circ 
  \Delta_{C(A_0 \to A_1 \to A_0)}] 
  ( \vec{\phi} | \vec{\psi} | \alpha ) \\
&= [id_B \otimes \Upsilon]
	\big( \sum_{\substack{I_1I_2 = \{1,2,\cdots,n\} \, \textrm{and} \\ 
						  J_1J_2 = \{1,2,\cdots,m\} \\
				          \textrm{as ordered sets}}} 
    \epsilon_{I_2,J_1}\cdot 
    (\vec{\phi}_{I_1} | \vec{\psi}_{J_1}) \otimes (\vec{\phi}_{I_2} | \vec{\psi}_{J_2} | \alpha) \, \big) \\
&= \sum_{\substack{I_1I_2I_3 = \{1,2,\cdots,n\} \, \textrm{and} \\ 
				   J_1J_2J_3 = \{1,2,\cdots,m\} \\
				   \textrm{as ordered sets}}} 
    \epsilon_{I_2I_3,J_1}\cdot \epsilon_{I_3,J_2}\cdot
    (\vec{\phi}_{I_1} | \vec{\psi}_{J_1}) \otimes 
    (\vec{\phi}_{I_2} | \vec{\psi}_{J_2}) \otimes 
    \upsilon_{|I_3|,|J_3|}(\vec{\phi}_{I_3} | \vec{\psi}_{J_3} | \alpha). \\
\end{align*}
On the other hand,
\begin{align*}
&\phantom{{}={}}
[\Delta_{\hat{\tau}^*C(A_1 \to A_0 \to A_1)} 
  \circ \Upsilon ]
  ( \vec{\phi} | \vec{\psi} | \alpha ) \\
&= \Delta_{\hat{\tau}^*C(A_1 \to A_0 \to A_1)}
	\big( \sum_{\substack{I_1I_2 = \{1,2,\cdots,n\} \, \textrm{and} \\ 
						  J_1J_2 = \{1,2,\cdots,m\} \\
				          \textrm{as ordered sets}}}
	\epsilon_{I_2,J_1} \cdot 
  (\vec{\phi}_{I_1} | \vec{\psi}_{J_1}) \otimes 
    \upsilon_{|I_2|,|J_2|}(\vec{\phi}_{I_2} | \vec{\psi}_{J_2} | \alpha) \, \big)\\
&= \sum_{\substack{I_1I_2I_3 = \{1,2,\cdots,n\} \, \textrm{and} \\ 
				   J_1J_2J_3 = \{1,2,\cdots,m\} \\
				   \textrm{as ordered sets}}} 
    \epsilon_{I_2I_3,J_1}\cdot \epsilon_{I_3,J_2}\cdot
    (\vec{\phi}_{I_1} | \vec{\psi}_{J_1}) \otimes 
    (\vec{\phi}_{I_2} | \vec{\psi}_{J_2}) \otimes 
    \upsilon_{|I_3|,|J_3|}(\vec{\phi}_{I_3} | \vec{\psi}_{J_3} | \alpha).   				          
\end{align*}
Clearly 
$(id_B \otimes \Upsilon) \circ 
\Delta_{C(A_0 \to A_1 \to A_0)} = 
\Delta_{\hat{\tau}^*C(A_1 \to A_0 \to A_1)} 
\circ \Upsilon$.

(2) We will show that $\Upsilon$ commutes with 
the differentials by direct computation. Since 
$\Upsilon$ is a map of cofree comodules, we only 
need to check that $\pi_1 \circ D(\Upsilon) = 0$ 
where $D(\Upsilon)$ is the differential applied 
to $\Upsilon$ as a linear map between complexes 
and $\pi_1$ denotes projection of a comodule 
onto its cogenerators. More explicitly, we want 
to check that
\begin{equation} \label{eq:upsilon}
\begin{aligned}
&\upsilon_{n, m} ( \tilde{\delta}(\vec{\phi}) | \vec{\psi} | \alpha ) \; + 
\upsilon_{n, m} ( \vec{\phi} | \tilde{\delta}(\vec{\psi}) | \alpha ) \; + 
\upsilon_{n-1, m} ( b^\prime(\vec{\phi}) | \vec{\psi} | \alpha ) \; + 
\upsilon_{n, m-1} ( \vec{\phi} | b^\prime(\vec{\psi}) | \alpha ) \; + \\
&\upsilon_{n, m} ( \vec{\phi} | \vec{\psi} | b(\alpha) ) \; + 
b \circ \upsilon_{n, m} ( \vec{\phi} | \vec{\psi} | \alpha ) \; + \\
& \sum \limits_{\substack{
  I_1I_2 = \{1,\smdots,n\}\\ 
  \textrm{as ordered sets}}}
  \epsilon_{I_2,\{1,\smdots,m-1\}}\cdot
  \upsilon_{|I_1|, m-1}(\vec{\phi}_{I_1} | \vec{\psi}_{\{1,\cdots, m-1\}} | \psi_{m} \{\vec{\phi}_{I_2}\} \cdot \alpha ) \; + \\
& \sum \limits_{\substack{
  J_1J_2 = \{1,\smdots,m\}\\ 
  \textrm{as ordered sets}}}
  \epsilon_{\{2,\smdots,n\},J_1}\cdot
\phi_1 \{ \psi_{J_1}\}\cdot \upsilon_{n-1, |J_2|}(\phi_{\{2,\cdots , n\}} | \psi_{J_2} | \alpha ) \; + \\  
& \epsilon_{\{n\},\{1,\smdots,m\}}\cdot
  \upsilon_{n-1, m}(\vec{\phi}_{\{1,\cdots, n-1\}} |\vec{\psi} | \phi_{n}\cdot \alpha) \; + \\
& \epsilon_{\{1,\smdots,n\},\{1\}}\cdot
\psi_1\cdot \upsilon_{n,m-1} ( \vec{\phi} | \vec{\psi}_{\{2,\cdots, m\}} | \alpha) \\ 
&= 0.
\end{aligned}
\end{equation}
In Equation \ref{eq:upsilon}, we will call the 
terms in rows 1-2 the ``standard terms'', 
and the terms in rows 3-6 the 
``extra terms''.

We compute the sum of the standard terms. 
In Table \ref{table:t1}, the leftmost column 
lists the expressions that don't cancel in 
the sum of the standard terms, the middle 
column gives the standard term from which 
the expression comes, and the rightmost 
column gives the term (extra or standard) 
that cancels the expression. 

All of the terms in Table \ref{table:t1} 
cancel, so $\Upsilon$ is a map of complexes.
\end{proof}
%
\begin{landscape}
\begin{center}
\begin{table}
  \begin{tabular}{ p{3in} | p{2in} | p{2.5in} }
    \hline
    Expression (Expansion) & 
      \breakcell{Comes from Standard Term\\in Equation \ref{eq:upsilon}} & 
      \breakcell{Cancelling Term\\in Equation \ref{eq:upsilon}} \\ \hline
    
    \breakcell{$f_0\psi_1(\lambda(\vec{\phi}_{I_2}) \mathfrak{a}_3 ) \cdot$\\
    $\phi_1(\lambda(\vec{\psi}_{\{2,\cdots, m\}} \lambda(\vec{\phi}_{I_3}) \mathfrak{a}_4, a_0, \mathfrak{a}_1) \otimes \lambda(\vec{\phi}_{I_1}) \mathfrak{a}_2$} &
    $\upsilon_{n, m} (\delta(\phi_1)\phi_2 \cdots \phi_n | \vec{\psi} | \alpha)$ & 
    $f_0 \psi_1 \cdot \upsilon_{n,m-1} ( \vec{\phi} | \vec{\psi}_{\{2,\cdots, m\}} | \alpha)$ \\ \hline

    \breakcell{$\phi_1( \lambda(\vec{\psi}_{\{1,\cdots, m-1\}}) \lambda(\vec{\phi}_{I_2}) \mathfrak{a}_3,$\\ 
    $\phantom{mo} \psi_{m} ( \lambda(\vec{\phi}_{I_3}) \mathfrak{a}_4) \cdot a_0, \mathfrak{a}_1 ) \otimes \lambda(\vec{\phi}_{I_1}) \mathfrak{a}_2$} &
    $\upsilon_{n, m} (\delta(\phi_1)\phi_2 \cdots \phi_n | \vec{\psi} | \alpha)$ &
    $\upsilon_{|I_1|, m-1}(\vec{\phi}_{I_1} | \vec{\psi}_{\{1,\cdots, m-1\}} | \psi_{m} \{\vec{\phi}_{I_2}\}\cdot \alpha )$ \\ \hline

    \breakcell{$\phi_1( \lambda(\vec{\psi}) \lambda(\vec{\phi}_{I_2}) \mathfrak{a}_3, g_m \phi_n(\mathfrak{a}_4) \cdot a_0, \mathfrak{a}_1)\otimes$\\ 
    $\otimes\, \lambda(\vec{\phi}_{I_1}) \mathfrak{a}_2$} &
    $\upsilon_{n, m} (\delta(\phi_1)\phi_2 \cdots \phi_n | \vec{\psi} | \alpha)$ &
    $\upsilon_{n-1, m}(\vec{\phi}_{\{1, \cdots, n-1\}} | \vec{\psi} | g_m \phi_{n} \cdot \alpha )$ \\ \hline

    \breakcell{$\phi_1( \lambda(\vec{\psi}) \lambda(\vec{\phi}_{I_2}) \mathfrak{a}_2) \cdot f_1(a_0) \otimes \lambda(\vec{\phi}_{I_1}) \mathfrak{a}_1$} &
    $\upsilon_{n, m} (\delta(\phi_1)\phi_2 \cdots \phi_n | \vec{\psi} | \alpha)$ &
    $\phi_1 \cdot \upsilon_{n-1, 0}(\vec{\phi}_{\{2, \cdots, n\}} | \vec{\psi} |\alpha )$ \\ \hline

    \breakcell{$f_0a_0 \cdot \phi_1(\mathfrak{a}_1) \otimes \lambda(\vec{\phi}_{\{1,\cdots,n-1\}}) \mathfrak{a}_2$} &
    \breakcell{$\upsilon_{n, m} (\delta(\phi_1)\phi_2 \cdots \phi_n | \vec{\psi} | \alpha)$ \\ if $\vec{\psi} = 1$} & 
    \breakcell{$b \circ \upsilon_{n, m} (\vec{\phi} | \vec{\psi} | \alpha)$ \\ if $\vec{\psi} =1$} \\ \hline

    \breakcell{$f_0 g_m \phi_n(\mathfrak{a}_2) f_0a_0 \otimes \lambda(\vec{\phi}_{\{1,\cdots,n-1\}}) \mathfrak{a}_1$} &
    \breakcell{$b \circ \upsilon_{n, m} (\vec{\phi} | \vec{\psi} | \alpha)$ \\ if $\vec{\psi} = 1$} &
    \breakcell{$\upsilon_{n-1, m}(\vec{\phi}_{\{1, \cdots, n-1\}} | \vec{\psi} | g_m \phi_{n} \cdot \alpha )$ \\ if $\vec{\psi} = 1$} \\ \hline

    \breakcell{$\phi_1(\lambda(\vec{\psi}) \lambda(\vec{\phi}_{I_2}) \mathfrak{a}_4, a_0, \mathfrak{a}_1) \cdot \phi_2(\mathfrak{a}_2) \otimes \lambda(\vec{\phi}_{I_1}) \mathfrak{a}_3$} &
    $b \circ \upsilon_{n, m} (\vec{\phi} | \vec{\psi} | \alpha)$ &
    $\upsilon_{n-1, m}(\phi_1 \cup \phi_2 \phi_3 \cdots \phi_n | \vec{\psi} | \alpha)$ \\ \hline
    
    \breakcell{$\phi_1(\lambda(\vec{\psi}_{J_1}) \lambda(\vec{\phi}_{I_2}) \mathfrak{a}_3) \phi_2(\lambda(\vec{\psi}_{J_2} \lambda(\vec{\phi}_{I_3}) \mathfrak{a}_3,$\\
    $\phantom{mo} a_0, \mathfrak{a}_1) \otimes \lambda(\vec{\phi}_{I_1}) \mathfrak{a}_2$} &
    $\upsilon_{n-1, m}(\phi_1 \cup \phi_2 \phi_3 \cdots \phi_n | \vec{\psi} | \alpha)$ &
     $\phi_1 \{ \vec{\psi}_{J_1} \} \cdot \upsilon_{n-1, |J_2|}(\vec{\phi}_{\{2, \cdots, n\}} | \vec{\psi}_{J_2} |\alpha )$\\ \hline

    \breakcell{$f_0 \psi_1(\lambda(\vec{\phi}_{I_2}) \mathfrak{a}_2) \cdot f_0a_0 \otimes \lambda(\vec{\phi}_{I_1}) \mathfrak{a}_1$} &  
    \breakcell{$f_0 \psi_1 \cdot \upsilon_{n, 0}(\vec{\phi} | 1 | \alpha)$ \\ if $\vec{\psi} = \psi_1$} &
    \breakcell{$ \upsilon_{|I_1|, 0} (\vec{\phi}_{I_1} | 1 | \psi_1 \{ \vec{\phi}_{I_2} \} \cdot \alpha )$ \\ if $\vec{\psi} = \psi_1$} \\ \hline
    \hline
  \end{tabular}
\caption{Expansion of terms in Equation \ref{eq:upsilon}}
\label{table:t1}
(Technically, the last term in the middle column is not a standard term, but we include it in the table for convenience.)
\end{table}
\end{center}
\end{landscape}
\begin{figure} \label{fig:upsilon}
\centerline{\xymatrix{
A_0 \ar@/^5pc/[r]^{f_0} 
\ar@/^2pc/[r]^{\big\Downarrow \phi_1}_{f_1} 
\ar@/_2pc/[r]^{\substack{\vdots\\ f_n}}
\ar@/_4pc/[rr]_{id}^{\substack{\alpha \\ \\ \\ \\ }}
& A_1 \ar@/^5pc/[r]^{g_0} 
\ar@/^2pc/[r]^{\big\Downarrow \psi_1}_{g_1} 
\ar@/_2pc/[r]^{\substack{\vdots\\ g_m}}
& A_0
}
$\overset{B}\longrightarrow$
\xymatrix{
A_0 \ar@/^5pc/[r]^{f_0} 
\ar@/^2pc/[r]^{\big\Downarrow \phi_1}_{f_1} 
\ar@/_2pc/[r]^{\substack{\vdots\\ f_n}}
\ar@/_4pc/[rr]_{id}^{\substack{\alpha \\ \\ \\ \\ }}
& A_1 \ar@/^5pc/[r]^{g_0} 
\ar@/^2pc/[r]^{\big\Downarrow \psi_1}_{g_1} 
\ar@/_2pc/[r]^{\substack{\vdots\\ g_m}}
& A_0
}}
\caption{A picture of the domain and target of $B$}
\end{figure}

\begin{prop}
$D(B) = \Upsilon^2 - Id$ where
\begin{align*}
B_{n, m} (\vec{\phi} | \vec{\psi} | \alpha) 
= 1 \otimes \lambda(\psi)\lambda(\phi) \mathfrak{a}_2 \otimes a_0 \otimes \mathfrak{a}_1                          
\end{align*}
\end{prop}

\begin{proof}
We must show: (1) $B$ is a map of comodules, and (2) $D(B) = \Upsilon^2 - Id$.

(1) This proof is standard for cofree comodules. 

(2) We show that $D(B) = \Upsilon^2 - Id$ by direct computation. Since all of the maps are maps of cofree comodules, we only need to check that $\pi_1(D(B) - \Upsilon^2 - Id) = 0$ where $\pi_1$ denotes projection of the comodule onto its degree-1 component, (i.e., $\pi_1: Bar(C^\bullet(A_0, A_1)) \otimes Bar(C^\bullet(A_1, A_0)) \otimes C_{-\bullet}(A_0, A_0) \rightarrow C_{-\bullet}(A_0, A_0)$). More explicitly, we want to check that
\begin{equation} \label{eq:upsilon_homotopy}
\begin{aligned}
&B_{n, m} ( \tilde{\delta}(\vec{\phi}) | \vec{\psi} | \alpha ) \; + 
B_{n, m} ( \vec{\phi} | \tilde{\delta}(\vec{\psi}) | \alpha ) \; + 
B_{n-1, m} ( b^\prime(\vec{\phi}) | \vec{\psi} | \alpha ) \; + 
B_{n, m-1} ( \vec{\phi} | b^\prime(\vec{\psi}) | \alpha ) \; + \\
&B_{n, m} ( \vec{\phi} | \vec{\psi} | b(\alpha) ) \; + 
b \circ B_{n, m} ( \vec{\phi} | \vec{\psi} | \alpha ) \; + \\
&B_{|I_1|, m-1}(\vec{\phi}_{I_1} | \vec{\psi}_{\{1,\cdots, m-1\}} | \psi_{m} \{\vec{\phi}_{I_2}\}\cdot \alpha ) \; + 
B_{n-1, m}(\vec{\phi}_{\{1,\cdots, n-1\}} |\vec{\psi}_{m} | \phi_{n} \cdot \alpha) \; + \\
&\phi_1 \{ \psi_{J_1}\} \cdot B_{\vec{\phi}|-1, |J_2|}(\phi_{\{2,\cdots , n\}} | \psi_{J_2} | \alpha ) \; + 
\psi_1 \cdot B_{n,m-1} ( \vec{\phi} | \vec{\psi}_{\{2,\cdots, |\vec{\psi}\}} | \alpha)  - \\ 
&\upsilon_{|J_1|, |I_1|} (\vec{\psi}_{J_1} | \vec{\phi}_{I_1} | \upsilon_{|I_2|, |J_2|} (\vec{\phi}_{I_2} | \vec{\psi}_{J_2} | \alpha ))  - \pi_1(\vec{\phi} | \vec{\psi} | \alpha) \\
&= 0.
\end{aligned}
\end{equation}

We will call the terms in the first two rows the ``standard terms'' in the computaion of $D(B)$, and the terms in the second two rows the ``extra terms'' in the computation of $D(B)$. The fifth row is $\pi_1(\Upsilon^2 - Id)$. 

Following the logic of Lemma [number], we compute the sum of the standard terms. In the chart below, the leftmost column lists the expressions that don't cancel in the sum of the standard terms, the middle column gives the standard term from which the expression comes, and the rightmost column gives the extra term that cancels the expression. 

\newpage

\begin{landscape}
\begin{center}
  \begin{tabular}{ p{3.25in} | p{2in} | p{2.5in} }
    \hline
    Expression & Comes from Standard Term & Cancels with Extra Term \\ \hline

    $\psi_1(\lambda(\vec{\phi}_{I_1}) \mathfrak{a}_2) \otimes \lambda(\vec{\psi}_{\{2,\cdots,m\}}) \lambda(\vec{\phi}_{I_2}) \mathfrak{a}_3 \otimes a_0 \otimes \mathfrak{a}_1$ &
    $b \circ B_{n,m} (\vec{\phi} | \vec{\psi} | \alpha)$ & 
    $\psi_1 \{ \vec{\phi}_{I_1} \} \cdot B_{|I_2|, m-1} (\vec{\phi}_{I_2} | \vec{\psi}_{\{2, \cdots, m \}} | \alpha)$ \\ \hline

    $g_0\phi_1( \mathfrak{a}_2 ) \otimes \lambda(\vec{\psi}) \lambda(\vec{\phi}_{\{2, \cdots, n\}}) \mathfrak{a}_3 \otimes a_0 \otimes \mathfrak{a}_1$ &
    $b \circ B_{n,m} (\vec{\phi} | \vec{\psi} | \alpha)$ & 
    $\phi_1 \cdot B_{n-1, m} (\vec{\phi}_{\{2, \cdots, n\}} | \vec{\psi} | \alpha)$ \\ \hline

    $1 \otimes \lambda(\vec{\psi}) \lambda(\vec{\phi}_{\{1, \cdots, n-1\}}) \mathfrak{a}_2 \otimes g_m \phi_n(\mathfrak{a}_3 \cdot a_0 \otimes \mathfrak{a}_1$ &
    $b \circ B_{n,m} (\vec{\phi} | \vec{\psi} | \alpha)$ & 
    $B_{n-1, m} (\vec{\phi}_{\{1, \cdots, n-1 \}} | \vec{\psi} | \phi_n \cdot \alpha)$ \\ \hline

    $1 \otimes \lambda(\vec{\psi}_{\{1, \cdots, m-1 \}}) \lambda(\vec{\phi}_{I_1}) \mathfrak{a}_2 \otimes g_m \psi_m( \lambda(\vec{\phi}_{I_2} \mathfrak{a}_3) \cdot a_0 \otimes \mathfrak{a}_1$ &
    $b \circ B_{n,m} (\vec{\phi} | \vec{\psi} | \alpha)$ & 
    $B_{|I_1|, m-1} (\vec{\phi}_{I_2} | \vec{\psi}_{\{1, \cdots, m-1\}} | \psi_m \{ \vec{\phi}_{I_2} \} \cdot \alpha)$ \\ \hline

    $g_0f_0a_0 \otimes \lambda(\vec{\psi}) \lambda(\vec{\phi}) \mathfrak{a}_1$ &
    $b \circ B_{n,m} (\vec{\phi} | \vec{\psi} | \alpha)$ & 
    $\upsilon_{|J_1|, |I_1|} (\vec{\psi}_{J_1} | \vec{\phi}_{I_1} | \upsilon_{|I_2|, |J_2|} (\vec{\phi}_{I_2} | \vec{\psi}_{J_2} | \alpha ))$ \\ \hline

    \hline
  \end{tabular}
\end{center}
(Technically, the last term in the right column is not an extra term, but we include it in the table for convenience.)
\end{landscape}

\newpage
\begin{landscape}

Now, we compute the remaining terms from the fifth row. In the chart below, the left column lists the remaining expressions that don't cancel in the fifth row, and the right column gives the extra term that cancels the expression. 

\begin{center}
  \begin{tabular}{ p{6.25in} | p{2.5in} }
    \hline
    Expression from Fifth Row & Cancels with Extra Term \\ \hline

    $\psi_1(\lambda(\vec{\phi}_{I_1}) \lambda(\vec{\psi}_{J_2}) \lambda(\vec{\phi}_{I_4}) \mathfrak{a}_4, \phi_{|I_1|+1} (\lambda(\vec{\psi}_{J_3}) \lambda(\vec{\phi}_{I_5}) \mathfrak{a}_5, a_0, \mathfrak{a}_1), \lambda(\vec{\phi}_{I_2 \backslash |I_1| + 1}) \mathfrak{a}_2) \otimes \lambda(\vec{\psi}_{J_1}) \lambda(\vec{\phi}_{I_3}) \mathfrak{a}_3$ &
     $\psi_1 \{ \vec{\phi}_{I_1} \} \cdot B_{|I_2|, m-1} (\vec{\phi}_{I_2} | \vec{\psi}_{\{2,\cdots,m\}} | \alpha)$ \\ \hline

    $\psi_1(\lambda(\vec{\phi}_{I_1}) \lambda(\vec{\psi}_{J_2}) \lambda(\vec{\phi}_{I_4}) \mathfrak{a}_4, f_{|I_1|+1}a_0, \lambda(\vec{\phi}_{I_2 \backslash |I_1| + 1}) \mathfrak{a}_1) \otimes \lambda(\vec{\psi}_{J_1}) \lambda(\vec{\phi}_{I_3}) \mathfrak{a}_2$ &
    $\phi_1 \cdot B_{n-1, m} (\vec{\phi}_{\{2,\cdots,n\}} | \vec{\psi} | \alpha)$ \\ \hline

    $g_0\phi_1(\lambda(\vec{\psi}_{J_2}) \lambda(\vec{\phi}_{I_2}) \mathfrak{a}_3, a_0, \mathfrak{a}_1) \otimes \lambda(\vec{\psi}_{J_1}) \lambda(\vec{\phi}_{I_1}) \mathfrak{a}_2$ &
    $\psi_1 \{ \vec{\phi}_{I_1} \} \cdot B_{|I_2|, m-1} (\vec{\phi}_{I_2} | \vec{\psi}_{\{2,\cdots,m\}} | \alpha)$ \\ \hline

    \hline
  \end{tabular}
\end{center}
\end{landscape}

All of the terms in the table describing the expansion of equation \ref{eq:upsilon_homotopy} cancel, so $D(B) = \Upsilon^2 - Id$.
\end{proof}
\begin{prop}
\label{prop:c3}
Let $\tau_{1!}(A_0,A_1): 
T(A_0 \to A_1 \to A_0) \longrightarrow
T(A_1 \to A_0 \to A_1)$ and 
$B(A_0,A_1): T(A_0 \to A_1 \to A_0) 
\longrightarrow T(A_0 \to A_1 \to A_0)$ 
be the maps defined in Propositions 
\ref{prop:c1} and \ref{prop:c2} above. 
Then, $$[\tau_{1!}, B] := 
\tau_{1!}(A_0,A_1) \circ B(A_0,A_1) - 
B(A_1,A_0) \circ \tau_{1!}(A_0,A_1) = 0.$$
\end{prop}
%
\begin{proof}
We show that $[\tau_{1!}, B] = 0$ by direct 
computation. Since all of the maps are maps 
of cofree comodules, we only need to check 
that $\pi_1([\tau_{1!}, B]) = 0$ where 
$\pi_1$ denotes projection of the comodule 
onto cogenerators. We check this directly.
%
\begin{align*}
&\phantom{=}
[\pi_1 \circ \tau_{1!}(A_0,A_1) \circ B(A_0,A_1)] 
  (\vec{\phi} | \vec{\psi} | \alpha ) \\
&= \sum \limits_{\substack{
  I_1I_2 = \{1,\smdots,n\}\\
  J_1J_2 = \{1,\smdots,m\}\\
  \textrm{as ordered sets}}}
\epsilon_{I_1,J_2} \cdot
  \tau_{1!}^{|I_1|, |J_1|} (\vec{\phi}_{I_1} | \vec{\psi}_{J_1} | 
    B^{|I_2|, |J_2|} (\vec{\phi}_{I_2} | \vec{\psi}_{J_2} | \alpha)) \\
&= 
\sum \limits_{\substack{
  I_1I_2 = \{1,\smdots,n\}\\
  J_1J_2 = \{1,\smdots,m\}\\
  \textrm{as ordered sets}}}
\begin{array}{l}  
\epsilon_{I_1,J_2} \cdot 
\eta_{\mathfrak{a_1},\mathfrak{a_2}} \cdot\\
\tau_{1!}^{|I_1|, |J_1|} (\vec{\phi}_{I_1} | \vec{\psi}_{J_1} | 
  1 \otimes \lambda(\vec{\psi}_{J_2}) \lambda(\vec{\phi}_{I_2}) 
  \mathfrak{a}_2, a_0, \mathfrak{a}_1)
\end{array} \\
&= 
\sum \limits_{\substack{
  I_1I_2 = \{1,\smdots,n\}\\
  J_1J_2 = \{1,\smdots,m\}\\
  \textrm{as ordered sets}}}
\epsilon_{I_1,J_2} \cdot 
\eta_{\mathfrak{a_1},\mathfrak{a_2}} \cdot
1 \otimes \lambda(\vec{\phi}_{I_1}) \big( 
  \lambda(\vec{\psi}) \lambda(\vec{\phi}_{I_2}) 
  \mathfrak{a}_2, a_0, \mathfrak{a}_1 \big)
\end{align*}
%
\begin{align*}
& \phantom{{}={}}
[\pi_1 \circ B(A_1,A_0) \circ \tau_{1!}(A_0,A_1)] 
  (\vec{\phi} | \vec{\psi} | \alpha ) \\
&=
\sum \limits_{\substack{
  I_1I_2 = \{1,\smdots,n\}\\
  J_1J_2 = \{1,\smdots,m\}\\
  \textrm{as ordered sets}}}
\epsilon_{I_1,J_2} \cdot
B^{|J_1|, |I_1|} (\vec{\psi}_{J_1} | \vec{\phi}_{I_1} | 
  \tau_{1!}^{|I_2|, |J_2|} (\vec{\phi}_{I_2} | \vec{\psi}_{J_2} | \alpha)) \\
&= 
\sum \limits_{\substack{
  I_1I_2 = \{1,\smdots,n\}\\
  J_1J_2 = \{1,\smdots,m\}\\
  \textrm{as ordered sets}}}
\begin{array}{l}{}
\epsilon_{I_1,J_2} \cdot
B^{|J_1|, |I_1|} \big(\vec{\psi}_{J_1} | \vec{\phi}_{I_1} | \phi_{|I_1|+1} (
  \lambda(\vec{\psi}_{J_2}) \lambda(\vec{\phi}_{I_3}) 
  \mathfrak{a}_3, a_0, \mathfrak{a}_1) \otimes 
  \lambda(\vec{\phi}_{I_2 \backslash |I_1| + 1}) 
  \mathfrak{a}_2 \; + \\
\hphantom{{}moveovermoveovermoveover{}} 
  + a_0 \otimes \lambda(\vec{\phi}_{I_2 \backslash |I_1| + 1}) 
  \mathfrak{a}_1 \; \; \; 
  \text{if }J_2 = \emptyset \big)
\end{array} \\
&= 
\sum \limits_{\substack{
  I_1I_2 = \{1,\smdots,n\}\\
  J_1J_2 = \{1,\smdots,m\}\\
  \textrm{as ordered sets}}}
\begin{array}{l}  
\epsilon_{I_1,J_2} \cdot  
\eta_{\mathfrak{a}_2,\mathfrak{a}_3} \cdot  
1 \otimes \lambda(\vec{\phi}_{I_1}) \lambda(\vec{\psi}_{J_1}) 
  \lambda(\vec{\phi}_{I_3}) \mathfrak{a}_3 \otimes 
  \phi_{|I_1|+1} (\lambda(\vec{\psi}_{J_2}) \lambda(\vec{\phi}_{I_4}) 
  \mathfrak{a}_4, a_0, \mathfrak{a}_1) \otimes \\
  \phantom{{}moveovermov{}}\otimes 
  \lambda(\vec{\phi}_{I_2 \backslash |I_1| + 1}) 
  \mathfrak{a}_2 \; + 
\end{array}\\
&\hphantom{{}moveovermove{}} 
  +\epsilon_{I_1,J_2} \cdot  
  \eta_{\mathfrak{a}_1,\mathfrak{a}_2} \cdot  
  1 \otimes \lambda(\vec{\phi}_{I_1}) \lambda(\vec{\psi}) 
  \lambda(\vec{\phi}_{I_3}) \mathfrak{a}_2 \otimes 
  a_0 \otimes \lambda(\vec{\phi}_{I_2}) \mathfrak{a}_1
\end{align*}
%
It's clear that $\pi_1 \circ \tau_{1!}(A_0,A_1) 
\circ B(A_0,A_1) =  \pi_1 \circ B(A_1,A_0) 
\circ \tau_{1!}(A_0,A_1)$: The final expansion of 
$\pi_1 \circ \tau_{1!}(A_0,A_1) \circ B(A_0,A_1)$ 
is the sum of the two terms in the final expansion 
of $\pi_1 \circ B(A_1,A_0) \circ \tau_{1!}(A_0,A_1)$, 
which is the sum of terms in which one of 
the $\phi$'s contains $a_0$ and the terms in which 
none of the $\phi$'s contains $a_0$).
\end{proof}
% \begin{figure} \label{fig:upsilon}
% \centerline{\xymatrix{
% A_0 \ar@/^5pc/[r]^{f_0} 
% \ar@/^2pc/[r]^{\big\Downarrow \phi_1}_{f_1} 
% \ar@/_2pc/[r]^{\substack{\vdots\\ f_n}}
% \ar@/_5pc/[rrr]_{id}^{\substack{\alpha \\ \\ \\ \\ }}
% & A_1 \ar@/^5pc/[r]^{g_0} 
% \ar@/^2pc/[r]^{\big\Downarrow \psi_1}_{g_1} 
% \ar@/_2pc/[r]^{\substack{\vdots\\ g_m}}
% & A_2 \ar@/^5pc/[r]^{h_0} 
% \ar@/^2pc/[r]^{\big\Downarrow \theta_1}_{h_1} 
% \ar@/_2pc/[r]^{\substack{\vdots\\ h_p}}
% & A_0
% }
% $\overset{\mathcal{B}}\longrightarrow$
% \xymatrix{
% A_2 \ar@/^5pc/[r]^{h_0} 
% \ar@/^2pc/[r]^{\big\Downarrow \theta_1}_{h_1} 
% \ar@/_2pc/[r]^{\substack{\vdots\\ h_p}}
% \ar@/_5pc/[rrr]_{id}^{\substack{\alpha \\ \\ \\ \\ }}
% & A_0 \ar@/^5pc/[r]^{f_0} 
% \ar@/^2pc/[r]^{\big\Downarrow \phi_1}_{f_1} 
% \ar@/_2pc/[r]^{\substack{\vdots\\ f_n}}
% & A_1 \ar@/^5pc/[r]^{g_0} 
% \ar@/^2pc/[r]^{\big\Downarrow \psi_1}_{g_1} 
% \ar@/_2pc/[r]^{\substack{\vdots\\ g_m}}
% & A_2
% }}
% \caption{A picture of the domain and target of $\mathcal{B}$}
% \end{figure}
%
\begin{prop} \label{prop:c4}
Let 
$$
\mathcal{B}_{A_0,A_1,A_2} = \mathcal{B}: 
C(A_0 \to A_1 \to A_2 \to A_0)
\to \hat{\tau}_2^{*2}C(A_1 \to A_2 \to A_0 \to A_1)
$$ 
be a map of comodules over 
$B(A_0 \to A_1 \to A_2 \to A_0)$ 
determined by the following maps to 
cogenerators:
\begin{equation}
\label{eq:def_sigma2}
\begin{split}
\mathcal{B}^{f_0, g_0,h_0}_{A_0,A_1,A_2}: 
  C(A_0 \to A_1 \to A_0)(h_0g_0f_0) 
&\to
\hat{\tau}_2^{*2}C(A_1 \to A_2 \to A_0 \to A_1)
  (f_0h_0g_0)\\
&\xrightarrow[cogenerators]{\textrm{project onto}}
C_{-\bullet}(A_1, _{f_0h_0g_0}{A_1}_{id})\\
\mathcal{B}_{n, m, p} (\vec{\phi} | \vec{\psi} | \vec{\theta} | \alpha) 
= & \sum_{\substack{I_1I_2 = \{1,2,\cdots,n\} \\
                          \textrm{as ordered sets}}}
  (-1)^{|\mathfrak{a}_1|(|\mathfrak{a}_1 + \mathfrak{a}_2|)}
  1 \otimes \lambda(\vec{\phi}_{I_1})\big( \lambda(\vec{\theta}) \lambda(\vec{\psi}) \lambda(\vec{\phi}_{I_2})
  \mathfrak{a}_2 \otimes a_0 \otimes \mathfrak{a}_1 \big)
\end{split}      
\end{equation}
Then, 
\begin{equation} \label{eq:prop4}
D(\mathcal{B}_{A_0,A_1,A_2}) = 
  \Upsilon_{A_2\bullet A_0, A_1} \circ
  \Upsilon_{A_0\bullet A_1, A_2} 
   - \Upsilon_{A_0, A_1\bullet A_2}.
\end{equation}
\end{prop}

\begin{proof}
We will show that Equation \ref{eq:prop4} 
holds by direct computation. Since all of 
the maps are maps of cofree comodules, we 
only need to check that $\pi_1($
Equation \ref{eq:prop4}$)$ holds where 
$\pi_1$ denotes projection of the comodule 
onto cogenerators. More explicitly, we 
want to check that
\begin{equation} \label{eq:prop4_expand}
\begin{aligned}
%hochschild cochain delta
\mathcal{B}_{n, m, p} ( \tilde{\delta}(\vec{\phi}) | \vec{\psi} | \vec{\theta} | \alpha ) \; + 
\mathcal{B}_{n, m, p} ( \vec{\phi} | \tilde{\delta}(\vec{\psi}) | \vec{\theta} | \alpha ) \; + 
\mathcal{B}_{n, m, p} ( \vec{\phi} | \vec{\psi} | \tilde{\delta}(\vec{\theta}) | \alpha ) \; + \\
%cochain b prime
\mathcal{B}_{n-1, m, p} ( b^\prime(\vec{\phi}) | \vec{\psi} | \vec{\theta} | \alpha ) \; + 
\mathcal{B}_{n, m-1, p} ( \vec{\phi} | b^\prime(\vec{\psi}) | \vec{\theta} | \alpha ) \; + 
\mathcal{B}_{n, m, p-1} ( \vec{\phi} | \vec{\psi} | b^\prime(\vec{\theta}) | \alpha ) \; + \\
%chain b
\mathcal{B}_{n, m, p} ( \vec{\phi} | \vec{\psi} | \vec{\theta} | b(\alpha) ) \; + 
b \circ \mathcal{B}_{n, m, p} ( \vec{\phi} | \vec{\psi} | \vec{\theta} | \alpha ) \; + \\
%twist before
\sum \limits_{\substack{
  I_1I_2 = \{1,\smdots,n\}\\
  J_1J_2 = \{1,\smdots,m\}\\
  \textrm{as ordered sets}}}
  \epsilon_{I_1, J_1,\{1,\smdots,p-1\}} \cdot
 \mathcal{B}_{|I_1|, |J_1|, p-1}(\vec{\phi}_{I_1} | \vec{\psi}_{J_1} | \vec{\theta}_{\{1,\cdots, p-1\}} |
     \theta_{p} \{\vec{\psi}_{J_2}\} \{\vec{\phi}_{I_2}\} \cdot \alpha ) \; + \\
%
\sum \limits_{\substack{
  I_1I_2 = \{1,\smdots,n\}\\
  \textrm{as ordered sets}}}
  \epsilon_{I_1,\{1,\smdots,m-1\},\{1,\smdots,p\}} \cdot   
 \mathcal{B}_{|I_1|, m-1, p}(\vec{\phi}_{I_1} | \vec{\psi}_{\{1,\cdots, m-1\}} | \vec{\theta} |
     \psi_{m} \{\vec{\phi}_{I_2}\}\cdot \alpha ) \; + \\
%     
\epsilon_{\{1,\smdots,n-1\},\{1,\smdots,m\},\{1,\smdots,p\}}\cdot
\mathcal{B}_{n-1, m, p}(\vec{\phi}_{\{1,\cdots, n-1\}} |\vec{\psi}_{m} | \vec{\theta} | 
     \phi_{n} \cdot \alpha) \; + \\
%twist after
\sum \limits_{\substack{
  J_1J_2 = \{1,\smdots,m\}\\
  K_1K_2 = \{1,\smdots,p\}\\
  \textrm{as ordered sets}}}
\epsilon_{\{1\},J_1,K_1} \cdot
\phi_1 \{\vec{\theta}_{K_1}\} \{\vec{\psi}_{J_1}\} \cdot
     \mathcal{B}_{n-1, |J_2|, |K_2|}
     (\vec{\phi}_{\{2,\cdots,n\}} | \vec{\psi}_{J_2} | \vec{\theta}_{K_2} | \alpha) \; + \\
\sum \limits_{\substack{
  J_1J_2 = \{1,\smdots,m\}\\
  \textrm{as ordered sets}}}
\epsilon_{\{\},J_1\{1\}} \cdot     
\theta_1 \{\vec{\psi}_{J_1}\} \cdot
     \mathcal{B}_{n, |J_2|, p-1}
     (\vec{\phi} | \vec{\psi}_{J_2} | \vec{\theta}_{\{2,\cdots,p\}} | \alpha) \; +\\
\epsilon_{\{\},\{1\},\{\}} \cdot        
\psi_1 \cdot
     \mathcal{B}_{n, m-1, p}
     (\vec{\phi} | \vec{\psi}_{\{2,\cdots,m\}} | \vec{\theta} | \alpha) \; + \\
%prop4
\upsilon_{n, p \leq * \leq m+p}(\vec{\phi} | \vec{\psi} \bullet \vec{\theta} | \alpha ) \; + \\
\sum \limits_{\substack{
  I_1I_2 = \{1,\smdots,n\}\\
  J_1J_2 = \{1,\smdots,m\}\\
  K_1K_2 = \{1,\smdots,p\}\\
  \textrm{as ordered sets}}}
\epsilon_{I_1,J_1,K_1} \cdot 
\upsilon_{|I_1| \leq * \leq |I_1| + |K_1|,|J_1|}(\vec{\theta}_{K_1} \bullet \vec{\phi}_{I_1}, \vec{\psi}_{J_1}, 
    \upsilon_{|J_2| \leq * \leq |I_2| + |J_2|,|K_2|}(\vec{\phi}_{I_2} \bullet \vec{\psi}_{J_2} | \vec{\theta}_{K_2} | \alpha )) \\
%
=0.
\end{aligned}
\end{equation}
where $\epsilon_{I_1,J_1,K_1} = 
(-1)^{(\sum \limits_{r \in I_2}|\phi_r|+1)
  ((\sum \limits_{s \in J_1}|\psi_s|+1) + 
  (\sum \limits_{t \in K_1}|\theta_t|+1)) + 
  (\sum \limits_{s \in J_2}|\psi_s|+1)
  (\sum \limits_{t \in K_1}|\theta_t|+1)}$.
In Equation \ref{eq:prop4_expand} above, 
we call the terms in rows 1-3 the 
``standard terms'' in the computation of 
$D(\mathcal{B}_{A_0,A_1,A_2})$, and the terms in rows 
4-9 the ``extra terms'' in the computation 
of $D(\mathcal{B}_{A_0,A_1,A_2})$. The terms in rows 10-11 
are $\pi_1$ of the righthand side of Equation 
\ref{eq:prop4}; we will call these the 
``10$^{th}$- and 11$^{th}$-row terms''.

We compute the sum of the standard terms. 
In Table \ref{table:t41}, the leftmost column lists 
the expressions that don't cancel in the sum 
of the standard terms, the middle column gives 
the standard term from which the expression comes, 
and the rightmost column gives the term that 
cancels the expression. Table \ref{table:t42} 
lists the remaining ninth row terms that aren't 
already listed in Table \ref{table:t41}. In 
Table \ref{table:t42}, the left column lists 
the remaining expressions that don't cancel in 
the ninth row, and the right column gives 
the extra term that cancels the expression. 

All of the terms in the tables describing 
the expansion of Equation \ref{eq:prop4_expand} 
cancel, so we're done.
\end{proof}
%
\begin{landscape}
\begin{center}
\begin{table}
  \begin{tabular}{ p{3.25in} | p{1.75in} | p{2.75in} }
    \hline
    Expression & Comes from Standard Term & Cancelling Term \\ \hline
    %
    % twist before
    $1 \otimes \lambda(\vec{\phi}_{I_1}) [
    \lambda(\vec{\theta}_{\{1,\cdots,p-1\}}
    \lambda(\vec{\psi}_{J_1})
    \lambda(\vec{\phi}_{I_2})
    \mathfrak{a}_2 \otimes 
    \theta_p(\lambda(\vec{\psi}_{J_2}) \lambda(\vec{\phi}_{I_3}) \mathfrak{a}_3) \cdot a_0 \otimes
    \mathfrak{a}_1 ]$ & 
    $b \circ \mathcal{B}_{n,m,p} (\vec{\phi} | \vec{\psi} | \vec{\theta} | \alpha)$ & 
    $\mathcal{B}_{|I_1|, |J_1|, p-1}(\vec{\phi}_{I_1} | \vec{\psi}_{J_1} | \vec{\theta}_{\{1,\cdots, p-1\}} |
     \theta_{p} \{\vec{\psi}_{J_2}\} \{\vec{\phi}_{I_2}\} \cdot \alpha )$ \\ \hline

    $1 \otimes \lambda(\vec{\phi}_{I_1}) [
    \lambda(\vec{\theta}
    \lambda(\vec{\psi}_{\{1,\cdots,m-1\}})
    \lambda(\vec{\phi}_{I_2})
    \mathfrak{a}_2 \otimes 
    \psi_m(\lambda(\vec{\phi}_{I_3}) \mathfrak{a}_3) \cdot a_0 \otimes
    \mathfrak{a}_1 ]$ & 
    $b \circ \mathcal{B}_{n,m,p} (\vec{\phi} | \vec{\psi} | \vec{\theta} | \alpha)$ & 
    $\mathcal{B}_{|I_1|, m-1, p}(\vec{\phi}_{I_1} | \vec{\psi}_{\{1,\cdots, m-1\}} | \vec{\theta} |
     \psi_m \{\vec{\phi}_{I_2}\} \cdot \alpha )$ \\ \hline

    $1 \otimes \lambda(\vec{\phi}_{I_1}) [
    \lambda(\vec{\theta}
    \lambda(\vec{\psi}
    \lambda(\vec{\phi}_{\{1,\cdots,n-1\}})
    \mathfrak{a}_2 \otimes 
    \psi_n(\mathfrak{a}_3) \cdot a_0 \otimes
    \mathfrak{a}_1 ]$ & 
    $b \circ \mathcal{B}_{n,m,p} (\vec{\phi} | \vec{\psi} | \vec{\theta} | \alpha)$ & 
    $\mathcal{B}_{n-1, m, p}(\vec{\phi}_{\{1,\cdots, n-1\}} | \vec{\psi} | \vec{\theta} |
     \phi_n \cdot \alpha )$ \\ \hline
    %
    %twist after
    $\phi_1(\lambda(\vec{\theta}_{K_1}) \lambda(\vec{\psi}_{J_1}) \lambda(\vec{\phi}_{I_2}) \mathfrak{a}_2)
    \otimes \lambda(\vec{\phi}_{I_1\backslash 1})[
    \lambda(\vec{\theta}_{K_2}) \lambda(\vec{\psi}_{J_3}) \lambda(\vec{\phi}_{I_3}) \mathfrak{a}_3
    \otimes a_0 \otimes \mathfrak{a}_1]$ &
    $b \circ \mathcal{B}_{n,m,p} (\vec{\phi} | \vec{\psi} | \vec{\theta} | \alpha)$ & 
    $\phi_1 \{\vec{\theta}_{K_1}\} \{\vec{\psi}_{J_1}\} \cdot
     \mathcal{B}_{n-1, |J_2|, |K_2|}
     (\vec{\phi}_{\{2,\cdots,n\}} | \vec{\psi}_{J_2} | \vec{\theta}_{K_2} | \alpha)$ \\ \hline

    $f_0\theta_1( \lambda(\vec{\psi}_{J_1}) \lambda(\vec{\phi}_{I_2}) \mathfrak{a}_2)
    \otimes \lambda(\vec{\phi}_{I_1})[
    \lambda(\vec{\theta}_{\{2,\cdots,p\}}) \lambda(\vec{\psi}_{J_2}) \lambda(\vec{\phi}_{I_3}) \mathfrak{a}_3
    \otimes a_0 \otimes \mathfrak{a}_1]$ &
    $b \circ \mathcal{B}_{n,m,p} (\vec{\phi} | \vec{\psi} | \vec{\theta} | \alpha)$ & 
    $\theta_1 \{\vec{\psi}_{J_1}\} \cdot
     \mathcal{B}_{n, |J_2|, p-1}
     (\vec{\phi} | \vec{\psi}_{J_2} | \vec{\theta}_{\{2,\cdots,p\}} | \alpha)$ \\ \hline

    $f_0h_0\psi_1( \lambda(\vec{\phi}_{I_2}) \mathfrak{a}_2)
    \otimes \lambda(\vec{\phi}_{I_1})[
    \lambda(\vec{\theta}) \lambda(\vec{\psi}_{\{2,\cdots,m\}}) \lambda(\vec{\phi}_{I_3}) \mathfrak{a}_3
    \otimes a_0 \otimes \mathfrak{a}_1]$ &
    $b \circ \mathcal{B}_{n,m,p} (\vec{\phi} | \vec{\psi} | \vec{\theta} | \alpha)$ & 
    $\psi_1 \cdot
     \mathcal{B}_{n, m-1, p}
     (\vec{\phi} | \vec{\psi}_{\{2,\cdots,m\}} | \vec{\theta} | \alpha)$ \\ \hline

    %wrap around
    $f_0h_0g_0 \phi_{i_1} ( \lambda(\vec{\theta}_{K_2}) \lambda(\vec{\psi}_{J_2}) \lambda(\vec{\phi}_{I_3})
       \mathfrak{a}_3 \otimes a_0 \otimes \mathfrak{a}_1) \otimes
       \lambda(\vec{\phi}_{I_1}) \lambda(\vec{\theta}_{K_1}) \lambda(\vec{\psi}_{J_1}) 
       \lambda(\vec{\phi}_{I_2 \backslash i_1}) \mathfrak{a}_2$ &
    $b \circ \mathcal{B}_{n,m,p} (\vec{\phi} | \vec{\psi} | \vec{\theta} | \alpha)$ & 
    11$^{th}$ row \\ \hline

    $f_0h_0g_0f_{i_1}a_0 \otimes \lambda(\vec{\phi}_{I_1}) \lambda(\vec{\theta}) 
       \lambda(\vec{\psi}_{J_1}) \lambda(\vec{\phi}_{I_2}) \mathfrak{a}_1$ &
    $b \circ \mathcal{B}_{n,m,p} (\vec{\phi} | \vec{\psi} | \vec{\theta} | \alpha)$ & 
    11$^{th}$ row \\ \hline

    $\phi_1( \lambda(\vec{\phi}_{I_1}) \lambda(\vec{\theta}) 
       \lambda(\vec{\psi}_{J_1}) \lambda(\vec{\phi}_{I_2}) \mathfrak{a}_3, a_0, \mathfrak{a}_1 ) \otimes
       \lambda(\vec{\phi}_{I_1 \backslash 1}) \mathfrak{a}_2$ &
    $b \circ \mathcal{B}_{n,m,p} (\vec{\phi} | \vec{\psi} | \vec{\theta} | \alpha)$ & 
    10$^{th}$ row \\ \hline

  \end{tabular}
\caption{Expansion of $``$standard terms'' in 
Equation \ref{eq:prop4_expand} and the 
terms that cancel them}
\label{table:t41}
\end{table}  
\end{center}
%
%
\begin{center}
\begin{table}
  \begin{tabular}{ p{6.25in} | p{2.5in} }
    \hline
    Expression from 11$^{th}$ Row & Cancels with Extra Term \\ \hline
    %twist after
    $\phi_1(\lambda(\vec{\theta}_{K_1}) \lambda(\vec{\psi}_{J_1}) \lambda(\vec{\phi}_{I_2}) [
      \lambda(\vec{\theta}_{K_3}) \lambda(\vec{\psi}_{J_4}) \lambda(\vec{\phi}_{I_5})
      \mathfrak{a}_3, a_0, \mathfrak{a}_1])
      \otimes \lambda(\vec{\phi}_{I_1\backslash 1}) \lambda(\vec{\theta}_{K_2}) 
      \lambda(\vec{\psi}_{J_3}) \lambda(\vec{\phi}_{I_4}) \mathfrak{a}_2$ &
    $\phi_1 \{\vec{\theta}_{K_1}\} \{\vec{\psi}_{J_1}\} \cdot
     \mathcal{B}_{n-1, |J_2|, |K_2|}
     (\vec{\phi}_{\{2,\cdots,n\}} | \vec{\psi}_{J_2} | \vec{\theta}_{K_2} | \alpha)$ \\ \hline

    $f_0\theta_1( \lambda(\vec{\psi}_{J_1}) \lambda(\vec{\phi}_{I_2}) [
      \lambda(\vec{\theta}_{K_2}) \lambda(\vec{\psi}_{J_3}) \lambda(\vec{\phi}_{I_4})
      \mathfrak{a}_3, a_0, \mathfrak{a}_1])
      \otimes \lambda(\vec{\phi}_{I_1}) \lambda(\vec{\theta}_{K_1 \backslash 1}) 
      \lambda(\vec{\psi}_{J_2}) \lambda(\vec{\phi}_{I_3}) \mathfrak{a}_2$ &
    $\theta_1 \{\vec{\psi}_{J_1}\} \cdot
     \mathcal{B}_{n, |J_2|, p-1}
     (\vec{\phi} | \vec{\psi}_{J_2} | \vec{\theta}_{\{2,\cdots,p\}} | \alpha)$ \\ \hline

    $f_0h_0\psi_1( \lambda(\vec{\phi}_{I_2}) [
      \lambda(\vec{\theta}_{K_2}) \lambda(\vec{\psi}_{J_2}) \lambda(\vec{\phi}_{I_4})
      \mathfrak{a}_3, a_0, \mathfrak{a}_1])
      \otimes \lambda(\vec{\phi}_{I_1}) \lambda(\vec{\theta}_{K_1}) 
      \lambda(\vec{\psi}_{J_1 \backslash 1}) \lambda(\vec{\phi}_{I_3}) \mathfrak{a}_2$ & 
    $\psi_1 \cdot
     \mathcal{B}_{n, m-1, p}
     (\vec{\phi} | \vec{\psi}_{\{2,\cdots,m\}} | \vec{\theta} | \alpha)$ \\ \hline

    \hline
  \end{tabular}
\caption{Expansion of remaining $``$11$^{th}$ row terms'' in 
Equation \ref{eq:prop4_expand} and the $``$extra terms''
that cancel them}
\label{table:t42}
\end{table}  
\end{center}
\end{landscape}
% \begin{figure} \label{fig:upsilon}
% \centerline{\xymatrix{
% A_0 \ar@/^5pc/[r]^{f_0} 
% \ar@/^2pc/[r]^{\big\Downarrow \phi_1}_{f_1} 
% \ar@/_2pc/[r]^{\substack{\vdots\\ f_n}}
% \ar@/_5pc/[rrr]_{id}^{\substack{\alpha \\ \\ \\ \\ }}
% & A_1 \ar@/^5pc/[r]^{g_0} 
% \ar@/^2pc/[r]^{\big\Downarrow \psi_1}_{g_1} 
% \ar@/_2pc/[r]^{\substack{\vdots\\ g_m}}
% & A_2 \ar@/^5pc/[r]^{h_0} 
% \ar@/^2pc/[r]^{\big\Downarrow \theta_1}_{h_1} 
% \ar@/_2pc/[r]^{\substack{\vdots\\ h_p}}
% & A_0
% }
% $\overset{[d_0^* \Upsilon, \mathcal{B}]}\longrightarrow$
% \xymatrix{
% A_0 \ar@/^5pc/[r]^{f_0} 
% \ar@/^2pc/[r]^{\big\Downarrow \phi_1}_{f_1} 
% \ar@/_2pc/[r]^{\substack{\vdots\\ f_n}}
% \ar@/_5pc/[rrr]_{id}^{\substack{\alpha \\ \\ \\ \\ }}
% & A_1 \ar@/^5pc/[r]^{g_0} 
% \ar@/^2pc/[r]^{\big\Downarrow \psi_1}_{g_1} 
% \ar@/_2pc/[r]^{\substack{\vdots\\ g_m}}
% & A_2 \ar@/^5pc/[r]^{h_0} 
% \ar@/^2pc/[r]^{\big\Downarrow \theta_1}_{h_1} 
% \ar@/_2pc/[r]^{\substack{\vdots\\ h_p}}
% & A_0
% }}
% \caption{A picture of the domain and target of $[d_0^* \Upsilon, \mathcal{B}]$}
% \end{figure}

\begin{prop}
\label{prop:c5}
Let $\Upsilon$ and $\mathcal{B}$ be as 
defined in the previous propositions. 
Then, $[\Upsilon, \mathcal{B}] := 
\Upsilon_{A_1\bullet A_2, A_0} 
\mathcal{B}_{A_0,A_1,A_2} - 
\mathcal{B}_{A_2, A_0, A_1} 
\Upsilon_{A_0 \bullet A_1, A_2} = 0$. 
(Note that $[\Upsilon, \mathcal{B}]$ is 
a map from $C(A_0 \to A_1 \to A_2 \to A_0)$ 
to itself.)
\end{prop}
%
\begin{proof}
We show the proposition by direct computation. 
Since all of the maps are maps of cofree 
comodules, we only need to check that 
$\pi_1([\Upsilon, \mathcal{B}]) = 0$ where 
$\pi_1$ denotes projection of the comodule 
onto cogenerators. We check this directly.
%
\begin{equation*}
\begin{aligned}
&\phantom{{}={}}
\pi_1 \circ \Upsilon_{A_1\bullet A_2, A_0} 
  \mathcal{B}_{A_0,A_1,A_2} 
  (\vec{\phi} | \vec{\psi} | \vec{\theta} | \alpha ) \\
&= 
\sum \limits_{\substack{
  I_1I_2 = \{1,\smdots,n\}\\
  J_1J_2 = \{1,\smdots,m\}\\
  K_1K_2 = \{1,\smdots,p\}\\
  \textrm{as ordered sets}}}
\epsilon_{I_2,J_1,J_2,K_1}\cdot
\upsilon_{|K_1| \leq * \leq |K_1|+|J_1|, |I_1|} (
   \vec{\psi}_{J_1} \bullet \vec{\theta}_{K_1} | 
   \vec{\phi}_{I_1} | 
   \mathcal{B}_{|I_2|, |J_2|, |K_2|} (
   \vec{\phi}_{I_2} | \vec{\psi}_{J_2} | \vec{\theta}_{K_2} | \alpha)) \\
&= 
\sum \limits_{\substack{
  I_1I_2 = \{1,\smdots,n\}\\
  J_1J_2 = \{1,\smdots,m\}\\
  K_1K_2 = \{1,\smdots,p\}\\
  \textrm{as ordered sets}}}
\begin{array}{l}  
\epsilon_{I_2,J_1,J_2,K_1}\cdot
\eta_{\mathfrak{a}_1, \mathfrak{a}_2} \cdot\\
\upsilon_{|K_1| \leq * \leq |K_1|+|J_1|, |I_1|} (
   \vec{\psi}_{J_1} \bullet \vec{\theta}_{K_1} | 
   \vec{\phi}_{I_1} |
   1 \otimes \lambda(\vec{\phi}_{I_2})[
     \lambda(\vec{\theta}_{K_2}) \lambda(\vec{\psi}_{J_2}) 
     \lambda(\vec{\phi}_{I_3}) 
     \mathfrak{a}_2, a_0, \mathfrak{a}_1] 
\end{array} \\
&= 
\sum \limits_{\substack{
  I_1I_2 = \{1,\smdots,n\}\\
  J_1J_2 = \{1,\smdots,m\}\\
  K_1K_2 = \{1,\smdots,p\}\\
  \textrm{as ordered sets}}}
\epsilon_{I_2,J_1,J_2,K_1}\cdot
\eta_{\mathfrak{a}_1, \mathfrak{a}_2} \cdot
1 \otimes \lambda(\vec{\theta}_{K_1}) \lambda(\vec{\psi}_{J_1}) 
  \lambda(\vec{\phi}_{I_1})[
     \lambda(\vec{\theta}_{K_2}) \lambda(\vec{\psi}_{J_2}) 
     \lambda(\vec{\phi}_{I_2})
     \mathfrak{a}_2, a_0, \mathfrak{a}_1]    
\end{aligned}
\end{equation*}
%
\begin{align*}
& \phantom{{}={}}
\pi_1 \circ \mathcal{B}_{A_2, A_0, A_1} 
  \Upsilon_{A_0 \bullet A_1, A_2} 
  (\vec{\phi} | \vec{\psi} | \vec{\theta} | \alpha ) \\
&= 
\sum \limits_{\substack{
  I_1I_2 = \{1,\smdots,n\}\\
  J_1J_2 = \{1,\smdots,m\}\\
  K_1K_2 = \{1,\smdots,p\}\\
  \textrm{as ordered sets}}}
\epsilon_{I_2,J_1,J_2,K_1}\cdot
B_{|K_1|, |I_1|, |J_1|} 
   (\vec{\theta}_{K_1} | \vec{\phi}_{I_1} | \vec{\psi}_{J_1} | 
   \upsilon_{|J_2| \leq * \leq |I_2|+|J_2|, |K_2|} 
   (\vec{\phi}_{I_2} \bullet \vec{\psi}_{J_2} | \vec{\theta}_{K_2} | \alpha)) \\
&= 
\sum \limits_{\substack{
  I_1I_2 = \{1,\smdots,n\}\\
  J_1J_2 = \{1,\smdots,m\}\\
  K_1K_2 = \{1,\smdots,p\}\\
  \textrm{as ordered sets}}}
\epsilon_{I_2,J_1,J_2,K_1}\cdot
\eta_{\mathfrak{a}_1, \mathfrak{a}_2} \cdot
1 \otimes \lambda(\vec{\theta}_{K_1}) \lambda(\vec{\psi}_{J_1}) 
  \lambda(\vec{\phi}_{I_1})[
     \lambda(\vec{\theta}_{K_2}) \lambda(\vec{\psi}_{J_2}) 
     \lambda(\vec{\phi}_{I_2})
     \mathfrak{a}_2, a_0, \mathfrak{a}_1]  
\end{align*}
It's clear that $\pi_1([\Upsilon, \mathcal{B}]) = 0$.
\end{proof}

\chapter{Notation}	% First appendix chapter, i.e., Appendix A.
	\subsection{Bar complexes}
Let $A$, $B$ be algebras over a field $k$ of characteristic zero. 
\begin{align*}
\bm{Bar(C^\bullet(A, B))} 
&= Bar_0(C^\bullet(A, B)) + \bigoplus_{\substack{n \geq 1}} Bar_n(C^\bullet(A, B)) \\
&= k \oplus 
  \bigoplus_{\substack{f_0, \cdots, f_n \\ 
					   \textrm{maps of algebras}, \\
						n \geq 1}} 
  C^\bullet(A, \, _{f_0}B_{f_1}) \otimes C^\bullet(A, \, _{f_1}B_{f_2}) \otimes \cdots \otimes C^\bullet(A, \, _{f_{n-1}}B_{f_n})
\end{align*}
where $_{f_i}B_{f_j}$ denotes $B$ as a bimodule over $A$ with left and right module structures given by $f_i$ and $f_j$, respectively.

$\bm{\vec{\phi}}$ denotes a typical element of $Bar(C^\bullet(A, B))$. In other words, $\vec{\phi} = \phi_1\otimes \cdots \otimes \phi_n$ where $\phi_i \in C^\bullet(A,  \, _{f_{i-1}}B_{f_i})$ for $1 \leq i \leq n$. We may also write $\bm{\vec{\phi}_{\{1,\cdots, n\}}}$ as shorthand to keep track of the subscript indices. ($\vec{\phi}_{\{\}} = 1 \in Bar_0(C^\bullet(A, B)) = k$.) When convenient, to reduce the number of unique variables, we denote the degree of $\vec{\phi}$ by $\bm{|\vec{\phi}|} = n$. $\newline$

$Bar(C^\bullet(A, B))$ is a dg coalgebra with the usual coproduct $\Delta$ and differential $d_{bar}$ for bar complexes. Namely, 
\begin{align*}
\Delta (\phi_1 \otimes \cdots \otimes \phi_n) 
= &1 \bigotimes \phi_1 \otimes \cdots \otimes \phi_n + \phi_1 \bigotimes \phi_2 \otimes \cdots \otimes \phi_n + \\
   &+ \cdots + \phi_1 \otimes \cdots \otimes \phi_i \bigotimes \phi_{i+1} \otimes \cdots \otimes \phi_n + \cdots + \\
  &+ \phi_1 \otimes \cdots \otimes \phi_{n-1} \bigotimes \phi_n + \phi_1 \otimes \cdots \otimes \phi_n \bigotimes 1
\end{align*}
\begin{align*}
d_{bar} (\phi_1 \otimes \cdots \otimes \phi_n) 
= & \tilde{\delta}(\phi_1 \otimes \cdots \otimes \phi_n) + b^\prime(\phi_1 \otimes \cdots \otimes \phi_n) \\
= &\bigoplus \pm \phi_1 \otimes \cdots \otimes \delta(\phi_i) \otimes \cdots \otimes \phi_n + \\
  &\bigoplus \pm \phi_1 \otimes \cdots \otimes \phi_i\cup \phi_{i+1} \otimes \cdots \otimes \phi_n
\end{align*}
where $\delta$ is the Hochschild cochain differential, $\tilde{\delta}$ is the extension of $\delta$ to the bar complex, $\cup$ is the cup product on Hochschild cochains, and $b^\prime$ is the extension of the cup product to the bar complex. $\newline$


\subsection{Comodules}

$\bm{(\vec{\phi} | \vec{\psi} | \alpha)}$ denotes a typical element of $Bar(C^\bullet(A_0, A_1)) \otimes Bar(C^\bullet(A_1, A_0)) \otimes C_{-\bullet}(A_0, A_0)$. In other words, ($\vec{\phi} | \vec{\psi} | \alpha$) = $\phi_1\otimes \cdots \otimes \phi_n \otimes \psi_1\otimes \cdots \otimes \psi_m \otimes \alpha$ where $\phi_i \in Bar(C^\bullet(A_0, A_1))$ for $1 \leq i \leq n$, $\psi_j \in Bar(C^\bullet(A_1, A_0))$ for $1 \leq j \leq m$, $\alpha \in C_{-\bullet}(A_0, A_0)$. $\newline$

\subsection{Elements of Hochschild chains}
Let $a_0 \otimes a_1 \otimes \cdots \otimes a_n$ denote a typical element of $C_{-\bullet}(A,A)$. At times, we wish to feed a portion of $a_0 \otimes a_1 \otimes \cdots \otimes a_n$ to a Hochschild cochain (or other map on chains) without specifying the degree of the cochain. To do this, we will rewrite $a_0 \otimes a_1 \otimes \cdots \otimes a_n = a_0 \otimes \mathfrak{a}_1 \otimes \cdots \otimes \mathfrak{a}_r$ where each $\mathfrak{a}_i = a_{j_i} \otimes a_{j_i+1} \otimes \cdots \otimes a_{j_{i+1}-1}$ and $\mathfrak{a}_i$ is an empty chain if $j_i = j_{i+1}$.

For example, if $\phi \in C^2(A,A)$, then we rewrite $\sum a_0 \otimes a_1 \otimes \cdots a_{i-1} \otimes \phi(a_i)$.


\subsection{Maps between bar complexes}

$\bm{\tau}$ denotes the flip map $Bar(C^\bullet(A_0, A_1)) \otimes Bar(C^\bullet(A_1, A_0)) \rightarrow Bar(C^\bullet(A_1, A_0)) \otimes Bar(C^\bullet(A_0, A_1))$, which switches the order of the two bar complexes. $\newline$

$\bm{\Upsilon}$ is a linear map $Bar(C^\bullet(A_0, A_1)) \otimes Bar(C^\bullet(A_1, A_0)) \otimes C_{-\bullet}(A_0, A_0) \rightarrow Bar(C^\bullet(A_1, A_0)) \otimes Bar(C^\bullet(A_0, A_1)) \otimes C_{-\bullet}(A_1, A_1)$. It is defined in the standard way so as to make it a map of comodules extending $\tau$. Explicitly, we define linear maps 
\begin{align*}
\upsilon_{n,m}: Bar_n(C^\bullet(A_0, A_1)) \otimes Bar_m(C^\bullet(A_1, A_0)) \otimes C_{-\bullet}(A_0, A_0) \rightarrow C_{-\bullet}(A_1, A_1).
\end{align*}
Then, we piece together the $\upsilon_{n,m}$ to define $\Upsilon$:
\begin{align*}
\Upsilon(\vec{\phi}_{\{1,\cdots,n\}} | \vec{\psi}_{\{1,\cdots,m\}} | \alpha) 
&= \sum_{\substack{I_1I_2 = \{1,\cdots,n\} \, \textrm{and} \\ 
				J_1J_2 = \{1,\cdots,m\} \\
				\textrm{as ordered sets}}}
	( \tau( \vec{\phi}_{I_1} | \vec{\psi}_{J_1} ) \, |  \, \upsilon_{|I_2|, |J_2|}( \vec{\phi}_{I_2} | \vec{\psi}_{J_2} | \alpha )) \\	
&= \sum_{\substack{I_1I_2 = \{1,\cdots,n\} \, \textrm{and} \\ 
				J_1J_2 = \{1,\cdots,m\} \\
				\textrm{as ordered sets}}}
	( \vec{\psi}_{I_1} \, | \vec{\phi}_{J_1} \, |  \, \upsilon_{|I_2|, |J_2|}( \vec{\phi}_{I_2} | \vec{\psi}_{J_2} | \alpha )). \\	
\end{align*} $\newline$

\subsection{Ordered sets}

Let $I_1$, $I_2$ be ordered sets. The degree of an ordered set, denoted $\bm{|\cdot|}$, is the number of elements the set contains.

$\bm{I_1I_2}$ denotes the concatenation of $I_1$ and $I_2$ as ordered sets. For example, if $I_1 = \{1,2,a\}$ and $I_2 = \{6,B,0\}$ are ordered sets, then $I_1I_2 = \{1,2,a,6,B,0\}$. 

When we write $I_1I_2 = \{1,\cdots,n\}$ as ordered sets, $I_1$ or $I_2$ may be empty. 



\end{document}

