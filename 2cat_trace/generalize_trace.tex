\section{Redefining the trace functor} \label{sec:redefine_trace}
In this section, we generalize Kaledin's 
definition of a trace functor on a 2-category 
to a trace functor on dg cocategories.
First, we transform the definition from 
the language from functors and natural 
transformations to the language of modules.
%
\begin{defn} \label{def:module_over_cat}
Let $\mathcal{C}$ be a $k$-linear category. 
A left module over $\mathcal{C}$ is a 
$k$-linear functor 
$\mathcal{C} \to k-mods$.
\end{defn}
%
Given the definition above, we can rewrite 
the definition of a trace functor on a 
2-category in the language of modules.
%
\begin{defn} \label{def:trace_module}
(Kaledin, reformulated): Let $\mathcal{C}$ 
be a category in $k$-linear categories. 
A trace functor on $\mathcal{C}$ is:
\begin{itemize}
\item for each $A \in Obj(\mathcal{C})$, a 
left module $T(A)$ over $\mathcal{C}(A,A)$
%
\item for each pair $A, B \in Obj(\mathcal{C})$, a
map of modules over $\mathcal{C}(A,B) \otimes \mathcal{C}(B,A)$
$$\tau_!(A,B): m_{ABA}^*T(A) \to \tau^*m_{BAB}^*T(B)$$ 
where $m_{ABA}$ is the composition functor 
$m_{ABA}: \mathcal{C}(A,B) \otimes \mathcal{C}(B,A) \to 
\mathcal{C}(A,A)$, $\tau$ is a flip functor, and 
pulling back along a functor means pre-composition.
%
\item for $A,B,C \in Obj(\mathcal{C})$, 
$$\tau^{*2} \tau_!(B,A) \circ \tau^* \tau_!(C,B) \circ 
\tau_!(A,C) = id.$$
\end{itemize}
\end{defn}
%
Now, we will translate from modules to dg comodules. 
Reversing the arrows in Definition 
\ref{def:module_over_cat}, we have the following 
definition for a dg comodule over a category 
in dg cocategories.
%
\begin{defn}\label{def:dg_comod} 
Let $\mathcal{C}$ be a dg 
cocategory. A dg comodule over 
$\mathcal{C}$ is: 
for each $f \in Obj(\mathcal{C})$, a complex 
$T^\bullet(f)$ and map of complexes
$$\Delta_f: T^\bullet(f) \to 
\prod\limits_{g \in Obj(\mathcal{C})} 
\mathcal{C}^\bullet(f,g) \otimes T^\bullet(g)
$$
such that the following two maps 
coincide (coassociativity):
$$
\xymatrix{
T^\bullet(f)
\ar[d]_{\Delta(f)}\\
\prod \limits_{g \in Obj(\mathcal{C})}
\mathcal{C}^\bullet(f,g) \otimes
T^\bullet(g)
\ar@/^1pc/[d]^{id \otimes \Delta(g)}
\ar@/_1pc/[d]_{\Delta_{\mathcal{C}(} \otimes id}\\
 \prod \limits_{g, g^\prime \in Obj(\mathcal{C})}
\mathcal{C}^\bullet(f,g) \otimes
\mathcal{C}^\bullet(g,g^\prime) \otimes
T^\bullet(g^\prime)  
}
$$
and the following diagram commutes 
(counitality):
$$
\xymatrixcolsep{5pc}
\xymatrixrowsep{5pc}
\xymatrix{
T^\bullet(f)
\ar[r]^{\Delta(f)}
\ar[rd]_{id}
& \prod \limits_{g \in Obj(\mathcal{C})}
\mathcal{C}^\bullet(f,g) \otimes
T^\bullet(g)
\ar[d]^{\epsilon_{\mathcal{C}}
\otimes id} \\
%
& T^\bullet(f).  
}
$$
\end{defn}
%
Finally, we can rewrite Definition 
\ref{def:trace_module} in terms of dg 
comodules.
%
\begin{defn} \label{def:trace_dg_comodule}
Let $\mathcal{C}$ be a category in 
dg cocategories. A trace functor on 
$\mathcal{C}$ is:
\begin{itemize}
\item for each $A \in Obj(\mathcal{C})$, a 
dg comodule $T(A)$ over $\mathcal{C}(A,A)$
%
\item for each pair $A, B \in Obj(\mathcal{C})$, a
map of dg comodules over $\mathcal{C}(A,B) \otimes \mathcal{C}(B,A)$
$$\tau_!(A,B): m_{ABA}^*T(A) \to \tau^*m_{BAB}^*T(B)$$ 
where $m_{ABA}$ is the composition functor 
$m_{ABA}: \mathcal{C}(A,B) \otimes \mathcal{C}(B,A) \to 
\mathcal{C}(A,A)$, $\tau$ is a flip functor. We 
can take any definition for the pullback that 
is a natural and satisifies 
\begin{equation}
F^*G^* = (GF)^*.
\end{equation}
%
\item for $A,B,C \in Obj(\mathcal{C})$, 
$$\tau^{*2} \tau_!(B,A) \circ \tau^* \tau_!(C,B) \circ 
\tau_!(A,C) = id.$$
\end{itemize}
\end{defn}



















