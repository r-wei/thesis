\documentclass[12pt]{article} %use [draft] to check on margins
\begin{document}

This dissertation builds on work in noncommutative differential geometry, and we start with a brief history of the subject. In 1962, Hochschild, Kostant and Rosenberg gave the first hint that, to build a notion of differential geometry for noncommutative algebras, one should start by studying Hochschild homology and cohomology. Their famous HKR theorem states that, for a regular, commutative algebra over a field of characteristic zero, (1) Hochschild homology is quasi-isomorphic to deRham forms on the algebra (with zero differential), and (2) Hochschild cohomology is quasi-isomorphic to poly-vector fields on the algebra (also with zero differential) [ref ??].

However, when the algebra is an algebra of smooth functions on a real manifold, we have much more differential-geometric structure. For example, (shifted) poly-vector fields form a dgla with the Schouten-Nijenhuis bracket, an extension of the Lie bracket on vector fields to poly-vector fields; and in 1997, Kontsevich proved that Hochschild cohomology with its known Gerstenhaber bracket is quasi-isomorphic to poly-vector fields, not as dglas, but as $L_\infty$ algebras [ref ??]. Moreover, the associative wedge product satisfies a Leibniz relation with the Schouten-Nijenhuis bracket, making poly-vector fields a Gerstenhaber algebra. In 1998, Tamarkin gave, for each Drinfeld associator, a quasi-isomorphism between Hochschild cochains and poly-vector fields as Gerstenhaber$_\infty$ algebras [ref ??].

But what about differential forms? Poly-vector fields act on forms in two ways: (1) via contraction as a dga, and (2) via the Lie derivative as a dgla. Together, these two actions satisfy Cartan's formula. We can axiomatize all of this structure (a Gerstenhaber algebra and the two actions on a module satsifying relations) into the notion of an algebra over the Calc operad. In 2008, Dolgushev, Tamarkin and Tsygan proved that the pair (Hochschild cochains, Hochschild chains) are quasi-isomorphic to (poly-vector fields, deRham forms) as Calc$_\infty$ algebras [ref ??].

Loosely given, the motivating question of this thesis is: Can we use the Calc$_\infty$ structure on Hochschild cochains and chains to create a cyclic object? First, we repackage and generalize the $L_\infty$ structure on Hochschild cochains to construct dg cocategories, $B(n)$ (Section ??). Then, we show that these dg cocategories form, not a cyclic, but a sheafy-cyclic object (Section ??). We introduce dg comodules $C(n)$ over $B(n)$, which are categorified versions of the bar construction of Hochschild chains as a module over Hochschild cochains via contraction (Section ??). Given this background, we ask the motivating question in a more technical way: Can we extend the sheafy-cyclic structure on dg cocategories to include the dg comodules?

In Sections ??, we establish the theoretical background needed to work with the dg comodules. Then, we give a reasonable candidate for a sheafy-cyclic structure on dg comodules (Section ??). Here, we find, surprisingly, that the formulas for cyclically rotating a dg comodule resemble the formulas for the Lie derivative of Hochschild cochains on chains. Unfortunately, our candidate only respects composition of morphisms up to homotopy. Yet, we are able to give these homotopies explicitly, and interestingly, they are also familiar formulas--they look like generalizations of the B operator. We show that no higher homotopies are needed, and repackage our $``$functor up to homotopy'' into an $A_\infty$ functor. Finally, we apply (categorified) Cobar to get a $``$homotopically sheafy-cyclic object in dg categories with dg modules'' (Section ??).

\end{document}