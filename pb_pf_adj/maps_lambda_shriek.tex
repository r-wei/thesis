\section{Maps $\lambda_!$}
\label{sec:shriek_maps}
In this section, for $\lambda \in 
\{ \textrm{generating 
morphisms in }\Lambda \}$ 
(see Appendix \ref{chap:lambda}),
we give maps $\lambda_!: 
C(\mathcal{A}) \to 
\hat{\lambda}^*C(\mathcal{A}^\prime)$ 
of dg comodules 
over $B(\mathcal{A})$. Showing that the 
$\lambda_!$ satisfy cyclic relations 
up to homotopy is the 
computational heart of this thesis, 
and will be done in the next chapter. 
For now, we introduce the $\lambda_!$'s.

$\newline$
Fix algebras $A_0, \dots A_{n+1}$. Let 
$B(n) := B(A_0 \to \dots \to A_n \to A_0)$, 
$C(n) := C(A_0 \to \dots \to A_n \to A_0)$ 
for this choice of algebras.
%
\subsection{Generating coboundaries 
  ${\delta_{j,n}}_!$ for $n \in \mathbb{N}$, 
  $0 \leq j < n$}
From Example \ref{eg:pb3}, we know that $C(n) \cong 
\hat{\delta}_{j,n}^*C(A_0 \to \smdots \to A_j 
\longrightarrow A_{j+2} \to \smdots \to A_n \to A_0)$. 
So, define
$\delta_{j,n!}:C(n)
\xrightarrow{id} C(n) \cong 
\hat{\delta}_{j,n}^*C(A_0 \to \smdots \to A_j 
\longrightarrow A_{j+2} \to \smdots \to A_n \to A_0)$ 
for $0 \leq j < n$.
%
\subsection{Generating codegeneracies 
  ${\sigma_{i,n}}_!$ for $n \in \mathbb{N}$, 
  $1 \leq i \leq n+1$}
From Example \ref{eg:pb4}, we know that $C(n) 
\cong \hat{\sigma}_{i,n}^*C(A_0 \to \smdots \to A_i 
\to A_i \to \smdots \to A_n \to A_0)$. 
So, define
$\sigma_{i,n!}:C(n)
\xrightarrow{id} C(n) \cong 
\hat{\sigma}_{i,n}^*C(A_0 \to \smdots \to A_i 
\to A_i \to \smdots \to A_n \to A_0)$ 
for $1 \leq i \leq n+1$.
%
\subsection{Generating rotations ${\tau_n}_!$}
\subsubsection{$n=0$}
Let ${\tau_0}_! = id: C(0) \to 
\hat{\tau}_0^*C(0) = \hat{id}^*C(0) = C(0)$.
%
\subsubsection{$n=1$}
We want to define a map of dg comodules over 
$B(1):= B(A_0 \to A_1 \to A_0)$
$$
{\tau_1}_!: 
C(1):= C(A_0 \to A_1 \to A_0) \to 
\hat{\tau}_1^*C(A_1 \to A_0 \to A_1).
$$
Example \ref{eg:pb5} describes the structure of 
$\hat{\tau}_1^*C(A_1 \to A_0 \to A_1)$. 
$\hat{\tau}_1^*C(A_1 \to A_0 \to A_1)$ is quasi-cofree 
over $B(1)$, so we can define ${\tau_1}_!$ by 
giving maps from $C(1)$ to the cogenerators of 
$\hat{\tau}_1^*C(A_1 \to A_0 \to A_1)$ and 
checking that the corresponding map of comodules 
commutes with the differentials. 

More explicitly, 
for $f = (f_{0,0}, f_{1,0}) \in Obj(B(1))$, we will 
give $k$-linear maps
\begin{align*}
\upsilon^f: C(1)^\bullet(f) 
& \to 
C_{-\bullet}(A_1, _{f_{0,0}f_{1,0}} {A_1}_{id})\\
(\phi_{0,1}\smdots \phi_{0,n_0}|
 \phi_{1,1}\smdots \phi_{0,n_1}| \alpha )
&\mapsto
\upsilon^f_{n_0,n_1}(\phi_{0,1}\smdots \phi_{0,n_0}|
 \phi_{1,1}\smdots \phi_{1,n_1}| \alpha ).
\end{align*}
Then, we lift $\{\upsilon_f | f \in Obj(B(1))\}$ 
to a map of comodules in the standard way: 
\begin{equation}
\label{eq:colift}
{\tau_{1!}}_f 
  (\phi_{0,1}\smdots \phi_{0,n_0}|
  \phi_{1,1}\smdots \phi_{1,n_1}| \alpha )
= \sum \limits_{\substack{
    0 \leq k_0 \leq n_0\\
    0 \leq k_1 \leq n_1}}
\phi_{0,1}\smdots \phi_{0,k_1}|\phi_{1,1} \smdots \phi_{1,k_0}|
\upsilon^{f_{0,k_1}, f_{1,k_0}}_{n_0-k_0,n_1-k_1} 
  (\phi_{0,k_0+1}\smdots \phi_{0,n_0}|
  \phi_{1,k_1+1}\smdots \phi_{1,n_1}| \alpha )
\end{equation}
(see Figure \ref{fig:phi|alpha} for notation). 
Finally, we will check by direct computation that 
${\tau_1}_!$ defined as such commutes with the 
differentials. To make the exposition smooth, 
all of this is done in Appendix Proposition 
\ref{prop:c1}.

\subsubsection{$n>1$}
For $n>1$, we define ${\tau_n}_!$ by 
pulling back ${\tau_1}_!$ along $\delta_{0,*}$'s 
as follows:
\begin{align*}
{\tau_n}_!: C(n) 
&\cong 
(\widehat{\delta_{0,2}\circ\dots\circ\delta_{0,n}})^*
  C(A_0 \to A_n \to A_0)\\
&\xrightarrow{
  (\widehat{\delta_{0,2}\circ\dots\circ\delta_{0,n}})^*
  {\tau_1}_!}
(\widehat{\delta_{0,2}\circ\dots\circ\delta_{0,n}})^*
  \hat{\tau}_1^*C(A_n \to A_0 \to A_n)\\
&\cong
(\widehat{\tau_1\circ\delta_{0,2}\circ\dots\circ
  \delta_{0,n}})^*C(A_n \to A_0 \to A_n)\\
&\cong
(\widehat{\delta_{1,2}\circ\dots\circ\delta_{1,n}
  \circ \tau_n})^*C(A_n \to A_0 \to A_n)\\
&\cong
\hat{\tau}_n^*
  (\widehat{\delta_{1,2}\circ\dots\circ\delta_{1,n}})^*
  C(A_n \to A_0 \to A_n)\\
&\cong 
\hat{\tau}_n^*C(A_n \to A_0 \to \smdots \to A_{n-1} \to A_n).  
\end{align*}

\subsection{Extra coboundary 
  ${\delta_{n,n}}_!$ for 
  $n \in \mathbb{N}$}
In $\Lambda$, we have $\delta_{n,n} = 
\delta_{0,n}\tau_n$, so we define 
\begin{align*}
{{\delta}_{n,n}}_! : C(n)
& \xrightarrow{{\tau_n}_!}
\hat{\tau}_n^*
  C(A_n \to A_0 \to \smdots \to A_{n-1} \to A_n)\\
& \xrightarrow{\hat{\tau}_n^*{\delta_{0,n}}_!}
\hat{\tau}_n^*\hat{\delta}_{0,n}^*
  C(A_n \to A_1 \to \smdots \to A_{n-1} \to A_n) \\
&\cong
(\widehat{\delta_{0,n}\tau_n})^*
  C(A_n \to A_1 \to \smdots \to A_{n-1} \to A_n)\\
&\cong
\hat{\delta}_{n,n}^*
C(A_n \to A_1 \to \smdots \to A_{n-1} \to A_n).
\end{align*}

\subsection{Extra codegeneracy 
  ${\sigma_{0,n}}_!$ for 
  $n \in \mathbb{N}$}
In $\Lambda$, we have $\sigma_{0,n} = 
\tau_{n+1} \sigma_{n+1,n}$, so we define 
\begin{align*}
{{\sigma}_{0,n}}_! : C(n)
& \xrightarrow{\sigma_{n+1,n!} = id}
\hat{\sigma}_{n+1,n}^*
  C(A_0 \to \smdots \to A_n \to A_0 \to A_0)\\
& \xrightarrow{\hat{\sigma}_{n+1,n}^*\tau_{n+1!}}
\hat{\sigma}_{n+1,n}^*\hat{\tau}_{n+1}^*
  C(A_0 \to A_0 \to \smdots \to A_n \to A_0) \\
&\cong
(\widehat{\tau_{n+1} \sigma_{n+1,n}})^*
  C(A_0 \to A_0 \to \smdots \to A_n \to A_0)\\
&\cong
\hat{\sigma}_{0,n}^*
C(A_0 \to A_0 \to \smdots \to A_n \to A_0).
\end{align*}
