\section{Pullbacks of dg comodules--examples}
%
\begin{eg}[Another definition of C(1)] \label{eg:pb}
Using $F$ and $G$ from Proposition 
\ref{prop:compute_pb}, we can induce 
differentials on $B(\mathcal{A}) \otimes_{\hat{\lambda}} T$ from
$\hat{\lambda}^*C$. We will compute this differential 
for a particular choice of $\lambda$. 
Let $\lambda = \delta_{0,1} \in \Lambda([1], [0])$. 
Fix algebras $A_0$ and $A_1$, and set 
\begin{align*}
\hat{\lambda}: B(1) 
&:= B(A_0 \to A_1 \to A_0) \to 
  B(A_0 \to A_0) =: B(0)\\
C(1) 
&:= C(A_0 \to A_1 \to A_0)\\
C(0) 
&:= C(A_0 \to A_0)
\end{align*}
($\hat{\lambda}$ is given by braces, 
see Section \ref{sec:def_G}.)

Note that 
$\hat{\delta}_{0,1}^* C(0) \cong
[B(1) \otimes_{\hat{\lambda}} 
  T(0)](f_{0,0},f_{1,0}) \cong
C(1)(f_{0,0},f_{1,0})$ 
as comodules where 
$(f_{0,0},f_{1,0}) \in Obj(B(1))$. 
Let 
$\phi_{0,1} \smdots \phi_{0,k_0} |
\phi_{1,1} \smdots \phi_{1,k_1} |
t$ be a typical element of 
$[B(1) \otimes_{\hat{\lambda}} 
  T(0)](f_{0,0},f_{1,0})$
(see \ref{fig:phi|alpha} for notational 
conventions). Then,
\begin{align*}
& \phantom{{}={}}
d_{B(1) \otimes_{\hat{\lambda}} T(0)}
(\phi_{0,1} \smdots \phi_{0,k_0} |
\phi_{1,1} \smdots \phi_{1,k_1} | t)\\
&=
Fd_{\hat{\lambda}^*C(0)}G
(\phi_{0,1} \smdots \phi_{0,k_0} |
\phi_{1,1} \smdots \phi_{1,k_1} | t) \\
&=
[F \circ (d_{B(1)}\otimes id_{C(0)} + 
  id_{B(1)} \otimes d_{C(0)})] \\
& \phantom{{}=  {}}
\big( \sum \limits_{\substack{1 \leq r_0 \leq k_0+1 \\ 
1 \leq r_1 \leq k_1+1}}
(\phi_{0,1} \smdots \phi_{0,r_0-1} |
\phi_{1,1} \smdots \phi_{1,r_1-1} ) \otimes 
((\phi_{0,r_0}\smdots \phi_{0,k_0}) \bullet 
(\phi_{1,r_1}\smdots \phi_{1,k_1}) | t) \big) \\
&= 
d_{C(1)(f_{0,0},f_{1,0})}
(\phi_{0,1} \smdots \phi_{0,k_0} |
\phi_{1,1} \smdots \phi_{1,k_1} | t)
\end{align*}
where the last equality holds by looking at which 
terms from $d_{B(1)}\otimes id_{C(0)} + 
id_{B(1)} \otimes d_{C(0)}$ 
are non-zero after projecting to cogenerators, and 
seeing that those are the same terms as in 
$d_{C(1)}$. So, $\hat{\delta}_{0,1}^*C(0) \cong C(1)$ as 
dg comodules. 
\end{eg}
%
\begin{eg}[Another definition of $C(n)$] 
  \label{eg:pb2}
Let $\lambda = \delta_{0,n} \in 
\Lambda([n],[n-1])$. Fix algebras 
$A_0, \dots A_n$, and set
\begin{align*}
\hat{\lambda}: B(n)
&:= B(A_0 \to A_1 \to \dots \to A_n \to A_0) 
  \to B(A_0 \to A_2 \to A_3 \to \dots \to A_n \to A_0)\\
C(n) 
&:= C(A_0 \to A_1 \to \dots \to A_n \to A_0)
\end{align*}
($\hat{\lambda}$ is given by bracing the 
first and second terms, see Section \ref{sec:def_G}.)

Example \ref{eg:pb} shows
\begin{equation} \label{eq:base_case}
C(1) \cong \hat{\delta}_{0,1}^*C(0)
\end{equation}
as dg comodules. Given Equation 
\ref{eq:base_case} above as a base 
case, we can show by induction that 
$$
C(n) \cong \hat{\delta}_{0,n}^*\dots\hat{\delta}_{0,1}^*C(0)
$$ 
as dg comodules. Suppose that 
$C(W_0 \to \dots \to W_{n-1} \to W_0) 
\cong \hat{\delta}_{0,n-1}^* \dots \hat{\delta}_{0,1}^*
C(W_0 \to W_0)$ for any 
choice of algebras $W_0, \dots W_{n-1}$.
(inductive hypothesis). Then, as 
comodules, we know
\begin{align*}
\hat{\delta}_{0,n}^*\dots\hat{\delta}_{0,1}^*C(0)
& \cong
  \hat{\delta}_{0,n}^*C(A_0 \to A_2 \to \smdots \to A_n \to A_0)
  \quad \textrm{(inductive hypothesis applied to algebras } 
  A_0, A_2,\dots,A_n)\\
& \cong 
  B(n) \otimes_{\hat{\delta}_{0,n}} 
  T(A_0 \to A_2 \to \smdots \to A_n \to A_0)
  \quad \textrm{(Proposition 
  \ref{prop:compute_pb})}\\
& \cong 
  B(n) \otimes_{\hat{\delta}_{0,n}} 
  T(A_0 \to A_0)
  \quad \textrm{(Definition of $T$)} \\
& \cong 
  C(n)
  \quad \textrm{(Definition of $C(n)$)}
\end{align*}
where $T(n)$ are the cogenerators of 
$C(n)$ (see Definition 
\ref{def:cogenerators}).

To show that the differentials coincide, 
we compute 
\begin{align*}
& \phantom{{}={}}
Fd_{\hat{\delta}_{0,n}^*\dots\hat{\delta}_{0,1}^*C(0)}G
  (\phi_{0,1} \smdots \phi_{0,k_0} | \smdots |
  \phi_{n,1} \smdots \phi_{n,k_n} | t) \\
&=
[F \circ (d_{B(n)}\otimes 
  id_{\hat{\delta}_{0,n-1}^*\dots\hat{\delta}_{0,1}^*C(0)} + 
  id_{B(n)} \otimes 
  d_{\hat{\delta}_{0,n-1}^*\dots\hat{\delta}_{0,1}^*C(0)})] \\
& \phantom{{}=  {}}
\big( \sum \limits_{\substack{1 \leq j \leq n\\
  1 \leq r_j \leq k_j+1}}
  (\phi_{0,1} \smdots \phi_{0,r_0-1} | \smdots |
  \phi_{n,1} \smdots \phi_{n,r_1-1} ) \otimes \\
&\phantom{{}= \big( \sum \sum {}}
  ((\phi_{0,r_0}\smdots \phi_{0,k_0}) \bullet 
  (\phi_{1,r_1}\smdots \phi_{1,k_1}) |
  \phi_{2,r_1}\smdots \phi_{2,k_2} | \smdots |
  \phi_{n,r_1}\smdots \phi_{n,k_n} | t) \big) \\
&=
[F \circ (d_{B(n)}\otimes 
  id_{C(A_0 \to A_2 \to \smdots \to A_n \to A_0)} + 
  id_{B(n)} \otimes 
  d_{C(A_0 \to A_2 \to \smdots \to A_n \to A_0)})] \\
& \phantom{{}=  {}}
\big( \sum \limits_{\substack{1 \leq j \leq n\\
  1 \leq r_j \leq k_j+1}}
  (\phi_{0,1} \smdots \phi_{0,r_0-1} | \smdots |
  \phi_{n,1} \smdots \phi_{n,r_1-1}) \otimes \\
&\phantom{{}= \big( \sum \sum {}}
  ((\phi_{0,r_0}\smdots \phi_{0,k_0}) \bullet 
  (\phi_{1,r_1}\smdots \phi_{1,k_1}) |
  \phi_{2,r_1}\smdots \phi_{2,k_2} | \smdots |
  \phi_{n,r_1}\smdots \phi_{n,k_n} | t) \big)
\end{align*}
where the last equality holds by the 
inductive hypothesis. The terms from 
$d_{B(n)}\otimes id_{C(A_0 \to A_2 \to 
\smdots \to A_n \to A_0)} + 
id_{B(n)} \otimes d_{C(A_0 \to A_2 \to 
\smdots \to A_n \to A_0)}$ that
are non-zero after projecting to cogenerators 
are exactly the terms in $d_{C(n)}$. 
So, $C(n) \cong \hat{\delta}_{0,n}^*\dots
\hat{\delta}_{0,1}^*C(0)$ as dg comodules. 
\end{eg}
%
\begin{eg}[Yet another description of C(n)] 
\label{eg:pb3}
Choose a sequence of generating coboundaries 
$\delta_{i_1,1}, \dots, \delta_{i_n,n}$ with 
$0 \leq i_j \leq j-1$, $1 \leq j \leq n$, 
$n>0$. 
Then,
$$
\delta_{i_1,1} \circ \dots \circ \delta_{i_n,n} 
= \delta_{0,1} \circ \dots \circ \delta_{0,n}
= \textrm{unique map in } \Delta([n],[0])
  \subset \Lambda([n],[0]).
$$
This implies that, as functors 
on categories of comodules, 
\begin{align*}
\hat{\delta}_{i_n,n}^* \dots \hat{\delta}_{i_1,1}^*
&=
\widehat{(\delta_{i_1,1} \circ \dots \circ 
  \delta_{i_n,n})}^*
  \quad \textrm{(Proposition \ref{prop:pbs_compose})}\\
&= 
\widehat{(\delta_{0,1} \circ \dots \circ 
  \delta_{0,n})}^*
  \quad \textrm{(Computation above)}\\
&= 
\hat{\delta}_{0,n}^* \dots \hat{\delta}_{0,1}^* 
  \quad \textrm{(Proposition \ref{prop:pbs_compose}).} 
\end{align*}
Since braces are associative, the 
differentials on $\hat{\delta}_{i_n,n}^* 
\dots \hat{\delta}_{i_1,1}^*C(A_0 \to A_0)$ and 
$\hat{\delta}_{0,n}^* 
\dots \hat{\delta}_{0,1}^*C(A_0 \to A_0)$ 
coincide. So, $\hat{\delta}_{i_n,n}^* 
\dots \hat{\delta}_{i_1,1}^*C(A_0 \to A_0)
\cong \hat{\delta}_{0,n}^* 
\dots \hat{\delta}_{0,1}^*C(A_0 \to A_0)$ 
as dg comodules.
\end{eg}
%
\begin{eg}[Pullbacks along codegeneracies]
\label{eg:pb4}
Fix algebras $A_0, \dots, A_n$ and let 
$\sigma_{i,n} \in \Lambda([n],[n+1])$, 
$0 \leq i \leq n$ be a generating 
codegeneracy. Set
\begin{align*}
\hat{\sigma}_{i,n}: B(n)
&:= 
B(A_0 \to \smdots \to A_n \to A_0) \to 
B(A_0 \to \dots \to A_i \to A_i 
    \to \dots A_n \to A_0)\\
C(n) 
&:= 
C(A_0 \to \dots \to A_n \to A_0)\\    
\end{align*}
From Proposition \ref{prop:compute_pb}, we 
know that $\hat{\sigma}_{i,n}^*
C(A_0 \to \dots \to A_i \to A_i \to \dots A_n \to A_0) 
\cong B(n) \otimes_{\hat{\sigma}_{i,n}} 
T(A_0 \to \dots \to A_i \to A_i \to \dots A_n \to A_0) 
\cong C(n)$ as comodules.
To show that the differentials coincide, 
we compute 
\begin{align*}
& \phantom{{}={}}
Fd_{\hat{\sigma}_{i,n}^*
  C(A_0 \to \dots \to A_i \to A_i \to \dots A_n \to A_0)}G
  (\phi_{0,1} \smdots \phi_{0,k_0} | \smdots |
  \phi_{n,1} \smdots \phi_{n,k_n} | t) \\
&=
[F \circ (d_{B(n)}\otimes 
  id_{\hat{\sigma}_{i,n}^*
  C(A_0 \to \dots \to A_i \to A_i \to \dots A_n \to A_0)} + 
  id_{B(n)} \otimes 
  d_{\hat{\sigma}_{i,n}^*
  C(A_0 \to \dots \to A_i \to A_i \to \dots A_n \to A_0)})] \\
& \phantom{{}=  {}}
\big( \sum \limits_{\substack{1 \leq j \leq n\\
  1 \leq r_j \leq k_j+1}}
  (\phi_{0,1} \smdots \phi_{0,r_0-1} | \smdots |
  \phi_{n,1} \smdots \phi_{n,r_1-1} ) \otimes \\
&\phantom{{}= \big( \sum \sum {}}
  (\phi_{0,r_0}\smdots \phi_{0,k_0}| \smdots |
  \phi_{0,r_{i-1}}\smdots \phi_{0,k_{i-1}}| 1 |
  \phi_{0,r_i}\smdots \phi_{0,k_i}| \smdots |
  \phi_{n,r_1}\smdots \phi_{n,k_n} | t) \big).
\end{align*}
Since 1 is a unit for braces, the terms from 
$d_{B(n)} \otimes id_{\hat{\sigma}_{i,n}^*
C(A_0 \to \dots \to A_i \to A_i \to \dots A_n \to A_0)} 
+ id_{B(n)} \otimes d_{\hat{\sigma}_{i,n}^*
C(A_0 \to \dots \to A_i \to A_i \to \dots A_n \to A_0)}$ 
that are non-zero after projecting to cogenerators 
are exactly the terms in $d_{C(n)}$. 
So, $\hat{\sigma}_{i,n}^*
C(A_0 \to \dots \to A_i \to A_i \to \dots A_n \to A_0) 
\cong C(n)$ 
as dg comodules.
\end{eg}
%
\begin{eg}[Pullbacks along rotations]
\label{eg:pb5}
Fix algebras $A_0, \dots, A_n$ and let 
$\tau_n \in \Lambda([n],[n])$ 
be a generating rotation. Set
\begin{align*}
\hat{\tau}_n: B(n) 
&:= B(A_0 \to \smdots \to A_n \to A_0) \to 
B(A_n \to A_0 \to \smdots \to A_n)\\
C(n) &:= 
C(A_0 \to \dots \to A_n \to A_0)
\end{align*}
From Proposition \ref{prop:compute_pb}, we 
know that $\hat{\tau}_n^*
C(A_n \to A_0 \to \smdots \to A_n) 
\cong B(n) \otimes_{\hat{\tau}_n} 
T(A_n \to A_0 \to \smdots \to A_n)$ 
as comodules. Unpacking the 
righthand side, we see that $B(n) 
\otimes_{\hat{\tau}_n} 
T(A_n \to A_0 \to \smdots \to A_n) 
\cong C(A_n \to A_0 \to \smdots \to A_n)$ 
as complexes--the 
isomorphism is given by $\hat{\tau}_n 
\otimes id_{T}$.
\end{eg}