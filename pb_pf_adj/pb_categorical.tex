\subsection{Category-theoretic definition of $\lambda^*$}
\label{sec:pb_defn}
Let $\lambda$ be as above, and let $C$ 
be a conilpotent dg 
comodule over $B_0$. We define 
$\lambda^* C$ as follows:
\begin{equation} \label{eq:pb_defn}
\lambda^*C := 
ker \big( B_1\otimes_\lambda C 
\mathrel{\mathop{\rightrightarrows}^
{\mathrm{id_{B_1}\otimes \Delta_{C}}}_
{\mathrm{(id_{B_1}\otimes \lambda \otimes id_C)
  \circ (\Delta_{B_1}\otimes id_C})}}
B_1 \otimes_\lambda B_0\otimes C \big)
\end{equation}
where $B_1\otimes_\lambda C$ and 
$B_1\otimes_\lambda B_0 \otimes C$ are dg comodules
over $B_1$ defined below. 
For $f \in Obj(B_1)$,
\begin{align*}
[B_1 \otimes_\lambda C](f) 
:&= 
\big(
\bigoplus \limits_{h \in Obj(B_1)} B_1^\bullet(f, h) 
\otimes C^\bullet(\lambda h), 
\Delta(f) = 
\bigoplus \limits_{h} \Delta_{B_1(f, h)} 
\otimes id_{C(\lambda h)}
\big) \\
[B_1 \otimes_\lambda B_0 \otimes C](f) 
:&= 
\big(
\bigoplus \limits_{\substack{h_1 \in Obj(B_1),\\ h_2 \in Obj(B_0)}}
B_1^\bullet(f, h_1) \otimes 
B_0^\bullet(\lambda h_1, h_2) \otimes
C^\bullet(h_2), \\
& \phantom{{}:=\big( {}}
\Delta(f) = 
\bigoplus \limits_{h_1,\,h_2} \Delta_{B_1(f, h_1)} 
\otimes id_{B_0(\lambda h_1, h_2)} 
\otimes id_{C(h_2)}
\big).
\end{align*}
The names of the maps in Equation \ref{eq:pb_defn} 
are also meant to be suggestive. In full detail, 
for $f \in Obj(B_1)$,
$$
{[id_{B_1}\otimes \Delta_{C}]}(f) 
:=
\bigoplus \limits_{h} id_{B_1(f, h)} \otimes \Delta_C(\lambda h)
$$
and 
\begin{align*}
[B_1 \otimes_\lambda C](f)
&\xrightarrow{
{[\Delta_{B_1}\otimes id_C]}(f) :=
\bigoplus \limits_{h} \Delta_{B_1}(f, h) 
  \otimes id_{C(\lambda h)}}
%
\bigoplus \limits_{h_1,h_2 \in Obj(B_1)}
B_1(f, h_1) \otimes B_1(h_1, h_2) 
\otimes C(\lambda h_2) \\
%
&\xrightarrow{
{[id_{B_1}\otimes \lambda \otimes id_C]}(f) := 
\bigoplus \limits_{h_1,\,h_2} id_{B_1(f, h_1)} \otimes \lambda(h_1,h_2)
\otimes id_{C(\lambda h)}}
[B_1 \otimes_\lambda B_0 \otimes C](f). 
\end{align*}
That the kernel is well-defined follows formally from the 
abelianness of the category of chain complexes, but it is also 
easy to check that the induced differentials from 
$[B_1\otimes_\lambda C](f)$ 
on the kernel are well-defined. 
Since $\Delta_{\lambda^*C}$ is 
induced by $\Delta_{B_1}$, we have
that $\Delta_{\lambda^*C}$ also satisfies 
coassociativity, counitality and 
conilpotency.

Next, we will define $\lambda^*$ on morphisms.
Let $F:C\to D$ be a map of conilpotent 
dg comodules over $B_0$. 
By the universal property of $\lambda^*D$, we can define a 
morphism $\lambda^*F:\lambda^*C \to \lambda^*D$ by giving a 
morphism from $(\lambda^*F)^\prime:\lambda^*C \to 
B_1 \otimes_\lambda D$ such that the two maps 
\begin{equation}\label{eqn:coker}
(id_{B_1}\otimes \Delta_D) \circ (\lambda^*F)^\prime, \>
(id_{B_1}\otimes \lambda \otimes id_D)\circ 
(\Delta_{B_1}\otimes id_D) \circ (\lambda^*F)^\prime: 
\lambda^*C \to B_1 \otimes_\lambda D 
\rightrightarrows
B_1 \otimes_\lambda B_0 \otimes D
\end{equation}
coincide. We define $(\lambda^*F)^\prime$ as follows:
$$
(\lambda^*F)^\prime:
\lambda^*C
\xrightarrow[inclusion]{canonical}
B_1 \otimes_\lambda C 
\xrightarrow{id_{B_1}\otimes F}
B_1 \otimes_\lambda D
$$
It's easy to check that the two maps in Equation 
\ref{eqn:coker} coincide: 
Let $b\otimes c$ be an arbitrary element of $\lambda^*C(f)
\hookrightarrow [B_1\otimes_\lambda C](f)$. 
Then,
\begin{align*}
[(id_{B_1}\otimes \Delta_D) \circ (\lambda^*F)^\prime] (b\otimes c) 
&= \sum \limits_{(Fc)} b \otimes (Fc)_{(1)} \otimes (Fc)_{(2)} \\
&= \sum \limits_{(c)} b \otimes Fc_{(1)} \otimes Fc_{(2)} 
\quad \textrm{(F is a map of comodules)}\\
&= [(id_{B_1}\otimes F \otimes F) \circ (id_{B_1}\otimes \Delta_C)] 
(b\otimes c) \\
&= [(id_{B_1}\otimes F\otimes F) \circ 
(id_{B_1}\otimes \lambda \otimes id_C)\circ 
(\Delta_{B_1}\otimes id_C)] (b\otimes c)\\
&\quad \textrm{($b\otimes c$ is in the kernel)}\\
&= \sum \limits_{(b)} b_{(1)}\otimes \lambda b_{(2)} \otimes Fc \\
&= [(id_{B_1}\otimes \lambda \otimes id_D)\circ 
(\Delta_{B_1}\otimes id_D) \circ (\lambda^*F)^\prime] 
(b \otimes c).
\end{align*}
So, $\lambda^*F$ is well-defined. In summary, we have commuting 
diagrams:
\begin{equation}\label{cd:lambda^*f}
\begin{CD}
\lambda^*C
@>\substack{\textrm{canonical}\\ \textrm{inclusion}}>>
B_1\otimes_\lambda C \\
@V\lambda^*FVV 
@VVid_{B_1}\otimes F = \textrm{ map inducing } \lambda^*FV \\
\lambda^*D
@>\substack{\textrm{canonical}\\ \textrm{inclusion}}>>
B_1\otimes_\lambda D
\end{CD}
\end{equation}
Finally, it is straightforward to see that $\lambda^*$ 
is a functor, i.e., that $\lambda^*$ preserves composition
of morphisms: Let $C \overset{F}{\to} D \overset{G}{\to} E$ be 
composable morphisms of dg comodules over $B_0$. 
The maps inducing 
$\lambda^*F$, $\lambda^*G$ and $\lambda^*(G\circ F)$ are 
$id_{B_1}\otimes F$, $id_{B_1}\otimes G$ and 
$id_{B_1}\otimes GF$, respectively. The inducing maps
respect composition--$(id_{B_1}\otimes G) \circ 
(id_{B_1}\otimes F) = id_{B_1}\otimes GF$--and by 
the commuting diagrams \ref{cd:lambda^*f}, the 
functor $\lambda^*$ does as well.
%
\begin{prop} \label{prop:pbs_compose}
Let $F:B_2 \to B_1$ and $G: B_1 \to B_0$ 
be functors between dg cocategories $B_2$, $B_1$ and 
$B_0$. Let $M$ be a dg comodule over $B_0$. Then,
$$
(GF)^*M \cong F^*G^*M.
$$
\end{prop}
%
\begin{proof}
We will prove the proposition by showing that 
$F^*G^*M$ satisfies the universal property of 
$(GF)^*M$. First, let $N$ be a dg comodule over $B_2$ 
and $H: N \to B_2 \otimes_{GF} M$ be a map of 
dg comodules such that the two maps 
\begin{equation}\label{eq:H_reln}
(id_{B_2} \otimes GF \otimes id_M) \circ 
(\Delta_{B_2}\otimes id_M) \circ H, \>
(id_{B_2} \otimes \Delta_M) \circ H: 
N \to B_2 \otimes_{GF} M
\rightrightarrows 
B_2 \otimes_{GF} \otimes B_0 \otimes M
\end{equation}
coincide. We will show that $H$ determines a 
map of dg comodules $\tilde{H}: N \to F^*G^*M$. 
Let $x \in Obj(B_2)$. Define 
\begin{align*}
H^\prime_x: N(x)
&\xrightarrow{H_x}
\bigoplus \limits_{y \in Obj(B_2)}
  B_2(x,y) \otimes M(GFy)\\
&\xrightarrow{F \otimes id_M}
\bigoplus \limits_{y \in Obj(B_2)}
  B_1(Fx,Fy) \otimes M(GFy)\\
&\subset
[B_1 \otimes_G M](Fx).  
\end{align*}
The image of $H^\prime_x$ lands in 
$G^*M(Fx)$, a subcomplex of 
$[B_1 \otimes_G M](Fx)$; checking 
this is straightforward using the 
universal property of $G^*M$, the 
fact that $F$ commutes with the 
coproducts, and Equation \ref{eq:H_reln}. 
So, for 
each $x \in Obj(B_2)$, we have a map  
of complexes $H^\prime_x: N(x) \to 
G^*M(Fx)$. Now define $\tilde{H}$ as 
follows:
\begin{align*}
\tilde{H}_x: N(x) 
&\xrightarrow{\Delta_N}
\prod \limits_{y \in Obj(B_2)}
  B_2(x,y) \otimes N(y)\\
&\xrightarrow{\prod id_{B_2} \otimes H^\prime_y}  
\prod \limits_{y \in Obj(B_2)}
  B_2(x,y) \otimes G^*M(Fy)\\
&\subset
[B_2 \otimes_F G^*M](x).
\end{align*}
Showing that $\tilde{H}$ lands in 
$G^*F^*M$, a subcomodule of 
$[B_2 \otimes_F G^*M]$, is also 
straightforward; we only need that 
$F$ and $H$ commute with the appropriate 
coproducts, and that the cocomposition 
on $B_2$ is coassociative. So, for each 
$x \in Obj(B_2)$, we have a map 
$\tilde{H}_x: N(x) \to G^*F^*M(x)$. 
It's clear that $\tilde{H}$ is a map 
of dg comodules since all of the maps in 
the composition of $\tilde{H}$ are maps 
of dg comodules.

Now, let $\tilde{H}: N \to F^*G^*M$ 
be a map of dg comodules over $B_2$. 
We will show 
that $\tilde{H}$ determines a map of dg 
comodules $H: N \to B_2 \otimes_GF M$ 
satisfying Equation \ref{eq:H_reln}. 
For $x \in Obj(B_2)$, let $H$ be 
defined as follows:
\begin{align*}
H_x: N(x)
&\xrightarrow{\tilde{H}_x}
F^*G^*M(x)\\
&\xrightarrow[inclusion]{canonical}
\bigoplus \limits_{\substack{
	y \in Obj(B_2)\\
	z_1 \in Obj(B_1)}}
  B_2(x,y) \otimes B_1(Fy, z_1) \otimes 
  M(Gz_1)\\
&\xrightarrow{id_{B_2} \otimes \epsilon_{B_1} 
   \otimes id_M}  
\bigoplus \limits_{y \in Obj(B_2)}
  B_2(x,y) \otimes M(GFy).
\end{align*}
The universal property of $G^*M$ implies 
that $(id_{B_2} \otimes \Delta_M) \circ H$ 
is equal to:
\begin{align*}
N(x)
&\xrightarrow{\tilde{H}_x} 
\bigoplus \limits_{\substack{
	y \in Obj(B_2)\\
	z_1 \in Obj(B_1)}}
  B_2(x,y) \otimes B_1(Fy, z_1) \otimes 
  M(Gz_1)\\
&\xrightarrow[(id_{B_2} \otimes \Delta_{B_1} 
  \otimes id_M)]{(id_{B_2} \otimes id_{B_1}
  \otimes G \otimes id_M)\circ}
\bigoplus \limits_{\substack{
	y \in Obj(B_2)\\
	y_1, z_1 \in Obj(B_1)}}
  B_2(x,y) \otimes B_1(Fy, y_1) \otimes 
  B_0(Gy_1, Gz_1) \otimes M(Gz_1) \\
&\xrightarrow{id_{B_2} \otimes \epsilon_{B_1}
  \otimes id_{B_0} \otimes id_M}
\bigoplus \limits_{\substack{
	y \in Obj(B_2)\\
	z_1 \in Obj(B_1)}}
  B_2(x,y) \otimes B_0(GFy, Gz_1) \otimes M(Gz_1).   
\end{align*}
On the other hand, the universal property 
of $F^*$ implies that $(id_{B_2} \otimes GF 
\otimes id_M) \circ (\Delta_{B_2}\otimes id_M) 
\circ H$ is equal to:
\begin{align*}
N(x)
&\xrightarrow{\tilde{H}_x} 
\bigoplus \limits_{\substack{
	y \in Obj(B_2)\\
	z_1 \in Obj(B_1)}}
  B_2(x,y) \otimes B_1(Fy, z_1) \otimes 
  M(Gz_1)\\
&\xrightarrow[(id_{B_2} \otimes \Delta_{B_1} 
  \otimes id_M)]{(id_{B_2} \otimes G
  \otimes id_{B_1} \otimes id_M)\circ}
\bigoplus \limits_{\substack{
	y \in Obj(B_2)\\
	y_1, z_1 \in Obj(B_1)}}
  B_2(x,y) \otimes B_0(GFy,Gy_1) \otimes
  B_1(y_1, z_1) \otimes M(Gz_1) \\
&\xrightarrow{id_{B_2} \otimes id_{B_0} 
  \otimes \epsilon_{B_1} \otimes id_M}
\bigoplus \limits_{\substack{
	y \in Obj(B_2)\\
	z_1 \in Obj(B_1)}}
  B_2(x,y) \otimes B_0(GFy, Gz_1) \otimes M(Gz_1).   
\end{align*}
So, the difference between the two maps 
in Equation \ref{eq:H_reln} comes down 
to the difference between 
$(\epsilon_{B_1} \otimes G) \circ \Delta_{B_1}$ 
and $(G \otimes \epsilon_{B_1}) \circ \Delta_{B_1}$. 
However, by the counitality of $B_1$, both of 
these maps are equal to $G$. So, $H$ satisfies 
Equation \ref{eq:H_reln}.
\end{proof}