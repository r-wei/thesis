\subsection{Explicit description of 
$\hat{\lambda}^*C(\mathcal{A}^\prime)$}
Let $\lambda$ be a morphism in $\Lambda$. 
Recall that from Section \ref{sec:def_G}
that we have a functor $\hat{\lambda}: 
B(\mathcal{A}) \to B(\mathcal{A}^\prime)$. 
Applying the constructions in 
Section \ref{sec:pb_defn} to
$\hat{\lambda}$, we get a functor 
$\hat{\lambda}^*$ from the 
category of conilpotent 
dg comodules over $B(\mathcal{A}^\prime)$ 
to the category 
of conilpotent dg comodules 
over $B(\mathcal{A})$. 
Below, we compute explicitly the complexes 
$[\hat{\lambda}^*C(\mathcal{A}^\prime)](f)$ 
for $f \in Obj(B(\mathcal{A}))$.
%
\begin{prop}\label{prop:compute_pb}
Fix $f_0 \in Obj(B(\mathcal{A}))$. As comodules,
\begin{equation}\label{eq:compute_pb}
[\hat{\lambda}^*C(\mathcal{A}^\prime)](f_0) \cong 
[B(\mathcal{A}) \otimes_{\hat{\lambda}} T(\mathcal{A}^\prime)](f_0) := 
\bigoplus \limits_{h \in Obj(B(\mathcal{A}))}
B(\mathcal{A})(f_0, h) \otimes T(\mathcal{A}^\prime)(\hat{\lambda} h)
\end{equation}
where $T(\mathcal{A}^\prime)(\hat{\lambda} h)$ are the cogenerators of 
$C(\mathcal{A}^\prime)(\hat{\lambda} h)$ 
(see Section \ref{def:cogenerators}).
\end{prop}
%
\begin{rem}
Proposition \ref{prop:compute_pb} holds 
for any quasi-cofree comodule over 
$B(\mathcal{A}^\prime)$. The proof is the same.
\end{rem}
%
\begin{proof}[Proof of Proposition \ref{prop:compute_pb}]
To simplify notation in this proof, 
we will drop all references to $f_0$ and,
when unambiguous, references to $\mathcal{A}^\prime$. 
In other words, in this proof only,
\begin{align*}
C := C(\mathcal{A}^\prime) 
&\textrm{ will denote } C(\mathcal{A}^\prime)(f_0), \\
\hat{\lambda}^*C := \hat{\lambda}^*C(\mathcal{A}^\prime) 
&\textrm{ will denote } [\hat{\lambda}^*C(\mathcal{A}^\prime)](f_0), \\
B(\mathcal{A}) \otimes_{\hat{\lambda}} T 
&\textrm{ will denote } [B(\mathcal{A}) \otimes_{\hat{\lambda}} T(\mathcal{A}^\prime)](f_0), \\
B(\mathcal{A}) \otimes_{\hat{\lambda}} C 
&\textrm{ will denote } [B(\mathcal{A}) \otimes_{\hat{\lambda}} C(\mathcal{A}^\prime)](f_0), \\
B(\mathcal{A}) \otimes_{\hat{\lambda}} B(\mathcal{A}^\prime) \otimes C 
&\textrm{ will denote } 
[B(\mathcal{A}) \otimes_{\hat{\lambda}} B(\mathcal{A}^\prime) \otimes C(\mathcal{A}^\prime)](f_0).
\end{align*}
%
To prove the proposition, 
we will give maps
$$
F: \hat{\lambda}^*C \rightleftarrows B(\mathcal{A}) \otimes_{\hat{\lambda}} T:G
$$
and show that $F\circ G = id_{B(\mathcal{A}) \otimes_{\hat{\lambda}} T}$ 
and $G \circ F = id_{\hat{\lambda}^*C}$.
We define $F$ as follows: 
$$
F:\hat{\lambda}^*C 
\xrightarrow[inclusion]{canonical}
B(\mathcal{A}) \otimes_{\hat{\lambda}} C
\xrightarrow[cogenerators]{project\;onto}
B(\mathcal{A}) \otimes_{\hat{\lambda}} T.
$$
To define $G$, we will give a map 
$G^\prime: B(\mathcal{A}) \otimes_{\hat{\lambda}} T \to 
B(\mathcal{A}) \otimes_{\hat{\lambda}} C$, and show that the image of 
$G^\prime$ lands in $\hat{\lambda}^*C$. We define 
$G^\prime$ as follows:
$$
G^\prime(b \otimes t) = 
\sum \limits_{(b)} b_{(1)} \otimes \hat{\lambda} b_{(2)} \cdot t
$$
where $b \otimes t \in B(\mathcal{A}) \otimes_{\hat{\lambda}} T$ and $\hat{\lambda} 
b_{(2)} \cdot t$ are elements of $C(\mathcal{A}^\prime)(\hat{\lambda} h)$ 
written in terms of cogenerators.

To prove that the image of $G^\prime$ lands in 
$\hat{\lambda}^*C$, we need to show that the two maps 
$$
(id_{B(\mathcal{A})}\otimes \Delta_{C}) \circ G^\prime, \>
(id_{B(\mathcal{A})}\otimes \hat{\lambda} \otimes id_C)\circ 
(\Delta_{B(\mathcal{A})}\otimes id_C) \circ G^\prime: 
B(\mathcal{A}) \otimes_{\hat{\lambda}} T
\to B(\mathcal{A}) \otimes_{\hat{\lambda}} C
\rightrightarrows
B(\mathcal{A}) \otimes_{\hat{\lambda}} B(\mathcal{A}^\prime) \otimes C
$$
coincide. We have
\begin{align*}
[(1\otimes \Delta_{C}) \circ G^\prime](b \otimes t) 
&= 
\sum \limits_{(b),\, (\hat{\lambda} b)} b_{(1)} \otimes 
(\hat{\lambda} b_{(2)})_{(1)} \otimes 
(\hat{\lambda} b_{(2)})_{(2)} \cdot t \\
&= 
\sum \limits_{(b)} b_{(1)} \otimes 
\hat{\lambda} b_{(2)} \otimes 
\hat{\lambda} b_{(3)} \cdot t \\
&= 
[(id_{B(\mathcal{A})}\otimes \hat{\lambda} \otimes id_C)\circ 
(\Delta_{B(\mathcal{A})}\otimes id_C) \circ G^\prime]
(b\otimes t)
\end{align*}
where the second equality holds since $\hat{\lambda}$ 
is a map of cocategories
and $\Delta_{B(\mathcal{A})}$ is coassociative.

It's clear from the definitions that $F$ and $G$ are 
maps of comodules and that 
$F\circ G = id_{B(\mathcal{A})\otimes_{\hat{\lambda}} T}$. All that remains 
is to show that $G \circ F = id_{\hat{\lambda}^*C}$. 
Let $\kappa = \Sigma_i b_i \otimes \beta_i \cdot t_i$ be an 
arbitrary element of $\hat{\lambda}^*C \hookrightarrow 
B(\mathcal{A}) \otimes_{\hat{\lambda}} C$ where $\beta_i \cdot t_i$ are elements 
of $C(\mathcal{A}^\prime)(\hat{\lambda} h)$ written in terms of cogenerators. 
Then, 
\begin{equation*}
GF(\kappa) = 
GF(\Sigma_i b_i \otimes \beta_i \cdot t_i) = 
\sum \limits_{\substack{i, \\ \beta_i = 1, \\ (b_i)}} 
{b_i}_{(1)} \otimes \hat{\lambda} {b_i}_{(2)} 
\cdot t_i.
\end{equation*}
We can divide the terms in $\kappa$ into two groups: 
(a) terms in which $\beta_i = 1 \in k$ and (b) terms in which $\beta_i
\neq 1 \in k$. Likewise, we can divide the terms in 
$GF(\kappa)$ into (a) terms in which $\hat{\lambda} {b_i}_{(2)} = 1$ 
and (b) terms in which $\hat{\lambda} {b_i}_{(2)} \neq 1$. 
From the definitions of $F$ and $G$, it's clear that 
the Group A terms in $\kappa$ are exactly the Group A 
terms in $GF(\kappa)$. 

To show that the Group B terms 
are the same, let $b_i \otimes \beta_i \cdot t_i$ be 
an arbitrary Group B term in $\kappa$.
Then, there is a term $b_i \otimes \beta_i \otimes t_i$ 
in $(id_{B(\mathcal{A})} \otimes \Delta_{C}) \kappa$. Since 
$(id_{B(\mathcal{A})} \otimes \Delta_{C}) \kappa = 
(id_{B(\mathcal{A})} \otimes \hat{\lambda} \otimes id_C) \circ 
(\Delta_{B(\mathcal{A})} \otimes id_C) \kappa$, 
there must be a Group A term, $b_{j_i} \otimes t_{j_i}$, 
in $\kappa$ such that $b_i \otimes \beta_i \otimes t_i$ 
is one of the terms in the sum 
$[(id_{B(\mathcal{A})} \otimes \hat{\lambda} \otimes id_C) \circ 
(\Delta_{B(\mathcal{A})} \otimes id_C)] (b_{j_i} \otimes t_{j_i}) = 
\sum \limits_{(b_{j_i})} {b_{j_i}}_{(1)} \otimes 
\hat{\lambda} {b_{j_i}}_{(2)} \otimes t_{j_i}$. Thus, 
$b_i \otimes \beta_i \cdot t_i$ is a Group B term in 
$GF(\kappa)$. 

Now let ${b_i}_{(1)} \otimes \hat{\lambda} {b_i}_{(2)} 
\cdot t_i$ be an arbitrary Group B term in $GF(\kappa)$. 
Then, ${b_i}_{(1)} \otimes \hat{\lambda} {b_i}_{(2)} \otimes t_i$ 
is a term in 
$(id_{B(\mathcal{A})} \otimes \hat{\lambda} \otimes id_C) \circ 
(\Delta_{B(\mathcal{A})} \otimes id_C) \kappa = 
(id_{B(\mathcal{A})} \otimes \Delta_{C}) \kappa 
$. So, there is a Group B term, $b_{j_i} \otimes 
\beta_{j_i} \cdot t_{j_i}$, in $\kappa$ such that 
${b_i}_{(1)} \otimes \hat{\lambda} {b_i}_{(2)} 
\otimes t_i$ is one of the terms in the sum 
$(id_{B(\mathcal{A})} \otimes \Delta_{C}) 
(b_{j_i} \otimes \beta_{j_i} \cdot t_{j_i}) = 
\sum \limits_{(\beta_{j_i})}
b_{j_i} \otimes {\beta_{j_i}}_{(1)} 
\otimes {\beta_{j_i}}_{(2)} \cdot t_{j_i}$. 
Since $t_i$ is a cogenerator, the only term in the sum 
that could be equal to ${b_i}_{(1)} \otimes 
\hat{\lambda} {b_i}_{(2)} \otimes t_i$ is 
$b_{j_i} \otimes \beta_{j_i} \otimes t_{j_i}$. 
Thus, ${b_i}_{(1)} \otimes \hat{\lambda} {b_i}_{(2)} 
\cdot t_i$ is a Group B term in $\kappa$.
\end{proof}