\section{A sheafy-cyclic object in dg cocategories}\label{sec:cyclic_B(n)}
We would like to say that we have a functor 
from Connes cyclic category $\Lambda$ 
(see Appendix \ref{chap:lambda} for 
generators and relations) to the category of dg 
cocategories where $[n] \mapsto B(n)$, 
but defining $B(n)$ involved choosing a 
sequence of algebras $A_0, \dots, A_n$. 

Instead, we have the following: 
Let $X: \Lambda \to Set$ be the functor 
that sends $[n]$ to the set of diagrams 
$A_0 \to A_1 \to \dots \to A_n \to A_0$ 
where the $A_i$'s are algebras. On 
generating morphisms in $\Lambda$, 
$X$ acts as follows: Let $\mathcal{A} = 
(A_0 \to \smdots \to A_n \to A_0)
\in X([n])$. 
\begin{align*}
X(\tau_n): \mathcal{A}
  &\mapsto (A_n \to A_0 \to \smdots \to A_{n-1} \to A_n)\\
X(\delta_{j,n}): \mathcal{A}
  &\mapsto 
    (A_0 \to \smdots \to A_j\longrightarrow A_{j+2 
    \textrm{ (mod n+1)}} \to \smdots \to A_n \to A_0) \\
X(\sigma_{i,n}): \mathcal{A}
  &\mapsto 
  \begin{cases}
    (A_0 \to \smdots \to 
  		   A_i \to A_i \to \smdots A_n \to A_0)
    & 1 \leq i \leq n\\
    (A_0 \to \smdots \to A_n \to A_0 \to A_0)
    & i=n+1	
  \end{cases}	   
\end{align*}
It's straightforward to 
check that $X$ respects composition of morphisms. 
Now, let $\chi$ be the category with objects given by
diagrams $A_0 \to \smdots \to A_n \to A_0$ 
where the $A_i$'s are algebras and $n \in \mathbb{N}$. 
Morphisms in $\chi$ are the pointwise images of 
$X$. In other words, the set of morphisms in
$\chi$ is $\{ X(\lambda)|_x: \lambda \in 
\Lambda([n],[m]), x \in X([n]) \}$. We will give 
a functor, $\mathcal{G}$, from $\chi$ to the 
category of dg 
cocategories; (this is our sheafy-cyclic object, 
i.e., a \textbf{sheafy-cyclic} object in a 
category $\mathcal{C}$ is a functor $\chi \to 
\mathcal{C}$).
%
\subsection{Aside on notation:} 
Fix $\lambda:[n] \to [m]$ in $\Lambda$ 
and $x \in X([n])$. To define $\mathcal{G}$, 
we will need to define a functor $\mathcal{G}
\big(X(\lambda)|_x \big): B(x) \to B(X(\lambda)(x))$. 
To simplify notation, we will denote 
$\hat{\lambda} := \mathcal{G}
\big(X(\lambda)|_x \big)$ and write 
$\hat{\lambda}: B(x) \to B(\lambda x)$. 
Technically, we are losing information 
about the $x$ when we write 
$\hat{\lambda}$ instead of $\mathcal{G}
\big(X(\lambda)|_x \big)$, but 
we will be clear 
about the source and target when needed.
%
\subsection{Definition of $\mathcal{G}$}
\label{sec:def_G}
Now, we will define $\mathcal{G}$. On objects, 
\begin{align*}
\mathcal{G}: (A_0 \to \smdots \to A_n \to A_0)
&\mapsto 
B(A_0 \to \smdots \to A_n \to A_0)\\
&\phantom{{}\mapsto{}}
\textrm{(see Section \ref{sec:B(n)} for 
definition of $B(\cdot)$)}
\end{align*}
On generating morphisms in $\chi$, set $\mathcal{A} = 
(A_0 \to \smdots \to A_n \to A_0) \in Obj(\chi)$, and 
define 
\begin{align*}
\hat{\tau}_n
  &\mapsto 
  \begin{cases}
  B(\mathcal{A}) \longrightarrow 
  B(A_n \to A_0 \to \smdots \to A_{n-1} \to A_n) \\
  \textrm{objects: } (f_0,f_1, \smdots, f_n) \mapsto 
  (f_n, f_0, \smdots, f_{n-1}) \\
  \textrm{morphisms: } \phi_{0,1}\smdots \phi_{0,k_0} | \smdots |
	\phi_{n,1}\smdots\phi_{n,k_n} \mapsto 
	\phi_{n,1}\smdots\phi_{n,k_n} | \smdots |
	\phi_{n-1,1}\smdots\phi_{n-1,k_{n-1}}
  \end{cases}\\	
%
\hat{\delta}_{j,n}
  &\mapsto 
  \begin{cases}
  B(\mathcal{A}) \longrightarrow 
  B(A_0 \to \smdots \to A_j\to A_{j+2 \textrm{ (mod n+1)}} 
      \to \smdots \to A_0) \\
  \textrm{objects: } (f_0,f_1, \smdots, f_n) \mapsto 
  (f_0, \smdots,f_{j+1}\circ f_j, \smdots, f_n) \\
  \textrm{morphisms: } \phi_{0,1}\smdots \phi_{0,k_0} | \smdots |
	\phi_{n,1}\smdots\phi_{n,k_n} \mapsto \\
  \phantom{{}morphisms: {}} 
	\phi_{0,1}\smdots \phi_{0,k_0} | \smdots |
	(\phi_{j,1}\smdots \phi_{j,k_j}) \bullet
	(\phi_{j+1,1}\smdots \phi_{j+1,k_{j+1}}) | \smdots |
	\phi_{n,1}\smdots\phi_{n,k_n} 
  \end{cases}\\
%
\underset{1 \leq i \leq n}{\hat{\sigma}_{i,n}}
  &\mapsto 
  \begin{cases}
  B(\mathcal{A}) \longrightarrow 
  B(A_0 \to \smdots \to A_i\to A_i
      \to \smdots \to A_0) \\
  \textrm{objects: } (f_0,f_1, \smdots, f_n) \mapsto 
  (f_0, \smdots,f_{i-1}, id_{A_i}, f_i, \smdots, f_n) \\
  \textrm{morphisms: } \phi_{0,1}\smdots \phi_{0,k_0} | \smdots |
	\phi_{n,1}\smdots\phi_{n,k_n} \mapsto \\
  \phantom{{}morphisms: {}} 
	\phi_{0,1}\smdots \phi_{0,k_0} | \smdots |
	\phi_{i-1,1}\smdots \phi_{i-1,k_{i-1}} | 1 |
	\phi_{i,1}\smdots \phi_{i,k_i} | \smdots |
	\phi_{n,1}\smdots\phi_{n,k_n} \\
  \phantom{{}morphisms: {}}	
  	1 \in k =\textrm{degree 0 component of }
  	Bar(C^\bullet(A_i, A_i))
  \end{cases}\\    
%
\hat{\sigma}_{n+1,n}
  &\mapsto 
  \begin{cases}
  B(\mathcal{A}) \longrightarrow 
  B(A_0 \to \smdots \to A_n\to A_0 \to A_0) \\
  \textrm{objects: } (f_0,f_1, \smdots, f_n) \mapsto 
  (f_0, \smdots, f_n, id_{A_0}) \\
  \textrm{morphisms: } \phi_{0,1}\smdots \phi_{0,k_0} | \smdots |
  \phi_{n,1}\smdots\phi_{n,k_n} \mapsto \\
  \phantom{{}morphisms: {}} 
  \phi_{0,1}\smdots \phi_{0,k_0} | \smdots |
  \phi_{n,1}\smdots\phi_{n,k_n} | 1 \\
  \phantom{{}morphisms: {}} 
    1 \in k =\textrm{degree 0 component of }
    Bar(C^\bullet(A_0, A_0))  
  \end{cases}    
\end{align*}
It's straightforward to check that $\mathcal{G}$ is a 
functor (i.e., that composition of morphisms and the relations 
are preserved). The only facts we need are that $\bullet$ is 
an associative map of complexes and that 
$1 \bullet (\phi_0 \smdots \phi_k) = 
(\phi_0 \smdots \phi_k) \bullet 1 = (\phi_0 \smdots \phi_k)$ 
where the 1's are in the degree 0 components of
$Bar(C^\bullet(A_i, A_i))$ for the appropriate $A_i$ 
(see Appendix Section \ref{sec:def_braces}).