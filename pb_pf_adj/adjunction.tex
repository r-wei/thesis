\section{Adjunction between $\lambda^*$ and $\lambda_\#$}\label{sec:cobar_ref}
In this section, we define $\lambda_\#$, 
the left adjoint to $\lambda^*$. 
More precisely, for any functor, $\lambda:
B_1 \to B_0$ between 
conilpotent dg cocategories, 
we define a functor 
$\lambda_\#$ from the category of 
conilpotent dg comodules 
over $B_1$ to the category of 
conilpotent dg 
comodules over $B_0$.
The adjunction will be used to show 
that structures we've established 
for $(B(n), C(n))$ still exist 
after we pass from cocategories and 
comodules to categories and modules
by applying (a categorified) $Cobar(-)$ 
to $(B(n), C(n))$ (see Section 
\ref{sec:cobar}). If a lesson 
of this thesis is that working with
cocategories is more tractable 
than with categories, then the reader 
may skip this section or save it until 
s/he is ready for Section 
\ref{sec:use_adjunction}.
%
\subsection{The functors $\lambda_\#$}
\label{sec:pf_defn}
Let $\lambda:B_1 \to B_0$ be a functor 
between conilpotent dg 
cocategories. Let $C$ be a 
conilpotent dg comodule over $B_1$.
We define $\lambda_\# C$ as follows: 
for $f \in Obj(B_0)$,
\begin{align*}
\lambda_\# C(f) := 
\big( \bigoplus \limits_{f^\prime \in \lambda^{-1}f}
C^\bullet(f^\prime),\\
\phantom{{}\lambda_\# C(f):=\big({}}
\Delta_{\lambda_\#C}(f):
\bigoplus \limits_{f^\prime \in \lambda^{-1}f}
C^\bullet(f^\prime)
&
\xrightarrow{\bigoplus \limits_{f^\prime} 
  \Delta_{C^\bullet}(f^\prime)}
\bigoplus \limits_{
  \substack{f^\prime \in \lambda^{-1}f\\h^\prime \in Obj(B_1)}}
 B_1^\bullet(f^\prime,h^\prime) \otimes C^\bullet(h^\prime) \\
&
\xrightarrow{\bigoplus \limits_{h^\prime,f^\prime} \lambda 
\otimes id_{C^\bullet(h^\prime)}}
\bigoplus \limits_{h^\prime \in Obj(B_1)}
  B_0^\bullet(f,\lambda h^\prime)
  \otimes C^\bullet(h^\prime) \\
&
\xrightarrow{include}
\bigoplus \limits_{h \in Obj(B_0)}
  B_0^\bullet(f,h) \otimes 
  (\bigoplus \limits_{h^\prime \in \lambda^{-1}h}
  C^\bullet(h^\prime))   
\big).
\end{align*}
To check that $\Delta_{\lambda_\#C}$ is well-defined, 
we need that the image of the first map, 
$\bigoplus \limits_{f^\prime} 
  \Delta_{C^\bullet}(f^\prime)$, 
is a finite sum. This is true since 
$C$ being conilpotent implies that 
the image of $\Delta_{C^\bullet}(f^\prime)$ 
is a finite sum for each $f^\prime \in
Obj(B_1)$. If $\lambda^{-1}f$ is empty, 
we set $\lambda_\# C(f) := 0$.
It is straightforward to check 
that $(\lambda_\#C, \Delta_{\lambda_\#C})$ 
is coassociative, conilpotent
and coaugmented. 
We will call $\lambda_\#$ 
``co-restriction of scalars''.

Let $F:C \to D$ be map of dg comodules 
over $B_1$. We define $\lambda_\# F$ as follows:
$$
(\lambda_\# F)_f : 
\lambda_\# C (f) = 
\bigoplus \limits_{f^\prime \in \lambda^{-1}f}
C^\bullet(f^\prime)
\xrightarrow{\bigoplus \limits_{f^\prime \in \lambda^{-1}f}
F_{f^\prime}} 
\bigoplus \limits_{f^\prime \in \lambda^{-1}f}
D^\bullet(f^\prime)
= \lambda_\# D (f).
$$
It's straightforward to check that $\lambda_\#$ 
is a functor (i.e., respects composition of 
morphisms).

%
\subsection{Adjunction}
\begin{prop}\label{prop:adjunction}
Given a functor between 
conilpotent dg cocategories, 
$\lambda: B_1 \to B_0$, let
\begin{equation*}
\lambda^*:
\let\scriptstyle\textstyle
\substack{
  \textrm{Category of}\\
  \textrm{conilpotent}\\
  \textrm{dg comodules over }B_0}
\leftrightarrows
\let\scriptstyle\textstyle
\substack{
  \textrm{Category of}\\
  \textrm{conilpotent}\\
  \textrm{dg comodules over }B_1}
: \lambda_\#
\end{equation*}
be the functors defined in Sections 
\ref{sec:pb_defn} and \ref{sec:pf_defn}. Then,
$\lambda_\#$ is left adjoint to $\lambda^*$.
\end{prop}
%
\begin{rem} \label{rem:adjunction}
Proposition \ref{prop:adjunction} 
is a categorified co-version of the 
adjunction between extension of scalars 
(left) and restriction of scalars (right) 
for modules over algebras.
\end{rem}
%
\begin{proof}[Proof of Proposition \ref{prop:adjunction}]
Let $C$ be a conilpotent dg comodule 
over $B_1$ and $D$ be a dg 
conilpotent dg comodule over $B_0$.
We want to show that
$$
Hom_{B_1}(C,\, \lambda^*D) = 
Hom_{B_0}(\lambda_\# C,\, D)
$$
as sets.

We will give maps 
$$
\Phi: Hom_{B_0}
(\lambda_\# C,\, D) \leftrightarrows Hom_{B_1}
(C,\, \lambda^*D): \Phi^{-1}
$$ 
satisfying 
$\Phi \circ \Phi^{-1} = id$ and 
$\Phi^{-1} \circ \Phi = id$. 

First, we define 
$\Phi$. Let 
$F$ be a morphism from $\lambda_\# C$ to $D$. 
By defintion, for $f \in Obj(B_0)$, we 
have maps of complexes
$$
F_f: 
\bigoplus \limits_{f^\prime \in \lambda^{-1}f}
C^\bullet(f^\prime)
\to D^\bullet(f).
$$
Define $\Phi F \in Hom_{B_1}(C,\, \lambda^*D)$ 
as follows: for $f^\prime \in Obj(B_1)$,
\begin{equation}\label{eqn:Phi_F}
\begin{split}
\Phi F_{f^\prime}:
C^\bullet(f^\prime)
&
\xrightarrow{\Delta_C}
\bigoplus \limits_{h^\prime \in Obj(B_1)}
  B_1^\bullet(f^\prime, h^\prime) 
  \otimes C^\bullet(h^\prime) \\
&
\xrightarrow{\bigoplus 
  \limits_{h^\prime}
  id_{B_1}\otimes 
  F_{\lambda h^\prime}|_{h^\prime}}
\bigoplus \limits_{h^\prime \in Obj(B_1)}
  B_1^\bullet(f^\prime, h^\prime) 
  \otimes D^\bullet(\lambda h^\prime) \\
&
\xrightarrow{include}
[B_1 \otimes_\lambda D](f^\prime).
\end{split}
\end{equation}
By the universal property of $\lambda^*D$, 
this defines a morphism $C \to \lambda^*D$ 
if the two maps
$$
(id_{B_1}\otimes \Delta_D) \circ \Phi F, \;
(id_{B_1}\otimes \lambda \otimes id_D)\circ 
  (\Delta_{B_1}\otimes id_D) \circ 
  \Phi F: 
C \rightrightarrows
B_1 \otimes_\lambda B_0\otimes D  
$$
coincide. In fact, on $f^\prime \in Obj(B_1)$,
both maps are equal to:
\begin{align*}
C^\bullet(f^\prime)
& \xrightarrow{\Delta_C}
\bigoplus \limits_{h^\prime \in Obj(B_1)}
  B_1^\bullet(f^\prime, h^\prime)
  \otimes C^\bullet(h^\prime)\\
& \xrightarrow{\bigoplus \limits_{h^\prime} 
  id_{B_1} \otimes \Delta_C}
\bigoplus \limits_{g^\prime, h^\prime \in Obj(B_1)}
  B_1^\bullet(f^\prime, g^\prime)
  \otimes B_1^\bullet(g^\prime, h^\prime)
  \otimes C^\bullet(h^\prime)\\
& \xrightarrow{\bigoplus \limits_{h^\prime,g^\prime}
  id_{B_1}\otimes \lambda \otimes 1_{C}}
\bigoplus \limits_{g^\prime, h^\prime \in Obj(B_1)}
  B_1^\bullet(f^\prime, g^\prime) \otimes
  B_0^\bullet(\lambda g^\prime, \lambda h^\prime)
  \otimes C^\bullet(h^\prime)\\
& \xrightarrow{\bigoplus \limits_{h^\prime,g^\prime}
  id_{B_1}\otimes id_{B_0} \otimes 
  F_{\lambda h^\prime}|_{h^\prime}}
\bigoplus \limits_{g^\prime, h^\prime \in Obj(B_1)}
  B_1^\bullet(f^\prime, g^\prime) \otimes
  B_0^\bullet(\lambda g^\prime, \lambda h^\prime)
  \otimes D^\bullet(\lambda h^\prime).
\end{align*}
This fact follows from $F$ being a map of comodules.
It's also clear that $\Phi F$ commutes with 
coproducts and differentials. So, we've 
shown $\Phi F \in Hom_{B_1}(C,\, \lambda^*D)$.

Second, we define $\Phi^{-1}$. Now, let 
$F \in Hom_{B_1}(C, \lambda^*D)$. 
For $f \in Obj(B_0)$, define 
\begin{align*}
\Phi^{-1}F_f : 
\bigoplus \limits_{f^\prime \in \lambda^{-1}f}
  C^\bullet(f^\prime)
&\xrightarrow{\bigoplus \limits_{f^\prime} 
  F_{f^\prime}}
\bigoplus \limits_{\substack{f^\prime \in \lambda^{-1}f,\\
  h^\prime \in Obj(B_1)}}
  B_1^\bullet(f^\prime, h^\prime)
  \otimes D^\bullet(\lambda h^\prime)\\    
&\xrightarrow{\bigoplus \limits_{f^\prime, h^\prime}
  \lambda \otimes id_D}
\bigoplus \limits_{h \in Obj(B_0)}
  B_0^\bullet(f, h)
  \otimes D^\bullet(h) \\
&\xrightarrow{\bigoplus \limits_h
  \epsilon_{B_0} \otimes id_D}
D^\bullet(f).
\end{align*}
It's clear that $\Phi^{-1}F$ commutes with the 
differentials. We will show that
$\Phi^{-1}F$ is a map of comodules.
Figure \ref{fig:delta_phi-1} gives a diagram 
showing that 
\begin{equation}\label{eq:delta_phi-1}
\Delta_D \circ \Phi^{-1}F_f = 
(\bigoplus \limits_{f^\prime, h^\prime, r^\prime}
  \epsilon_{B_0} \lambda \otimes \lambda 
  \otimes id_{D}) \circ 
(\bigoplus \limits_{f^\prime, h^\prime}
  \Delta_{B_1} \otimes id_D) \circ  
(\bigoplus \limits_{f^\prime}
 F_{f^\prime}).
\end{equation}
On the other hand, Figure 
 \ref{fig:phi-1_delta} gives a diagram 
showing that 
\begin{equation} \label{eq:phi-1_delta}
(id_{B_1}\otimes 
\Phi^{-1}F) \circ \Delta_{\lambda_\# C} = 
(\bigoplus \limits_{f^\prime, h^\prime, r^\prime}
  \lambda \otimes \epsilon_{B_0} \lambda 
  \otimes id_{D}) \circ
(\bigoplus \limits_{f^\prime, h^\prime}
  \Delta_{B_1} \otimes id_D) \circ  
(\bigoplus \limits_{f^\prime}
 F_{f^\prime}).
\end{equation}
We see that the righthand sides of 
Equations \ref{eq:delta_phi-1} and 
\ref{eq:phi-1_delta} are the same 
except for the $B_0$ factor on which 
$\epsilon_{B_0}$ acts. However, in general,
for $\lambda: B_1 \to B_0$ a map 
of dg cocategories, we have
\begin{align*}
(\lambda \otimes \epsilon_{B_0} \lambda)
  \circ \Delta_{B_1}
&=
(id_{B_0} \otimes \epsilon_{B_0}) \circ
  \Delta_{B_0} \circ \lambda
\quad \textrm{($\lambda$ commutes with coproduct)}\\
&= 
id_{B_0} \circ \lambda  
\quad \textrm{(definition of cocategory)}\\
&= 
(\epsilon_{B_0} \otimes id_{B_0}) \circ
  (\Delta_{B_0}) \circ \lambda
\quad \textrm{(definition of cocategory)}\\
&=
(\epsilon_{B_0} \lambda \otimes \lambda)
  \circ \Delta_{B_1}
\quad \textrm{($\lambda$ commutes with coproduct).}  
\end{align*}
So, $(id_{B_1}\otimes \Phi^{-1}F) \circ 
\Delta_{\lambda_\# C} = \Delta_D \circ \Phi^{-1}F$, 
and $\Phi^{-1}F \in Hom_{B_0}(\lambda_\#C, D)$.

For $F: C \to \lambda^*D$ a map of 
dg comodules and $f^\prime 
\in B_1$, Figure \ref{fig:phi_phi-1} 
shows that $\Phi\Phi^{-1}F_{f^\prime}
 = F_{f^\prime}$.
For $F: \lambda_\# C \to D$ a map of 
dg comodules and $f 
\in B_0$, Figure \ref{fig:phi-1_phi} 
shows that $\Phi^{-1}\Phi F_f= F_f$. 
Thus, we have $\Phi\Phi^{-1} = id$ 
and $\Phi^{-1}\Phi = id$.
\end{proof}
%
\begin{landscape}
\begin{figure}
\centerline{\xymatrixrowsep{5pc}
\xymatrix{
\bigoplus \limits_{f^\prime \in \lambda^{-1}f}
  C^\bullet(f^\prime)
\ar@[red][d]^{\bigoplus \limits_{f^\prime} F_{f^\prime}}\\
%
\bigoplus \limits_{\substack{
  f^\prime \in \lambda^{-1}f\\
  h^\prime \in Obj(B_1)}}
  B_1^\bullet(f^\prime, h^\prime)
  \otimes D^\bullet(\lambda h^\prime)
\ar@[red][r]^{\bigoplus \limits_{f^\prime, h^\prime} 
  \lambda \otimes id_{D}}  
\ar@/_/[d]_{\bigoplus \limits_{f^\prime, h^\prime, r^\prime}
  \substack{
  (\Delta_{B_1} \otimes id_D) \circ \\
  (id_{B_1} \otimes \lambda \otimes id_D)}}
\ar@/^/[d]^{\bigoplus \limits_{f^\prime, h^\prime} 
  id_{B_1} \otimes \Delta_{D}}
& \bigoplus \limits_{h \in Obj(B_0)}
  B_0^\bullet(f,h) \otimes D^\bullet(h)
\ar@[red][r]^{\bigoplus \limits_h 
  \epsilon_{B_0} \otimes id_D} 
& D^\bullet(f)
\ar@[red][d]^{\Delta_D} \\
%
\bigoplus \limits{\substack{
  f^\prime \in \lambda^{-1}f,\\
  h^\prime \in Obj(B_1),\\
  r \in Obj(B_0)}}
  B_0^\bullet(f^\prime, h^\prime) \otimes 
  B_1^\bullet(\lambda h^\prime, r) 
  \otimes D^\bullet(r)
\ar[rr]_{\bigoplus \limits_{f^\prime, h^\prime, r}
  \epsilon_{B_0} \lambda \otimes id_{B_1} 
  \otimes id_D}
& & \bigoplus \limits_{h \in Obj(B_0)}
  B_0^\bullet(f,h) \otimes D^\bullet(h) 
}}
\caption{Commuting diagram 
involving $\Delta_D \circ 
\Phi^{-1}F$} \label{fig:delta_phi-1}
$\Delta_D \circ 
\Phi^{-1}F$ = composition 
of red arrows. 
The fact that $F: C \to \lambda^*D$ and 
the universal property of $\lambda^*D$ 
imply that the diagram commutes.  
\end{figure}
%
\begin{figure}
\centerline{\xymatrixrowsep{5pc}
\xymatrix{
\bigoplus \limits_{f^\prime \in \lambda^{-1}f}
  C^\bullet(f^\prime)
\ar[d]^{\bigoplus \limits_{f^\prime} F_{f^\prime}}
\ar@[red][r]^{\bigoplus \limits_{f^\prime} \Delta_C}
& \bigoplus \limits_{\substack{
  f^\prime \in \lambda^{-1}f\\
  h^\prime \in Obj(B_1)}}
  B_1^\bullet(f^\prime, h^\prime)
  \otimes C^\bullet(h^\prime)
\ar@[red][r]^{\bigoplus \limits_{f^\prime, h^\prime} 
  \lambda \otimes id_{C}}  
\ar[d]^{\bigoplus \limits_{f^\prime, h^\prime}
  id_{B_1} \otimes 
  F_{\lambda h^\prime}|_{h^\prime}}
& \bigoplus \limits_{h^\prime \in Obj(B_1)}
  B_0^\bullet(f,\lambda h^\prime) 
  \otimes C^\bullet(h^\prime)
\ar@[red][d]^{\bigoplus \limits_{h^\prime}
  id_{B_0} \otimes 
  F_{\lambda h^\prime}|_{h^\prime}}\\
%  
\bigoplus \limits{\substack{
  f^\prime \in \lambda^{-1}f,\\
  r^\prime \in Obj(B_1)}}
  B_1^\bullet(f^\prime, r^\prime) 
  \otimes D^\bullet(\lambda r^\prime)
\ar[r]^{\substack{\Delta_{\lambda^*D} =\\
  \bigoplus \limits_{f^\prime, r^\prime}
  \Delta_{B_1} \otimes id_D}}
& \bigoplus \limits{\substack{
  f^\prime \in \lambda^{-1}f,\\
  h^\prime, r^\prime \in Obj(B_1)}}
  B_1^\bullet(f^\prime, h^\prime) \otimes
  B_1^\bullet(h^\prime, r^\prime) 
  \otimes D^\bullet(\lambda r^\prime)
\ar[r]^{\bigoplus 
  \limits_{f^\prime, h^\prime, r^\prime}
  \lambda \otimes id_{B_0} \otimes id_D}
& \bigoplus \limits_{h^\prime, r^\prime 
  \in Obj(B_1)}
  B_0^\bullet(f,\lambda h^\prime) \otimes
  B_1^\bullet(h^\prime, r^\prime)
  \otimes D^\bullet(\lambda r^\prime)  
\ar@[red][d]_{\bigoplus \limits_{h^\prime, r^\prime}
  id_{B_0} \otimes 
  \epsilon_{B_0} \lambda \otimes id_{D}}\\
%
& & \bigoplus \limits_{h\in Obj(B_0)}
  B_0^\bullet(f,h) \otimes 
  D^\bullet(h)
}}
\caption{Commuting diagram
involving $(id_{B_1}\otimes 
\Phi^{-1}F) \circ \Delta_{\lambda_\# C}$} \label{fig:phi-1_delta}
$(id_{B_1}\otimes 
\Phi^{-1}F) \circ \Delta_{\lambda_\# C}$
= composition of red arrows. 
The fact that $F$ respects coproducts  
implies that the left square commutes. 
\end{figure}
%
\begin{figure}
\centerline{\xymatrixrowsep{5pc}
\xymatrix{
C^\bullet(f^\prime)
\ar@[red][r]^{\Delta_C}
\ar[d]^{F_{f^\prime}}
& \bigoplus \limits_{g^\prime \in Obj(B_1)}
  B_1^\bullet(f^\prime, g^\prime)
  \otimes C^\bullet(g^\prime)
\ar@[red][d]^{\bigoplus \limits_{g^\prime}
  id_{B_1}\otimes F_{g^\prime}}\\
%
\bigoplus \limits_{h^\prime \in Obj(B_1)}  
  B_1^\bullet(f^\prime, h^\prime)
  \otimes D^\bullet(\lambda h^\prime)
\ar[r]^{\Delta_{\lambda^* D} = \bigoplus
  \limits_{h^\prime} \Delta_{B_1} \otimes id_D}
\ar@/_3pc/[rr]^{id}  
& \bigoplus \limits_{g^\prime, h^\prime \in Obj(B_1)}
  B_1^\bullet(f^\prime, g^\prime)
  \otimes B_1^\bullet(g^\prime, h^\prime)
  \otimes D^\bullet(\lambda h^\prime)
\ar@[red][r]^{\bigoplus \limits_{g^\prime, h^\prime}
  id_{B_1} \otimes 
  (\epsilon_{B_0}\lambda = \epsilon_{B_1})
  \otimes id_D}
& \bigoplus \limits_{g^\prime \in Obj(B_1)}
  B_1^\bullet(f^\prime, g^\prime)
  \otimes D^\bullet(\lambda g^\prime)
}}
\caption{Commuting diagram 
involving $\Phi\Phi^{-1}F_{f^\prime}$}  \label{fig:phi_phi-1}
$\Phi\Phi^{-1}F_{f^\prime}$
= composition of red arrows. 
The square commutes because $F$ 
respects coproducts; 
the composition of the bottom row 
of horizontal arrows is equal to 
the identity because $\lambda_\#D$ 
satisfies counitality.
\end{figure}
%
\begin{figure}
\centerline{
\xymatrixrowsep{5pc}
\xymatrixcolsep{5pc}
\xymatrix{
\bigoplus \limits_{f^\prime \in \lambda^{-1}f}
  C^\bullet(f^\prime)
\ar@[red][r]^{\bigoplus \limits_{f^\prime}
  \Delta_C}
\ar[dd]_{F_f}
& \bigoplus \limits_{\substack{
  f^\prime \in \lambda^{-1}f,\\
  g^\prime \in Obj(B_1)}}
  B_1^\bullet(f^\prime, g^\prime)
  \otimes C^\bullet(g^\prime)
\ar@[red][r]^{\bigoplus \limits_{
  f^\prime, g^\prime} 
  id_{B_1} \otimes 
  F_{\lambda g^\prime}|_{g^\prime}}
\ar[d]^{\bigoplus \limits_{
  f^\prime, g^\prime} 
  \lambda \otimes id_C}
& \bigoplus \limits_{\substack{
  f^\prime \in \lambda^{-1}f,\\
  g^\prime \in Obj(B_1)}} 
  B_1^\bullet(f^\prime, g^\prime)
  \otimes D\bullet(\lambda g^\prime)
\ar@[red][d]^{\bigoplus \limits_{
  f^\prime, g^\prime} 
  \lambda \otimes id_D}\\
%
& \bigoplus \limits_{g^\prime \in Obj(B_1)}
  B_0^\bullet(f, \lambda g^\prime)
  \otimes C^\bullet(g^\prime)
\ar[r]^{\bigoplus \limits_{g^\prime}
  id_{B_0} \otimes 
  F_{\lambda g^\prime}|_{g^\prime}}
& \bigoplus \limits_{g \in Obj(B_0)}
  B_0^\bullet(f, g)
  \otimes D^\bullet(g)
\ar@[red][r]^{\bigoplus \limits_g
  \epsilon_{B_0} \otimes id_D}
& D^\bullet(f)\\
%
D^\bullet(f)
\ar[rru]_{\Delta_D}  
\ar@/_3pc/[rrru]_{id}
}}
\caption{Commuting diagram 
involving $\Phi^{-1}\Phi F_f$}  \label{fig:phi-1_phi}
$\Phi^{-1}\Phi F_f$ = composition of red arrows. 
The concave pentagon on the left 
side commutes because $F$ 
respects coproducts; the triangle in the 
bottom right corner commutes 
because $D$ satisfies counitality.
\end{figure}
\end{landscape}