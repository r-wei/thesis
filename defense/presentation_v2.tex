\documentclass[t]{beamer}
\usetheme{AnnArbor}
\usecolortheme{dolphin}

\usepackage{color}
\usepackage[all]{xy}
\usepackage{bm}
\usepackage{amsthm}
\usepackage{amsmath}
\usepackage{ulem} %\sout
\usepackage{relsize} %\mathlarger


\theoremstyle{plain}
\newtheorem{prop}[theorem]{Proposition}
\theoremstyle{example}
\newtheorem{eg}{Example}[section]
\theoremstyle{definition}
\newtheorem{defn}{Definition}[section]

%make dots smaller
\newcommand \smdots{\makebox[1em][c]{.\hfil.\hfil.}}
\let \dots \smdots

\newcommand{\leftmapsto}{\leftarrow\!\shortmid} %\left mapsto

%extensible rightrightarrows
\makeatletter
\newcommand*{\relrelbarsep}{.386ex}
\newcommand*{\relrelbar}{%
  \mathrel{%
    \mathpalette\@relrelbar\relrelbarsep
  }%
}
\newcommand*{\@relrelbar}[2]{%
  \raise#2\hbox to 0pt{$\m@th#1\relbar$\hss}%
  \lower#2\hbox{$\m@th#1\relbar$}%
}
\providecommand*{\rightrightarrowsfill@}{%
  \arrowfill@\relrelbar\relrelbar\rightrightarrows
}
\providecommand*{\leftleftarrowsfill@}{%
  \arrowfill@\leftleftarrows\relrelbar\relrelbar
}
\providecommand*{\xrightrightarrows}[2][]{%
  \ext@arrow 0359\rightrightarrowsfill@{#1}{#2}%
}
\providecommand*{\xleftleftarrows}[2][]{%
  \ext@arrow 3095\leftleftarrowsfill@{#1}{#2}%
}
\makeatother

%info for title slide
\title[]{What do algebras form?}
%\subtitle{Subtitle}
\author{Rebecca Wei}
\institute{Northwestern University}
\date{Jan 25, 2017}
%\subject{Subject}

\begin{document}

\frame{\titlepage}
%\frame{\tableofcontents}

\frame{
	\frametitle{Outline}
	\begin{itemize}
	\item \textbf{Question:} What do algebras form?
	$\newline$
	\item \textbf{Answer 1:} A category in categories ($HH^0$)
	\item \textbf{Derived Answer 1:} A category in dg cocategories (Hochschild cochains...)
	$\newline$
	\item \textbf{Answer 2:} A 2-category with a trace functor ($HH_0$)
	\item \textbf{Derived Answer 2:} A category in dg cocategories with a trace functor (Hochschild chains...)
	%
	\only<2>
	{\textcolor{blue}{up to homotopy}}
	\end{itemize}
}

\frame{
	\textbf{Answer 1:} Algebras form a \textbf{2-category}.\\
	$\newline$
	\begin{itemize}
	\item Objects: algebras $A, B, \dots$
	\item 1-Morphisms: 
		\only<1>{bimodules $_AM_B$}
		\only<2->{\textcolor{blue}{$_fB$, $f:A \to B$ map of algebras}}
	\item 1-Composition: $_AM_B \otimes_B { _B}N_C$
	\item 2-Morphisms: 
		\only<1-2>{morphisms of bimodules}
		\only<3->{\textcolor{blue}{
		\begin{align*}
		\{\textrm{maps of bimodules } _fB \to _gB \}
		&\cong
		Z_A(_gB_f) \cong HH^0(A,\, _gB_f) \\
		M
		&\mapsto
		M(1)\\
		(M_b: b^\prime \mapsto b \cdot b^\prime)
		&\leftmapsto 
		b\\
		\end{align*}}}
	\end{itemize}
	%
	\only<4->
	{\begin{center}
	Can we use Hochschild cohomology or cochains instead of $HH^0$?
	\end{center}}
}

\frame{
	\textbf{Derived Answer 1:} Algebras form a category in dg cocategories.
	$\newline$
	\begin{itemize}
	\item Objects: algebras $A, B, \dots$
	\item Morphisms: a dg cocategory $Bar(Hoch(A,B))$
	\item Composition: $\bullet: Bar(Hoch(A,B)) \otimes Bar(Hoch(B,C)) \to 
	Bar(Hoch(A,C))$ associative map of dg cocategories
	\end{itemize}
}

\frame{
	\textbf{Defining $Bar(Hoch(A,B))$}\\
	\begin{enumerate}
	%
	\item $Hoch(A,B)$ is a dg category with 
	\begin{itemize}
		\item Objects: algebra maps $f:A \to B$ 
		\item Morphisms: $Hoch(A)(f,g) = 
		(C^\bullet(A,\, _fB_g),\, _f\delta_g)$
		%
		\item Composition: cup product on cochains
		\only<1>
		{\begin{align*}
		\phi &\in C^p(A, _fB_g)\\
		\psi &\in C^q(A, _gB_h)\\
		(\phi \cup \psi)(a_1,\dots,a_{p+q})
		&= \pm \phi(a_1, \dots, a_p)\psi(a_{p+1},\dots,a_q)
		\end{align*}}
	\end{itemize}
	%
	\only<2->
	{\item $Bar: DGCat \to DGCocat$\\}
	%
	\only<3>
	{Bar(Hoch(A,B)) has the same objects as Hoch(A,B).\\
	\begin{columns}
	\begin{column}{0.4\textwidth}
   	\resizebox{0.45\vsize}{!}
	   {\xymatrix{
		A 
		\ar@/^4pc/[rrr]^{f_0}_{\substack{\\\textcolor{blue}{\Downarrow \phi_1}}}
		\ar@/^2pc/[rrr]^{f_1}_{\substack{\\\textcolor{blue}{\Downarrow \phi_2}}}
		\ar@/^0pc/[rrr]^{f_2}_{\textcolor{blue}{\vdots}}
		\ar@/_3pc/[rrr]^{f_{n-1}}_{\substack{\\\textcolor{blue}{\Downarrow \phi_n}}}
		\ar@/_4pc/[rrr]_{f_n} 
		&&&	B}}
	\end{column}
	\begin{column}{0.6\textwidth} 
		A morphism from $f_0$ to $f_n$ in Bar(Hoch(A,B)) \\ $\newline$
		$\Delta(\phi_1 \dots \phi_n) = \sum \limits_{0 \leq i \leq n} \pm 
		\phi_1 \dots \phi_i \otimes \phi_{i+1} \dots \phi_n$\\
		$|\phi_1 \dots \phi_n| = \sum \limits_{1 \leq i \leq n} |\phi_i| - n$\\
		$d_{Bar(Hoch(A,B))} = \tilde{d}_{Hoch(A,B)} + d_\cup$
	\end{column}
	\end{columns}
	}
	\end{enumerate}
}

\frame{
	\textbf{Derived Answer 1:} Algebra form a category in dg cocategories.
	$\newline$
	\begin{itemize}
	\item Objects: algebras $A, B, \dots$
	\item Morphisms: a dg cocategory $Bar(Hoch(A,B))$
	\item Composition: $\bullet: Bar(Hoch(A,B)) \otimes Bar(Hoch(B,C)) \to 
	Bar(Hoch(A,C))$ associative map of dg cocategories
	\end{itemize}
}

\frame{
	$Bar(Hoch(A,C))$ is the cofree dg cocategory on $Hoch(A,C)$:
	$$\resizebox{.9\vsize}{!}
	  {\xymatrix{
	  Bar(Hoch(A,B)) \otimes Bar(Hoch(B,C)) 
	  \ar[r]^{\quad\quad\quad\quad\quad \bullet}
	  \ar@{.>}[rd]_\bullet
	  & Hoch(A,C)\\
	  & Bar(Hoch(A,C))
	  \ar@{->>}[u]}}$$
	%
	\only<2->
	{$$\resizebox{.95\vsize}{!}
	  {\xymatrix{
		A 
		\ar@/^4pc/[rrrr]^{f_0 \in Obj(Bar(Hoch(A,B)))}_{\substack{\\\textcolor{blue}{\Downarrow \phi_1}}}
		\ar@/^2pc/[rrrr]^{f_1}_{\substack{\\\textcolor{blue}{\Downarrow \phi_2}}}
		\ar@/^0pc/[rrrr]^{f_2}_{\textcolor{blue}{\vdots}}
		\ar@/_3pc/[rrrr]^{f_{n-1}}_{\substack{\\\textcolor{blue}{\Downarrow \phi_n}}}
		\ar@/_4pc/[rrrr]_{f_n \in Obj(Bar(Hoch(A,B)))} 
		&&&&	B 
		\ar@/^1pc/[rrr]^{g_0 \in Obj(Bar(Hoch(B,C)))}_{\substack{\\\\\textcolor{blue}{\Downarrow \psi}}}
		\ar@/_1pc/[rrr]_{g_1\in Obj(Bar(Hoch(B,C)))} 
		&&& C 
		\ar@{|->}[r]^{\bullet}
		& A
		\ar@/^2pc/[rr]^{g_0f_0 \in Obj(Hoch(A,C))}_{\substack{\\\\\\ \textcolor{blue}{\psi\{\phi_1, \dots, \phi_n\}}\\ \textcolor{blue}{\Downarrow}}}
		\ar@/_2pc/[rr]_{g_1f_n \in Obj(Hoch(A,C))}
		&& C
	  }}$$
	}
	%
	\only<3>
	{\scalebox{.75}{\parbox{1\linewidth}{%
	\begin{align*}
	\psi\{\phi_1, \dots, \phi_n\}(a_1,\dots,a_q) = 
	\sum \pm \psi(
	&f_0a_1, \dots, f_0a_{i_1}, \phi_1(a_{i_1+1}, \dots), 
	  f_1a_*, \dots, f_1a_*, \\
	&\phi_2(a_*, \dots), f_2a_*, \dots
	, f_na_q)
	\end{align*}}}}
	%
	\only<4->
	{$$\resizebox{1.0\vsize}{!}
	  {\xymatrix{
		A 
		\ar@/^1pc/[r]^{f_0}_{\substack{\\\\\textcolor{blue}{\Downarrow \phi}}}
		\ar@/_1pc/[r]_{f_n} 
		&	B 
		\ar@/^1pc/[r]^{g_0}_{\substack{\\\\\textcolor{blue}{\Downarrow 1 \in k}}}
		\ar@/_1pc/[r]_{g_1} 
		& C 
		\ar@{|->}[r]^{\bullet}
		& A
		\ar@/^1pc/[r]^{g_0f_0}_{\substack{\\\\ \textcolor{blue}{\Downarrow \phi}}}
		\ar@/_1pc/[r]_{g_1f_n}
		& C
	  } \quad \quad
	  \xymatrix{
		A 
		\ar@/^1pc/[r]^{f_0}_{\substack{\\\\\textcolor{blue}{\Downarrow 1 \in k}}}
		\ar@/_1pc/[r]_{f_n} 
		&	B 
		\ar@/^1pc/[r]^{g_0}_{\substack{\\\\\textcolor{blue}{\Downarrow \psi}}}
		\ar@/_1pc/[r]_{g_1} 
		& 	C 
		\ar@{|->}[r]^{\bullet}
		& A
		\ar@/^1pc/[r]^{g_0f_0}_{\substack{\\\\ \textcolor{blue}{\Downarrow \psi}}}
		\ar@/_1pc/[r]_{g_1f_n}
		& C
	  }}$$}
	%
	\only<5->
	{Braces are associative. 
	(Getzler-Jones; Voronov-Gerstenhaber, Lyubashenko-Manzyuk; 
	Keller)}
}

\frame{
	\frametitle{Outline}
	\begin{itemize}
	\item \textbf{Question:} What do algebras form?
	$\newline$
	\item \textbf{Answer 1:} A category in categories ($HH^0$)
	\item \textbf{Derived Answer 1:} A category in dg cocategories (Hochschild cochains...)
	$\newline$
	\only<2->
	{\item \textcolor{blue}{Brief background on non-commutative calculus}}
	$\newline$
	\item \textbf{Answer 2:} A 2-category with a trace functor ($HH_0$)
	\item \textbf{Derived Answer 2:} A category in dg cocategories with a trace functor (Hochschild chains...)
	\textcolor{blue}{up to homotopy}
	\end{itemize}
}

\frame{
	\textbf{Non-commutative calculus}
	%
	\only<1->
	{\begin{theorem}(Hochschild-Kostant-Rosenberg, '62) Let $A$ be a regular, commutative algebra 
	over a field $k$ of characteristic 0. Then, 
	\begin{align*}
	(C_\bullet(A, A), b) 
	& \xrightarrow{\sim}
	\Omega^\bullet_{A/k}\\
	(C^\bullet(A,A), \delta)
	& \xrightarrow{\sim}
	\wedge^\bullet T_A 
	= \wedge^\bullet(Der_k(A,A)).
	\end{align*}
	\end{theorem}}
	%
	\only<2>
	{\begin{theorem}(Kontsevich, '97) Let $A=C^\infty(M)$ 
	for $M$ a smooth real manifold. Then, there is an 
	$L_\infty$ map $$(C^{\bullet+1}(A,A), \delta, [,]_{Ger})
	\xrightarrow{\sim} (\wedge^{\bullet+1} T_A, d=0, [,]_{SN}).$$
	\end{theorem}}
	%
	\only<3>
	{\begin{theorem}(Tamarkin, '98) Dependent on the 
	choice of a Drinfeld associator, there is a 
	$Ger_\infty$ map $$(C^{\bullet+1}(A,A), \delta, 
	[,]_{Ger}, \cup, \dots)
	\xrightarrow{\sim} 
	(\wedge^\bullet T_A, d=0, [,]_{SN}, \wedge).$$
	\end{theorem}}
	%
	\only<4-5>
	{\begin{theorem}(Dolgushev-Tamarkin-Tsygan, '08) 
	There is a $Calc_\infty$ map 
	$$(C^\bullet(A,A), C_{-\bullet}(A,A))
	\xrightarrow{\sim} 
	(\wedge^\bullet T_A, \Omega^\bullet_{A/k}).$$
	\end{theorem}}
}

\frame{
	\textbf{Answer 2:} Algebras form a 2-category with a trace functor
	\begin{defn} (Kaledin): A \underline{trace functor} on a 2-category $\mathcal{C}$ is:
	\begin{itemize}
		\item for each $A \in Obj(\mathcal{C})$, a functor 
		$TR_A: \mathcal{C}(A,A) \to k-mod$
		%
		\item for each pair $A, B \in Obj(\mathcal{C})$, a natural transformation
		$\tau_!(A,B)$
		\only<1>
		{$$\resizebox{0.9\vsize}{!}
		{\xymatrix{
		\mathcal{C}(A,B) \otimes \mathcal{C}(B,A)
		\ar[rr]^{\tau = flip}
		\ar[d]_{m}
		&& \mathcal{C}(B,A) \otimes \mathcal{C}(A,B)
		\ar[d]_{m} \\
		\mathcal{C}(A,A)
		\ar[rd]_{TR_A}
		& \textcolor{blue}{\substack{\Rightarrow\\ \tau_!(A,B)}}
		& \mathcal{C}(B,B) 
		\ar[ld]^{TR_B}\\
		& k-mod
		}}$$}
		%
		\item such that $\tau_!(B,A) \circ \tau_!(C,B) \circ \tau_!(A,C) = id$
		\only<2>
		{$$\resizebox{1.3\vsize}{!}
		{\xymatrixcolsep{0.25pc}
		\xymatrix{
		& \mathcal{C}(C,A) \otimes \mathcal{C}(A,C) \otimes \mathcal{C}(B,C) 
		\ar[rd]^{\tau}
		\ar[dd]_{\textcolor{blue}{\substack{\Leftarrow\\ \tau_!(B,A)}}}\\
		\mathcal{C}(A,B) \otimes \mathcal{C}(B,C) \otimes \mathcal{C}(C,A) 
		\ar[ru]^{\tau}
		\ar[rd]^{\textcolor{blue}{\quad \quad \quad \substack{\Rightarrow\\ \tau_!(A,C)}}}
		&& \mathcal{C}(B,C) \otimes \mathcal{C}(C,A) \otimes \mathcal{C}(A,B)
		\ar@/_1pc/[ll]_{\quad \quad \quad \quad \quad \tau}
		\ar[ld]_{\textcolor{blue}{\substack{\Rightarrow\\ \tau_!(C,B)}}\quad \quad\quad \quad}\\
		& k-mod
		}}$$}
	\end{itemize}
	\end{defn}
}

\frame{
	\textbf{Answer 2:} Algebras form 2-category with a trace functor
	\begin{defn} (Kaledin): A \underline{trace functor} on a 2-category $\mathcal{C}$ is:
	\begin{itemize}
		\item for each $A \in Obj(\mathcal{C})$, a functor 
		$TR_A: \mathcal{C}(A,A) \to k-mod$\\
		\textcolor{blue}{$TR_A: \textrm{bimodule } _AM_A \mapsto M/[A,M] \cong HH_0(A,M)$}
		%
		\only<2->
		{\item for each pair $A, B \in Obj(\mathcal{C})$, a natural transformation
		$\tau_!(A,B)$
		\begin{columns}[T]
		\begin{column}{0.55\textwidth}
			$$\resizebox{0.75\vsize}{!}
			{\xymatrix{
			\mathcal{C}(A,B) \otimes \mathcal{C}(B,A)
			\ar[rr]^{\tau = flip}
			\ar[d]_{m}
			&& \mathcal{C}(B,A) \otimes \mathcal{C}(A,B)
			\ar[d]_{m} \\
			\mathcal{C}(A,A)
			\ar[rd]_{TR_A}
			& \textcolor{blue}{\substack{\Rightarrow\\ \tau_!(A,B)}}
			& \mathcal{C}(B,B) 
			\ar[ld]^{TR_B}\\
			& k-mod
			}} $$
		\end{column}
		\begin{column}{0.45\textwidth} 
			\scalebox{.75}{\parbox{1\linewidth}{%
			\textcolor{blue}{\begin{align*}
			\frac{_AM_B \otimes_B { _B}N_A}{[A, M \otimes_B N]}
			&\xrightarrow{\tau_!(A,B)} 
			\frac{_BN_A \otimes_A { _A}M_B}{[B, N \otimes_A M]}\\
			m \otimes n
			&\mapsto
			n \otimes m
			\end{align*}}
			%
			such that\\ $\tau_!(B,A) \circ \tau_!(C,B) \circ \tau_!(A,C) = id$}}
		\end{column}
		\end{columns}
		}
	\end{itemize}
	\end{defn}
	%
	\only<3->
	{{\begin{center}
	Can we use Hochschild homology or chains instead of $HH_0$ 
	to extend this to a trace functor on the category in dg cocategories?
	\end{center}}}
}

\frame{
	\textbf{Massaging the definition of a trace functor}
	\begin{defn} (Kaledin): A \underline{trace functor} on a 
		\only<1>{2-category}
		\only<2-5>{\textcolor{blue}{category in k-linear categories}} 
		\only<6->{\textcolor{red}{category in dg cocategories}}
		$\mathcal{C}$ is:
	\begin{itemize}
		\item for each $A \in Obj(\mathcal{C})$, a 
		\only<1-2>{functor $TR_A: \mathcal{C}(A,A) \to k-mod$}
		\only<3-6>{\textcolor{blue}{left module $T(A)$ over $\mathcal{C}(A,A)$}}
		\only<7->{\textcolor{blue}{left }
		\textcolor{red}{dg comodule }
		\textcolor{blue}{$T(A)$ over $\mathcal{C}(A,A)$}}
		%
		\only<1-6>
		{$$\mathcal{C}(A,A)(f,g) \otimes_k TR_A(g) \to TR_A(f)$$}
		\only<7->{$$
		\textcolor{red}{\prod \limits_{g \in Obj(\mathcal{C})}}
		\mathcal{C}(A,A)
		\textcolor{red}{^\bullet}(f,g) \otimes_k TR_A
		\textcolor{red}{^\bullet}(g) 
		\textcolor{red}{\leftarrow} TR_A
		\textcolor{red}{^\bullet}(f)$$}
		%
		\item for each pair $A, B \in Obj(\mathcal{C})$, a
		\only<1-3>{natural transformation $\tau_!(A,B)$}
		\only<4-7>{\textcolor{blue}{map of modules 
		$\tau_!(A,B): m^*T(A) \to \tau^*m^*T(B)$ 
		over $\mathcal{C}(A,B) \otimes \mathcal{C}(B,A)$}}
		\only<8->{\textcolor{red}{map of dg comodules}
		\textcolor{blue}{$\tau_!(A,B): m^*T(A) \to \tau^*m^*T(B)$ 
		over $\mathcal{C}(A,B) \otimes \mathcal{C}(B,A)$}}
		%
		$$\resizebox{0.75\vsize}{!}
		{\xymatrix{
		\mathcal{C}(A,B) \otimes \mathcal{C}(B,A)
		\ar[rr]^{\tau}
		\ar[d]_{m}
		&& \mathcal{C}(B,A) \otimes \mathcal{C}(A,B)
		\ar[d]_{m} \\
		\mathcal{C}(A,A)
		\ar[rd]_{TR_A}
		& \textcolor{blue}{\substack{\Rightarrow\\ \tau_!(A,B)}}
		& \mathcal{C}(B,B) 
		\ar[ld]^{TR_B}\\
		& k-mod
		}}$$
		%
		\item
		such that 
		\only<1-4>
		{$\tau_!(B,A) \circ \tau_!(C,B) \circ \tau_!(A,C) = id$}
		\only<5->
		{\textcolor{blue}{$
		\tau^{*2} \tau_!(B,A) \circ 
		\tau^* \tau_!(C,B) \circ 
		\tau_!(A,C) = id$}}
	\end{itemize}
	\end{defn}
}

\frame{
	\textbf{Unraveling the definition of a trace functor}
	%
	\begin{defn} Let $\mathcal{C}$ be a category in dg cocategories. 
	Let $\chi(\mathcal{C})$ be the dg category with
	\begin{itemize}
	\item Objects = $\{A_0 \to \dots \to A_n \to A_0: A_i \in Obj(\mathcal{C}), n\geq 0\}$
	\item Morphisms = $\{$linear combinations of compositions of 
	{\small  
  	\setlength{\abovedisplayskip}{6pt}
  	\setlength{\belowdisplayskip}{\abovedisplayskip}
  	\setlength{\abovedisplayshortskip}{0pt}
  	\setlength{\belowdisplayshortskip}{3pt}
  	\begin{align*}
	rotations\; \tau_n
	&: 
	\mathcal{A}
	\mapsto (A_n \to A_0 \to \dots \to A_n) \\
	coboundaries\; \delta_{j,n}
	&: 
	\mathcal{A}\mapsto 
	(A_0 \to \dots \to A_j \to A_{j+2\,(mod\,n+1)} 
	\to \dots \to A_0) \\
	codegeneracies\; \sigma_{i,n}
	&: \mathcal{A}
	\mapsto 
	(A_0 \to \dots \to A_i \to A_i \to \dots \to A_0)
	\end{align*}}
	where $\mathcal{A}:= (A_0 \to \dots \to A_n \to A_0)$,
	subject to the cyclic relations$\}[0]$
	\end{itemize}
	\end{defn}
	%
	\only<2->
	{A trace functor on a category $\mathcal{C}$ 
	in dg cocategories will give: a dg functor from 
	$\chi(\mathcal{C}) \to \mathcal{D}$.}
}

\frame{
	\textbf{Unraveling the definition of a trace functor}
	%
	\begin{defn} Let $\mathcal{D}$ be the dg category with 
	\begin{itemize}
	\item Objects = $\{(\underset{B}{\textrm{dg cocategory}}, 
	\underset{C}{\textrm{dg comodule}})\}$
	\item Morphisms:
	\begin{align*}
	\mathcal{D}^p \big((B_1, C_1), (B_0,C_0) \big)
	&:= 
	\begin{Bmatrix}
	F: B_1 \to B_0 \; dg\; functor,\\
	F_!: C_1 \to F^*C_0 \; degree\textrm{-}p \; linear \; map
	\end{Bmatrix}\\
	&\phantom{:=}
	d_{\mathcal{D}}(F, F_!) = 
	(F, [d,F_!] = 
	{\scriptstyle d_{F^*C_0}\circ F_! \pm F_! \circ d_{C_1}})
	\end{align*}
	\end{itemize}
	\end{defn}
	%
	\only<2->
	{For us, $F^*C_0$ is the categorified version of 
	co-extension of scalars:
	$$F^*C_0 = ker(B_1 \otimes C_0 
	\xrightrightarrows[1 \otimes \Delta_{C_0}]{(1 \otimes F \otimes 1)
		(\Delta_{B_1} \otimes 1 )} 
	B_1 \otimes B_0 \otimes C_0)$$}
}

\frame{
	\textbf{Unraveling the definition of a trace functor}\\
	%
	Let $\mathcal{C}$ be a category in dg cocategories. 
	A trace functor on $\mathcal{C}$ gives a dg functor
	\begin{align*}
	\chi(\mathcal{C})
	&\to 
	\mathcal{D}\\
	%
	\only<2->
	{{\scriptstyle (A_0 \to \dots \to A_n \to A_0)}
	&\mapsto 
	\begin{pmatrix}
		\mathcal{C}(A_0,A_1) \otimes \dots \otimes \mathcal{C}(A_n,A_0),\\
		m^{*n} T(A_0),\;
		{\scriptstyle m^n: \mathcal{C}(A_0,A_1) \otimes \dots \otimes \mathcal{C}(A_n,A_0) \to \mathcal{C}(A_0,A_0)}
	\end{pmatrix}\\
	}
	%
	\only<3-5>
	{\delta_{j,n}
	&\mapsto
	\begin{pmatrix}
		{\scriptstyle \dots \otimes \mathcal{C}(A_j,A_{j+1})
		\otimes \mathcal{C}(A_{j+1},A_{j+2}) \otimes \dots
		\xrightarrow{\hat{\delta}_{j,n}=m}
		\dots \otimes \mathcal{C}(A_j,A_{j+2}) \otimes \dots}\\
		{\scriptstyle m^{*n}T(A_0)
		\xrightarrow{\delta_{j,n!} = id}
		\hat{\delta}_{j,n}^* m^{*n-1}T(A_0)
		\cong (m^{n-1}\hat{\delta}_{j,n})^* T(A_0)}
	\end{pmatrix}\\
	}
	%
	\only<4-5>
	{\sigma_{i,n}
	&\mapsto
	\begin{pmatrix}
		{\scriptstyle \dots \otimes \mathcal{C}(A_i,A_{i+1})
		 \otimes \dots
		\xrightarrow{\hat{\sigma}_{i,n}}
		\dots \otimes \mathcal{C}(A_i,A_i) 
		\otimes \mathcal{C}(A_i,A_{i+1}) \otimes \dots}\\
		{\scriptstyle m^{*n}T(A_0)
		\xrightarrow{\sigma_{i,n!} = id}
		\hat{\sigma}_{i,n}^* m^{*n+1}T(A_0)
		\cong (m^{n+1}\hat{\sigma}_{i,n})^* T(A_0)}
	\end{pmatrix}\\
	}
	%
	\only<5->
	{\tau_n
	&\mapsto
	\begin{pmatrix}
		{\scriptstyle \mathcal{C}(A_0,A_1) \otimes \dots 
		 \otimes \mathcal{C}(A_n,A_0)
		\xrightarrow{\hat{\tau}_n}
		\mathcal{C}(A_n,A_0) \otimes \dots 
		 \otimes \mathcal{C}(A_{n-1},A_n)}\\
		{\scriptstyle m^{*n}T(A_0)
		\xrightarrow{\tau_{n!} = m^{*n-1} \tau_!(A_0,A_n)}
		\hat{\tau}_n^* m^{*n}T(A_n)}\\
		{\scriptstyle m^{n-1}: 
		\big( \mathcal{C}(A_0,A_1) \otimes \dots \otimes 
		\mathcal{C}(A_{n-1},A_n) \big) \otimes \mathcal{C}(A_n,A_0)
		\to \mathcal{C}(A_0,A_n) \otimes \mathcal{C}(A_n,A_0)}
	\end{pmatrix}\\
	}
	\end{align*}
	%
	\only<6>
	{$\tau_n^{n+1} = id$ is preserved: 
	\begin{itemize}
	\item n=2 cocyle relation, 
	\item $n>2$ pullback of cocycle relation, 
	\item n=1 cocycle relation for $A,B,C=B$ and the fact that $\sigma_{1,1!}$ is an identity map
	\end{itemize}}
	%
	\only<7>
	{Functor is DG: $\delta_{j,n!} = id$, $\sigma_{i,n!} = id$, 
	$\tau_{n!} = m^{*n-1} \tau_!$ are maps of DG comodules.}
}

\frame{
	\textbf{Question:} Can we give a dg functor
	\begin{align*}
	\chi(\mathcal{C}) 
	&\to 
	\mathcal{D} \\
	{\scriptstyle \textrm{where }(A_0 \to \dots \to A_n \to A_0)}
	& \mapsto
	\begin{pmatrix}
	{\scriptstyle B(A_0 \to \dots \to A_n \to A_0) := }\\
	{\scriptstyle := Bar(Hoch(A_0, A_1)) \otimes \dots \otimes Bar(Hoch(A_n,A_0)),}\\
	{\scriptstyle C(A_0 \to \dots \to A_n \to A_0) := m^{*n}T(A_0)}
	\end{pmatrix}?
	\end{align*}
	%
	\only<2->
	{$\newline$\textbf{Answer: }
	No, but we can give an $A_\infty$-functor.\\
	This will imply that algebras form a category in dg cocategories 
	with a trace functor \textcolor{blue}{up to homotopy.}\\}
	%
	\only<3->
	{$\newline$\textbf{Rest of this talk: }
	\begin{itemize}
	\item Define dg comodules $C(A_0 \to \dots \to A_0)$ using Hochschild chains
	\item Describe the $A_\infty$-functor: $\tau_{1!}$, 
	$\tau_{n!}^{n+1} = m^{*n-1}\tau_{1!} \sim id$
	\end{itemize}}
}

\frame{
	Fix algebras $A_0,\dots,A_n$.
	Let $\mathcal{A} = (A_0 \to \dots \to A_n \to A_0)$.\\
	Define a dg comodule $C(\mathcal{A})$ over $B(\mathcal{A})$:
	$$C(\mathcal{A})^\bullet
	(\underbrace{
	A_0 \overset{f_{0,0}}{\to} \dots 
	\to A_n \overset{f_{n,0}}{\to} A_0 }_{\in Obj(B(\mathcal{A}))}):=$$
	%
	\begin{align*}
	\only<2->
	{&:= 
	\{
	\resizebox{.65\vsize}{!}
	{\xymatrix{
	A_0 \ar@/^4pc/[r]^{f_{0,0}} 
	\ar@/^2pc/[r]^{\textcolor{blue}{\Downarrow \phi_{0,1}}}_{f_{0,1}} 
	\ar@/_1pc/[r]^{\substack{\textcolor{blue}{\vdots}\\ f_{0,k_0}}}
	\ar@/_3pc/[rrrr]^{\substack{\mathlarger{\textcolor{blue}{\alpha}} \\\\}}_{ id_{A_0}}
	& A_1 \ar@{.>}@/^4pc/[r] 
	\ar@{.>}@/^2pc/[r]_{\substack{\\ \\ \textcolor{blue}{\vdots}}} 
	\ar@{.>}@/_1pc/[r]
	& \cdots \ar@{.>}@/^4pc/[r] 
	\ar@{.>}@/^2pc/[r]_{\substack{\\ \\ \textcolor{blue}{\vdots}}} 
	\ar@{.>}@/_1pc/[r]
	& A_n \ar@/^4pc/[r]^{f_{n,0}} 
	\ar@/^2pc/[r]^{\textcolor{blue}{\Downarrow \phi_{n,1}}}_{f_{n,1}} 
	\ar@/_1pc/[r]^{\substack{\textcolor{blue}{\vdots}\\ f_{n,k_n}}}
	& A_0 
	}}
	=
	\substack{(\phi_{0,1}|\dots|\phi_{0,k_0}) \otimes \dots \otimes
		(\phi_{n,1}|\dots|\phi_{n,k_n}) \otimes \alpha : \\\\
		s.t.\; \phi_{i,j} \in C^\bullet(A_i, _{f_{j-1}} {A_{i+1}}_{f_j}),\\
 		\alpha \in C_{-\bullet}(A_0, _{f_{n,k_n}\dots f_{0,k_0}} A_0)}\}\\}
 	%
 	\only<3->
 	{&\phantom{moveov}d_{C(A_0 \to \dots \to A_0)} = 
 	d_{B} \otimes id_{C_{-\bullet}} +  
 	id_{B} \otimes b + \tilde{\iota}}
	\end{align*}
}

\frame{
	where $\tilde{\iota}$ is given as follows:
	$$\resizebox{1.2\vsize}{!}
	{\xymatrix{
	C(A_0 \to \dots \to A_0)^\bullet
	  (f_{0,0},\dots,f_{n,0})
	\ar[r]^{\tilde{\iota}}
	\ar[rd]_{\substack{\textrm{extend as a}\\
		\textrm{coderivation}}}^{\tilde{\iota}}
	& C_{-\bullet}(A_0, _{f_{n,0}\dots f_{0,0}} A_0)\\
	& C(A_0 \to \dots \to A_0)^\bullet
	  (f_{0,0},\dots,f_{n,0})
	\ar@{->>}[u]
	}}
	$$

	\only<2->
	{\begin{align*}
	&\tilde{\iota}({\scriptstyle (\phi_{0,1}|\dots|\phi_{0,k_0}) 
	  \otimes \dots \otimes 
	  (\phi_{n,1}|\dots|\phi_{n,k_n}) \otimes \alpha})
	= \iota_{\scriptstyle (\phi_{0,1}|\dots|\phi_{0,k_0}) 
	  \bullet \dots \bullet
	 (\phi_{n,1}|\dots|\phi_{n,k_n})} \alpha\\
	& \iota_\phi(a_0 \otimes \dots a_p)
	= \pm \phi(a_{d+1}, \dots, a_p)\cdot a_0 \otimes 
	a_1 \otimes \dots \otimes a_d \quad 
	\textrm{where } |\phi|=p-d
	\end{align*}}
}

\frame{
	\textbf{Defining $\tau_{1!}$}
	%
	\begin{align*}
	\only<1->
	{C(A_0 \to A_1 \to A_0)^\bullet(f,g) 
	&\xrightarrow{\tau_{1!}} C(A_1 \to A_0 \to A_1)^\bullet(g,f)\\}
	%
	\only<1>
	{\resizebox{.35\vsize}{!}
	{\xymatrix{
	A_0 \ar@/^4pc/[r]^{f_{0,0}} 
	\ar@/^2pc/[r]^{\textcolor{blue}{\Downarrow \phi_{0,1}}}_{f_{0,1}} 
	\ar@/_1pc/[r]^{\substack{\textcolor{blue}{\vdots}\\ f_{0,k_0}}}
	\ar@/_2pc/[rr]^{\substack{\mathlarger{\textcolor{blue}{\alpha}} \\}}_{ id_{A_0}}
	& A_1 \ar@/^4pc/[r]^{f_{1,0}} 
	\ar@/^2pc/[r]^{\textcolor{blue}{\Downarrow \phi_{1,1}}}_{f_{1,1}} 
	\ar@/_1pc/[r]^{\substack{\textcolor{blue}{\vdots}\\ f_{1,k_1}}}
	& A_0 
	}}
	&\mapsto 
	\sum
	\resizebox{.35\vsize}{!}
	{\xymatrix{
	A_1 \ar@/^4pc/[r]^{f_{1,0}} 
	\ar@/^2pc/[r]^{\textcolor{blue}{\Downarrow \phi_{1,1}^\prime}}_{f_{1,1}} 
	\ar@/_1pc/[r]^{\substack{\textcolor{blue}{\vdots}\\ f_{1,k_1}}}
	\ar@/_2pc/[rr]^{\substack{\mathlarger{\textcolor{blue}{\alpha^\prime}} \\}}_{ id_{A_1}}
	& A_0 \ar@/^4pc/[r]^{f_{0,0}} 
	\ar@/^2pc/[r]^{\textcolor{blue}{\Downarrow \phi_{0,1}^\prime}}_{f_{0,1}} 
	\ar@/_1pc/[r]^{\substack{\textcolor{blue}{\vdots}\\ f_{0,k_0}}}
	& A_1 
	}}}
	%
	\only<2>
	{\resizebox{.35\vsize}{!}
	{\xymatrix{
	A_0 \ar@/^4pc/[r]^{f_{0,0}} 
	\ar@/^2pc/[r]^{\textcolor{blue}{\Downarrow \phi_{0,1}}}_{f_{0,1}} 
	\ar@/_1pc/[r]^{\substack{\textcolor{blue}{\vdots}\\ f_{0,k_0}}}
	\ar@/_2pc/[rr]^{\substack{\mathlarger{\textcolor{blue}{\alpha}} \\}}_{ id_{A_0}}
	& A_1 \ar@/^4pc/[r]^{f_{1,0}} 
	\ar@/^2pc/[r]^{\textcolor{blue}{\Downarrow \phi_{1,1}}}_{f_{1,1}} 
	\ar@/_1pc/[r]^{\substack{\textcolor{blue}{\vdots}\\ f_{1,k_1}}}
	& A_0 
	}}
	&\mapsto 
	\resizebox{.35\vsize}{!}
	{\xymatrix{
	A_1 
	\ar[r]^{f_{1,0}} 
	\ar@/_2pc/[rr]^{\substack{\mathlarger{\textcolor{blue}{\alpha^\prime}} \\}}_{ id_{A_1}}
	& A_0 
	\ar[r]^{f_{0,0}} 
	& A_1 
	}}\\}
	%
	\only<3-4>
	{\resizebox{.35\vsize}{!}
	{\xymatrix{
	A_0 
	\ar[r]^{f_{0,0}} 
	\ar@/_2pc/[rr]^{\substack{\mathlarger{\textcolor{blue}{\alpha}} \\}}_{ id_{A_0}}
	& A_1 
	\ar[r]^{f_{1,0}} 
	& A_0 }}
	&\overset{?}{\mapsto}
	\resizebox{.35\vsize}{!}
	{\xymatrix{
	A_1 
	\ar[r]^{f_{1,0}} 
	\ar@/_2pc/[rr]^{\substack{\mathlarger{\textcolor{blue}{\alpha^\prime}} \\}}_{ id_{A_1}}
	& A_0 
	\ar[r]^{f_{0,0}} 
	& A_1 
	}}\\}
	%
	\only<4>
	{\alpha = a_0 \otimes \dots \otimes a_n
	&\mapsto 
	\alpha^\prime = f_{0,0}(a_0) \otimes \dots 
	\otimes f_{0,0}(a_n)\\}
	%
	\only<5-9>
	{\resizebox{.35\vsize}{!}
	{\xymatrix{
	A_0 
	\ar@/^2pc/[r]^{f_{0,0}}_{\textcolor{blue}{\Downarrow \phi}}
	\ar[r]^{f_{0,1}}
	\ar@/_2pc/[rr]^{\substack{\mathlarger{\textcolor{blue}{\alpha}} \\}}_{ id_{A_0}}
	& A_1 
	\ar[r]^{f_{1,0}} 
	& A_0 }}
	&\overset{?}{\mapsto}
	\resizebox{.35\vsize}{!}
	{\xymatrix{
	A_1 
	\ar[r]^{f_{1,0}} 
	\ar@/_2pc/[rr]^{\substack{\mathlarger{\textcolor{blue}{\alpha^\prime}} \\}}_{ id_{A_1}}
	& A_0 
	\ar[r]^{f_{0,0}} 
	& A_1 
	}}\\}
	%
	\only<6>
	{\overline{\tau_{1!} \circ d} (\phi \otimes \alpha)
	& = 
	\overline{d\circ \tau_{1!}}(\phi \otimes \alpha)\\}
	%
	\only<7-9>
	{[b,\bar{\tau}_{1!}](\phi \otimes \alpha) \pm
	\bar{\tau}_{1!}(\delta\phi \otimes \alpha)
	&= 
	[\bar{\tau}_{1!}, \iota_{\phi}](\alpha)\\}
	\end{align*}
	%
	\begin{align*}
	\only<8>
	{L_\phi(\alpha) = 
	&\sum \limits_{k\geq 1}
	\pm a_0 \otimes \dots \otimes \phi(a_k,\dots)
	\otimes a_r \otimes \dots \otimes a_n +\\
	&\sum \pm \phi(a_k, \dots, a_n, a_0, \dots) \otimes 
	a_s \otimes \dots \otimes a_{k-1}\\
	[b,L_\phi] 
	&\pm L_{\delta\phi} = 0}
	%
	\only<9>
	{\textcolor{blue}{\bar{\tau}_{1!}}
	(\phi \otimes \alpha) = 
	&\sum \limits_{k\geq 1}
	\pm \textcolor{blue}{f_{0,0}}a_0 \otimes 
	\dots \otimes \phi(a_k,\dots) \otimes 
	\textcolor{blue}{f_{0,1}}a_r \dots \otimes 
	\textcolor{blue}{f_{0,1}}a_n +\\
	&\sum \pm \phi(\textcolor{blue}{f_{1,0}f_{0,1}} a_k, 
	\dots, \textcolor{blue}{f_{1,0}f_{0,1}}a_n, 
	a_0, \dots) \otimes 
	\textcolor{blue}{f_{0,1}}a_s \otimes \dots \otimes
	\textcolor{blue}{f_{0,1}} a_{k-1}}
	\end{align*}
}

\frame{	
	\textbf{Defining $\tau_{1!}$}
	%
	\begin{align*}
	&\bar{\tau}_{1!} \big(
	(\phi_{0,1}|\dots|\phi_{0,k_0}) \otimes 
	(\phi_{1,1}|\dots|\phi_{1,k_1}) \otimes 
	\alpha \big) = \\
	=  &\sum_{\substack{
			1\leq i \leq j_1 \leq \dots \leq j_{2k_1} \leq k_0,
			\\p}} 		
	  \pm \phi_{0,1} \big(
	  \substack{
	  f_{1,0}f_{0,i}a_p, \dots, 
	  f_{1,0}\phi_{0,i+1}(a_*, \dots), \\
	  f_{1,0}f_{0,i+1}a_*, \dots,
	  f_{1,0}\phi_{0,j_1}(a_*, \dots),\\
	  f_{1,0}f_{0,j_1}a_*, \dots,
	  \phi_{1,1}(f_{0,j_1}a_*, \dots, 
	  \phi_{0,j_1+1}(a_*, \dots), \dots), \dots,\\
	  \phi_{1,k_1}(f_{0,j_{2k_1-1}}a_*, \dots, 
	  \phi_{0,j_{2k_1-1}+1}(a_*, \dots), \dots), 
	  \dots, a_0, \dots} \big) \otimes \\
	& \phantom{moveovermove}
	\otimes f_{0,1}a_* \otimes \dots \otimes 
	  \phi_{0,2}(a_*, \dots ) \otimes 
	  f_{0,2}a_* \otimes \dots \otimes \\ 
	& \phantom{moveovermove} \otimes
	  \phi_{0,i}(a_*, \dots ) \otimes 
	  f_{0,i}a_* \otimes \dots f_{0,i}a_{p-1} + \\
	& \bigg( \sum \pm f_{0,0}a_0 \otimes \dots 
	\otimes \phi_{0,1}(a_*, \dots) \otimes \dots \otimes 
	 \phi_{0,n_0}(a_*,\dots) \otimes \\
	& \phantom{move}
	\otimes f_{0,n_0}a_* \otimes \dots \otimes 
	f_{0,n_0}a_n \quad \textrm{if $k_1 = 0$} \bigg)
	\end{align*}
}

\frame{
	\textbf{First homotopy: $\tau_{1!}^2 \sim id$} 
	\only<1->
	{$$\resizebox{1.2\vsize}{!}
	{\xymatrix{
	C(A_0 \to A_1 \to A_0)
	\ar[r]^{\tau_{1!}} 
	\ar@/_2pc/[rr]_{id}
	& \hat{\tau}_1^*C(A_1 \to A_0 \to A_1)
	\ar[r]^{\hat{\tau}_1^*\tau_{1!}} 
	& \hat{\tau}_1^{*2}C(A_0 \to A_1 \to A_0)
	}}$$}
	%
	\only<2-3>
	{$\resizebox{.35\vsize}{!}
	{\xymatrix{
	A_0 
	\ar[r]^{f_{0,0}} 
	\ar@/_2pc/[rr]^{\substack{\mathlarger{\textcolor{blue}{\alpha}} \\}}_{ id_{A_0}}
	& A_1 
	\ar[r]^{f_{1,0}} 
	& A_0 }}$\\
	$\alpha = a_0 \otimes \dots \otimes a_n 
	\overset{\tau_{1!}}{\mapsto} 
	f_{0,0}a_0 \otimes \dots \otimes f_{0,0}a_n
	\overset{\hat{\tau}_{1!}^*\tau_{1!}}{\mapsto}  
	f_{1,0}f_{0,0}a_0 \otimes \dots \otimes f_{1,0}f_{0,0}a_n$}
	%
	\only<3>
	{$\newline$ 
	\begin{align*}
	f_{1,0}f_{0,0}\alpha - \alpha
	&= [b, B] (\alpha)\\
	B(a_0 \otimes \dots \otimes a_n) 
	&= 
	\sum \limits_{0\leq i \leq n} \pm
	1 \otimes f_{1,0}f_{0,0}a_i \otimes 
	\dots \otimes f_{1,0}f_{0,0}a_n \otimes 
	a_0 \otimes \dots \otimes a_{i-1}
	\end{align*}}
	%
	\only<4>
	{\begin{align*}
	&B \big(
	(\phi_{0,1}|\dots|\phi_{0,k_0}) \otimes 
	(\phi_{1,1}|\dots|\phi_{1,k_1}) \otimes 
	\alpha \big) = \\
	=  &\sum_{\substack{
	   0 \leq j_1 \leq \dots \leq j_{2k_1} \leq k_0 
	   \\p}}
	\pm 1 \otimes  		
	  f_{1,0}f_{0,0}a_p \otimes \dots \otimes 
	  f_{1,0}\phi_{0,1}(a_*, \dots) \otimes \\
	&\phantom{moveovermove}  \otimes
	  f_{1,0}f_{0,1}a_* \otimes \dots \otimes
	  f_{1,0}\phi_{0,j_1}(a_*, \dots) \otimes \\
	&\phantom{moveovermove}  \otimes
	  f_{1,0}f_{0,j_1}a_* \otimes \dots \otimes
	  \phi_{1,1}(f_{0,j_1}a_*, \dots, 
	  \phi_{0,j_1+1}(a_*, \dots), \dots) \otimes \\
	&\phantom{moveovermove} \otimes	  
	  \dots \otimes \phi_{1,k_1}(f_{0,j_{2k_1-1}}a_*, \dots, 
	  \phi_{0,j_{2k_1-1}+1}(a_*, \dots), \dots) \otimes 
	  \dots \otimes \\
	&\phantom{moveovermove} \otimes	
	  a_0 \otimes \dots \otimes a_{p-1}
	  \end{align*}}
}

\frame{
	\textbf{In the language of $A_\infty$-functors}
	\begin{align*}
	\chi(\mathcal{C}) 
	&\to 
	\mathcal{D} \\
	{\scriptstyle \mathcal{A}:= (A_0 \to A_1 \to A_0)}
	& \mapsto
	\begin{pmatrix}
	{\scriptstyle B(\mathcal{A}), }\\
	{\scriptstyle C(\mathcal{A})}
	\end{pmatrix}\\
	%
	\tau_1
	&\mapsto 
	\begin{pmatrix}
	\hat{\tau}_1: B(\mathcal{A}) \to 
	  B(\tau_1 \mathcal{A})\\
	\tau_{1!}: C(\mathcal{A}) \to 
	  \hat{\tau}_1^*C(\tau_1\mathcal{A})
	\end{pmatrix}\\
	%
	(\tau_1, \tau_1)
	&\mapsto
	\begin{pmatrix}
	id: B(\mathcal{A}) \to B(\mathcal{A})\\
	B: C(\mathcal{A}) \to C(\mathcal{A})
	\end{pmatrix}\\
	%
	(\tau_1, \tau_1, \tau_1)
	&\mapsto
	\begin{pmatrix}
	\hat{\tau}_1: B(\mathcal{A}) \to 
	  B(\tau_1 \mathcal{A})\\
	0: C(\mathcal{A}) \to 
	  \hat{\tau}_1^*C(\tau_1\mathcal{A})
	\end{pmatrix}\\
	%
	&\vdots
	\end{align*}
}

\frame{
	\textbf{The $A_\infty$ relations mean:}\\
	%
	\begin{itemize}
	\item $\tau_{1!}$ is a map of complexes
	\item $B^2 = 0$
	\item The following diagram commutes:
	$$\xymatrix{
	C(\mathcal{A})
	\ar[r]^{\tau_{1!}}
	\ar[d]^{B}
	& \hat{\tau}_1^* C(\tau_1\mathcal{A})
	\ar[d]^{\hat{\tau}_1^*B}\\
	C(\mathcal{A})
	\ar[r]_{\tau_{1!}}
	& \hat{\tau}_1^* C(\tau_1\mathcal{A})
	}
	$$
	\end{itemize}
}		


\frame{
	For higher $n >1$, we want to find a 
	homotopy between $``\tau_{n!}^{n+1}$'' 
	and $id$.\\
	%
	\only<2->
	{However, it is sufficient to find a 
	homotopy between $\hat{\tau}_n^{*2} \delta_{0,n!} 
	\circ \hat{\tau}_n^* \tau_{n!} \circ \tau_{n!}$ and 
	$\hat{\delta}_{n-1,n}^* \tau_{n-1!} \circ 
	\delta_{n-1,n!}$.
	$$
	\xymatrix{
	[n]
	\ar[r]^{\tau_n}
	\ar[d]^{\delta_{n-1,n}}
	& [n]
	\ar[r]^{\tau_n}
	& [n]
	\ar[dl]^{\delta_{0,n}}\\
	[n-1]
	\ar[r]_{\tau_{n-1}}
	& [n-1]
	}
	$$}
	%
	\only<3->
	{\textbf{Strategy:} Find such a homotopy, 
	$\mathcal{B}$, for $n=2$, and use 
	$\hat{\delta}_0^{*n-2}\mathcal{B}$ for $n>2$.}
}

\frame{
	\begin{align*}
		\chi(\mathcal{C})
		&\to 
		\mathcal{D}\\
		%
		\mathcal{A}
		&\mapsto
		(B(\mathcal{A}), C(\mathcal{A}))\\
		%
		\mu = \tau_{n-1}\circ \delta_{n-1,n} = 
		\delta_{0,n}\circ \tau_n^2
		&\mapsto 
		\begin{pmatrix}
		\hat{\tau}_{n-1} \circ \hat{\delta}_{n-1,n}
		= \hat{\delta}_{0,n}\circ \hat{\tau}_n^2\\
		\hat{\tau}_n^{*2} \delta_{0,n!} \circ 
		  \hat{\tau}_n^*\tau_{n!} \circ \tau_{n!}
		\end{pmatrix}\\
		%
		(\delta_{0,n}, \tau_n^2)
		&\mapsto
		\begin{pmatrix}
		\hat{\delta}_{0,n}\circ \hat{\tau}_n^2\\
		0
		\end{pmatrix}\\
		%
		(\tau_{n-1}, \delta_{n-1,n})
		&\mapsto
		\begin{pmatrix}
		\hat{\tau}_{n-1}\circ \hat{\delta}_{n-1,n}\\
		\mathcal{B}
		\end{pmatrix}\\
		%
		(\tau_{n-1}, \delta_{n-1,n}, \lambda)
		&\mapsto
		\begin{pmatrix}
		\hat{\tau}_{n-1}\circ \hat{\delta}_{n-1,n} 
		  \circ \hat{\lambda} \\
		0
		\end{pmatrix}\\
		%
		&\vdots
	\end{align*}
}

\frame{
	For $n>1$, the $A_\infty$ relations mean:\\
	%
	\begin{itemize}
	\item $\tau_{n!}$ is a map of complexes
	\item $\mathcal{B}^2 = 0$
	%
	\only<1>
	{\item We have a pair of homotopic maps:
	$$\xymatrixrowsep{5pc}
	\xymatrix{
	C(A_0 \to \smdots \to A_n \to A_0) 
	 \ar@/^1pc/[d]^{\substack{
	   (\widehat{\delta_{n-2,n-1}\delta_{n-1,n}})^*
	   \tau_{n-2!}\\\\\\
	   \textrm{$``$brace together the last 3 algebras,}\\
	   \textrm{then apply $\tau_{n-2!}$ once''}}}
	 \ar@/_1pc/[d]_{\substack{
	   \hat{\tau}_n^{*2} \tau_{n!}\circ
	   \hat{\tau}_n^*\tau_{n!}\circ \tau_{n!}\\\\\\
	   \textrm{$``$apply $\tau_{n!}$ 3 times''}}}\\
	C(A_{n-2} \to A_{n-1} \to A_n \to 
	A_0 \to \smdots \to A_{n-2}).
	}$$}
	%
	\only<2>
	{\item They are homotopic via two homotopies:
	$$\resizebox{1.1\vsize}{!}
	{\xymatrix{
	\underset{\substack{\\\\
	  \textrm{$``$brace together $A_{n-2}, A_{n-1}, A_n$,}\\
	  \textrm{then apply $\tau_{n-2!}$''}}}
	  {(\widehat{\delta_{n-2,n-1}
	   \delta_{n-1,n}})^*\tau_{n-2!}}
	\ar[r]^{\cong}
	\ar[d]_{\cong}   
	& \hat{\delta}_{n-1,n}^*(
	  \hat{\delta}_{n-2,n-1}^*\tau_{n-2!})   
	\ar[r]^{\hat{\delta}_{n-1,n}^*\mathcal{B}_{n-1}}
	& \hat{\delta}_{n-1,n}^*(
	  \hat{\tau}_{n-1}^*\tau_{n-1!} \circ \tau_{n-1!})
	\ar[d]^{\cong}  \\
	(\widehat{\delta_{n-2,n-1}\delta_{n-2,n}})^*
	  \tau_{n-2!}  
	\ar[d]_{\hat{\delta}_{n-2,n}^*\mathcal{B}_{n-1}}
	&& \underset{\substack{ \\\\
	    \textrm{$``$brace together $A_{n-1}, A_n$}\\
	    \textrm{and apply $\tau_{n-1!}$,}\\
	    \textrm{then apply $\tau_{n!}$''}}}
	    {\hat{\tau}_n^{*2}\tau_{n!} \circ 
	    \hat{\delta}_{n-1,n}^*\tau_{n-1!}} 
	\ar[d]^{\substack{\\\\\\\\
	        \tau_n^{*2}\tau_{n!} \circ \mathcal{B}_n}}\\
	\hat{\delta}_{n-2,n}^*(
	  \hat{\tau}_{n-1}^*\tau_{n-1!} \circ \tau_{n-1!})
	\ar[r]_{\cong}                   
	& \underset{\substack{\\\\
	          \textrm{$``$apply $\tau_{n!}$,}\\
	          \textrm{then brace together $A_{n-1}, A_{n-2}$}\\
	          \textrm{and apply $\tau_{n-1!}$''}}}
	         {\hat{\tau}_n^*(\hat{\delta}_{n-1,n}^*
	          \tau_{n-1!}) \circ \tau_{n!}}
	\ar[r]^{\hat{\tau}_n^*\mathcal{B}_n \circ \tau_{n!}}
	& \underset{\substack{\\\\
	    \textrm{$``$apply $\tau_{n!}$ three times''}}}
	    {\hat{\tau}_n^{*2} \tau_{n!}\circ
	     \hat{\tau}_n^*\tau_{n!}\circ \tau_{n!}}
	}}$$}
	\end{itemize}
}

\frame{
	\textbf{Summary:}
	%
	We have a given an $A_\infty$-functor 
	$\chi(\mathcal{C}) \to \mathcal{D}$, 
	which implies that algebras form a 
	\textcolor{blue}{category in dg cocategories with 
	a trace up to homotopy}.\\
	%
	$\newline$
	To get a category in \textcolor{blue}{dg categories} 
	with a trace up to homotopy, apply (categorified) 
	Cobar(-).
}

\frame{
	$\newline\newline\newline\newline\newline$
	\begin{center} 
	\textbf{Thank you!}
	\end{center}
}

\end{document}